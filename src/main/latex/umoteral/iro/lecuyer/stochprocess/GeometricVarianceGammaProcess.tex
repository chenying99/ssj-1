%
%  Pierre Tremblay, July 2003.
%

\defclass {GeometricVarianceGammaProcess}

This class represents a \emph{geometric variance gamma} process $S(t)$
(see \cite[page 86]{fMAD98a}). This stochastic process is defined by the
equation
\eq
S(t) = S(0) \mbox{ exp}(\mu t + X(t; \sigma, \nu, \theta) + \omega t),
\label{GeoVGeqn}
\endeq
where $X$ is a variance gamma process and
\eq
\omega = (1/\nu) \mbox{ ln}( 1 - \theta \nu - \sigma^{2} \nu /2).
\label{omegaEqn}
\endeq

\bigskip\hrule\bigskip

%%%%%%%%%%%%%%%%%%%%%%%%%%%%%%%%%%%%%%%%%%%%%%%%%%%%%%%%%%%%%%%%%%
\begin{code}
\begin{hide}
/*
 * Class:        GeometricVarianceGammaProcess
 * Description:  
 * Environment:  Java
 * Software:     SSJ 
 * Copyright (C) 2001  Pierre L'Ecuyer and Université de Montréal
 * Organization: DIRO, Université de Montréal
 * @author       Pierre Tremblay
 * @since        July 2003

 * SSJ is free software: you can redistribute it and/or modify it under
 * the terms of the GNU General Public License (GPL) as published by the
 * Free Software Foundation, either version 3 of the License, or
 * any later version.

 * SSJ is distributed in the hope that it will be useful,
 * but WITHOUT ANY WARRANTY; without even the implied warranty of
 * MERCHANTABILITY or FITNESS FOR A PARTICULAR PURPOSE.  See the
 * GNU General Public License for more details.

 * A copy of the GNU General Public License is available at
   <a href="http://www.gnu.org/licenses">GPL licence site</a>.
 */
\end{hide}
package umontreal.iro.lecuyer.stochprocess;\begin{hide}
import umontreal.iro.lecuyer.rng.*;
import umontreal.iro.lecuyer.probdist.*;
import umontreal.iro.lecuyer.randvar.*;

\end{hide}

public class GeometricVarianceGammaProcess extends StochasticProcess \begin{hide} {
    protected VarianceGammaProcess vargamma;
    protected double        theta,
                            nu,
                            mu,
                            sigma,
                            omega,
                            muPlusOmega;
    protected double[]      mudt;
\end{hide}
\end{code}
%%%%%%%%%%%%%%%%%%%%%%%%%%%%%%%%%%%%%%%%%%%%%%%%%%%%%%%%%%%%%%%%
\subsubsection* {Constructors}
\begin{code}

   public GeometricVarianceGammaProcess (double s0, double theta,
                                         double sigma, double nu,
                                         double mu, RandomStream stream) \begin{hide} {
        vargamma = new VarianceGammaProcess (0.0, theta, sigma, nu, stream);
        setParams (s0, theta, sigma, nu, mu);
    }\end{hide}
\end{code}
\begin{tabb}
Constructs a new \texttt{GeometricVarianceGammaProcess} with parameters
$\theta = \texttt{theta}$, $\sigma = \texttt{sigma}$, $\nu = \texttt{nu}$,
$\mu = \texttt{mu}$ and initial value $S(t_{0}) = \texttt{s0}$.
The \texttt{stream} is  used to generate the \class{VarianceGammaProcess} object used to implement
$X$ in (\ref{GeoVGeqn}).
\end{tabb}
\begin{code}

   public GeometricVarianceGammaProcess (double s0, double mu,
                                         VarianceGammaProcess vargamma)
 \begin{hide} {
        this.vargamma = vargamma;
        setParams (s0, vargamma.getTheta (), vargamma.getSigma (),
                   vargamma.getNu (), mu);
    }\end{hide}
\end{code}
\begin{tabb}
Constructs a new \texttt{GeometricVarianceGammaProcess}.
The parameters $\theta, \sigma, \nu$ are set to the parameters of the
 \class{VarianceGammaProcess} \texttt{vargamma}. The parameter $\mu$
is set to \texttt{mu} and the initial values $S(t_{0}) = \texttt{s0}$.
\end{tabb}
%%%%%%%%%%%%%%%%%%%%%%%%%%%%%%%%%%%%%%
\subsubsection* {Methods}
\begin{hide}\begin{code}

   public double nextObservation() {
        double nextX  = vargamma.nextObservation();
        observationIndex = vargamma.getCurrentObservationIndex();
        // Could be different than simply 'observationIndex++' because of the
        // possibility of Gamma/Brownian bridge
        observationCounter++;

        double s = x0 * Math.exp (muPlusOmega * (t[observationIndex] - t[0])
                                  + nextX);
        path[observationIndex] = s;
        return s;
    }

   public double[] generatePath() {
        double s = x0;
        resetStartProcess();
        double[] vgpath = vargamma.generatePath();
        for (int i = 0; i < d; i++) {
            s *= Math.exp (mudt[i] + vgpath[i+1] - vgpath[i]);
            path[i+1] = s;
        }
        observationIndex = d;
        observationCounter++;
        return path;
    }
    // allows the user to create a path by specifiying the uniform random numbers to be used
   public double[] generatePath (double[] uniform01) {
        double s = x0;
        resetStartProcess();

        double[] vgpath = vargamma.generatePath(uniform01);
        for (int i = 0; i < d; i++) {
            s *= Math.exp (mudt[i] + vgpath[i+1] - vgpath[i]);
            path[i+1] = s;
        }
        observationIndex = d;
        observationCounter++;
        return path;
    }


    // method not verified by JS...  old stuff
   public double getCurrentUpperBound()  {
        // Find index for last observation generated (chronologically)
        int j = 0; // By default, t0 !
        int i = observationCounter - 1;
        double tForIthObserv;
        while (i > 0) {
            tForIthObserv = t[observationIndexFromCounter[i]];
            if (tForIthObserv <= t[observationCounter] && tForIthObserv > t[j])
                j = i;
            i--;
        }

        // Calculate bound following recipe
        double u = 0.0;
        GammaProcess gpos = ((VarianceGammaProcessDiff) vargamma).getGpos();
        double[] gposPath = gpos.getPath();
        double deltaGpos = gposPath[observationIndex] - gposPath[j];
        double s = path[observationIndex];
        if (muPlusOmega < 0)
             u = s * Math.exp (deltaGpos);
        else u = s * Math.exp (muPlusOmega * (t[observationIndex] - t[j])
                               + deltaGpos);
        return u;
    }
\end{code}
% \begin{tabb}
% Returns the upper bound defined as $U_{i}$ in eqns (11) and (12) of
% (CITE:AVRAM03), for $i$ the current observation index.
% It has the property that $S(t) \leq U_{i}$
% for $t_{j} \leq t \leq t_{i}$, where $j$ is the index of .
% \emph{Warning:} The variance gamma process $X$ must be an instance
% of \texttt{VarianceGammaProcessDiff} in order for this upper bound to be well defined.
% \end{tabb}
\end{hide}\begin{code}

   public void resetStartProcess() \begin{hide} {
        observationIndex   = 0;
        observationCounter = 0;
        vargamma.resetStartProcess();
    }\end{hide}
\end{code}
\begin{tabb} Resets the \texttt{GeometricaVarianceGammaProcess},
but also applies the \texttt{resetStartProcess} method to the
\class{VarianceGammaProcess} object used to generate this process.
\end{tabb}
\begin{code}

   public void setParams (double s0, double theta, double sigma, double nu,
                          double mu) \begin{hide} {
        this.x0    = s0;
        this.theta = theta;
        this.sigma = sigma;
        this.nu    = nu;
        this.mu    = mu;
        if (observationTimesSet) init(); // Otherwise no need to.
    }\end{hide}
\end{code}
\begin{tabb}
Sets the parameters
$S(t_{0}) = \texttt{s0}$, $\theta = \texttt{theta}$, $\sigma = \texttt{sigma}$,
$\nu = \texttt{nu}$ and $\mu = \texttt{mu}$ of the process.
\emph{Warning}: This method will recompute some quantities stored internally,
which may be slow if called repeatedly.
\end{tabb}
\begin{code}

   public double getTheta() \begin{hide} { return theta; }\end{hide}
\end{code}
\begin{tabb}
Returns the value of the parameter $\theta$.
\end{tabb}
\begin{code}

   public double getMu() \begin{hide} { return mu; }\end{hide}
\end{code}
\begin{tabb}
Returns the value of the parameter $\mu$.
\end{tabb}
\begin{code}

   public double getNu() \begin{hide} { return nu; }\end{hide}
\end{code}
\begin{tabb}
Returns the value of the parameter $\nu$.
\end{tabb}
\begin{code}

   public double getSigma() \begin{hide} { return sigma; }\end{hide}
\end{code}
\begin{tabb}
Returns the value of the parameter $\sigma$.
\end{tabb}
\begin{code}

   public double getOmega() \begin{hide} { return omega; }\end{hide}
\end{code}
\begin{tabb}
Returns the value of the quantity $\omega$ defined in (\ref{omegaEqn}).
\end{tabb}
\begin{code}

   public VarianceGammaProcess getVarianceGammaProcess() \begin{hide} {
        return vargamma;
}\end{hide}
\end{code}
\begin{tabb} Returns a reference to the variance gamma process $X$ defined
in the constructor.
\end{tabb}
\begin{code}\begin{hide}

    protected void init() {
        super.init();
        if (1 <= theta*nu + sigma*sigma*nu / 2.0)
           throw new IllegalArgumentException ("theta*nu + sigma*sigma*nu / 2 >= 1");
        omega = Math.log (1 - theta*nu - sigma*sigma*nu / 2.0) / nu;
        muPlusOmega = mu + omega;

        if (observationTimesSet) {
            // Telling the variance gamma proc. about the observ. times
            vargamma.setObservationTimes (t, d);

            // We need to know in which order the observations are generated
            this.observationIndexFromCounter
                = vargamma.getArrayMappingCounterToIndex();

            mudt = new double[d];
            for (int i = 0; i < d; i++)
                mudt[i] = muPlusOmega * (t[i+1] - t[i]);
        }
    }


   public void setStream (RandomStream stream)  {
        vargamma.setStream(stream);
   }


   public RandomStream getStream () {
      return vargamma.getStream();
   }

}
\end{hide}
\end{code}
