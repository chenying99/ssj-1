\defclass {GeometricLevyProcess}

Abstract class used as a parent class for the exponentiation
of a L\'evy process $X(t)$:
\begin{equation}
S(t) = S(0) \exp\left(X(t) + (r - \omega_{RN}) t\right).
\end{equation}
The interest is here denoted $r$ and is refered to
as \texttt{muGeom} in the class below.
The risk neutral correction is given by $\omega_{RN}$
and takes into account risk aversion in the pricing of
assets; its value depends on the specific L\'evy process
that is used.

\class{GeometricNormalInverseGaussianProcess} is
implemented as a child of this class
and so could \class{GeometricVarianceGammaProcess}
and \class{GeometricBrownianMotion}.


\bigskip\hrule\bigskip

%%%%%%%%%%%%%%%%%%%%%%%%%%%%%%%%%%%%%%%%%%%%%%%%%%%%%%%%%%%%%%%%%%
\begin{code}
\begin{hide}
/*
 * Class:        GeometricLevyProcess
 * Description:  
 * Environment:  Java
 * Software:     SSJ 
 * Copyright (C) 2001  Pierre L'Ecuyer and Université de Montréal
 * Organization: DIRO, Université de Montréal
 * @author       
 * @since

 * SSJ is free software: you can redistribute it and/or modify it under
 * the terms of the GNU General Public License (GPL) as published by the
 * Free Software Foundation, either version 3 of the License, or
 * any later version.

 * SSJ is distributed in the hope that it will be useful,
 * but WITHOUT ANY WARRANTY; without even the implied warranty of
 * MERCHANTABILITY or FITNESS FOR A PARTICULAR PURPOSE.  See the
 * GNU General Public License for more details.

 * A copy of the GNU General Public License is available at
   <a href="http://www.gnu.org/licenses">GPL licence site</a>.
 */
\end{hide}
package umontreal.iro.lecuyer.stochprocess;\begin{hide}
import umontreal.iro.lecuyer.rng.*;
import umontreal.iro.lecuyer.probdist.*;
import umontreal.iro.lecuyer.randvar.*;

\end{hide}

public abstract class GeometricLevyProcess extends StochasticProcess \begin{hide} {

    protected StochasticProcess levyProcess;
    protected double omegaRiskNeutralCorrection;
    protected double muGeom;  // usually the interest rate
    protected double[] muGeomRNdt; // risk neutral corrected 
    protected double[] muGeomRNdT; // risk neutral corrected, from time t0.

    protected void init()
    {
        super.init();
        if (observationTimesSet)
        {
            // Telling the variance gamma proc. about the observ. times
            levyProcess.setObservationTimes (t, d);

            // We need to know in which order the observations are generated
            this.observationIndexFromCounter 
                = levyProcess.getArrayMappingCounterToIndex();

            muGeomRNdt = new double[d];
            for (int i = 0; i < d; i++){
                muGeomRNdt[i] = (muGeom-omegaRiskNeutralCorrection) * 
                                (t[i+1] - t[i]);
            }
            muGeomRNdT = new double[d+1];
            for (int i = 0; i <= d; i++){
                muGeomRNdT[i] = (muGeom-omegaRiskNeutralCorrection) * 
                                (t[i] - t[0]);                
            }
        }
    }
\end{hide}
\end{code}
%%%%%%%%%%%%%%%%%%%%%%%%%%%%%%%%%%%%%%
\subsubsection* {Methods}
\begin{code}

   public double[] generatePath() \begin{hide} {
        double s = x0;
        resetStartProcess();
        double[] arithmPath = levyProcess.generatePath();
        for (int i = 0; i < d; i++)
        {
            s *= Math.exp (muGeomRNdt[i] + arithmPath[i+1] - arithmPath[i]);
            path[i+1] = s;
        }
        observationIndex = d;
        return path;
    }\end{hide}
\end{code}
\begin{tabb} Generates a path.
\end{tabb}
\begin{code}

   public double nextObservation() \begin{hide} {
       double levy = levyProcess.nextObservation();
       observationIndex = levyProcess.getCurrentObservationIndex();
       path[observationIndex] = x0 * 
             Math.exp( muGeomRNdT[observationIndex] + levy );
       return path[observationIndex];
    }\end{hide}
\end{code}
\begin{tabb} 
Returns the next observation. 
It will also work on a L\'evy process which is sampled using
the bridge order, but it will return the observations in 
the bridge order.
If the underlying L\'evy process is of the PCA type, this
method is not usable.
\end{tabb}
\begin{code}

   public void resetStartProcess() \begin{hide} {
        super.init();
        levyProcess.resetStartProcess();
    }\end{hide}
\end{code}
\begin{tabb} Resets the step counter of the geometric process and 
the underlying L\'evy process to the start value.
\end{tabb}
\begin{code}

   public void setObservationTimes(double[] time, int d) \begin{hide} {
        super.setObservationTimes(time, d);
        levyProcess.setObservationTimes(time, d);
    }\end{hide}
\end{code}
\begin{tabb} Sets the observation times on the geometric process
and the underlying L\'evy process.
\end{tabb}
\begin{code}

   public double getOmega() \begin{hide} {
        return omegaRiskNeutralCorrection;
    }\end{hide}
\end{code}
\begin{tabb} Returns the risk neutral correction.
\end{tabb}
\begin{code}

   public double getMuGeom() \begin{hide} {
        return muGeom;
    }\end{hide}
\end{code}
\begin{tabb} Returns the geometric drift parameter,
which is usually the interest rate, $r$.
\end{tabb}
\begin{code}

   public void setMuGeom (double muGeom) \begin{hide} {
        this.muGeom = muGeom;
    }\end{hide}
\end{code}
\begin{tabb} Sets the drift parameter (interest rate) of the geometric term.
\end{tabb}
\begin{code}

   public StochasticProcess getLevyProcess() \begin{hide} {
        return levyProcess;
    }\end{hide}
\end{code}
\begin{tabb} Returns the L\'evy process.
\end{tabb}
\begin{code}

   public void resetRiskNeutralCorrection (double omegaRN) \begin{hide} {
       omegaRiskNeutralCorrection = omegaRN;
       init();
    }\end{hide}
\end{code}
\begin{tabb}
Changes the value of $\omega_{RN}$.
There should usually be no need to redefine the risk neutral
correction from the value set by the constructor.  However it
is sometimes not unique, e.g. in \class{GeometricNormalInverseGaussianProcess}
\cite{fALB04a}.
\end{tabb}
\begin{code}

   public RandomStream getStream() \begin{hide} {
        return levyProcess.getStream();
    }\end{hide}
\end{code}
\begin{tabb} Returns the stream from the underlying L\'evy process.
If the underlying L\'evy process has multiple streams, it returns
what the \texttt{getStream()} method of that process was made to return.
\end{tabb}
\begin{code}

   public void setStream (RandomStream stream) \begin{hide} {
        levyProcess.setStream(stream);
    }\end{hide}
\end{code}
\begin{tabb} Resets the stream in the underlying L\'evy process.
If the underlying L\'evy process has multiple streams, it sets
the streams on this process in the same way as \texttt{setStream()}
for that process.
\end{tabb}
\begin{code}
\begin{hide}
}
\end{hide}
\end{code}
