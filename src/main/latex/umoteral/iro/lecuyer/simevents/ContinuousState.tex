\defclass {ContinuousState}

Represents the portion of the simulator's state associated with
continuous-time simulation.
Any simulator, including the default static one, can have an associate continuous state
which is obtained using the \texttt{continuousState()} method
of the \class{Simulator} class.
This state includes all active integration variables as well as the current integration method.

One of the methods \method{selectEuler}{}, \method{selectRungeKutta2}{} or
\method{selectRungeKutta4}{} must be called before starting 
any integration.
These methods permit one to select the numerical integration method 
and the step size \texttt{h} (in time units) that will be used
for \emph{all} continuous-time variables linked to the simulator.
For all the methods, an integration step at time $t$ changes 
the values of the variables from their old values at time $t-h$ to their
new values at time $t$.

Each integration step is scheduled as an event and added to the event list.

\bigskip\hrule

%%%%%%%%%%%%%%%%%%%%%%%%%%%%%%%%%%%%%%%%%%%%%%%%%%%%%%%%%%%%%%%%%%
\begin{code}
\begin{hide}
/*
 * Class:        ContinuousState
 * Description:  Represents the portion of the simulator's state associated
                 with continuous-time simulation.
 * Environment:  Java
 * Software:     SSJ 
 * Copyright (C) 2001  Pierre L'Ecuyer and Université de Montréal
 * Organization: DIRO, Université de Montréal
 * @author       
 * @since

 * SSJ is free software: you can redistribute it and/or modify it under
 * the terms of the GNU General Public License (GPL) as published by the
 * Free Software Foundation, either version 3 of the License, or
 * any later version.

 * SSJ is distributed in the hope that it will be useful,
 * but WITHOUT ANY WARRANTY; without even the implied warranty of
 * MERCHANTABILITY or FITNESS FOR A PARTICULAR PURPOSE.  See the
 * GNU General Public License for more details.

 * A copy of the GNU General Public License is available at
   <a href="http://www.gnu.org/licenses">GPL licence site</a>.
 */
\end{hide}
package umontreal.iro.lecuyer.simevents;\begin{hide}

import java.util.List;
import java.util.Collections;
import java.util.ArrayList;\end{hide}


public class ContinuousState \begin{hide} {
\end{hide}

   // Integration methods
   public enum IntegMethod{ 
      EULER,            // Euler integration method
      RUNGEKUTTA2,      // Runge-Kutta integration method of order 2
      RUNGEKUTTA4       // Runge-Kutta integration method of order 4
   }\begin{hide}

   private double stepSize;            // Integration step size.
   private IntegMethod integMethod;    // Integration method in use.
   private int order;                  // Order of the method in use.
   private double[] A = new double[4];
   private double[] B = new double[4];
   private double[] C = new double[4];

   // The event that actually executes integration steps.
   private StepEvent stepEv = null;

 // Class of event that executes an integration step.
   private class StepEvent extends Event {
      public StepEvent(Simulator sim) { super(sim); }
      public void actions() {
         switch (integMethod) {
            case EULER:       oneStepEuler();  break;
            case RUNGEKUTTA2: oneStepRK();  break;
            case RUNGEKUTTA4: oneStepRK();  break;
            default: throw new IllegalArgumentException 
                ("Integration step with undefined method");
         }
         this.schedule (stepSize);
         // if (afterInteg != null) afterInteg.actions();
      }

      public String toString() {
         return "Integration step for continuous variable ";
      }
   }

   private List<Continuous> list;
   private Simulator sim;
\end{hide}
\end{code}

%%%%%%%%%%%%%%%%%%%%%%%%%%%%%%%%%%%
\subsubsection* {Constructor}
\begin{code}
   protected ContinuousState (Simulator sim) \begin{hide} {
      this.list = new ArrayList<Continuous>();
      this.sim = sim;
      assert sim != null;
   }
\end{hide}
\end{code} 
 \begin{tabb}  Creates a new \class{ContinuousState} object linked to the given simulator. 
 Usually, the user should not call this constructor directly since a new object 
 is created automatically by the \texttt{continuousState()} method of
 class \class{Simulator}.
 \end{tabb}
%%%%%%%%%%%%%%%%%%%%%%%%%%%%%%%%%%%
\subsubsection* {Methods}
\begin{code}

   public List<Continuous> getContinuousVariables()\begin{hide} {
       return Collections.unmodifiableList (list);
   }\end{hide}
\end{code}
\begin{tabb}  Returns the list of continuous-time variables currently
  integrated by the simulator.
  The returned list is updated automatically as variables are added or removed, but it
  cannot be modified directly. One must instead use
  \texttt{startInteg} or \texttt{stopInteg} in class \class{Continuous} to add
  or remove variables.
\end{tabb}
 \begin{hide} \begin{code}

   protected void startInteg(Continuous c) {
      // The following is done only the first time this method is called.
      if (stepEv == null) 
         stepEv = new StepEvent(sim);
      c.active = true;
      // Inserts this in list of active variables.
      if (list.isEmpty()) {
         stepEv.schedule (stepSize);
      }   // There was no active variable.
      list.add(c);
   }
\end{code}
 \begin{tabb}  Starts the integration process that will change the state of
  \class{Continuous} variable at each integration step.
 \end{tabb}
 \begin{code}

   protected void stopInteg(Continuous c) {
      c.active = false;
      list.remove(c);
      if (list.isEmpty()) stepEv.cancel();
   }
\end{code}
 \begin{tabb}  Stops the integration process for \class{Continuous} variable.
  The variable keeps the value it took at the last integration step
  before calling \texttt{stopInteg}.
 \end{tabb}\end{hide}
 \begin{code}

   public IntegMethod integMethod ()\begin{hide} {
      return integMethod;
   }\end{hide}
\end{code}
 \begin{tabb}  Return an integer that represent the integration method in use.
 \end{tabb}
\begin{htmlonly}
   \return{Interger that represent the integration method in use.}
\end{htmlonly}
\begin{code}

   public void selectEuler (double h)\begin{hide} {
      integMethod = IntegMethod.EULER;
      stepSize = h;
   }\end{hide}
\end{code}
 \begin{tabb}  Selects the Euler method as the integration method,
  with the integration step size \texttt{h}, in time units.
 \end{tabb}
\begin{htmlonly}
   \param{h}{integration step, in simulation time units}
\end{htmlonly}
\begin{code}

   public void selectRungeKutta2 (double h)\begin{hide} {
      integMethod = IntegMethod.RUNGEKUTTA2;
      stepSize = h;
      order = 2;
      A[0] = 1.0;  A[1] = 0.0;
      B[0] = 0.5;  B[1] = 0.5;
      C[0] = 0.0;  C[1] = 1.0;
   }\end{hide}
\end{code}
 \begin{tabb}  Selects a Runge-Kutta method of order~2 as the integration
  method to be used, with step size \texttt{h}.
 \end{tabb}
\begin{htmlonly}
   \param{h}{integration step, in simulation time units}
\end{htmlonly}
\begin{code}

   public void selectRungeKutta4 (double h)\begin{hide} {
      integMethod = IntegMethod.RUNGEKUTTA4;
      stepSize = h;
      order = 4;
      A[0] = 0.5;  A[1] = 0.5;  A[2] = 1.0;  A[3] = 0.0;
      B[0] = 1.0/6.0;   B[1] = 1.0/3.0;
      B[2] = 1.0/3.0;   B[3] = 1.0/6.0;
      C[0] = 0.0;  C[1] = 0.5;  C[2] = 0.5;  C[3] = 1.0;
   }\end{hide}
\end{code}
 \begin{tabb}  Selects a Runge-Kutta method of order~4 as the integration
  method to be used, with step size \texttt{h}.
 \end{tabb}
\begin{htmlonly}
   \param{h}{integration step, in simulation time units}
\end{htmlonly}

\begin{code}\begin{hide}

   private void oneStepEuler()  {
     Continuous v;
      double t = sim.time() - stepSize;
      int current;
      current = list.size();
      while (current > 0) {
         v = list.get(--current);
         v.phi = v.value + stepSize * v.derivative (t);
      }
      current = list.size();
      while (current > 0) {
         v = list.get(--current);
         v.value = v.phi;
         if (v.ev != null) 
            v.ev.scheduleNext();
         v.afterEachStep();
      }
   }

   private void oneStepRK() {
      Continuous v;
      double t = sim.time() - stepSize;
      int current = list.size();
      while (current > 0) {
         v = list.get(--current);
         v.buffer = v.value;
         v.sum = 0.0;
         v.pi = 0.0;
      }
      for (int i = 1; i <= order-1; i++) {
         current = list.size();
         while (current > 0) {
            v = list.get(--current);
            v.pi = v.derivative (t + stepSize * C[i-1]);
            v.sum = v.sum + v.pi * B[i-1];
            v.phi = v.buffer + stepSize * v.pi * A[i-1];
         }
         current = list.size();
         while (current > 0) { 
            v = list.get(--current);
            v.value = v.phi;
         }
      } 
      current = list.size();
      while (current > 0) {
         v = list.get(--current);
         v.pi = v.derivative (t + stepSize * C[order-1]);
         v.value = v.buffer + stepSize * (v.sum + v.pi * B[order-1]);
         if (v.ev != null) v.ev.scheduleNext();
         v.afterEachStep();
      }
   }
}\end{hide}
\end{code}
