\defclass {RandomStream}

This interface defines the basic structures to handle multiple streams
of uniform (pseudo)\-random numbers and convenient
tools to move around within and across these streams.
The actual random number generators (RNGs) are provided in classes
that implement this \texttt{RandomStream} interface.
%A variety of such RNGs are available, in classes 
%\texttt{RandMrg, RandTaus}, etc.
Each stream of random numbers is an object of the class that implements
this interface, and can be viewed as a virtual random number generator.
% The random number streams can also be instantiated as points from
% HUPS (see module \texttt{Qrand}).

For each type of base RNG (i.e., each implementation of the 
\texttt{RandomStream} interface), the full period of the generator
is cut into adjacent {\em streams\/} (or segments) of length $Z$, 
and each of these streams is partitioned into $V$ {\em substreams\/}
of length $W$, where $Z = VW$.
The values of $V$ and $W$ depend on the specific RNG, but are usually
larger than $2^{50}$.
Thus, the distance $Z$ between the starting points of two successive 
streams provided by an RNG usually exceeds $2^{100}$.
The initial seed of the RNG is the starting point of the first stream.  
It has a default value for each type of RNG,
but this initial value can be changed by calling \texttt{setPackageSeed} 
for the corresponding class.
Each time a new \texttt{RandomStream} is created, its starting point
(initial seed) is computed automatically,
$Z$ steps ahead of the starting point of the previously created stream
of the same type, and its current state is set equal to this starting point.

For each stream, one can advance by one step and generate one value,
or go ahead to the beginning of the next substream within this stream, 
or go back to the beginning of the current substream, or to the beginning
of the stream, or jump ahead or back by an arbitrary number of steps.
Denote by $C_g$ the current state of a stream $g$,
$I_g$ its initial state, $B_g$ the state at the beginning of the
current substream, and $N_g$ the state at the beginning of the next substream.
\latex{The following diagram shows an example of a stream whose
  state is at the 6th value of the third substream, i.e., $2W+5$
  steps ahead of its initial state $I_g$ and 5 steps ahead of its
  state $B_g$.}
The form of the state of a stream depends on its type.
For example, the state of a stream of class \class{MRG32k3a} is a vector
of six 32-bit integers represented internally as floating-point numbers
(in \texttt{double}).

\begin{latexonly}
%% see The TeXbook, Exercice 10.4
\def\tick#1{\vrule height 0pt depth #1pt}
\def\enskip{\hskip.5em\relax}
\def\ld{\hbox to 0.24 in{\vtop{\kern3.0pt\hbox{\dotfill}}}}
\def\ts{\enskip\tick4}
\def\suba{\hbox to 1.2in {\vtop
 {\hbox to 1.2in{\hbox to .66in{\hrulefill}\hbox to .30in{\dotfill}\hrulefill}
  \hbox to 1.2in{\tick9\ts\ts\ts\ts\ts\ts\ts\hfill\ts\enskip}}}}
\def\subb{\hbox to 1.2in {\vtop
 {\hbox to 1.2in{\hbox to .84in {\hrulefill}
         \hbox to .24in {\dotfill}\hbox to .12in {\hrulefill}}
  \hbox to 1.2in{\tick9\ts\ts\ts\ts\ts\ts\ts\ts\hfill\ts\enskip}}}}

$$
\vbox{\offinterlineskip
\hbox to 5.5in{\hskip 2.4in \hbox to 1.2in{\hfil\hskip1.07em $C_g$\hfil}\hfill}
\vskip .1pt
\hbox to 5.5in{\hskip 2.4in \hbox to 1.2in{\hfil\hskip1.07em $\Downarrow$\hfil}
    \hfill}
\vtop {\offinterlineskip\hskip 0.0in \hbox to 5.5 true in
       {\suba\suba\subb\suba\tick9
        \hbox to .12in {\hrulefill}\hbox to .24in{\dotfill}}}\vskip .1pt
\hskip -0.65in\hbox to 1.2 in{\hfil$I_g$\hfil}\hskip 1.25in
  \hbox to 1.2 in{\hfil$B_g$\hfil}\hbox to 1.2 in{\hfil$N_g$\hfil}\hfill
\vskip .1pt }
$$
\end{latexonly}

The methods for manipulating the streams and generating random
numbers are implemented differently for each type of RNG.
The methods whose formal parameter types do not depend
on the RNG type are specified in the interface \texttt{RandomStream}.
The others (e.g., for setting the seeds) are given only in the
classes that implement the specific RNG types.

\latex{See \cite{sLAW00a,rLEC91a,rLEC02a} 
  %and Section~\ref{sec:timeshared}
  for examples of 
  situations where the multiple streams offered here are useful.}

\iffalse  %%%%%%%%%%%%%%%
\begin {itemize}
\item[(a)]
You want to compare two or more similar systems, via simulation
with common random numbers, with $n$ simulation runs for each system.
% (For more details about common random numbers and other variance
%  reduction techniques, see for example \cite{sBRA87a,sLAW91a}.)
To guarantee that the same random numbers are used across the systems
with good synchronization, assign different streams to different
places where random numbers are needed in the model
(e.g., to compare queuing systems, use one stream for the
inter-arrival times, one stream for the service times at each queue,
one stream for routing decisions, etc.).
To make sure that each stream starts from
the same state across the different systems, assign run $i$ to the
$i$th substream, for each stream.
The experiment then proceeds as follows.
For the first system, simulate run 1 by starting all the streams from
their initial seed, $I_g$.
Before each new run, advance all the streams to the initial state
of their next substream, $N_g$.
After the $n$th run, reset all the streams to their initial state, $I_g$.
Repeat for each system that you want to compare.
%
\item[(b)]
You want to run simulations on several processors, in parallel,
and you want each processor to have its own (virtual)
random number generator.
In this case, you can simply use one stream for each processor.
You need a central authority, or {\em monitor\/}, to manage the
allocation of the streams, but once a stream is allocated,
the processors can go their own way, independently of each other.
In this setup, the processors use in fact the same generator,
but with different {\em seeds\/}.
This makes the implementation easier than if a different
generator is used on each processor.
\end {itemize}
\fi  %%%%%%%%%%%%%%%%%%%%

Methods for generating random variates from non-uniform distributions
are provided in the \externalclass{umontreal.iro.lecuyer}{randvar} package.
%The class \texttt{DiscreteDist} can be used to construct an arbitrary 
%discrete distribution over a finite set of values, and to generate from it.

\bigskip\hrule

%%%%%%%%%%%%%%%%%%%%%%%%%%%%%%%%%%%%%%%%%%%%%%%%%%%%%%%%%%%%%%%%%%
\begin{code}
\begin{hide}
/*
 * Class:        RandomStream
 * Description:  basic structures to handle multiple streams of uniform
                 (pseudo)-random numbers and tools to move around within
                 and across these streams
 * Environment:  Java
 * Software:     SSJ 
 * Copyright (C) 2001  Pierre L'Ecuyer and Université de Montréal
 * Organization: DIRO, Université de Montréal
 * @author       Pierre L'Ecuyer
 * @since        2000

 * SSJ is free software: you can redistribute it and/or modify it under
 * the terms of the GNU General Public License (GPL) as published by the
 * Free Software Foundation, either version 3 of the License, or
 * any later version.

 * SSJ is distributed in the hope that it will be useful,
 * but WITHOUT ANY WARRANTY; without even the implied warranty of
 * MERCHANTABILITY or FITNESS FOR A PARTICULAR PURPOSE.  See the
 * GNU General Public License for more details.

 * A copy of the GNU General Public License is available at
   <a href="http://www.gnu.org/licenses">GPL licence site</a>.
 */
\end{hide}
package umontreal.iro.lecuyer.rng;

public interface RandomStream \begin{hide} { \end{hide}
\end{code}

%%%%%%%%%%%%%%%%%%%%%%%%%%%%%%%%%
\subsubsection* {Methods}
\begin{code}

   public void resetStartStream();
\end{code}
 \begin{tabb} Reinitializes the stream to its initial state $I_g$:
   $C_g$ and $B_g$ are set to $I_g$.
 \end{tabb}
\begin{code}

   public void resetStartSubstream();
\end{code}
 \begin{tabb} Reinitializes the stream to the beginning of its current
   substream: $C_g$ is set to $B_g$.
 \end{tabb}
\begin{code}

   public void resetNextSubstream();
\end{code}
 \begin{tabb} Reinitializes the stream to the beginning of its next
   substream: $N_g$ is computed, and
   $C_g$ and $B_g$ are set to $N_g$.
 \end{tabb}
\begin{code}

   public String toString();
\end{code}
  \begin{tabb} Returns a string containing the current state of this stream.
  \end{tabb}
\begin{htmlonly}
   \return{the state of the generator formated as a string}
\end{htmlonly}
\begin{code}

   public double nextDouble();
\end{code}
  \begin{tabb} Returns a (pseudo)random number from the uniform distribution
   over the interval $(0,1)$, using this stream, after advancing its
   state by one step.  The generators programmed in SSJ never 
   return the values 0 or 1.
  \end{tabb}
\begin{htmlonly}
   \return{the next generated uniform}
\end{htmlonly}
\begin{code}

   public void nextArrayOfDouble (double[] u, int start, int n);
\end{code}
  \begin{tabb} Generates \texttt{n} (pseudo)random numbers from the
   uniform distribution and stores them into the array \texttt{u}
   starting at index \texttt{start}.
  \end{tabb}
\begin{htmlonly}
   \param{u}{array that will contain the generated uniforms}
   \param{start}{starting index, in the array \texttt{u}, to write uniforms from}
   \param{n}{number of uniforms to generate}
\end{htmlonly}
\begin{code}

   public int nextInt (int i, int j);
\end{code}
 \begin{tabb} Returns a (pseudo)random number from the discrete uniform 
   distribution over the integers $\{i,i+1,\dots,j\}$,
   using this stream.  (Calls \texttt{nextDouble} once.)
 \end{tabb}
\begin{htmlonly}
   \param{i}{smallest integer that can be generated}
   \param{j}{greatest integer that can be generated}
   \return{the generated integer}
\end{htmlonly}
\begin{code}

   public void nextArrayOfInt (int i, int j, int[] u, int start, int n);
\end{code}
  \begin{tabb} Generates \texttt{n} (pseudo)random numbers
   from the discrete uniform 
   distribution over the integers $\{i,i+1,\dots,j\}$,
   using this stream and stores the result in the array \texttt{u}
   starting at index \texttt{start}.  (Calls \texttt{nextInt} \texttt{n} times.)
  \end{tabb}
\begin{htmlonly}
   \param{i}{smallest integer that can be generated}
   \param{j}{greatest integer that can be generated}
   \param{u}{array that will contain the generated values}
   \param{start}{starting index, in the array \texttt{u}, to write integers from}
   \param{n}{number of values being generated}
\end{htmlonly}
\begin{code}\begin{hide} 
}
\end{hide}
\end{code}
