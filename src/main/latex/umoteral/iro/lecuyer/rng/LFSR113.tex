\defmodule {LFSR113}

Extends \class{RandomStreamBase} using a composite linear feedback
shift register (LFSR) (or Tausworthe) RNG as defined in
\cite{rLEC96a,rTEZ91b}.
This generator is the \texttt{LFSR113} proposed by \cite{rLEC99a}.
It has four 32-bit components combined by a bitwise xor.
Its period length is \html{approximatively 2^{113}}
\latex{$\rho\approx 2^{113}$}. The values of $V$, $W$ and $Z$ are $2^{35}$,
$2^{55}$ and $2^{90}$ respectively (see \class{RandomStream} for their
definition). The seed of the RNG, and the state of a stream at any given
step, are four-dimensional vectors of 32-bit integers.
The default initial seed of the RNG is $(12345, 12345, 12345, 12345)$.
The \texttt{nextValue} method returns numbers with 32 bits of precision.


%%%%%%%%%%%%%%%%%%%%%%%%%%%%%%%%%%%%%%%%%%
\bigskip\hrule

\begin{code}
\begin{hide}
/*
 * Class:        LFSR113
 * Description:  32-bit composite linear feedback shift register proposed by L'Ecuyer
 * Environment:  Java
 * Software:     SSJ 
 * Copyright (C) 2001  Pierre L'Ecuyer and Université de Montréal
 * Organization: DIRO, Université de Montréal
 * @author       
 * @since

 * SSJ is free software: you can redistribute it and/or modify it under
 * the terms of the GNU General Public License (GPL) as published by the
 * Free Software Foundation, either version 3 of the License, or
 * any later version.

 * SSJ is distributed in the hope that it will be useful,
 * but WITHOUT ANY WARRANTY; without even the implied warranty of
 * MERCHANTABILITY or FITNESS FOR A PARTICULAR PURPOSE.  See the
 * GNU General Public License for more details.

 * A copy of the GNU General Public License is available at
   <a href="http://www.gnu.org/licenses">GPL licence site</a>.
 */
\end{hide}
package umontreal.iro.lecuyer.rng; \begin{hide}

import java.io.Serializable;
\end{hide}

public class LFSR113 extends RandomStreamBase \begin{hide} {

   private static final long serialVersionUID = 70510L;
   // La date de modification a l'envers, lire 10/05/2007

   // generator constant: make sure that double values 0 and 1 never occur
   private static final double NORM = 1.0 / 0x100000001L;   // 2^32 + 1


   // state variables:
   private int z0;
   private int z1;
   private int z2;
   private int z3;

   //stream and substream variables :
   private int[] stream;
   private int[] substream;
   private static int[] curr_stream = {12345, 12345, 12345, 12345};

 \end{hide}
\end{code}

%%%%%%%%%%%%%%%%%%%%%%%%%%%%%%%%%%%%%%%%%%%
\subsubsection* {Constructors}

\begin{code}
   public LFSR113() \begin{hide} {
      name = null;

      stream = new int[4];
      substream = new int[4];

      for(int i = 0; i < 4; i++)
         stream[i] = curr_stream[i];

      resetStartStream();


      // Les operations qui suivent permettent de faire sauter en avant
      // de 2^90 iterations chacunes des composantes du generateur.
      // L'etat interne apres le saut est cependant legerement different
      // de celui apres 2^90 iterations puisqu'il ignore l'etat dans
      // lequel se retrouvent les premiers bits de chaque composantes,
      // puisqu'ils sont ignores dans la recurrence. L'etat redevient
      // identique a ce que l'on aurait avec des iterations normales
      // apres un appel a nextValue().

      int z, b;

      z = curr_stream[0] & -2;
      b = (z <<  6) ^ z;
      z = (z) ^ (z << 2) ^ (z << 3) ^ (z << 10) ^ (z << 13) ^
         (z << 16) ^ (z << 19) ^ (z << 22) ^ (z << 25) ^
         (z << 27) ^ (z << 28) ^
         (b >>> 3) ^ (b >>> 4) ^ (b >>> 6) ^ (b >>> 9) ^ (b >>> 12) ^
         (b >>> 15) ^ (b >>> 18) ^ (b >>> 21);
      curr_stream[0] = z;


      z = curr_stream[1] & -8;
      b = (z <<  2) ^ z;
      z = (b >>> 13) ^ (z << 16);
      curr_stream[1] = z;


      z = curr_stream[2] & -16;
      b = (z <<  13) ^ z;
      z = (z << 2) ^ (z << 4) ^ (z << 10) ^ (z << 12) ^ (z << 13) ^
         (z << 17) ^ (z << 25) ^
         (b >>> 3) ^ (b >>> 11) ^ (b >>> 15) ^ (b >>> 16) ^ (b >>> 24);
      curr_stream[2] = z;


      z = curr_stream[3] & -128;
      b = (z <<  3) ^ z;
      z = (z << 9) ^ (z << 10) ^ (z << 11) ^ (z << 14) ^ (z << 16) ^
         (z << 18) ^ (z << 23) ^ (z << 24) ^
         (b >>> 1) ^ (b >>> 2) ^ (b >>> 7) ^ (b >>> 9) ^ (b >>> 11) ^
         (b >>> 14) ^ (b >>> 15) ^ (b >>> 16) ^ (b >>> 23) ^ (b >>> 24);
      curr_stream[3] = z;
   }\end{hide}
\end{code}
\begin{tabb} Constructs a new stream.
\end{tabb}
\begin{code}

   public LFSR113 (String name) \begin{hide} {
      this();
      this.name = name;
   }\end{hide}
\end{code}
\begin{tabb} Constructs a new stream with the identifier \texttt{name}.
\end{tabb}
\begin{htmlonly}
  \param{name}{name of the stream}
\end{htmlonly}

%%%%%%%%%%%%%%%%%%%%%%%%%%%%%%%%%%%%
\subsubsection* {Methods}
\begin{code}
   public static void setPackageSeed (int[] seed) \begin{hide} {
      checkSeed (seed);
      for(int i = 0; i < 4; i++)
         curr_stream[i] = seed[i];
   }

   private static void checkSeed  (int[] seed) {
      if (seed.length < 4)
         throw new IllegalArgumentException("Seed must contain 4 values");
      if ((seed[0] >= 0 && seed[0] < 2)  ||
          (seed[1] >= 0 && seed[1] < 8)  ||
          (seed[2] >= 0 && seed[2] < 16) ||
          (seed[3] >= 0 && seed[3] < 128))
         throw new IllegalArgumentException
         ("The seed elements must be either negative or greater than 1, 7, 15 and 127 respectively");
   }\end{hide}
\end{code}
\begin{tabb} Sets the initial seed for the class \texttt{LFSR113} to the four
  integers of the vector \texttt{seed[0..3]}.
  This will be the initial state (or seed) of the next created stream.
  The default seed for the first stream is $(12345, 12345, 12345, 12345)$.
  The first, second, third and fourth integers of \texttt{seed}
  must be either negative, or greater than or equal to
   2, 8, 16 and 128 respectively.
\end{tabb}
\begin{htmlonly}
  \param{seed}{array of 4 elements representing the seed}
\end{htmlonly}
\begin{code}

   public void setSeed (int[] seed) \begin{hide} {
      checkSeed  (seed);
      for(int i = 0; i < 4; i++)
         stream[i] = seed[i];
      resetStartStream();
   } \end{hide}
\end{code}
\begin{tabb} This method is discouraged for normal use.
  Initializes the stream at the beginning of a stream with the initial
  seed \texttt{seed[0..3]}. The seed must satisfy the same conditions
  as in \texttt{setPackageSeed}.
  This method only affects the specified stream; the others are not
  modified, so the beginning of the streams will not be
  spaced $Z$ values apart.
  For this reason, this method should only be used in very
  exceptional cases; proper use of the \texttt{reset...} methods
  and of the stream constructor is preferable.
\end{tabb}
\begin{htmlonly}
  \param{seed}{array of 4 elements representing the seed}
\end{htmlonly}
\begin{code}

   public int[] getState() \begin{hide} {
      return new int[]{z0, z1, z2, z3};
   } \end{hide}
\end{code}
\begin{tabb} Returns the current state of the stream, represented as
  an array of four integers.
\end{tabb}
\begin{htmlonly}
  \return{the current state of the stream}
\end{htmlonly}
\begin{code}

   public LFSR113 clone() \begin{hide} {
      LFSR113 retour = null;
      retour = (LFSR113)super.clone();
      retour.stream = new int[4];
      retour.substream = new int[4];
      for (int i = 0; i<4; i++) {
         retour.substream[i] = substream[i];
         retour.stream[i] = stream[i];
      }
      return retour;
   }\end{hide}
\end{code}
 \begin{tabb} Clones the current generator and return its copy.
 \end{tabb}
 \begin{htmlonly}
   \return{A deep copy of the current generator}
\end{htmlonly}
\begin{code}
  \begin{hide}

   public void resetStartStream() {
      for(int i = 0; i < 4; i++)
         substream[i] = stream[i];
      resetStartSubstream();
   }

   public void resetStartSubstream() {
      z0 = substream[0];
      z1 = substream[1];
      z2 = substream[2];
      z3 = substream[3];
   }
/*
   // La version de Mario: beaucoup plus lent que l'ancienne version: on garde
   // l'ancienne version.
   public void resetNextSubstream() {
      byte [] c0 = new byte[] {0, 1, 0, 1, 1, 0, 0, 0, 0, 0, 0, 1, 1, 0, 0,
                            0, 1, 1, 1, 0, 1, 0, 1, 0, 0, 1, 0, 0, 0, 1, 1};
      byte [] c1 = new byte[] {1, 0, 0, 0, 1, 1, 1, 1, 1, 1, 1, 0, 0, 0, 0,
                            0, 0, 0, 0, 1, 0, 1, 0, 1, 0, 1, 0, 1, 0, 0, 0};
      byte [] c2 = new byte[] {0, 0, 0, 0, 0, 0, 0, 1, 0, 1, 1, 0, 1, 0, 1,
                            0, 1, 0, 1, 0, 1, 0, 1, 0, 1, 0, 0, 0, 0, 0, 0};
      byte [] c3 = new byte[] {1, 0, 0, 1, 1, 1, 1, 1, 0, 0, 1, 0, 0, 1, 0,
                            0, 0, 0, 0, 1, 0, 0, 0, 0, 0, 0, 0, 0, 0, 0, 0};
      int x0 = 0;
      int x1 = 0;
      int x2 = 0;
      int x3 = 0;
      resetStartSubstream();
      for (int i = 0; i < 31; ++i) {
         if (c0[i] == 1) x0 ^= z0;
         if (c1[i] == 1) x1 ^= z1;
         if (c2[i] == 1) x2 ^= z2;
	 if (c3[i] == 1) x3 ^= z3;

         int b;
         b  = (((z0 <<   6) ^ z0) >>> 13);
         z0 = (((z0 &   -2) << 18) ^ b);
         b  = (((z1 <<   2) ^ z1) >>> 27);
         z1 = (((z1 &   -8) <<  2) ^ b);
         b  = (((z2 <<  13) ^ z2) >>> 21);
         z2 = (((z2 &  -16) <<  7) ^ b);
         b  = (((z3 <<   3) ^ z3) >>> 12);
         z3 = (((z3 & -128) << 13) ^ b);
      }
      substream[0] = x0;
      substream[1] = x1;
      substream[2] = x2;
      substream[3] = x3;
      resetStartSubstream();
   }
*/

   public void resetNextSubstream() {
      // Les operations qui suivent permettent de faire sauter en avant
      // de 2^55 iterations chacunes des composantes du generateur.
      // L'etat interne apres le saut est cependant legerement different
      // de celui apres 2^55 iterations puisqu'il ignore l'etat dans
      // lequel se retrouvent les premiers bits de chaque composantes,
      // puisqu'ils sont ignores dans la recurrence. L'etat redevient
      // identique a ce que l'on aurait avec des iterations normales
      // apres un appel a nextValue().

      int z, b;

      z = substream[0] & -2;
      b = (z <<  6) ^ z;
      z = (z) ^ (z << 3) ^ (z << 4) ^ (z << 6) ^ (z << 7) ^
         (z << 8) ^ (z << 10) ^ (z << 11) ^ (z << 13) ^ (z << 14) ^
         (z << 16) ^ (z << 17) ^ (z << 18) ^ (z << 22) ^
         (z << 24) ^ (z << 25) ^ (z << 26) ^ (z << 28) ^ (z << 30);
      z ^= (b >>> 1) ^ (b >>> 3) ^ (b >>> 5) ^ (b >>> 6) ^
         (b >>> 7) ^ (b >>> 9) ^ (b >>> 13) ^ (b >>> 14) ^
         (b >>> 15) ^ (b >>> 17) ^ (b >>> 18) ^ (b >>> 20) ^
         (b >>> 21) ^ (b >>> 23) ^ (b >>> 24) ^ (b >>> 25) ^
         (b >>> 26) ^ (b >>> 27) ^ (b >>> 30);
      substream[0] = z;


      z = substream[1] & -8;
      b = z ^ (z << 1);
      b ^= (b << 2);
      b ^= (b << 4);
      b ^= (b << 8);

      b <<= 8;
      b ^= (z << 22) ^ (z << 25) ^ (z << 27);
      if((z & 0x80000000) != 0) b ^= 0xABFFF000;
      if((z & 0x40000000) != 0) b ^= 0x55FFF800;
      z = b ^ (z >>> 7) ^ (z >>> 20) ^ (z >>> 21);
      substream[1] = z;


      z = substream[2] & -16;
      b = (z <<  13) ^ z;
      z = (b >>> 3) ^ (b >>> 17) ^
         (z << 10) ^ (z << 11) ^ (z << 25);
      substream[2] = z;


      z = substream[3] & -128;
      b = (z <<  3) ^ z;
      z = (z << 14) ^ (z << 16) ^ (z << 20) ^
         (b >>> 5) ^ (b >>> 9) ^ (b >>> 11);
      substream[3] = z;

      resetStartSubstream();
   }


   public String toString()  {
      if (name == null)
         return "The state of the LFSR113 is: { " +
                z0 + ", " + z1 + ", " + z2 + ", " + z3 + " }";
      else
         return "The state of " + name + " is: { " +
                z0 + ", " + z1 + ", " + z2 + ", " + z3 + " }";
   }

   private long nextNumber() {
      int b;
      b  = (((z0 <<   6) ^ z0) >>> 13);
      z0 = (((z0 &   -2) << 18) ^ b);
      b  = (((z1 <<   2) ^ z1) >>> 27);
      z1 = (((z1 &   -8) <<  2) ^ b);
      b  = (((z2 <<  13) ^ z2) >>> 21);
      z2 = (((z2 &  -16) <<  7) ^ b);
      b  = (((z3 <<   3) ^ z3) >>> 12);
      z3 = (((z3 & -128) << 13) ^ b);

      long r = (z0 ^ z1 ^ z2 ^ z3);

      if (r <= 0)
         r += 0x100000000L;      //2^32

      return r;
   }

   protected double nextValue() {
      // Make sure that double values 0 and 1 never occur
      return nextNumber() * NORM;
   }

   public int nextInt (int i, int j) {
      if (i > j)
         throw new IllegalArgumentException(i + " is larger than " + j + ".");
      long d = j-i+1L;
      long q = 0x100000000L / d;
      long r = 0x100000000L % d;
      long res;

      do {
         res = nextNumber();
      } while (res >= 0x100000000L - r);

      return (int) (res / q) + i;
   }

}\end{hide}
\end{code}
