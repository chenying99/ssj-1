\defmodule {WELL1024}

Implements the \class{RandomStream} interface via inheritance from
\class{RandomStreamBase}. The backbone generator is a Well Equidistributed
Long period Linear Random Number Generator (WELL),  proposed by F.
 Panneton\html{,}\latex{ in \cite{rPAN06b,rPAN04t},} and which has a state size
of 1024 bits and a period length of \html{approximatively}
\latex{$\rho\approx$} $2^{1024}$. The values of $V$, $W$ and $Z$ are $2^{300}$,
$2^{400}$ and $2^{700}$ respectively (see \class{RandomStream} for their
definition). The seed of the RNG, and the state of a stream at any given
step, is a 16-dimensional vector of 32-bit integers.
% The default initial seed of the RNG is \texttt{(0xDE410B75, 0x904FA5C7,
% 0x8BD4701E, 0x011EA361, 0x6EB189E0, 0x7A2B0CE1, 0xE02631CA,
% 0x72EBA132, 0x5189DA0F, 0x3EB72A2C, 0x51ABE513, 0x6D9EA57C,
% 0x4D690BF1, 0x84217FCA, 0x7290DE1A, 0x429F5A48, 0x6EC42EF3,
% 0x960AB315, 0x72C3A743, 0x48E13BF1, 0x8917EAC8, 0x284AE026,
% 0x357BF240, 0x913B51AC, 0x136AF195, 0x361ABC18, 0x731AB725,
% 0x63D3A7C9, 0xE5F32A18, 0x91A8E164, 0x04EA61B5, 0xC72A6091)}.
The output of \texttt{nextValue} has 32 bits of precision.

This implementation requires the use of about
250K of memory to run. This memory is shared between all instances of the
class, and is only loaded when the first instance is created.

%%%%%%%%%%%%%%%%%%%%%%%%%%%%%%%%%%%%%%%%
\bigskip\hrule

\begin{code}
\begin{hide}
/*
 * Class:        WELL1024
 * Description:  a Well Equidistributed Long period Linear Random Number
                 Generator with a state size of 1024 bits
 * Environment:  Java
 * Software:     SSJ 
 * Copyright (C) 2001  Pierre L'Ecuyer and Université de Montréal
 * Organization: DIRO, Université de Montréal
 * @author       
 * @since

 * SSJ is free software: you can redistribute it and/or modify it under
 * the terms of the GNU General Public License (GPL) as published by the
 * Free Software Foundation, either version 3 of the License, or
 * any later version.

 * SSJ is distributed in the hope that it will be useful,
 * but WITHOUT ANY WARRANTY; without even the implied warranty of
 * MERCHANTABILITY or FITNESS FOR A PARTICULAR PURPOSE.  See the
 * GNU General Public License for more details.

 * A copy of the GNU General Public License is available at
   <a href="http://www.gnu.org/licenses">GPL licence site</a>.
 */
\end{hide}
package umontreal.iro.lecuyer.rng; \begin{hide}

import umontreal.iro.lecuyer.util.BitVector;
import umontreal.iro.lecuyer.util.BitMatrix;

import java.io.Serializable;
import java.io.ObjectOutputStream;
import java.io.FileOutputStream;
import java.io.ObjectInputStream;
import java.io.InputStream;

import java.io.FileNotFoundException;
import java.io.IOException;
\end{hide}

public class WELL1024 extends RandomStreamBase \begin{hide} {

   private static final long serialVersionUID = 70510L;
   //La date de modification a l'envers, lire 10/05/2007

   private static final double NORM = 2.32830643653869628906e-10;

   private static final int MASK = 0x1F;

   private static final int W = 32;
   private static final int R = 32;
   private static final int P = 0;
   private static final int M1 = 3;
   private static final int M2 = 24;
   private static final int M3 = 10;

   private static final int A1 = 0xDB79FB31;
   private static final int B1 = 0x0FDAA2C1;
   private static final int B2 = 0x27C5A58D;
   private static final int C1 = 0x71E993FE;


   private int state_i = 0;
   private int[] state;

   //stream and substream variables
   private int[] stream;
   private int[] substream;
   private static int[] curr_stream;

   //state transition matrices
   private static BitMatrix Apw;
   private static BitMatrix Apz;

   //if the generator was initialised
   private static boolean initialised = false;

   // P(z) = {0x00000001, 0x00000000, 0x00000000, 0x00000000,
   //         0x028a0008, 0x02288020, 0x2baaa20a, 0x0209aa00,
   //         0x3f871248, 0x80172a7b, 0xee101d14, 0xef2221f3,
   //         0xb5bf7be1, 0xab57e80c, 0xfa24ee53, 0x37dab9aa,
   //         0xd353180b, 0xf1c5d9ed, 0xd6465866, 0x7a048625,
   //         0x892b7ef6, 0x2ca9170f, 0xa8a3f324, 0x36be065f,
   //         0x57aee2ab, 0xb20f4dd9, 0xa0eaa2ee, 0xa678c37a,
   //         0x5792d2ae, 0xac449456, 0x51549f89, 0x0, 0x1}

   // Ce tableau represente les 1024 coefficients du polynome suivant
   // (z^(2^400) mod P(z)) mod 2
   // P(z) est le polynome caracteristique du generateur.
   private static final int [] pw = new int[]
                        { 0xe44294e, 0xef237eff, 0x5e8b6bfb, 0xa724e67a,
                         0x59994cfd, 0x6f7c3de1, 0x6735d50d, 0x4bfe199a,
                         0x39c28e61, 0xfd075266, 0x96cc6d1f, 0x5dc1a685,
                         0xd67fa444, 0xccc01b86,  0x8ff861c, 0xce113725,
                         0x66707603, 0x38abb0fd,  0x7681f64, 0x104535c5,
                         0xce4ae5f4, 0x50e37105, 0xd0c5f77f, 0x74c1ebf6,
                         0x2ccf1505, 0xd1f21b86, 0x9a6c402e, 0xea34a31c,
                         0x65e13d13, 0xde8f2f05, 0x89db804f, 0x8dc387f2};

   // Ce tableau represente les 1024 coefficients du polynome suivant
   // (z^(2^700) mod P(z)) mod 2
   // P(z) est le polynome caracteristique du generateur.
   private static final int [] pz = new int[]
                        {0x7cab7da4, 0xef28b275, 0x18ffa66a, 0x2aa41e52,
                         0x15b6bd86, 0x560d0d76, 0xcdeda011, 0x96231727,
                         0xeec6a7f2, 0x99fd2be6, 0x92afa886, 0xcca777f0,
                         0x972eff38, 0xa29f8e49, 0x22b4b9b6, 0x1089c898,
                         0x6d569b25, 0x879044c2, 0x5e41b523, 0x33f19dd6,
                         0x7c005fc5, 0x7f9a1907, 0x39bf9eed, 0x4bd86a74,
                          0xe1e47e3, 0x96ead7ac, 0xc834f9ee, 0xd9ff4a4f,
                         0x717f044c, 0xfd0e15e6,  0x6c18ef3, 0xbfdd2942};

   private static void initialisation() {

      curr_stream = new int[] {0xDE410B75, 0x904FA5C7, 0x8BD4701E, 0x011EA361,
                               0x6EB189E0, 0x7A2B0CE1, 0xE02631CA, 0x72EBA132,
                               0x5189DA0F, 0x3EB72A2C, 0x51ABE513, 0x6D9EA57C,
                               0x4D690BF1, 0x84217FCA, 0x7290DE1A, 0x429F5A48,
                               0x6EC42EF3, 0x960AB315, 0x72C3A743, 0x48E13BF1,
                               0x8917EAC8, 0x284AE026, 0x357BF240, 0x913B51AC,
                               0x136AF195, 0x361ABC18, 0x731AB725, 0x63D3A7C9,
                               0xE5F32A18, 0x91A8E164, 0x04EA61B5, 0xC72A6091};

      //reading the state transition matrices

      try {
         InputStream is = WELL1024.class.getClassLoader().
                          getResourceAsStream("umontreal/iro/lecuyer/rng/WELL1024.dat");
         if(is == null)
            throw new FileNotFoundException("Couldn't find WELL1024.dat");

         ObjectInputStream ois = new ObjectInputStream(is);

         Apw = (BitMatrix)ois.readObject();
         Apz = (BitMatrix)ois.readObject();

         ois.close();
      } catch(FileNotFoundException e) {
         e.printStackTrace();
         System.exit(1);
      } catch(IOException e) {
         e.printStackTrace();
         System.exit(1);
      } catch(ClassNotFoundException e) {
         e.printStackTrace();
         System.exit(1);
      }

      initialised = true;
   }

/*
   private void advanceSeed(int[] seed, BitMatrix bm) {
      BitVector bv = new BitVector(seed, 1024);

      bv = bm.multiply(bv);

      for(int i = 0; i < R; i++)
         seed[i] = bv.getInt(i);
   }
*/

   private void advanceSeed(int[] seed, int [] p) {
      int b;
      int [] x = new int[R];

      for (int i = 0; i < R; i++) {
         state[i] = seed[i];
      }
      state_i = 0;

      for (int j = 0; j < R; ++j) {
         b = p[j];
         for (int k = 0; k < W; ++k) {
            if ((b & 1) == 1) {
               for (int i = 0; i < R; i++) {
                  x[i] ^= state[(state_i + i) & MASK];
               }
            }
            b >>= 1;
            nextValue();
         }
      }

      for (int i = 0; i < R; i++) {
         seed[i] = x[i];
      }
   }

   private static void verifySeed(int[] seed) {
      if (seed.length < R)
         throw new IllegalArgumentException("Seed must contain " + R +
                                            " values");

      for(int i = 0; i < R; i++)
         if (seed[i] != 0)
            return;

      throw new IllegalArgumentException
      ("At least one of the element of the seed must not be 0.");
   }

   private WELL1024(int i) {
      //unit vector (to build the state transition matrices)
      state = new int[R];
      for(int j = 0; j < R; j++)
         state[j] = 0;
      state[i / W] = 1 << (i % W);
      state_i = 0;
   }
 \end{hide}
\end{code}

%%%%%%%%%%%%%%%%%%%%%%%%%%%%%%%%%%%%%%%%%%%%%%%%%%%%
\subsubsection* {Constructors}

\begin{code}
   public WELL1024() \begin{hide} {
      if(!initialised)
         initialisation();

      state = new int[R];
      stream = new int[R];
      substream = new int[R];

      for(int i = 0; i < R; i++)
         stream[i] = curr_stream[i];

 //     advanceSeed(curr_stream, Apz);
      advanceSeed(curr_stream, pz);

      resetStartStream();
   } \end{hide}
\end{code}
\begin{tabb} Constructs a new stream.
\end{tabb}
\begin{code}

   public WELL1024 (String name) \begin{hide} {
      this();
      this.name = name;
   } \end{hide}
\end{code}
\begin{tabb} Constructs a new stream with the identifier \texttt{name}
  (used in the \texttt{toString} method).
\end{tabb}
\begin{htmlonly}
  \param{name}{name of the stream}
\end{htmlonly}

%%%%%%%%%%%%%%%%%%%%%%%%%%%%%%%%%%%%%
\subsubsection* {Methods}
\begin{code}
   public static void setPackageSeed (int seed[]) \begin{hide} {
      verifySeed (seed);
      if(!initialised)
         initialisation();

      for(int i = 0 ; i < R; i++)
         curr_stream[i] = seed[i];
   } \end{hide}
\end{code}
\begin{tabb} Sets the initial seed of this class to the 32
  integers of array \texttt{seed[0..31]}.
  This will be the initial seed of the class and of the next created stream.
  At least one of the integers must be non-zero.
\end{tabb}
\begin{htmlonly}
  \param{seed}{array of 32 elements representing the seed}
\end{htmlonly}
\begin{code}

   public void setSeed (int seed[]) \begin{hide} {
      verifySeed (seed);

      for(int i = 0 ; i <  R; i ++)
         stream[i] = seed[i];

      resetStartStream();
   } \end{hide}
\end{code}
\begin{tabb} This method is discouraged for normal use.
  Initializes the stream at the beginning of a stream with the initial
  seed \texttt{seed[0..31]}. The seed must satisfy the same
  conditions as in \texttt{setPackageSeed}.
  This method only affects the specified stream; the others are not
  modified.  Hence after calling this method, the beginning of the streams
  will no longer be spaced $Z$ values apart.
  For this reason, this method should only be used in very exceptional cases;
  proper use of the \texttt{reset...} methods and of the stream constructor is
  preferable.
\end{tabb}
\begin{htmlonly}
  \param{seed}{array of 32 elements representing the seed}
\end{htmlonly}
\begin{code}

   public int[] getState() \begin{hide} {
      int[] result = new int[R];
      for(int i = 0 ; i < R; i ++)
         result[i] = state[(state_i + i) & MASK];
      return result;
   } \end{hide}
\end{code}
\begin{tabb} Returns the current state of the stream, represented as an
  array of 32 integers.
\end{tabb}
\begin{htmlonly}
  \return{the current state of the stream}
\end{htmlonly}
\begin{code}

\begin{hide}
   public void resetStartStream() {
      for(int i = 0; i < R; i++)
         substream[i] = stream[i];
      resetStartSubstream();
   }

   public void resetStartSubstream() {
      state_i = 0;
      for(int i = 0; i < R; i++)
         state[i] = substream[i];
   }

   public void resetNextSubstream() {
//      advanceSeed(substream, Apw);
      advanceSeed(substream, pw);
      resetStartSubstream();
   }

   public String toString()  {
      StringBuffer sb = new StringBuffer();

      if(name == null)
         sb.append("The state of this WELL1024 is : {");
      else
         sb.append("The state of " + name + " is : {");
      for(int i = 0; i < R - 1; i++)
         sb.append(state[(state_i + i) & MASK] + ", ");
      sb.append(state[(state_i + R - 1) & MASK] + "}");

      return sb.toString();
   }

   protected double nextValue() {
      int z0, z1, z2;

      z0    = state[(state_i + 31) & MASK];
      z1    = state[state_i] ^ (state[(state_i + M1) & MASK] ^
                                (state[(state_i + M1) & MASK] >>> 8));
      z2    = (state[(state_i + M2) & MASK] ^
               (state[(state_i + M2) & MASK] << 19)) ^
              (state[(state_i + M3) & MASK] ^
               (state[(state_i + M3) & MASK] << 14));
      state[state_i] = z1 ^ z2;
      state[(state_i + 31) & MASK] = (z0 ^ (z0 << 11)) ^
                                     (z1 ^ (z1 << 7)) ^ (z2 ^ (z2 << 13));
      state_i = (state_i + 31) & MASK;

      long result = state[state_i];

      return ((double) (result > 0 ? result : result + 0x100000000L) * NORM);

   }


 \end{hide}
\end{code}
\unmoved\begin{htmlonly}
  This method is only meant to be used during the compilation process.
  It is used to create the resource file the class need in order to
  run.
\end{htmlonly}
\begin{code}

   public WELL1024 clone() \begin{hide} {
      WELL1024 retour = null;
      retour = (WELL1024)super.clone();
         retour.state = new int[R];
         retour.substream = new int[R];
         retour.stream = new int[R];
      for (int i = 0; i<R; i++) {
         retour.state[i] = state[i];
         retour.substream[i] = substream[i];
         retour.stream[i] = stream[i];
      }
      return retour;
   }\end{hide}
\end{code}
 \begin{tabb} Clones the current generator and return its copy.
 \end{tabb}
 \begin{htmlonly}
   \return{A deep copy of the current generator}
\end{htmlonly}
\begin{code}
  \begin{hide}
   public static void main(String[] args) {
      if(args.length < 1) {
         System.err.println("Must provide the output file.");
         System.exit(1);
      }

      //computes the state transition matrices

      System.out.println("Creating the WELL1024 state " +
                         "transition matrices.");

      //the state transition matrices
      BitMatrix STp0, STpw, STpz;

      BitVector[] bv = new BitVector[1024];
      WELL1024 well;
      int[] vect = new int[R];

      for(int i = 0; i < 1024; i++) {
         well = new WELL1024(i);

         well.nextValue();
         for(int j = 0; j < R; j++)
            vect[j] = well.state[(well.state_i + j) & MASK];

         bv[i] = new BitVector(vect, 1024);
      }

      STp0 = (new BitMatrix(bv)).transpose();

      STpw = STp0.power2e(400);
      STpz = STpw.power2e(300);


      try {
         FileOutputStream fos = new FileOutputStream(args[0]);
         ObjectOutputStream oos = new ObjectOutputStream(fos);
         oos.writeObject(STpw);
         oos.writeObject(STpz);
         oos.close();
      } catch(FileNotFoundException e) {
         System.err.println("Couldn't create " + args[0]);
         e.printStackTrace();
      } catch(IOException e) {
         e.printStackTrace();
      }

   }

}\end{hide}
\end{code}

