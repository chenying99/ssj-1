\defclass {BetaSymmetricalDist}

Specializes the class \class{BetaDist} to the case of a \emph{symmetrical}
{\em beta\/} distribution over the interval $[0,1]$,
with shape parameters $\alpha = \beta$.
A faster inversion method is implemented here for this special case.
Because of the symmetry around 1/2, four series are used to compute the
\texttt{cdf}, two around $x = 0$ and two around $x = 1/2$.
Given $u$, one then solves each series for $x$ by using the
 Newton-Raphson method which shows quadratic convergence when the
starting iterate is close enough to the solution $x$.
% one uses a series expansion of the distribution around 0 and the Newton
% method of iteration for the solution of equations.
% ^^ Unclear. Expansion of the distribution?  Solution of which equations?

\bigskip\hrule

%%%%%%%%%%%%%%%%%%%%%%%%%%%%%%%%%%%%%%%%
\begin{code}\begin{hide}
/*
 * Class:        BetaSymmetricalDist
 * Description:  Symmetrical beta distribution
 * Environment:  Java
 * Software:     SSJ 
 * Copyright (C) 2001  Pierre L'Ecuyer and Université de Montréal
 * Organization: DIRO, Université de Montréal
 * @author       Richard Simard
 * @since        April 2005

 * SSJ is free software: you can redistribute it and/or modify it under
 * the terms of the GNU General Public License (GPL) as published by the
 * Free Software Foundation, either version 3 of the License, or
 * any later version.

 * SSJ is distributed in the hope that it will be useful,
 * but WITHOUT ANY WARRANTY; without even the implied warranty of
 * MERCHANTABILITY or FITNESS FOR A PARTICULAR PURPOSE.  See the
 * GNU General Public License for more details.

 * A copy of the GNU General Public License is available at
   <a href="http://www.gnu.org/licenses">GPL licence site</a>.
 */
\end{hide}
package  umontreal.iro.lecuyer.probdist;
\begin{hide}
import umontreal.iro.lecuyer.util.*;
import umontreal.iro.lecuyer.functions.MathFunction;
import optimization.*;
\end{hide}

public class BetaSymmetricalDist extends BetaDist\begin{hide} {
   private static final double PI_2 = 1.5707963267948966;   // Pi/2
   private static final int MAXI = 11;   // Max number of Newton iterations
   private static final int MAXIB = 50;  // Max number of bisection iterations
   private static final int MAXJ = 2000; // Max number of terms in series
   private static final double EPSSINGLE = 1.0e-5;
   private static final double EPSBETA = 1.0e-10; // < 0.75 sqrt(DBL_EPSILON)
   private static final double SQPI_2 = 0.88622692545275801; // Sqrt(Pi)/2
   private static final double LOG_SQPI_2 = -0.1207822376352453; // Ln(Sqrt(Pi)/2)
   private static final double ALIM1 = 100000.0; // limit for normal approximation
   private static final double LOG2 = 0.6931471805599453;   // Ln(2)
   private static final double LOG4 = 1.3862943611198906;   // Ln(4)
   private static final double INV2PI = 0.6366197723675813; // 2 / PI
   private double Ceta;
   private double logCeta;

   private static class Function implements MathFunction
   {
      protected int n;
      protected double sum;

      public Function (double sum, int n) {
         this.n = n;
         this.sum = sum;
      }

      public double evaluate (double x) {
         if (x <= 0.0) return 1.0e100;
         return (- 2.0 * n * Num.digamma (x) + n * 2.0 * Num.digamma (2.0 * x) + sum);
      }
   }


   private static double belog (double x)
   /*
    * Used in the Peizer and Pratt normal approximation of BetaSymCDF.
    *
    * This is the function   (1 - x*x + 2*x*log(x)) / ((1 - x)*(1 - x))
    */
   {
      if (x > 1.0)
         return -belog(1.0/x);
      if (x < 1.0e-200)
         return 1.0;
      if (x < 0.9)
         return (1.0 - x*x + 2.0*x*Math.log(x)) / ((1.0 - x)*(1.0 - x));
      if (x == 1.0)
         return 0.0;

      /* For x near 1, use a series expansion to avoid loss of precision. */
      double term;
      final double EPS = 1.0e-12;
      final double Y = 1.0 - x;
      double ypow = 1.0;
      double sum = 0.0;
      int j = 2;
      do {
         ypow *= Y;
         term = ypow / (j * (j + 1));
         sum += term;
         j++;
      } while (Math.abs (term / sum) > EPS);

      return 2.0 * sum;
   }
\end{hide}
\end{code}

%%%%%%%%%%%%%%%%%%%%%
\subsubsection* {Constructors}

\begin{code}

   public BetaSymmetricalDist (double alpha)\begin{hide} {
      super(alpha, alpha);
      setParams (alpha, 14);
   }\end{hide}
\end{code}
  \begin{tabb}  Constructs a \texttt{BetaSymmetricalDist} object with parameters
   $\alpha = \beta =$ \texttt{alpha}, over the unit interval $(0,1)$.
  \end{tabb}
\begin{code}

   public BetaSymmetricalDist (double alpha, int d)\begin{hide} {
      super(alpha, alpha, d);
      setParams (alpha, d);
   }\end{hide}
\end{code}
\begin{tabb}
   Same as \texttt{BetaSymmetricalDist (alpha)}, but using approximations of
   roughly \texttt{d} decimal digits of precision when computing the
   distribution, complementary distribution, and inverse functions.
\end{tabb}

%%%%%%%%%%%%%%%%%%%%%%%%%
\subsubsection* {Methods}

\begin{code}\begin{hide}

   public double cdf (double x) {
      return calcCdf (alpha, x, decPrec, logFactor, logBeta, logCeta, Ceta);
   }

   public double barF (double x) {
      return calcCdf (alpha, 1.0 - x, decPrec, logFactor, logBeta,
                      logCeta, Ceta);
   }

   public double inverseF (double u) {
      return calcInverseF (alpha, u, decPrec, logFactor, logBeta,
                           logCeta, Ceta);
   }\end{hide}

   public static double density (double alpha, double x)\begin{hide} {
      return density (alpha, alpha, 0.0, 1.0, x);
   }\end{hide}
\end{code}
\begin{tabb}
  Returns the density evaluated at $x$.
\end{tabb}
\begin{code}

   public static double cdf (double alpha, int d, double x)\begin{hide} {
      return calcCdf (alpha, x, d, Num.DBL_MIN, 0.0, 0.0, 0.0);
   }\end{hide}
\end{code}
\begin{tabb}  Same as
 \method{cdf}{double,double,int,double}~\texttt{(alpha, alpha, d, x)}.
\end{tabb}
\begin{code}

   public static double barF (double alpha, int d, double x)\begin{hide} {
      return cdf (alpha, d, 1.0 - x);
   }\end{hide}
\end{code}
\begin{tabb} Returns the complementary distribution function.
\end{tabb}
\begin{code}

   public static double inverseF (double alpha, double u)\begin{hide} {
      return calcInverseF (alpha, u, 14, Num.DBL_MIN, 0.0, 0.0, 0.0);
   }

   /*----------------------------------------------------------------------*/

   private static double series1 (double alpha, double x, double epsilon)
   /*
    * Compute the series for F(x).
    * This series is used for alpha < 1 and x close to 0.
    */
   {
      int j;
      double sum, term;
      double poc = 1.0;
      sum = 1.0 / alpha;
      j = 1;
      do {
         poc *= x * (j - alpha) / j;
         term = poc / (j + alpha);
         sum += term;
         ++j;
      } while ((term > sum * epsilon) && (j < MAXJ));

      return sum * Math.pow (x, alpha);
   }


   /*------------------------------------------------------------------------*/

   private static double series2 (double alpha, double y, double epsilon)
   /*
    * Compute the series for G(y).   y = 0.5 - x.
    * This series is used for alpha < 1 and x close to 1/2.
    */
   {
      int j;
      double term, sum;
      double poc;
      final double z = 4.0 * y * y;

      /* Compute the series for G(y) */
      poc = sum = 1.0;
      j = 1;
      do {
         poc *= z * (j - alpha) / j;
         term = poc / (2 * j + 1);
         sum += term;
         ++j;
      } while ((term > sum * epsilon) && (j < MAXJ));

      return sum * y;
   }


   /*------------------------------------------------------------------------*/

   private static double series3 (double alpha, double x, double epsilon)
   /*
    * Compute the series for F(x).
    * This series is used for alpha > 1 and x close to 0.
    */
   {
      int j;
      double sum, term;
      final double z = -x / (1.0 - x);

      sum = term = 1.0;
      j = 1;
      do {
         term *= z * (j - alpha) / (j + alpha);
         sum += term;
         ++j;
      } while ((Math.abs (term) > sum * epsilon) && (j < MAXJ));

      return sum * x;
   }


   /*------------------------------------------------------------------------*/

   private static double series4 (double alpha, double y, double epsilon)
   /*
    * Compute the series for G(y).   y = 0.5 - x.
    * This series is used for alpha > 1 and x close to 1/2.
    */
   {
      int j;
      double term, sum;
      final double z = 4.0 * y * y;

      term = sum = 1.0;
      j = 1;
      do {
         term *= z * (j + alpha - 0.5) / (0.5 + j);
         sum += term;
         ++j;
      } while ((term > sum * epsilon) && (j < MAXJ));

      return sum * y;
   }

   /*------------------------------------------------------------------------*/

   private static double Peizer (double alpha, double x)
   /*
    * Normal approximation of Peizer and Pratt
    */
   {
      final double y = 1.0 - x;
      double z;
      z = Math.sqrt ((1.0 - y * belog (2.0 * x) -
         x * belog (2.0 * y)) / ((2.0*alpha - 5.0 / 6.0) * x * y)) *
         (2.0*x - 1.0) * (alpha - 1.0 / 3.0 + 0.025 / alpha);

      return NormalDist.cdf01 (z);
   }

   /*-------------------------------------------------------------------------*/

   private static double inverse1 (
      double alpha,                // Shape parameter
      double bu,                   // u * Beta(alpha, alpha)
      int d                        // Digits of precision
      )
   /*
    * This method is used for alpha < 1 and x close to 0.
    */
   {
      int i, j;
      double x, xnew, poc, sum, term;
      final double EPSILON = EPSARRAY[d];

      // First term of series
      x = Math.pow (bu * alpha, 1.0 / alpha);

      // If T1/T0 is very small, neglect all terms of series except T0
      term = alpha * (1.0 - alpha) * x / (1.0 + alpha);
      if (term < EPSILON)
         return x;

      x = bu * alpha / (1.0 + term);
      xnew = Math.pow (x, 1.0/alpha);  // Starting point of Newton's iterates

      i = 0;
      do {
         ++i;
         x = xnew;

         /* Compute the series for F(x) */
         poc = 1.0;
         sum = 1.0 / alpha;
         j = 1;
         do {
            poc *= x * (j - alpha) / j;
            term = poc / (j + alpha);
            sum += term;
            ++j;
         } while ((term > sum * EPSILON) && (j < MAXJ));
         sum *= Math.pow (x, alpha);

         /* Newton's method */
         term = (sum - bu) * Math.pow (x*(1.0 - x), 1.0 - alpha);
         xnew = x - term;

      } while ((Math.abs (term) > EPSILON) && (i <= MAXI));

      return xnew;
   }

   /*----------------------------------------------------------------------*/

   private static double inverse2 (
      double alpha,                // Shape parameter
      double w,                    // (0.5 - u)B/pow(4, 1 - alpha)
      int d                        // Digits of precision
      )
   /*
    * This method is used for alpha < 1 and x close to 1/2.
    */
   {
      int i, j;
      double term, y, ynew, z, sum;
      double poc;
      final double EPSILON = EPSARRAY[d];

      term = (1.0 - alpha) * w * w * 4.0 / 3.0;
      /* If T1/T0 is very small, neglect all terms of series except T0 */
      if (term < EPSILON)
         return 0.5 - w;

      ynew = w / (1 + term);     /* Starting point of Newton's iterates */
      i = 0;
      do {
         ++i;
         y = ynew;
         z = 4.0 * y * y;

         // Compute the series for G(y)
         poc = sum = 1.0;
         j = 1;
         do {
            poc *= z * (j - alpha) / j;
            term = poc / (2 * j + 1);
            sum += term;
            ++j;
         } while ((term > sum * EPSILON) && (j < MAXJ));

         sum *= y;

         // Newton's method
         ynew = y - (sum - w) * Math.pow (1.0 - z, 1.0 - alpha);

      } while ((Math.abs (ynew - y) > EPSILON) && (i <= MAXI));

      return 0.5 - ynew;
   }


   /*---------------------------------------------------------------------*/

   private static double bisect (
      double alpha,                // Shape parameter
      double logBua,               // Ln(alpha * u * Beta(alpha, alpha))
      double a,                    // x is presumed in [a, b]
      double b,
      int d                        // Digits of precision
)
   /*
    * This method is used for alpha > 1 and u very close to 0. It will almost
    * never be called, if at all.
    */
   {
      int i, j;
      double z, sum, term;
      double x, xprev;
      final double EPSILON = EPSARRAY[d];

      if (a >= 0.5 || a > b) {
         a = 0.0;
         b = 0.5;
      }

      x = 0.5 * (a + b);
      i = 0;
      do {
         ++i;
         z = -x / (1 - x);

         /* Compute the series for F(x) */
         sum = term = 1.0;
         j = 1;
         do {
            term *= z * (j - alpha) / (j + alpha);
            sum += term;
            ++j;
         } while ((Math.abs (term/sum) >  EPSILON) && (j < MAXJ));
         sum *= x;

         /* Bisection method */
         term = Math.log (x * (1.0 - x));
         z = logBua - (alpha - 1.0) * term;
         if (z > Math.log(sum))
            a = x;
         else
            b = x;
         xprev = x;
         x = 0.5 * (a + b);

      } while ((Math.abs(xprev - x) > EPSILON) && (i < MAXIB));

      return x;
   }

   /*---------------------------------------------------------------------*/

   private static double inverse3 (
      double alpha,                // Shape parameter
      double logBua,               // Ln(alpha * u * Beta(alpha, alpha))
      int d                        // Digits of precision
      )
   /*
    * This method is used for alpha > 1 and x close to 0.
    */
   {
      int i, j;
      double z, x, w, xnew, sum = 0., term;
      double eps = EPSSINGLE;
      final double EPSILON = EPSARRAY[d];
      // For alpha <= 100000 and u < 1.0/(2.5 + 2.25*sqrt(alpha)), X0 is always
      // to the right of the solution, so Newton is certain to converge.
      final double X0 = 0.497;

      /* Compute starting point of Newton's iterates */
      w = logBua / alpha;
      x = Math.exp (w);
      term = (Math.log1p(-x) + logBua) / alpha;
      z = Math.exp (term);
      if (z >= 0.25)
          xnew = X0;
      else if (z > 1.0e-6)
          xnew = (1.0 - Math.sqrt(1.0 - 4.0*z)) / 2.0;
      else
          xnew = z;

      i = 0;
      do {
         ++i;
         if (xnew >= 0.5)
             xnew = X0;
         x = xnew;

         sum = Math.log (x * (1.0 - x));
         w = logBua - (alpha - 1.0) * sum;
         if (Math.abs (w) >= Num.DBL_MAX_EXP * Num.LN2) {
            xnew = X0;
            continue;
         }
         w = Math.exp (w);
         z = -x / (1 - x);

         /* Compute the series for F(x) */
         sum = term = 1.0;
         j = 1;
         do {
            term *= z * (j - alpha) / (j + alpha);
            sum += term;
            ++j;
         } while ((Math.abs (term/sum) > eps) && (j < MAXJ));
         sum *= x;

         /* Newton's method */
         term = (sum - w) / alpha;
         xnew = x - term;
         if (Math.abs(term) < 32.0*EPSSINGLE)
            eps = EPSILON;

      } while ( (Math.abs (xnew - x) > sum * EPSILON) &&
               (Math.abs (xnew - x) > EPSILON) && (i <= MAXI));

      /* If Newton has not converged with enough precision, call bisection
         method. It is very slow, but will be called very rarely. */
      if (i >= MAXI && Math.abs (xnew - x) > 10.0 * EPSILON)
         return bisect (alpha, logBua, 0.0, 0.5, d);
      return xnew;
   }


   /*---------------------------------------------------------------------*/

   private static double inverse4 (
      double alpha,        // Shape parameter
      double logBva,       // Ln(B) + Ln(1/2 - u) + (alpha - 1)*Ln(4)
      int d                // Digits of precision
      )
   /*
    * This method is used for alpha > 1 and x close to 1/2.
    */
   {
      int i, j;
      double term, sum, y, ynew, z;
      double eps = EPSSINGLE;
      final double EPSILON = EPSARRAY[d];

      ynew = Math.exp (logBva);    // Starting point of Newton's iterates
      i = 0;
      do {
         ++i;
         y = ynew;

         /* Compute the series for G(y) */
         z = 4.0 * y * y;
         term = sum = 1.0;
         j = 1;
         do {
            term *= z * (j + alpha - 0.5) / (0.5 + j);
            sum += term;
            ++j;
         } while ((term > sum * eps) && (j < MAXJ));
         sum *= y * (1.0 - z);

         /* Newton's method */
         term = Math.log1p (-z);
         term = sum - Math.exp (logBva - (alpha - 1.0) * term);
         ynew = y - term;
         if (Math.abs(term) < 32.0*EPSSINGLE)
            eps = EPSILON;

      } while ((Math.abs (ynew - y) > EPSILON) &&
               (Math.abs (ynew - y) > sum*EPSILON) && (i <= MAXI));

      return 0.5 - ynew;
   }


   /*---------------------------------------------------------------------*/

   private static double PeizerInverse (double alpha, double u)
   {
      /* Inverse of the normal approximation of Peizer and Pratt */
      double t1, t3, xprev;
      final double C2 = alpha - 1.0 / 3.0 + 0.025 / alpha;
      final double z = NormalDist.inverseF01 (u);
      double x = 0.5;
      double y = 1.0 - x;
      int i = 0;

      do {
         i++;
         t1 = (2.0 * alpha - 5.0 / 6.0) * x * y;
         t3 = 1.0 - y * belog (2.0 * x) - x * belog (2.0 * y);
         xprev = x;
         x = 0.5 + 0.5 * z * Math.sqrt(t1 / t3) / C2;
         y = 1.0 - x;
      } while (i <= MAXI && Math.abs (x - xprev) > EPSBETA);

      return x;
   }

   /*---------------------------------------------------------------------*/

   private static void CalcB4 (double alpha, double [] bc, double epsilon)
   {
      double temp;
      double pB = 0.0;
      double plogB = 0.0;
      double plogC = 0.0;

      /* Compute Beta(alpha, alpha) or Beta(alpha, alpha)*4^(alpha-1). */

      if (alpha <= EPSBETA) {
         /* For a -> 0, B(a,a) = (2/a)*(1 - 1.645*a^2 + O(a^3)) */
         pB = 2.0 / alpha;
	      plogB = Math.log(pB);
         plogC = plogB + (alpha - 1.0)*LOG4;

      } else if (alpha <= 1.0) {
         temp = Num.lnGamma(alpha);
	      plogB = 2.0 * temp - Num.lnGamma(2.0*alpha);
         plogC = plogB + (alpha - 1.0)*LOG4;
         pB = Math.exp(plogB);

      } else if (alpha <= 10.0) {
         plogC = Num.lnGamma(alpha) - Num.lnGamma(0.5 + alpha) + LOG_SQPI_2;
         plogB = plogC - (alpha - 1.0)*LOG4;
         pB = Math.exp(plogB);

      } else if (alpha <= 200.0) {
         /* Convergent series for Gamma(x + 0.5) / Gamma(x) */
         double term = 1.0;
         double sum = 1.0;
         int i = 1;
         while (term > epsilon*sum) {
            term *= (i - 1.5)*(i - 1.5) /(i*(alpha + i - 1.5));
            sum += term;
            i++;
         }
         temp = SQPI_2 / Math.sqrt ((alpha - 0.5)*sum);
         plogC = Math.log(temp);
         plogB = plogC - (alpha - 1.0)*LOG4;
         pB = Math.exp(plogB);

      } else {
         /* Asymptotic series for Gamma(a + 0.5) / (Gamma(a) * Sqrt(a)) */
         double z = 1.0 / (8.0*alpha);
         temp = 1.0 + z*(-1.0 + z*(0.5 + z*(2.5 - z*(2.625 + 49.875*z))));
         /* This is 4^(alpha - 1)*B(alpha, alpha) */
         temp = SQPI_2 / (Math.sqrt(alpha) * temp);
         plogC = Math.log(temp);
         plogB = plogC - (alpha - 1.0)*LOG4;
         pB = Math.exp(plogB);
      }
      bc[0] = pB;
      bc[1] = plogB;
      bc[2] = plogC;
   }

   /*---------------------------------------------------------------------*/

   private static double calcInverseF (double alpha, double u, int d,
          double logFact, double logBeta, double logCeta, double Ceta) {
      if (alpha <= 0.0)
         throw new IllegalArgumentException ("alpha <= 0");
      if (u > 1.0 || u < 0.0)
         throw new IllegalArgumentException ("u not in [0,1]");
      if (u == 0.0) return 0.0;
      if (u == 1.0) return 1.0;
      if (u == 0.5) return 0.5;

      // Case alpha = 1 is the uniform law
      if (alpha == 1.0) return u;

      // Case alpha = 1/2 is the arcsin law
      double temp;
      if (alpha == 0.5) {
         temp = Math.sin (u * PI_2);
         return temp * temp;
      }

      if (alpha > ALIM1)
         return PeizerInverse (alpha, u);

      boolean isUpper;             // True if u > 0.5
      if (u > 0.5) {
         isUpper = true;
         u = 1.0 - u;
      } else
         isUpper = false;

      double x;
      double C = 0.0, B = 0.0, logB = 0.0, logC = 0.0;

      if (logFact == Num.DBL_MIN) {
         double [] bc = new double[] {0.0, 0.0, 0.0};
         CalcB4 (alpha, bc, EPSARRAY[d]);
         B = bc[0]; logB = bc[1]; logC = bc[2];
	 C = Math.exp(logC);
      } else {
         B = 1.0/ Math.exp(logFact);
         logB = logBeta;
         logC = logCeta;
         C = Ceta;
      }

      if (alpha <= 1.0) {
         // First term of integrated series around 1/2
	 double y0 = C * (0.5 - u);
         if (y0 > 0.25)
            x = inverse1 (alpha, B * u, d);
         else
            x = inverse2 (alpha, y0, d);

      } else {
         if (u < 1.0 / (2.5 + 2.25*Math.sqrt(alpha))) {
            double logBua = logB + Math.log (u * alpha);
            x = inverse3 (alpha, logBua, d);
         } else {
            // logBva = Ln(Beta(a,a) * (0.5 - u)*pow(4, a - 1)
            double logBva = logC - LOG2 + Math.log1p (-2.0*u);

            x = inverse4 (alpha, logBva, d);
         }
      }

      if (isUpper)
         return 1.0 - x - Num.DBL_EPSILON;
      else
         return x;
   }

   /*---------------------------------------------------------------------*/

   private static double calcCdf (double alpha, double x, int d,
           double logFact, double logBeta, double logCeta, double Ceta) {
      double temp, u, logB = 0.0, logC = 0.0, C = 0.0;
      boolean isUpper;                   /* True if x > 0.5 */
      double B = 0.0;                    /* Beta(alpha, alpha) */
      double x0;
      final double EPSILON = EPSARRAY[d];  // Absolute precision

      if (alpha <= 0.0)
         throw new IllegalArgumentException ("alpha <= 0");

      if (x <= 0.0) return 0.0;
      if (x >= 1.0) return 1.0;
      if (x == 0.5) return 0.5;
      if (alpha == 1.0) return x;         /* alpha = 1 is the uniform law */
      if (alpha == 0.5)                   /* alpha = 1/2 is the arcsin law */
         return INV2PI * Math.asin(Math.sqrt(x));

      if (alpha > ALIM1)
         return Peizer (alpha, x);

      if (x > 0.5) {
         x = 1.0 - x;
         isUpper = true;
      } else
         isUpper = false;

      if (logFact == Num.DBL_MIN) {
         double [] bc = new double[3];
         bc[0] = B; bc[1] = logB; bc[2] = logC;
         CalcB4 (alpha, bc, EPSILON);
         B = bc[0]; logB = bc[1]; logC = bc[2];
 	 C = Math.exp(logC);
     } else {
         B = 1.0/ Math.exp(logFact);
         logB = logBeta;
         logC = logCeta;
         C = Ceta;
      }

      if (alpha <= 1.0) {
         /* For x = x0, both series use the same number of terms to get the
            required precision */
         if (x > 0.25) {
            temp = -Math.log (alpha);
            if (alpha >= 1.0e-6)
               x0 = 0.25 + 0.005 * temp;
            else
               x0 = 0.13863 + .01235 * temp;
         } else
           x0 = 0.25;

         if (x <= x0)
            u = (series1 (alpha, x, EPSILON)) / B;
         else
            u = 0.5 - (series2 (alpha, 0.5 - x, EPSILON)) / C;

      } else {                        /* 1 < alpha < ALIM1 */
         if (alpha < 400.0)
            x0 = 0.5 - 0.9 / Math.sqrt(4.0*alpha);
         else
            x0 = 0.5 - 1.0 / Math.sqrt(alpha);
         if (x0 < 0.25)
            x0 = 0.25;

         if (x <= x0) {
            temp = (alpha - 1.0) * Math.log (x * (1.0 - x))  - logB;
            u = series3 (alpha, x, EPSILON) * Math.exp(temp) / alpha;

         } else {
            final double y = 0.5 - x;
            if (y > 0.05) {
               temp = Math.log(1.0 - 4.0*y*y);
            } else {
               u = 4.0*y*y;
               temp = -u * (1.0 + u * (0.5 + u *(1.0/3.0 + u*(0.25 +
                      u*(0.2 + u*(1.0/6.0 + u*1.0/7.0))))));
            }
            temp = alpha * temp - logC;
            u = 0.5 - (series4 (alpha, y, EPSILON)) * Math.exp(temp);
         }
      }

      if (isUpper)
         return 1.0 - u;
      else
         return u;
   }

   public double getMean() {
      return 0.5;
   }

   public double getVariance() {
      return BetaSymmetricalDist.getVariance (alpha);
   }

   public double getStandardDeviation() {
      return BetaSymmetricalDist.getStandardDeviation (alpha);
   }

\end{hide}\end{code}
\begin{tabb}
  Returns the inverse distribution function evaluated at $u$, for the
  symmetrical beta distribution over the interval $[0,1]$, with shape
  parameters $0 < \alpha = \beta$ = \texttt{alpha}.
  Uses four different hypergeometric series
  to compute the distribution $u = F(x)$
  (for the four cases $x$ close to 0 and $\alpha < 1$,
     $x$ close to 0 and $\alpha > 1$,  $x$ close to 1/2 and $\alpha < 1$,
  and  $x$ close to 1/2 and $\alpha > 1$),
  which are then solved by Newton's method for the solution of equations.
  For $\alpha > 100000$, uses a normal approximation given in \cite{tPEI68a}.
\hpierre{To be done: It should be possible to make the non-static
  method more efficient by precomputing certain quantities.}
\end{tabb}
\begin{code}

   public static double[] getMLE (double[] x, int n)\begin{hide} {
      if (n <= 0)
         throw new IllegalArgumentException ("n <= 0");

      double var = 0.0;
      double sum = 0.0;
      for (int i = 0; i < n; i++)
      {
         var += ((x[i] - 0.5) * (x[i] - 0.5));
         if (x[i] > 0.0 && x[i] < 1.0)
            sum += Math.log (x[i] * (1.0 - x[i]));
         else
            sum -= 709.0;
      }
      var /= (double) n;

      Function f = new Function (sum, n);

      double[] parameters = new double[1];
      double alpha0 = (1.0 - 4.0 * var) / (8.0 * var);

      double a = alpha0 - 5.0;
      if (a <= 0.0)
         a = 1e-15;

      parameters[0] = RootFinder.brentDekker (a, alpha0 + 5.0, f, 1e-5);

      return parameters;
   }\end{hide}
\end{code}
\begin{tabb}
   Estimates the parameter $\alpha$ of the symmetrical beta  distribution
   over the interval [0, 1] using the maximum likelihood method, from the
   $n$ observations $x[i]$, $i = 0, 1, \ldots, n-1$. The estimate is returned
   in element 0 of the returned array.
   \begin{detailed}
   The maximum likelihood estimator $\hat{\alpha}$ satisfies the equation
   \begin{eqnarray*}
  \psi(\hat\alpha) - \psi(2\hat\alpha) = \frac1{2n} \sum_{i=1}^{n} \ln(x_i(1 - x_i))
   \end{eqnarray*}
   where  $\bar{x}_n$ is the average of $x[0], \ldots, x[n-1]$, and
   $\psi$ is the logarithmic derivative of the Gamma function
   $\psi(x) = \Gamma'(x) / \Gamma(x)$.
   \end{detailed}
\end{tabb}
\begin{htmlonly}
   \param{x}{the list of observations to use to evaluate parameters}
   \param{n}{the number of observations to use to evaluate parameters}
   \return{returns the parameter [$\hat{\alpha}$]}
\end{htmlonly}
\begin{code}

   public static BetaSymmetricalDist getInstanceFromMLE (double[] x, int n)\begin{hide} {
      double parameters[] = getMLE (x, n);
      return new BetaSymmetricalDist (parameters[0]);
   }\end{hide}
\end{code}
\begin{tabb}
   Creates a new instance of a symmetrical beta distribution with parameter $\alpha$
   estimated using the maximum likelihood method based on the $n$ observations
   $x[i]$, $i = 0, 1, \ldots, n-1$.
\end{tabb}
\begin{htmlonly}
   \param{x}{the list of observations to use to evaluate parameters}
   \param{n}{the number of observations to use to evaluate parameters}
\end{htmlonly}
\begin{code}

   public static double getMean (double alpha)\begin{hide} {
      if (alpha <= 0.0)
         throw new IllegalArgumentException ("alpha <= 0");

      return 0.5;
   }\end{hide}
\end{code}
\begin{tabb}  Computes and returns the mean $E[X] = 1/2$ of the symmetrical beta
   distribution with parameter $\alpha$.
\end{tabb}
\begin{htmlonly}
   \return{the mean of the symmetrical beta distribution $E[X] = 1/2$}
\end{htmlonly}
\begin{code}

   public static double getVariance (double alpha)\begin{hide} {
      if (alpha <= 0.0)
         throw new IllegalArgumentException ("alpha <= 0");

      return (1 / (8 * alpha + 4));
   }\end{hide}
\end{code}
\begin{tabb}  Computes and returns the variance, $\mbox{Var}[X] = 1/(8\alpha + 4)$,
   of the symmetrical beta distribution with parameter $\alpha$.
\end{tabb}
\begin{htmlonly}
   \return{the variance of the symmetrical beta distribution
    $\mbox{Var}[X] = 1 / [4 (2\alpha + 1)]$}
\end{htmlonly}
\begin{code}

   public static double getStandardDeviation (double alpha)\begin{hide} {
      if (alpha <= 0.0)
         throw new IllegalArgumentException ("alpha <= 0");

      return (1 / Math.sqrt(8 * alpha + 4));
   }\end{hide}
\end{code}
\begin{tabb}  Computes and returns the standard deviation of the
   symmetrical beta distribution with parameter $\alpha$.
\end{tabb}
\begin{htmlonly}
   \return{the standard deviation of the symmetrical beta distribution}
\end{htmlonly}
\begin{code}\begin{hide}

   public void setParams (double alpha, double beta, double a, double b, int d) {
      // We don't want to calculate Beta, logBeta and logFactor twice.
      if (a >= b)
         throw new IllegalArgumentException ("a >= b");
      this.decPrec = d;
      supportA = this.a = a;
      supportB = this.b = b;
      bminusa = b - a;
    }

   private void setParams (double alpha, int d) {
      if (alpha <= 0.0)
         throw new IllegalArgumentException ("alpha <= 0");
      this.alpha = alpha;
      beta = alpha;

      double [] bc = new double[] {0.0, 0.0, 0.0};
      CalcB4 (alpha, bc, EPSARRAY[d]);
      Beta = bc[0]; logBeta = bc[1]; logCeta = bc[2];
      Ceta = Math.exp (logCeta);
      if (Beta > 0.0)
         logFactor = -logBeta - (2.0*alpha - 1) * Math.log(bminusa);
      else
         logFactor = 0.0;
    }\end{hide}
\end{code}
\begin{code}

   public double[] getParams ()\begin{hide} {
      double[] retour = {alpha};
      return retour;
   }\end{hide}
\end{code}
\begin{tabb}
   Return a table containing the parameter of the current distribution.
\end{tabb}
\begin{hide}\begin{code}

   public String toString ()\begin{hide} {
      return getClass().getSimpleName() + " : alpha = " + alpha;
   }\end{hide}
\end{code}
\begin{tabb}
   Returns a \texttt{String} containing information about the current distribution.
\end{tabb}\end{hide}
\begin{code}\begin{hide}
}\end{hide}
\end{code}
