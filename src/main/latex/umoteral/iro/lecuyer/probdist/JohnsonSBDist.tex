\defclass {JohnsonSBDist}

Extends the class \class{ContinuousDistribution} for
the {\em Johnson $S_B$\/} distribution\latex{ (see \cite[page 314]{sLAW00a})}
with shape parameters $\gamma$ and $\delta > 0$, location parameter $\xi$, 
and scale parameter $\lambda>0$.
Denoting $y=(x-\xi)/\lambda$, the density is
\eq
 f(x) = \html{\delta/(}\latex{\frac {\delta}} {\lambda y(1 - y)\sqrt{2\pi}}
  \exp(-(1/2)\left[\gamma+\delta\ln \left(y/(1 - y)\right) \right]^2)\html{)}
  \mbox { for } \xi < x < \xi+\lambda,     \eqlabel{eq:JohnsonSB-density}
\endeq
and 0 elsewhere.  The distribution function is
\eq
 F(x) = \Phi [\gamma + \delta\ln (y/(1-y))],
     \mbox { for } \xi < x < \xi+\lambda,   \eqlabel{eq:JohnsonSB-dist}
\endeq
where $\Phi$ is the standard normal distribution function.
The inverse distribution function is
\eq
  F^{-1}(u) = \xi + \lambda (1/(1+e^{-v(u)})) \qquad\mbox{for }  0 \le u \le 1,
                                      \eqlabel{eq:JohnsonSB-inverse}
\endeq  
where
\eq
  v(u) = [\Phi^{-1}(u) - \gamma]/\delta.
\endeq

This class relies on the methods  \clsexternalmethod{}{NormalDist}{cdf01}{} and
  \clsexternalmethod{}{NormalDist}{inverseF01}{}
of \class{NormalDist} to approximate $\Phi$ and $\Phi^{-1}$.


\bigskip\hrule

%%%%%%%%%%%%%%%%%%%%%%%%%%%%%%%%%%%%%%%%
\begin{code}
\begin{hide}
/*
 * Class:        JohnsonSBDist
 * Description:  Johnson S_Bdistribution
 * Environment:  Java
 * Software:     SSJ 
 * Copyright (C) 2001  Pierre L'Ecuyer and Université de Montréal
 * Organization: DIRO, Université de Montréal
 * @author       
 * @since

 * SSJ is free software: you can redistribute it and/or modify it under
 * the terms of the GNU General Public License (GPL) as published by the
 * Free Software Foundation, either version 3 of the License, or
 * any later version.

 * SSJ is distributed in the hope that it will be useful,
 * but WITHOUT ANY WARRANTY; without even the implied warranty of
 * MERCHANTABILITY or FITNESS FOR A PARTICULAR PURPOSE.  See the
 * GNU General Public License for more details.

 * A copy of the GNU General Public License is available at
   <a href="http://www.gnu.org/licenses">GPL licence site</a>.
 */
\end{hide}
package umontreal.iro.lecuyer.probdist;\begin{hide}
import umontreal.iro.lecuyer.util.Num;
\end{hide}

public class JohnsonSBDist extends ContinuousDistribution\begin{hide} {
   private double gamma;
   private double delta;
   private double xi;
   private double lambda;\end{hide}
\end{code}
   
%%%%%%%%%%%%%%%%%%%%%%%%%%%%%%%%%%%%%%%%%
\subsubsection* {Constructor}

\begin{code}

   public JohnsonSBDist (double gamma, double delta,
                         double xi, double lambda)\begin{hide} {
      setParams (gamma, delta, xi, lambda);
   }\end{hide}
\end{code}
  \begin{tabb} Constructs a \texttt{JohnsonSBDist} object
   with shape parameters $\gamma$ and $\delta$,
   location parameter $\xi$ and scale parameter $\lambda$.
  \end{tabb}

%%%%%%%%%%%%%%%%%%%%%%%%%%%%%%%%%%%
\subsubsection* {Methods}
\begin{hide}
\begin{code}

   public double density (double x) {
      return density (gamma, delta, xi, lambda, x);
   }
       
   public double cdf (double x) {
      return cdf (gamma, delta, xi, lambda, x);
   }
     
   public double barF (double x) {
      return barF (gamma, delta, xi, lambda, x);
   }

   public double inverseF (double u){
      return inverseF (gamma, delta, xi, lambda, u);
   }
\end{code}
\end{hide}\begin{code}

   public static double density (double gamma, double delta,
                                 double xi, double lambda, double x)\begin{hide} {
      if (lambda <= 0)
         throw new IllegalArgumentException ("lambda <= 0");
      if (delta <= 0)
         throw new IllegalArgumentException ("delta <= 0");
      if (x <= xi || x >= (xi+lambda))
         return 0.0;
      double y = (x - xi)/lambda;
      double t1 = gamma + delta*Math.log (y/(1.0 - y));
      return delta/(lambda*y*(1.0 - y)*Math.sqrt (2.0*Math.PI))*
           Math.exp (-t1*t1/2.0);
   }\end{hide}
\end{code}
\begin{tabb} Computes the density function (\ref{eq:JohnsonSB-density}).
\end{tabb}
\begin{code}      

   public static double cdf (double gamma, double delta,
                             double xi, double lambda, double x)\begin{hide} {
      if (lambda <= 0)
         throw new IllegalArgumentException ("lambda <= 0");
      if (delta <= 0)
         throw new IllegalArgumentException ("delta <= 0");
      if (x <= xi)
         return 0.0;
      if (x >= xi+lambda)
         return 1.0;
      double y = (x - xi)/lambda;
      return NormalDist.cdf01 (gamma + delta*Math.log (y/(1.0 - y)));
   }\end{hide}
\end{code}
 \begin{tabb}
  Computes the distribution function (\ref{eq:JohnsonSB-dist}).
 \end{tabb}
\begin{code}

   public static double barF (double gamma, double delta,
                              double xi, double lambda, double x)\begin{hide} {
      if (lambda <= 0)
         throw new IllegalArgumentException ("lambda <= 0");
      if (delta <= 0)
         throw new IllegalArgumentException ("delta <= 0");
      if (x <= xi)
         return 1.0;
      if (x >= xi+lambda)
         return 0.0;
      double y = (x - xi)/lambda;
      return NormalDist.barF01 (gamma + delta*Math.log (y/(1.0 - y)));
   }\end{hide}
\end{code}
  \begin{tabb}
  Computes the complementary distribution.
 \end{tabb}
\begin{code}

   public static double inverseF (double gamma, double delta,
                                  double xi, double lambda, double u)\begin{hide} {
      if (lambda <= 0)
         throw new IllegalArgumentException ("lambda <= 0");
      if (delta <= 0)
         throw new IllegalArgumentException ("delta <= 0");
      if (u > 1.0 || u < 0.0)
          throw new IllegalArgumentException ("u not in [0,1]");

      double v, z;

      if (u >= 1.0)    // if u == 1, in fact
          return xi+lambda;
      if (u <= 0.0)    // if u == 0, in fact
          return xi;

      z = NormalDist.inverseF01 (u);
      v = (z - gamma)/delta;

      if ((z >= XBIG) || (v >= Num.DBL_MAX_EXP*Num.LN2))
            return xi + lambda;

        if ((z <= -XBIG) || (v <= -Num.DBL_MAX_EXP*Num.LN2))
            return xi;

        v = Math.exp (v);
        return (xi + (xi+lambda)*v)/(1.0 + v);
   }\end{hide}
\end{code}
  \begin{tabb}
  Computes the inverse of the distribution (\ref{eq:JohnsonSB-inverse}).
 \end{tabb}
\begin{code}

   public double getGamma()\begin{hide} {
      return gamma;
   }\end{hide}
\end{code}
  \begin{tabb}
  Returns the value of $\gamma$ for this object.
 \end{tabb}
\begin{code}      

   public double getDelta()\begin{hide} {
      return delta;
   }\end{hide}
\end{code}
  \begin{tabb}
  Returns the value of $\delta$ for this object.
 \end{tabb}
\begin{code}      

   public double getXi()\begin{hide} {
      return xi;
   }\end{hide}
\end{code}
  \begin{tabb}
  Returns the value of $\xi$ for this object.
 \end{tabb}
\begin{code}      

   public double getLambda()\begin{hide} {
      return lambda;
   }
\end{hide}
\end{code}
  \begin{tabb}
  Returns the value of $\lambda$ for this object.
 \end{tabb}
\begin{code}      

   public void setParams(double gamma, double delta, double xi,
                         double lambda)\begin{hide} {
      if (lambda <= 0)
         throw new IllegalArgumentException ("lambda <= 0");
      if (delta <= 0)
         throw new IllegalArgumentException ("delta <= 0");
      this.gamma = gamma;
      this.delta = delta;
      this.xi = xi;
      this.lambda = lambda;
      supportA = xi;
      supportB = xi + lambda;
   }\end{hide}
\end{code}
  \begin{tabb}
  Sets the value of the parameters $\gamma$, $\delta$, $\xi$ and
  $\lambda$ for this object.
 \end{tabb}
 \begin{code}

   public double[] getParams ()\begin{hide} {
      double[] retour = {gamma, delta, xi, lambda};
      return retour;
   }\end{hide}
\end{code}
\begin{tabb}
   Return a table containing the parameters of the current distribution.
   This table is put in regular order: [$\gamma$, $\delta$, $\xi$, $\lambda$].
\end{tabb}
\begin{hide}\begin{code}

   public String toString ()\begin{hide} {
      return getClass().getSimpleName() + " : gamma = " + gamma + ", delta = " + delta + ", xi = " + xi + ", lambda = " + lambda;
   }\end{hide}
\end{code}
\begin{tabb}
   Returns a \texttt{String} containing information about the current distribution.
\end{tabb}\end{hide}
\begin{code}\begin{hide}
}\end{hide}
\end{code}
