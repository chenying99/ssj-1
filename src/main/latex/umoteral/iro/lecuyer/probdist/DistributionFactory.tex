\defclass{DistributionFactory}

This class implements a string API for the package \texttt{probdist}.
It uses Java Reflection to allow the creation of probability distribution
objects from a string.  This permits one to obtain distribution specifications 
from a file or dynamically from user input during program execution.
This string API is similar to that of 
\htmladdnormallink{UNURAN}{http://statistik.wu-wien.ac.at/unuran/} 
\cite{iLEY02a}.

The (static) methods of this class invoke the constructor specified 
in the string.  For example,
\begin{vcode}
 d = DistributionFactory.getContinuousDistribution ("NormalDist (0.0, 2.5)");
\end{vcode}
is equivalent to
\begin{vcode}
 d = NormalDist (0.0, 2.5);
\end{vcode}

The string that specifies the distribution (i.e., the formal parameter
\texttt{str} of the methods) must be a valid call of the constructor
of a class that extends \class{ContinuousDistribution} or
\class{DiscreteDistribution}, and all parameter values must be numerical
values (variable names are not allowed).
\hpierre{Because reconstructing the constructor call with variable names 
   would be more difficult.}
% Spaces in the string are ignored.
%
\begin{hide} %%%
If no parentheses follow, the default parameters are used, i.e.\ a
no-parameter constructor is searched and invoked.  Distribution parameters
are surrounded with parentheses and are separated from each other using
commas.  The order and types of the parameters should be the same
as the corresponding class constructor. When specifying decimal values, the dot
should be used for decimal separation, not the comma.  When specifying a 
 \texttt{float} or a \texttt{double}, one can use \texttt{-infinity} or 
 \texttt{infinity} to denote $-\infty$ and $+\infty$,
respectively.  This is parsed to \texttt{Double.NEGATIVE\_INFINITY}
and \texttt{Double.POSITIVE\_INFINITY}, respectively.  However, this
is not accepted by all probability distributions.
\hpierre{Vraiment utile?  Peut-etre pour specifier les intervalles 
   pour les distributions tronquees.}

For example, if one uses the string \texttt{Normal},
a \class{NormalDist} object with $\mu = 0$ and $\sigma = 1$ will be created.
If one uses \texttt{Exponential(2.5)}, an \class{ExponentialDist}
object with $\lambda=2.5$ will be constructed.
\end{hide}  %%%%%%%%%%%

The distribution parameters can also be estimated from a set of observations
instead of being passed to the constructor.  In that case, one passes the
vector of observations, and the constructor estimates the parameters by
the maximum likelihood method.


\bigskip\hrule

\begin{code}
\begin{hide}
/*
 * Class:        DistributionFactory
 * Description:  allows the creation of distribution objects from a string
 * Environment:  Java
 * Software:     SSJ 
 * Copyright (C) 2001  Pierre L'Ecuyer and Université de Montréal
 * Organization: DIRO, Université de Montréal
 * @author       
 * @since

 * SSJ is free software: you can redistribute it and/or modify it under
 * the terms of the GNU General Public License (GPL) as published by the
 * Free Software Foundation, either version 3 of the License, or
 * any later version.

 * SSJ is distributed in the hope that it will be useful,
 * but WITHOUT ANY WARRANTY; without even the implied warranty of
 * MERCHANTABILITY or FITNESS FOR A PARTICULAR PURPOSE.  See the
 * GNU General Public License for more details.

 * A copy of the GNU General Public License is available at
   <a href="http://www.gnu.org/licenses">GPL licence site</a>.
 */
\end{hide}
package umontreal.iro.lecuyer.probdist;
\begin{hide}
import java.lang.reflect.*;
import java.util.StringTokenizer;\end{hide}

public class DistributionFactory\begin{hide} {
   private DistributionFactory() {}   //  ????   Utile?

   public static Distribution getDistribution (String str) {
      // Extracts the name of the distribution.
      // If there is an open parenthesis, the name contains all the 
      // non-space characters preceeding it.If not,the name is the full string.

      int i = 0;
      str = str.trim();

      int idx = str.indexOf ('(', i);
      String distName;
      if (idx == -1)
         distName = str.substring (i).trim();
      else
         distName = str.substring (i, idx).trim();
 
      // Try to find the class in probdist package.
      Class<?> distClass;
      if (distName.equals ("String"))
         throw new IllegalArgumentException ("Invalid distribution name: " 
                                             + distName);
      try {
         distClass = Class.forName ("umontreal.iro.lecuyer.probdist." 
                                    + distName);
      }
      catch (ClassNotFoundException e) {
         // Look for a fully qualified classname whose constructor
         //  matches this string.
         try {
            distClass = Class.forName (distName);
            // We must check if the class implements Distribution 
            if (Distribution.class.isAssignableFrom(distClass) == false)
               throw new IllegalArgumentException 
                  ("The given class is not a Probdist distribution class.");
         }
         catch (ClassNotFoundException ex) {
            throw new IllegalArgumentException ("Invalid distribution name: " 
                                                + distName);
         }
      }

      String paramStr = "";
      if (idx != -1) {
         // Get the parameters from the string.
         int parFrom = idx;
         int parTo = str.lastIndexOf (')');
         // paramStr will contain the parameters without parentheses.
         paramStr = str.substring (parFrom + 1, parTo).trim();
         if (paramStr.indexOf ('(') != -1 || paramStr.indexOf (')') != -1)
            //All params are numerical,so parenthesis nesting is forbidden here
            throw new IllegalArgumentException ("Invalid parameter string: " 
                                                + paramStr);
      }

      if (paramStr.equals ("")) {
         // No parameter is given to the constructor.
         try {
            return (Distribution) distClass.newInstance();
         }
         catch (IllegalAccessException e) {
            throw new IllegalArgumentException 
                                         ("Default parameters not available");
         }
         catch (InstantiationException e) {
            throw new IllegalArgumentException 
                                         ("Default parameters not available");
         }
      }

      // Find the number of parameters and try to find a matching constructor.
      // Within probdist, there are no constructors with the same
      // number of arguments but with different types.
      // This simplifies the constructor selection scheme.
      StringTokenizer paramTok = new StringTokenizer (paramStr, ",");
      int nparams = paramTok.countTokens();
      Constructor[] cons = distClass.getConstructors();
      Constructor distCons = null;
      Class[] paramTypes = null;
      // Find a public constructor with the correct number of parameters.
      for (i = 0; i < cons.length; i++) {
         if (Modifier.isPublic (cons[i].getModifiers()) &&
             ((paramTypes = cons[i].getParameterTypes()).length == nparams)) {
            distCons = cons[i];
            break;
         }
      }
      if (distCons == null)
         throw new IllegalArgumentException ("Invalid parameter number");

      // Create the parameters for the selected constructor.
      Object[] instParams = new Object[nparams];
      for (i = 0; i < nparams; i++) {
         String par = paramTok.nextToken().trim();
         try {
            // We only need a limited set of parameter types here.
            if (paramTypes[i] == int.class)
               instParams[i] = new Integer (par);
            else if (paramTypes[i] == long.class)
               instParams[i] = new Long (par);
            else if (paramTypes[i] == float.class) {
               if (par.equalsIgnoreCase ("infinity") || par.equalsIgnoreCase 
                                                                 ("+infinity"))
                  instParams[i] = new Float (Float.POSITIVE_INFINITY);
               else if (par.equalsIgnoreCase ("-infinity"))
                  instParams[i] = new Float (Float.NEGATIVE_INFINITY);
               else
                  instParams[i] = new Float (par);
            }
            else if (paramTypes[i] == double.class) {
               if (par.equalsIgnoreCase ("infinity") || par.equalsIgnoreCase
                                                                 ("+infinity"))
                  instParams[i] = new Double (Double.POSITIVE_INFINITY);
               else if (par.equalsIgnoreCase ("-infinity"))
                  instParams[i] = new Double (Double.NEGATIVE_INFINITY);
               else
                  instParams[i] = new Double (par);
            }
            else
               throw new IllegalArgumentException
                  ("Parameter " + (i+1) + " type " + paramTypes[i].getName() +
                   "not supported");
         }
         catch (NumberFormatException e) {
            throw new IllegalArgumentException
               ("Parameter " + (i+1) + " of type " +
                paramTypes[i].getName()+" could not be converted from String");
         }
      }

      // Try to instantiate the distribution class.
      try {
         return (Distribution) distCons.newInstance (instParams);
      }
      catch (IllegalAccessException e) {
         return null;
      }
      catch (InstantiationException e) {
         return null;
      }
      catch (InvocationTargetException e) {
         return null;
      }
   }\end{hide}
\end{code}
\begin{hide} 
  Uses the Java Reflection API to construct a \class{ContinuousDistribution}{}
  , \class{DiscreteDistributionInt} or \class{DiscreteDistribution}{} object,
  as specified by the string \texttt{str}.
  This method throws exceptions
  if it cannot parse the given string and returns \texttt{null} if
  the distribution object simply could not be created due to a Java-specific
  instantiation problem.
\begin{htmlonly}
   \param{str}{distribution specification string}
   \return{a distribution object or \texttt{null} if it could not be instantiated}
   \exception{IllegalArgumentException}{if parsing problems occured when 
   reading \texttt{str}}
\end{htmlonly}
\end{hide}
\begin{code}

\begin{hide} @SuppressWarnings("unchecked")
\end{hide}   public static ContinuousDistribution getDistributionMLE
                    (String distName, double[] x, int n)\begin{hide} {

      Class<?> distClass;
      try
      {
         distClass = Class.forName ("umontreal.iro.lecuyer.probdist." + distName);
      }
      catch (ClassNotFoundException e)
      {
         try
         {
            distClass = Class.forName (distName);
         }
         catch (ClassNotFoundException ex)
         {
            throw new IllegalArgumentException ("Invalid distribution name: " 
                                                + distName);
         }
      }

      return getDistributionMLE ((Class<? extends ContinuousDistribution>)distClass, x, n);
   }\end{hide}
\end{code}
\begin{tabb}
   Uses the Java Reflection API to construct a \class{ContinuousDistribution}
   object by estimating parameters of the distribution using the maximum likelihood
   method based on the $n$ observations in table $x[i]$, $i = 0, 1, \ldots, n-1$.
\end{tabb}
\begin{htmlonly}
   \param{distName}{the name of the distribution to instanciate}
   \param{x}{the list of observations to use to evaluate parameters}
   \param{n}{the number of observations to use to evaluate parameters}
\end{htmlonly}
\begin{code}

\begin{hide} @SuppressWarnings("unchecked")
\end{hide}   public static DiscreteDistributionInt getDistributionMLE
                    (String distName, int[] x, int n)\begin{hide} {

      Class<?> distClass;
      try
      {
         distClass = Class.forName ("umontreal.iro.lecuyer.probdist." + distName);
      }
      catch (ClassNotFoundException e)
      {
         try
         {
            distClass = Class.forName (distName);
         }
         catch (ClassNotFoundException ex)
         {
            throw new IllegalArgumentException ("Invalid distribution name: " 
                                                + distName);
         }
      }

      return getDistributionMLE ((Class<? extends DiscreteDistributionInt>)distClass, x, n);
   }\end{hide}
\end{code}
\begin{tabb}
   Uses the Java Reflection API to construct a \class{DiscreteDistributionInt}
   object by estimating parameters of the distribution using the maximum likelihood
   method based on the $n$ observations in table $x[i]$, $i = 0, 1, \ldots, n-1$.
\end{tabb}
\begin{htmlonly}
   \param{distName}{the name of the distribution to instanciate}
   \param{x}{the list of observations to use to evaluate parameters}
   \param{n}{the number of observations to use to evaluate parameters}
\end{htmlonly}
\begin{code}

\begin{hide} @SuppressWarnings("unchecked")
\end{hide}   public static <T extends ContinuousDistribution> T getDistributionMLE
                    (Class<T> distClass, double[] x, int n)\begin{hide} {
      if (ContinuousDistribution.class.isAssignableFrom(distClass) == false)
               throw new IllegalArgumentException 
                  ("The given class is not a Probdist distribution class.");

      Method m;
      try
      {
         m = distClass.getMethod ("getInstanceFromMLE", double[].class, int.class);
      }
      catch (NoSuchMethodException e) {
         throw new IllegalArgumentException
         ("The given class does not provide the static method getInstanceFromMLE (double[],int)");
      }
      if (!Modifier.isStatic (m.getModifiers()) ||
          !distClass.isAssignableFrom (m.getReturnType()))
         throw new IllegalArgumentException
         ("The given class does not provide the static method getInstanceFromMLE (double[],int)");
      
      try
      {
         return (T)m.invoke (null, x, n);
      }
      catch (IllegalAccessException e) {
         return null;
      }
      catch (IllegalArgumentException e) {
         return null;
      }
      catch (InvocationTargetException e) {
         return null;
      }      
   }\end{hide}
\end{code}
\begin{tabb}
   Uses the Java Reflection API to construct a \class{ContinuousDistribution}
   object by estimating parameters of the distribution using the maximum likelihood
   method based on the $n$ observations in table $x[i]$, $i = 0, 1, \ldots, n-1$.
\end{tabb}
\begin{htmlonly}
   \param{distClass}{the class of the distribution to instanciate}
   \param{x}{the list of observations to use to evaluate parameters}
   \param{n}{the number of observations to use to evaluate parameters}
\end{htmlonly}
\begin{code}

\begin{hide} @SuppressWarnings("unchecked")
\end{hide}   public static <T extends DiscreteDistributionInt> T getDistributionMLE
                    (Class<T> distClass, int[] x, int n)\begin{hide} {
      if (DiscreteDistributionInt.class.isAssignableFrom(distClass) == false)
               throw new IllegalArgumentException 
                  ("The given class is not a discrete distribution class over integers.");
      
      Method m;
      try
      {
         m = distClass.getMethod ("getInstanceFromMLE", int[].class, int.class);
      }
      catch (NoSuchMethodException e) {
         throw new IllegalArgumentException
         ("The given class does not provide the static method getInstanceFromMLE (int[],int)");
      }
      if (!Modifier.isStatic (m.getModifiers()) ||
          !distClass.isAssignableFrom (m.getReturnType()))
         throw new IllegalArgumentException
         ("The given class does not provide the static method getInstanceFromMLE (int[],int)");
      
      try
      {
         return (T)m.invoke (null, x, n);
      }
      catch (IllegalAccessException e) {
         return null;
      }
      catch (IllegalArgumentException e) {
         return null;
      }
      catch (InvocationTargetException e) {
         return null;
      }      
   }\end{hide}
\end{code}
\begin{tabb}
   Uses the Java Reflection API to construct a \class{DiscreteDistributionInt}
   object by estimating parameters of the distribution using the maximum likelihood
   method based on the $n$ observations in table $x[i]$, $i = 0, 1, \ldots, n-1$.
\end{tabb}
\begin{htmlonly}
   \param{distClass}{the class of the distribution to instanciate}
   \param{x}{the list of observations to use to evaluate parameters}
   \param{n}{the number of observations to use to evaluate parameters}
\end{htmlonly}
\begin{code}

   public static ContinuousDistribution getContinuousDistribution (String str)\begin{hide} {
      return (ContinuousDistribution)getDistribution (str);
   }\end{hide}
\end{code}
\begin{tabb}  
  Uses the Java Reflection API to construct a \class{ContinuousDistribution}
  object by executing the code contained in the string \texttt{str}.
  This code should be a valid invocation of the constructor of a 
  \class{ContinuousDistribution} object.
  This method throws exceptions if it cannot parse the given string and 
  returns \texttt{null} if the distribution object could not be created due to 
  a Java-specific instantiation problem.
\hpierre{So the user must always verify if \texttt{null} was returned?} 
% Same as calling \method{getDistribution}{}
% and casting the result to \class{ContinuousDistribution}.
\end{tabb}
\begin{htmlonly}
   \param{str}{string that contains a call to the constructor of a continuous 
     distribution}
   \return{a continuous distribution object or \texttt{null} if it could not 
     be instantiated}
   \exception{IllegalArgumentException}{if parsing problems occured 
     when reading \texttt{str}}
   \exception{ClassCastException}{if the distribution string does not represent
     a continuous distribution}
\end{htmlonly}
\begin{code}

   public static DiscreteDistribution getDiscreteDistribution (String str)\begin{hide} {
      return (DiscreteDistribution)getDistribution (str);
   }
\end{hide}
\end{code}
\begin{tabb}  
  Same as \method{getContinuousDistribution}{}, but for discrete distributions
  over the real numbers.
\end{tabb}
\begin{htmlonly}
   \param{str}{string that contains a call to the constructor of a discrete
     distribution}
   \return{a discrete distribution object, or \texttt{null} if it could not 
     be instantiated}
   \exception{IllegalArgumentException}{if parsing problems occured when 
     reading \texttt{str}}
   \exception{ClassCastException}{if the distribution string does not represent
     a discrete distribution}
\end{htmlonly}
\begin{code}

   public static DiscreteDistributionInt getDiscreteDistributionInt (String str)\begin{hide} {
      return (DiscreteDistributionInt)getDistribution (str);
   }
}\end{hide}
\end{code}
\begin{tabb}  
  Same as \method{getContinuousDistribution}{}, but for discrete distributions
  over the integers.
\end{tabb}
\begin{htmlonly}
   \param{str}{string that contains a call to the constructor of a discrete
     distribution}
   \return{a discrete distribution object, or \texttt{null} if it could not 
     be instantiated}
   \exception{IllegalArgumentException}{if parsing problems occured when 
     reading \texttt{str}}
   \exception{ClassCastException}{if the distribution string does not represent
     a discrete distribution}
\end{htmlonly}
