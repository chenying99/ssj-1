\defclass {InverseGaussianDist}

Extends the class \class{ContinuousDistribution} for 
the {\em inverse Gaussian\/} distribution with location parameter
$\mu > 0$ and scale parameter $\lambda > 0$.
Its density is
\begin{htmlonly}
\eq
   f(x) = \sqrt{\lambda / (2\pi x^{3})} e^{-\lambda(x - \mu)^2 / (2\mu^2x)},
\qquad\mbox{ for } x > 0.
\endeq
The distribution function is given by
\eq
   F(x) = \Phi\left(\sqrt{\lambda/{x}}({x}/{\mu} - 1)\right) +
  e^{{2\lambda}/{\mu}}\Phi\left(-\sqrt{{\lambda}/{x}}({x}/{\mu} + 1)\right),
\endeq
where $\Phi$ is the standard normal distribution function.
\end{htmlonly}%
\begin{latexonly}%
\eq
 f(x) = \sqrt{\frac{\lambda}{2\pi x^{3}}}\; e^{{-\lambda(x - \mu)^2}/{(2\mu^2x)}},
\qquad\mbox {for } x > 0.
\eqlabel{eq:fInverseGaussian}
\endeq
The distribution function is given by
\eq
   F(x) = \Phi\left(\sqrt{\frac{\lambda}{x}}\left(\frac{x}{\mu} - 1\right)\right) +
          e^{({2\lambda}/{\mu})}\Phi\left(-\sqrt{\frac{\lambda}{x}}\left(\frac{x}{\mu} + 1\right)\right),
\eqlabel{eq:FInverseGaussian}
\endeq
where $\Phi$ is the standard normal distribution function.
\end{latexonly}%

The non-static versions of the methods \texttt{cdf}, \texttt{barF}, 
and \texttt{inverseF} call the static version of the same name.


\bigskip\hrule

%%%%%%%%%%%%%%%%%%%%%%%%%%%%%%%%%%%%%%%%
\begin{code}
\begin{hide}
/*
 * Class:        InverseGaussianDist
 * Description:  inverse Gaussian distribution
 * Environment:  Java
 * Software:     SSJ 
 * Copyright (C) 2001  Pierre L'Ecuyer and Université de Montréal
 * Organization: DIRO, Université de Montréal
 * @author       
 * @since

 * SSJ is free software: you can redistribute it and/or modify it under
 * the terms of the GNU General Public License (GPL) as published by the
 * Free Software Foundation, either version 3 of the License, or
 * any later version.

 * SSJ is distributed in the hope that it will be useful,
 * but WITHOUT ANY WARRANTY; without even the implied warranty of
 * MERCHANTABILITY or FITNESS FOR A PARTICULAR PURPOSE.  See the
 * GNU General Public License for more details.

 * A copy of the GNU General Public License is available at
   <a href="http://www.gnu.org/licenses">GPL licence site</a>.
 */
\end{hide}
package umontreal.iro.lecuyer.probdist;
\begin{hide}
import umontreal.iro.lecuyer.probdist.NormalDist;
import umontreal.iro.lecuyer.util.*;
import umontreal.iro.lecuyer.functions.MathFunction;
 \end{hide}

public class InverseGaussianDist extends ContinuousDistribution\begin{hide} {
   protected double mu;
   protected double lambda;

   private static class Function implements MathFunction {
      protected double mu;
      protected double lambda;
      protected double u;

      public Function (double mu, double lambda, double u) {
         this.mu = mu;
         this.lambda = lambda;
         this.u = u;
      }

      public double evaluate (double x) {
         return u - cdf(mu, lambda, x);
      }
   }
\end{hide}
\end{code}
%%%%%%%%%%%%%%%%%%%%%%%%%%%%%%%%%%%%%%%%%
\subsubsection* {Constructor}

\begin{code}

   public InverseGaussianDist (double mu, double lambda)\begin{hide} {
      setParams (mu, lambda);
   }\end{hide}
\end{code}
\begin{tabb}
   Constructs the {\em inverse Gaussian\/} distribution with parameters $\mu$ and $\lambda$.
\end{tabb}

%%%%%%%%%%%%%%%%%%%%%%%%%%%%%%%%%%%
\subsubsection* {Methods}

\begin{code}\begin{hide}

   public double density (double x) {
      return density (mu, lambda, x);
   }

   public double cdf (double x) {
      return cdf (mu, lambda, x);
   }

   public double barF (double x) {
      return barF (mu, lambda, x);
   }

   public double inverseF (double u) {
      return inverseF (mu, lambda, u);
   }

   public double getMean() {
      return getMean (mu, lambda);
   }

   public double getVariance() {
      return getVariance (mu, lambda);
   }

   public double getStandardDeviation() {
      return getStandardDeviation (mu, lambda);
   }\end{hide}

   public static double density (double mu, double lambda, double x)\begin{hide} {
      if (mu <= 0.0)
         throw new IllegalArgumentException ("mu <= 0");
      if (lambda <= 0.0)
         throw new IllegalArgumentException ("lambda <= 0");
      if (x <= 0.0)
         return 0.0;

      double sqrtX = Math.sqrt (x);

      return (Math.sqrt (lambda / (2 * Math.PI)) / (sqrtX * sqrtX * sqrtX) *
              Math.exp (-lambda * (x - 2 * mu + (mu * mu / x)) / (2 * mu * mu)));
   }\end{hide}
\end{code}
\begin{tabb} Computes the density function (\ref{eq:fInverseGaussian}) for the
     inverse gaussian distribution with parameters $\mu$ and $\lambda$,
     evaluated at $x$.
\end{tabb}
\begin{code}

   public static double cdf (double mu, double lambda, double x)\begin{hide} {
      if (mu <= 0.0)
         throw new IllegalArgumentException ("mu <= 0");
      if (lambda <= 0.0)
         throw new IllegalArgumentException ("lambda <= 0");
      if (x <= 0.0)
         return 0.0;
      double temp = Math.sqrt (lambda / x);
      double z = temp * (x / mu - 1.0);
      double w = temp * (x / mu + 1.0);

      // C'est bien un + dans    exp (2 * lambda / mu)
      return (NormalDist.cdf01 (z) +
              Math.exp (2 * lambda / mu) * NormalDist.cdf01 (-w));
   }\end{hide}
\end{code}
\begin{tabb}
   Computes the distribution function (\ref{eq:FInverseGaussian})
   of the inverse gaussian distribution with parameters $\mu$ and
   $\lambda$, evaluated at $x$.
 \end{tabb}
\begin{code}
   
   public static double barF (double mu, double lambda, double x)\begin{hide} {
      return 1.0 - cdf (mu, lambda, x);
   }\end{hide}
\end{code}
\begin{tabb}
   Computes the complementary distribution function of the inverse gaussian distribution
   with parameters $\mu$ and $\lambda$, evaluated at $x$.
 \end{tabb}
\begin{code}
   
   public static double inverseF (double mu, double lambda, double u)\begin{hide} {
      if (mu <= 0.0)
         throw new IllegalArgumentException ("mu <= 0");
      if (lambda <= 0.0)
         throw new IllegalArgumentException ("lambda <= 0");
      if (u < 0.0 || u > 1.0)
         throw new IllegalArgumentException ("u must be in [0,1]");
      if (u == 1.0)
         return Double.POSITIVE_INFINITY;
      if (u == 0.0)
         return 0.0;

      Function f = new Function (mu, lambda, u);

      // Find interval containing root = x*
      double sig = getStandardDeviation(mu, lambda);
      double x0 = 0.0;
      double x = mu;
      double v = cdf(mu, lambda, x);
      while (v < u) {
         x0 = x;
         x += 3.0*sig;
         v = cdf(mu, lambda, x);
      }

      return RootFinder.brentDekker (x0, x, f, 1e-12);
   }\end{hide}
\end{code}
\begin{tabb}
   Computes the inverse of the inverse gaussian distribution
   with parameters $\mu$ and $\lambda$.
 \end{tabb}
\begin{code}

   public static double[] getMLE (double[] x, int n)\begin{hide} {
      if (n <= 0)
         throw new IllegalArgumentException ("n <= 0");

      double parameters[];
      parameters = new double[2];
      double sum = 0;
      for (int i = 0; i < n; i++) {
         sum += x[i];   
      }
      parameters[0] = sum / (double) n;

      sum = 0;
      for (int i = 0; i < n; i++) {
         sum += ((1.0 / (double) x[i]) - (1.0 / parameters[0]));
      }
      parameters[1] = (double) n / (double) sum;

      return parameters;
   }\end{hide}
\end{code}
\begin{tabb}
   Estimates the parameters $(\mu, \lambda)$ of the inverse gaussian distribution
   using the maximum likelihood method, from the $n$ observations
   $x[i]$, $i = 0, 1,\ldots, n-1$. The estimates are returned in a two-element
    array, in regular order: [$\mu$, $\lambda$]. 
   \begin{detailed}
   The maximum likelihood estimators are the values $(\hat\mu , \hat\lambda)$ 
   that satisfy the equations:
   \begin{eqnarray*}
      \hat{\mu} & = & \bar{x}_n \\[5pt]
      \frac{1}{\hat{\lambda}} & = & \frac{1}{n} \sum_{i=1}^{n}
    \left(\frac{1}{x_i} - \frac{1}{\hat{\mu}}\right),
   \end{eqnarray*}
   where $\bar x_n$ is the average of $x[0],\dots,x[n-1]$, 
    \cite[page 271]{tJOH95a}.
   \end{detailed}
\end{tabb}
\begin{htmlonly}
   \param{x}{the list of observations used to evaluate parameters}
   \param{n}{the number of observations used to evaluate parameters}
   \return{returns the parameters [$\hat{\mu}$, $\hat{\lambda}$]}
\end{htmlonly}
\begin{code}

   public static InverseGaussianDist getInstanceFromMLE (double[] x, int n)\begin{hide} {
      double parameters[] = getMLE (x, n);
      return new InverseGaussianDist (parameters[0], parameters[1]);
   }\end{hide}
\end{code}
\begin{tabb}
   Creates a new instance of an inverse gaussian distribution with parameters
   $\mu$ and $\lambda$ estimated using the maximum likelihood method based on
   the $n$ observations $x[i]$, $i = 0, 1, \ldots, n-1$.
\end{tabb}
\begin{htmlonly}
   \param{x}{the list of observations to use to evaluate parameters}
   \param{n}{the number of observations to use to evaluate parameters}
\end{htmlonly}
\begin{code}

   public static double getMean (double mu, double lambda)\begin{hide} {
      if (mu <= 0.0)
         throw new IllegalArgumentException ("mu <= 0");
      if (lambda <= 0.0)
         throw new IllegalArgumentException ("lambda <= 0");

      return mu;      
   }\end{hide}
\end{code}
\begin{tabb} Returns the mean $E[X] = \mu$ of the
   inverse gaussian distribution with parameters $\mu$ and $\lambda$.
\end{tabb}
\begin{htmlonly}
   \return{the mean of the inverse gaussian distribution $E[X] = \mu$}
\end{htmlonly}
\begin{code}

   public static double getVariance (double mu, double lambda)\begin{hide} {
      if (mu <= 0.0)
         throw new IllegalArgumentException ("mu <= 0");
      if (lambda <= 0.0)
         throw new IllegalArgumentException ("lambda <= 0");

      return (mu * mu * mu / lambda);
   }\end{hide}
\end{code}
\begin{tabb}  Computes and returns the variance $\mbox{Var}[X] = \mu^3/\lambda$ of
   the inverse gaussian distribution with parameters $\mu$ and $\lambda$.
\end{tabb}
\begin{htmlonly}
   \return{the variance of the inverse gaussian distribution
   $\mbox{Var}[X] = \mu^3 / \lambda$
\end{htmlonly}
\begin{code}

   public static double getStandardDeviation (double mu, double lambda)\begin{hide} {
      return Math.sqrt (getVariance (mu, lambda));
   }\end{hide}
\end{code}
\begin{tabb}  Computes and returns the standard deviation
   of the inverse gaussian distribution with parameters $\mu$ and $\lambda$.
\end{tabb}
\begin{htmlonly}
   \return{the standard deviation of the inverse gaussian distribution}
\end{htmlonly}
\begin{code}

   public double getLambda()\begin{hide} {
      return lambda;
   }\end{hide}
\end{code}
 \begin{tabb} Returns the parameter $\lambda$ of this object.
 \end{tabb}
\begin{code}

   public double getMu()\begin{hide} {
      return mu;
   }\end{hide}
\end{code}
 \begin{tabb} Returns the parameter $\mu$ of this object.
 \end{tabb}
\begin{code}

   public void setParams (double mu, double lambda)\begin{hide} {
      if (mu <= 0.0)
         throw new IllegalArgumentException ("mu <= 0");
      if (lambda <= 0.0)
         throw new IllegalArgumentException ("lambda <= 0");

      this.mu = mu;
      this.lambda = lambda;
      supportA = 0.0;
   }\end{hide}
\end{code}
\begin{tabb}
   Sets the parameters $\mu$ and $\lambda$ of this object.
\end{tabb}
\begin{code}

   public double[] getParams ()\begin{hide} {
      double[] retour = {mu, lambda};
      return retour;
   }\end{hide}
\end{code}
\begin{tabb}
   Return a table containing the parameters of the current distribution.
   This table is put in regular order: [$\mu$, $\lambda$].
\end{tabb}
\begin{hide}\begin{code}

   public String toString ()\begin{hide} {
      return getClass().getSimpleName() + " : mu = " + mu + ", lambda = " + lambda;
   }\end{hide}
\end{code}
\begin{tabb}
   Returns a \texttt{String} containing information about the current distribution.
\end{tabb}\end{hide}
\begin{code}\begin{hide}
}\end{hide}
\end{code}
