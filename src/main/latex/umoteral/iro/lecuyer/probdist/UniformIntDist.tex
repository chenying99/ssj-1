\defclass {UniformIntDist}

Extends the class \class{DiscreteDistributionInt} for
the {\em discrete uniform\/} distribution over the range $[i,j]$.
Its mass function is given by
\begin{htmlonly}
\eq
   p(x) = 1 / (j - i + 1)  \qquad\mbox{ for } x = i, i + 1, \ldots, j
   \eqlabel{eq:fmassuniformint}
\endeq
and 0 elsewhere.  The distribution function is
\eq
   F(x) = (\lfloor x\rfloor -i+1)/(j-i+1) \qquad\mbox { for } i\le x\le j
                                          \eqlabel{eq:cdfuniformint}
\endeq
and its inverse is
\eq
   F^{-1}(u) = i + (j - i + 1)u
      \qquad\mbox{for }0 \le u \le 1.     \eqlabel{eq:invuniformint}
\endeq
\end{htmlonly}%
\begin{latexonly}
\eq
   p(x) = \frac{1}{j - i + 1}  \qquad\mbox{ for } x = i, i + 1, \ldots, j
   \eqlabel{eq:fmassuniformint}
\endeq
and 0 elsewhere.  The distribution function is
\eq
   F(x) = \left\{\begin{array}{ll}
     0, & \mbox { for } x < i\\[5pt]
     \displaystyle \frac{\lfloor x\rfloor -i+1}{j-i+1},  & \mbox { for } i\le x < j\\[12pt]
     1, & \mbox { for } x \ge j.
  \end{array}\right.
                                          \eqlabel{eq:cdfuniformint}
\endeq
and its inverse is
\eq
   F^{-1}(u) = i + \lfloor (j - i + 1)u\rfloor
      \qquad\mbox{for }0 \le u \le 1.     \eqlabel{eq:invuniformint}
\endeq
\end{latexonly}

\bigskip\hrule

%%%%%%%%%%%%%%%%%%%%%%%%%%%%%%%%%%%%%%%%
\begin{code}
\begin{hide}
/*
 * Class:        UniformIntDist
 * Description:  discrete uniform distribution
 * Environment:  Java
 * Software:     SSJ
 * Copyright (C) 2001  Pierre L'Ecuyer and Université de Montréal
 * Organization: DIRO, Université de Montréal
 * @author
 * @since

 * SSJ is free software: you can redistribute it and/or modify it under
 * the terms of the GNU General Public License (GPL) as published by the
 * Free Software Foundation, either version 3 of the License, or
 * any later version.

 * SSJ is distributed in the hope that it will be useful,
 * but WITHOUT ANY WARRANTY; without even the implied warranty of
 * MERCHANTABILITY or FITNESS FOR A PARTICULAR PURPOSE.  See the
 * GNU General Public License for more details.

 * A copy of the GNU General Public License is available at
   <a href="http://www.gnu.org/licenses">GPL licence site</a>.
 */
\end{hide}
package umontreal.iro.lecuyer.probdist;


public class UniformIntDist extends DiscreteDistributionInt\begin{hide} {
   protected int i;
   protected int j;
\end{hide}\end{code}
%%%%%%%%%%%%%%%%%%%%%%%%%%%%%%%%%%%%%%%%%
\subsubsection* {Constructor}

\begin{code}

   public UniformIntDist (int i, int j)\begin{hide} {
      setParams (i, j);
   }\end{hide}
\end{code}
  \begin{tabb} Constructs a discrete uniform distribution over the interval $[i,j]$.
  \end{tabb}


%%%%%%%%%%%%%%%%%%%%%%%%%%%%%%%%%%%
\subsubsection* {Methods}

\begin{code}\begin{hide}

   public double prob (int x) {
      return prob (i, j, x);
   }

   public double cdf (int x) {
      return cdf (i, j, x);
   }

   public double barF (int x) {
      return barF (i, j, x);
   }

   public int inverseFInt (double u) {
      return inverseF (i, j, u);
   }

   public double getMean() {
      return getMean (i, j);
   }

   public double getVariance() {
      return getVariance (i, j);
   }

   public double getStandardDeviation() {
      return getStandardDeviation (i, j);
   }\end{hide}

   public static double prob (int i, int j, int x)\begin{hide} {
      if (j < i)
         throw new IllegalArgumentException ("j < i");
      if (x < i || x > j)
         return 0.0;

      return (1.0 / (j - i + 1.0));
   }\end{hide}
\end{code}
\begin{tabb} Computes the discrete uniform probability $p(x)$%
\latex{ defined in  (\ref{eq:fmassuniformint})}.
\end{tabb}
\begin{code}

   public static double cdf (int i, int j, int x)\begin{hide} {
      if (j < i)
         throw new IllegalArgumentException ("j < i");
      if (x < i)
         return 0.0;
      if (x >= j)
         return 1.0;

      return ((x - i + 1) / (j - i + 1.0));
   }\end{hide}
\end{code}
 \begin{tabb}
  Computes the discrete uniform distribution function
defined in (\ref{eq:cdfuniformint}).
 \end{tabb}
\begin{code}

   public static double barF (int i, int j, int x)\begin{hide} {
      if (j < i)
        throw new IllegalArgumentException ("j < i");
      if (x <= i)
         return 1.0;
      if (x > j)
         return 0.0;

      return ((j - x + 1.0) / (j - i + 1.0));
   }\end{hide}
\end{code}
 \begin{tabb}
  Computes the discrete uniform complementary distribution function
  $\bar{F}(x)$.
  \emph{WARNING:} The complementary distribution function is defined as
    $\bar F(x) = P[X \ge x]$.
 \end{tabb}
\begin{code}

   public static int inverseF (int i, int j, double u)\begin{hide} {
      if (j < i)
        throw new IllegalArgumentException ("j < i");

       if (u > 1.0 || u < 0.0)
           throw new IllegalArgumentException ("u not in [0, 1]");

       if (u <= 0.0)
           return i;
       if (u >= 1.0)
           return j;

       return i + (int) (u * (j - i + 1.0));
   }\end{hide}
\end{code}
  \begin{tabb}
  Computes the inverse of the discrete uniform distribution function
 (\ref{eq:invuniformint}).
 \end{tabb}
\begin{code}

   public static double[] getMLE (int[] x, int n)\begin{hide} {
      if (n <= 0)
         throw new IllegalArgumentException ("n <= 0");

      double parameters[] = new double[2];
      parameters[0] = (double) Integer.MAX_VALUE;
      parameters[1] = (double) Integer.MIN_VALUE;
      for (int i = 0; i < n; i++) {
         if ((double) x[i] < parameters[0])
            parameters[0] = (double) x[i];
         if ((double) x[i] > parameters[1])
            parameters[1] = (double) x[i];
      }
      return parameters;
   }\end{hide}
\end{code}
\begin{tabb}
   Estimates the parameters $(i, j)$ of the uniform distribution
   over integers using the maximum likelihood method, from the $n$ observations
   $x[k]$, $k = 0, 1, \ldots, n-1$. The estimates are returned in a two-element
    array, in regular order: [$i$, $j$].
   \begin{detailed}
   The maximum likelihood estimators are the values
   $(\hat{\imath}$, $\hat{\jmath})$ that satisfy the equations
   \begin{eqnarray*}
      \hat{\imath} & = & \mbox{min} \{x_k\}\\
      \hat{\jmath} & = & \mbox{max} \{x_k\}
   \end{eqnarray*}
   where  $\bar x_n$ is the average of $x[0],\dots,x[n-1]$.
   \end{detailed}
\end{tabb}
\begin{htmlonly}
   \param{x}{the list of observations used to evaluate parameters}
   \param{n}{the number of observations used to evaluate parameters}
   \return{returns the parameters [$\hat{\imath}$, $\hat{\jmath}$]}
\end{htmlonly}
\begin{code}

   public static UniformIntDist getInstanceFromMLE (int[] x, int n)\begin{hide} {

      double parameters[] = getMLE (x, n);

      return new UniformIntDist ((int) parameters[0], (int) parameters[1]);
   }\end{hide}
\end{code}
\begin{tabb}
   Creates a new instance of a discrete uniform distribution over integers with parameters
   $i$ and $j$ estimated using the maximum likelihood method based on the $n$ observations
   $x[k]$, $k = 0, 1, \ldots, n-1$.
\end{tabb}
\begin{htmlonly}
   \param{x}{the list of observations to use to evaluate parameters}
   \param{n}{the number of observations to use to evaluate parameters}
\end{htmlonly}
\begin{code}

   public static double getMean (int i, int j)\begin{hide} {
      if (j < i)
        throw new IllegalArgumentException ("j < i");

      return ((i + j) / 2.0);
   }\end{hide}
\end{code}
\begin{tabb}  Computes and returns the mean $E[X] = (i + j)/2$
   of the discrete uniform distribution.
\end{tabb}
\begin{htmlonly}
   \return{the mean of the discrete uniform distribution}
\end{htmlonly}
\begin{code}

   public static double getVariance (int i, int j)\begin{hide} {
      if (j < i)
        throw new IllegalArgumentException ("j < i");

      return (((j - i + 1.0) * (j - i + 1.0) - 1.0) / 12.0);
   }\end{hide}
\end{code}
\begin{tabb}  Computes and returns the variance
   $\mbox{Var}[X] = [(j - i + 1)^2 - 1]/{12}$
   of the discrete uniform distribution.
\end{tabb}
\begin{htmlonly}
   \return{the variance of the discrete uniform distribution}
\end{htmlonly}
\begin{code}

   public static double getStandardDeviation (int i, int j)\begin{hide} {
      return Math.sqrt (UniformIntDist.getVariance (i, j));
   }\end{hide}
\end{code}
\begin{tabb}  Computes and returns the standard deviation
   of the discrete uniform distribution.
\end{tabb}
\begin{htmlonly}
   \return{the standard deviation of the discrete uniform distribution}
\end{htmlonly}
\begin{code}

   public int getI()\begin{hide} {
      return i;
   }\end{hide}
\end{code}
  \begin{tabb}
  Returns the parameter $i$.
 \end{tabb}
\begin{code}

   public int getJ()\begin{hide} {
      return j;
   }\end{hide}
\end{code}
  \begin{tabb}
  Returns the parameter $j$.
 \end{tabb}
\begin{code}

   public void setParams (int i, int j)\begin{hide} {
      if (j < i)
        throw new IllegalArgumentException ("j < i");

      supportA = this.i = i;
      supportB = this.j = j;
   }\end{hide}
\end{code}
  \begin{tabb}
  Sets the parameters $i$ and $j$ for this object.
 \end{tabb}
 \begin{code}

   public double[] getParams ()\begin{hide} {
      double[] retour = {i, j};
      return retour;
   }\end{hide}
\end{code}
\begin{tabb}
   Return a table containing the parameters of the current distribution.
   This table is put in regular order: [$i$, $j$].
\end{tabb}
\begin{hide}\begin{code}

   public String toString ()\begin{hide} {
      return getClass().getSimpleName() + " : i = " + i + ", j = " + j;
   }\end{hide}
\end{code}
\begin{tabb}
   Returns a \texttt{String} containing information about the current distribution.
\end{tabb}\end{hide}
\begin{code}\begin{hide}
}\end{hide}
\end{code}
