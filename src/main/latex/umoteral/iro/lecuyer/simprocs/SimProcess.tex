\defclass{SimProcess}

This abstract class provides process scheduling tools.
Each type of process should be defined as a subclass of the
class \texttt{SimProcess}, and must provide an implementation of the method
\method{actions}{} which describes the life of a process of this class.
Whenever a process instance starts, its \method{actions}{} method
begins executing.

Just like an event, a process must first be constructed, then scheduled.
Scheduling a process is equivalent to placing an event in the event list
that will start this process when
the simulation clock reaches the specified time.
The \method{toString}{} method can be overridden to return
information about the process.  This information will
be returned when formating the event list as a string, if the
process is delayed.

A process can be in one of the following states: \texttt{INITIAL},
\texttt{EXECUTING}, \texttt{DELAYED}, \texttt{SUSPENDED}, and \texttt{DEAD}
\latex{(see the diagram)}.
At most {\em one\/} process can be in the \texttt{EXECUTING} state at
any given time, and when there is one, this executing process
(called the {\em current process\/}) is said to {\em have control\/}
and is executing one of the instructions of its \method{actions}{} method.
% Note that the simulation executive (in the class \texttt{Sim}) is also
% a process and it is the one who has control, e.g, when an \texttt{Event}
% is being executed.
%  ^^^^  No longer true.
\begin{latexonly}
\begin{figure}[htb]
\caption{The possible states and state transitions of a \texttt{SimProcess}}
%
\setlength{\unitlength}{1in}
\begin{picture}(6.4,5.5)(0.0,0.0)\thicklines\tt
%  Les boites et le texte:
\put(2.0,0.5){\makebox(0,0)\texttt{EXECUTING}}
\put(3.5,2.0){\makebox(0,0)\texttt{SUSPENDED}}
\put(2.0,3.5){\makebox(0,0)\texttt{DELAYED}}
\put(2.0,5.0){\makebox(0,0)\texttt{INITIAL}}
\put(2.0,0.5){\oval(3.0,0.4)}
\put(3.5,2.0){\oval(3.0,0.4)}
\put(2.0,3.5){\oval(3.0,0.4)}
\put(2.0,5.0){\oval(3.0,0.4)}
\put(0.0,5.0){\vector(1,0){0.5}}
\put(0.25,5.2){\makebox(0,0){\rm \texttt{new}}}
\put(1.0,4.25){\makebox(0,0){\rm \texttt{schedule}}}
\put(1.0,4.05){\vector(0,-1){.34}}
\put(1.0,4.80){\line(0,-1){.40}}
\put(3.0,4.25){\makebox(0,0){\rm \texttt{scheduleNext}}}
\put(3.0,4.05){\vector(0,-1){.34}}
\put(3.0,4.80){\line(0,-1){.40}}
\put(0.8,2.5){\makebox(0,0){\rm ({\em event})}}
\put(0.8,2.35){\vector(0,-1){1.64}}
\put(0.8,2.65){\line(0,1){.65}}
\put(1.6,1.5){\makebox(0,0){delay}}
\put(1.6,1.65){\vector(0,1){1.64}}
\put(1.6,1.35){\line(0,-1){.65}}
\put(3.2,1.4){\makebox(0,0){suspend}}
\put(3.2,0.7){\line(0,1){0.53}}
\put(3.2,1.55){\vector(0,1){.24}}
\put(3.55,2.55){\makebox(0,0){suspend, cancel}}
\put(3.3,3.3){\line(0,-1){0.6}}
\put(3.3,2.4){\vector(0,-1){.19}}
\put(2.4,2.95){\makebox(0,0){reschedule}}
\put(2.4,2.75){\makebox(0,0){resume}}
\put(2.3,2.2){\line(0,1){0.40}}
\put(2.3,3.1){\vector(0,1){.19}}
\put(2.9,3.3){\line(0,-1){0.7}}
\put(2.7,2.6){\oval(0.4,0.2)[b]}
\put(2.5,3.1){\vector(0,1){.19}}
\put(5.5,0.5){\makebox(0,0){kill}}
\put(3.5,0.5){\line(1,0){1.8}}
\put(5.7,0.5){\vector(1,0){0.25}}
\put(5.5,2.0){\makebox(0,0){kill}}
\put(5.0,2.0){\line(1,0){0.3}}
\put(5.7,2.0){\vector(1,0){.25}}
\put(5.5,3.5){\makebox(0,0){kill}}
\put(3.5,3.5){\line(1,0){1.8}}
\put(5.7,3.5){\vector(1,0){.25}}
\put(5.5,5.0){\makebox(0,0){kill}}
\put(3.5,5.0){\line(1,0){1.8}}
\put(5.7,5.0){\vector(1,0){.25}}
\end{picture}
\end{figure}
\end{latexonly}
A process that has been constructed but not yet scheduled is in
the \texttt{INITIAL} state.
A process is in the \texttt{DELAYED} state if there is a planned event
in the event list to activate it (give it control).
When a process is scheduled, it is placed in the \texttt{DELAYED} state.
It is in the \texttt{SUSPENDED} state if it is waiting to be reactivated
(i.e., obtain control) without having an event to do so in the event list.
A process can suspends itself by calling \method{suspend}{} directly or
indirectly (e.g., by requesting a busy \class{Resource}).
Usually, a \texttt{SUSPENDED} process must be reactivated by another process
or event via the \method{resume}{} method.
A process in the \texttt{DEAD} state no longer exists.

To construct new processes, the user needs to extend \class{SimProcess}.

Note: the user needs to ensure that the \texttt{actions}
method of any process can be terminated, i.e., no infinite loops.
For example, using threads process-oriented simulator,
if such a method never terminates, threads will not be recycled,
causing memory problems.

\bigskip\hrule

%%%%%%%%%%%%%%%%%%%%%%%%%%%%%%%%%%%%%%%%%%%%%%%%%%%%%%%%%%%%%%%%%%
\begin{code}
\begin{hide}
/*
 * Class:        SimProcess
 * Description:  Abstract class providing process scheduling tools
 * Environment:  Java
 * Software:     SSJ 
 * Copyright (C) 2001  Pierre L'Ecuyer and Université de Montréal
 * Organization: DIRO, Université de Montréal
 * @author       
 * @since

 * SSJ is free software: you can redistribute it and/or modify it under
 * the terms of the GNU General Public License (GPL) as published by the
 * Free Software Foundation, either version 3 of the License, or
 * any later version.

 * SSJ is distributed in the hope that it will be useful,
 * but WITHOUT ANY WARRANTY; without even the implied warranty of
 * MERCHANTABILITY or FITNESS FOR A PARTICULAR PURPOSE.  See the
 * GNU General Public License for more details.

 * A copy of the GNU General Public License is available at
   <a href="http://www.gnu.org/licenses">GPL licence site</a>.
 */
\end{hide}
package umontreal.iro.lecuyer.simprocs;\begin{hide}

import umontreal.iro.lecuyer.simevents.Event;
import umontreal.iro.lecuyer.simevents.eventlist.EventList;
import umontreal.iro.lecuyer.simevents.Simulator;
import umontreal.iro.lecuyer.simprocs.ProcessSimulator;\end{hide}

public class SimProcess \begin{hide} {
\end{hide}
\end{code}
\subsubsection*{Possible states of a process}

\begin{code}

   public static final int INITIAL   = 0;
\end{code}
\begin{tabb}   The process has been created but not yet scheduled. \end{tabb}
\begin{code}

   public static final int EXECUTING = 1;
\end{code}
\begin{tabb}   The process is the one currently executing its
   \method{actions}{} method. \end{tabb}
\begin{code}

   public static final int DELAYED   = 2;
\end{code}
\begin{tabb}   The process is not executing but has an event in the event list to
   reactivate it later on.  \end{tabb}
\begin{code}

   public static final int SUSPENDED = 3;
\end{code}
\begin{tabb}   The process is not executing and will have to be reactivated by another
   process or event later on.  \end{tabb}
\begin{code}

   public static final int DEAD      = 4;
\end{code}
\begin{tabb}   The process has terminated its execution.  \end{tabb}
\begin{code}\begin{hide}
   // The state of a process is not kept explicitly, but
   // can be recovered from its eventTime variable.
    public static final double STARTING = -20.0;
    public static final double WAITING  = -10.0;
    // Variables

   protected ProcessSimulator sim;
   // Simulator linked with this simProcess.

   protected Event scheduledEvent = null;
   // the next scheduled event
\end{hide}
\end{code}

%%%%%%%%%%%%%%%%%%%%%%%%%%%%%%%%%%%
\subsubsection* {Constructor}
\begin{code}

   public SimProcess() \begin{hide} {
      this((ProcessSimulator)Simulator.getDefaultSimulator());
   }\end{hide}
\end{code}
 \begin{tabb}  Constructs a new process without scheduling it
  and associates this new process with the default simulator; one
  can get additional knowledge with
 \externalclass{umontreal.iro.lecuyer.simevents}{Simulator} static methods.
   It will have to be scheduled later on.
   The process is in the \texttt{INITIAL} state.
 \end{tabb}
\begin{code}

   public SimProcess (ProcessSimulator sim) \begin{hide} {
      this.sim = sim;
      scheduledEvent = sim.createControlEvent(this);
   }\end{hide}
\end{code}
 \begin{tabb}  Constructs a new process associated with \texttt{sim}
  without scheduling it. It will have to be scheduled later on.
   The process is in the \texttt{INITIAL} state.
 \end{tabb}
 \begin{htmlonly}
   \param{sim}{Link the current variable to ProcessSimulator \texttt{sim}}
\end{htmlonly}

%%%%%%
%%%%%%%%

%%%%%%%%%%%%%%%%%%%%%%%%%%%%%%%%%%%
\subsubsection* {Methods}
\begin{code}

   public void schedule (double delay) \begin{hide} {
      if (scheduledEvent == null)
         throw new IllegalStateException ("Cannot schedule a dead process");
      if (scheduledEvent.time() != STARTING)
         throw new IllegalStateException ("Only a process in INITIAL state can call schedule");

      scheduledEvent.schedule (delay);
   } \end{hide}
\end{code}
\begin{tabb}
   Schedules the process to start in \texttt{delay} time units.
   This puts the process in the \texttt{DELAYED} state.
  \end{tabb}
\begin{htmlonly}
   \param{delay}{delay, in simulation time, before the process starts}
\end{htmlonly}
\begin{code}

   public void scheduleNext() \begin{hide} {
      if (scheduledEvent == null)
         throw new IllegalStateException ("Cannot schedule a dead process");
      if (scheduledEvent.time() != STARTING)
         throw new IllegalStateException
            ("Only a process in INITIAL state can call scheduleNext");
      scheduledEvent.scheduleNext();
   } \end{hide}
\end{code}
\begin{tabb}  Schedules this process to start at the current time, by placing
   it at the beginning of the event list.
   This puts the process in the \texttt{DELAYED} state.
  \end{tabb}
\begin{code}

   public Event scheduledEvent() \begin{hide} {
      return scheduledEvent;
   } \end{hide}
\end{code}
\begin{tabb}
   Returns the \texttt{Event} associated with the current variable.
  \end{tabb}
\begin{code}

   public void setScheduledEvent (Event scheduledEvent) \begin{hide} {
      this.scheduledEvent = scheduledEvent;
   } \end{hide}
\end{code}
\begin{tabb}  Sets the event associated to the current variable.
  \end{tabb}
\begin{htmlonly}
   \param{scheduledEvent}{new scheduledEvent for the current variable}
\end{htmlonly}
\begin{code}

   public double priority() \begin{hide} {
      return scheduledEvent.priority();
   } \end{hide}
\end{code}
\begin{tabb}
   Returns the priority of the current variable.
\end{tabb}
\begin{htmlonly}
   \return{priority of the current variable.}
\end{htmlonly}
\begin{code}

   public void setPriority (double priority) \begin{hide} {
      scheduledEvent.setPriority(priority);
   } \end{hide}
\end{code}
\begin{tabb}
   Sets the priority assigned to the current variable in the simulation.
   This method should never be called after the event has been scheduled, otherwise
   the events will not execute in ascending priority order anymore.
  \end{tabb}
\begin{htmlonly}
   \param{priority}{priority assigned to the current variable in the simulation}
\end{htmlonly}
\begin{code}

   public final boolean isAlive() \begin{hide} {
      return scheduledEvent != null;
   } \end{hide}
\end{code}
  \begin{tabb} Returns \texttt{true} if the process is alive, otherwise \texttt{false}.
  \end{tabb}
\begin{htmlonly}
   \return{\texttt{true} if the process is not in the \texttt{DEAD} state}
\end{htmlonly}
\begin{code}

   public int getState() \begin{hide} {
      if (scheduledEvent == null)                return DEAD;
      else if (sim.currentProcess() == this)             return EXECUTING;
      else if (scheduledEvent.time() >= 0.0)  return DELAYED;
      else if (scheduledEvent.time() == STARTING) return INITIAL;
      else return SUSPENDED;
   } \end{hide}
\end{code}
  \begin{tabb}  Returns the state of the process.
  \end{tabb}
\begin{htmlonly}
   \return{one of the process states \method{INITIAL}{},
      \method{EXECUTING}{}, \method{DELAYED}{}, \method{SUSPENDED}{}, or
      \method{DEAD}{}.}
\end{htmlonly}
\begin{code}

   public double getDelay() \begin{hide} {
      if (scheduledEvent == null)
         throw new IllegalStateException ("Trying to getDelay() on a dead process");
      if (scheduledEvent.time() < 0.0)
         throw new IllegalStateException
                   ("Calling getDelay for a process not in DELAYED state");
      return scheduledEvent.time() - sim.time();
   } \end{hide}
\end{code}
  \begin{tabb}  If the process is in the \texttt{DELAYED} state, returns
   the remaining time until the planned occurrence of its
   activating event.
   Otherwise, an illegal state exception will be thrown printing an error message.
  \end{tabb}
\begin{htmlonly}
   \return{remaining delay before the process starts}
   \exception{IllegalStateException}{if the process is not in \texttt{DELAYED} state}
\end{htmlonly}
\begin{code}

   public void reschedule (double delay) \begin{hide} {
      if (sim.currentProcess() == this)
         throw new IllegalStateException
                   ("reschedule() for a process in EXECUTING state");
      if (scheduledEvent == null)
         throw new IllegalStateException ("reschedule() for a dead process ");
      if (delay < 0.0) throw new IllegalArgumentException
                                 ("Calling reschedule() with negative delay");
      scheduledEvent.reschedule (delay);
   } \end{hide}
\end{code}
  \begin{tabb}  If the process is in the \texttt{DELAYED} state, removes it from
   the event list and reschedules it in \texttt{delay} units of time.
   Example: If the process \texttt{p} has called \method{delay}{}\texttt{(5.0)}
   at time 10.0, and another process invokes \method{reschedule}{}\texttt{(p, 6.2)}
   at time 13.5, then the process \texttt{p} will resume at time
   $13.5 + 6.2 = 19.7$.
  \end{tabb}
\begin{htmlonly}
   \param{delay}{new delay, in simulation time units, before the process starts or is resumed}
\end{htmlonly}
\begin{code}

   public void resume() \begin{hide} {
      if (scheduledEvent == null)   // DEAD state
         throw new IllegalStateException ("calling resume() for a dead process");
      if (scheduledEvent.time() >= 0.0)  scheduledEvent.cancel();
      scheduledEvent.scheduleNext();
   } \end{hide}
\end{code}
  \begin{tabb} Places this process at the beginning of the event list
   to resume its execution.  If the process was \texttt{DELAYED}, cancels
   its earlier activating event.
  \end{tabb}
\begin{code}

   public boolean cancel() \begin{hide} {
      if (scheduledEvent == null) // DEAD state
         throw new IllegalStateException ("calling cancel() for a dead process");
      if (scheduledEvent.time() < 0.0 )
         throw new IllegalStateException
                   ("cancel() for a process not in DELAYED state");
      boolean removed = scheduledEvent.cancel();
      scheduledEvent.setTime (WAITING);
      return removed;
   } \end{hide}
\end{code}
  \begin{tabb}  Cancels the activating event that was supposed to resume
   this process, and places the process in the \texttt{SUSPENDED} state.
   This method can be invoked only for a process in the \texttt{DELAYED}
   state.
  \end{tabb}
\begin{htmlonly}
   \return{\texttt{true} if the process was canceled successfully}
\end{htmlonly}
\begin{code}

   public void delay (double delay) \begin{hide} {
      sim.delay(this, delay);
   } \end{hide}
\end{code}
  \begin{tabb}  Suspends the execution of the currently executing process and
    schedules it to resume its execution in \texttt{delay} units of simulation
    time.  It moves in the \texttt{DELAYED} state.
    Only the process in the \texttt{EXECUTING} state can call this method.
  \end{tabb}
\begin{code}

   public void suspend() \begin{hide} {
      sim.suspend(this);
   } \end{hide}
\end{code}
  \begin{tabb} This method can only be invoked for the \texttt{EXECUTING}
   or a \texttt{DELAYED} process.
   If the process is \texttt{EXECUTING}, suspends execution.
   If the process is \texttt{DELAYED}, cancels its activating event.
   This places the process in the \texttt{SUSPENDED} state.
  \end{tabb}
\begin{code}

   public void kill() \begin{hide} {
      sim.kill(this);
   } \end{hide}
\end{code}
  \begin{tabb}  Terminates the life of this process and sets its state to
   \texttt{DEAD}, after canceling its activating event if there is one.
  \end{tabb}
\begin{code}

   public void actions()\begin{hide} { }; \end{hide}
\end{code}
  \begin{tabb}  This is the method that is called when this process is
   executing. Every subclass of \texttt{SimProcess} that is to be
    instantiated must provide an implementation of this method.
  \end{tabb}
\begin{code}

   public static void init() \begin{hide} {
      ProcessSimulator.initDefault();
   } \end{hide}
\end{code}
  \begin{tabb}  This method calls \texttt{ProcessSimulator.initDefault()},
    which initializes the default simulator to use processes.
  \end{tabb}
\begin{code} \begin{hide}
}\end{hide}
\end{code}
