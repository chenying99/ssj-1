\defclass{ListOfTalliesWithCovariance}

Extends \class{ListOfTallies} to add support for the computation
of the sample covariance between each pair of elements
in a list, without storing all observations.
This list of tallies contains internal structures to keep track of
$\bar X_{n, i}$ for $i=0, \ldots, d-1$, and
$\sum_{k=0}^{n-1} (X_{i, k} - \bar X_{k, i})(X_{j, k} - \bar X_{k, j})/n$, for
$i=0,\ldots, d-2$ and $j=1,\ldots,d-1$, with $j>i$.
Here, $\bar X_{n, i}$ is the $i$th component of $\barboldX_n$, the average vector,
and $\bar X_{0, i}=0$ for $i=0,\ldots,d-1$.
The value $X_{i,k}$ corresponds to the $i$th component of the $k$th observation
$\boldX_k$.
These sums are updated every time a vector is added to this list, and
are used to estimate the covariances.

Note: the size of the list of tallies must remain fixed because of the
data structures used for computing sample covariances.
As a result, the first call to \texttt{init} makes this list
unmodifiable.

Note: for the sample covariance to be computed between a pair
of tallies, the number of observations in each tally
should be the same.  It is therefore recommended to always
add complete vectors of observations to this list.
Moreover, one must use
the \method{add}{} method in this class to add vectors of observations for
the sums used for covariance estimation to be updated correctly.
Failure to use this method, e.g., adding observations
to each individual tally in the list, will result in an incorrect
estimate of the covariances, unless the tallies
in the list can store observations.
For example, the following code, which adds the vector \texttt{v} in
the list of tallies \texttt{list}, works correctly only if the list
contains instances of \class{TallyStore}:
\begin{vcode}
   for (int i = 0; i < v.length; i++)
      list.get (i).add (v[i]);
\end{vcode}
But the following code is always correct:
\begin{vcode}
   list.add (v);
\end{vcode}

\bigskip\hrule

\begin{code}
\begin{hide}
/*
 * Class:        ListOfTalliesWithCovariance
 * Description:  List of tallies with covariance
 * Environment:  Java
 * Software:     SSJ 
 * Copyright (C) 2001  Pierre L'Ecuyer and Université de Montréal
 * Organization: DIRO, Université de Montréal
 * @author       Éric Buist 
 * @since        2007

 * SSJ is free software: you can redistribute it and/or modify it under
 * the terms of the GNU General Public License (GPL) as published by the
 * Free Software Foundation, either version 3 of the License, or
 * any later version.

 * SSJ is distributed in the hope that it will be useful,
 * but WITHOUT ANY WARRANTY; without even the implied warranty of
 * MERCHANTABILITY or FITNESS FOR A PARTICULAR PURPOSE.  See the
 * GNU General Public License for more details.

 * A copy of the GNU General Public License is available at
   <a href="http://www.gnu.org/licenses">GPL licence site</a>.
 */
\end{hide}
package umontreal.iro.lecuyer.stat.list;\begin{hide}

import umontreal.iro.lecuyer.stat.Tally;
import umontreal.iro.lecuyer.stat.TallyStore;
import java.util.logging.Level;
import java.util.logging.Logger;
import cern.colt.matrix.DoubleMatrix1D;
\end{hide}


public class ListOfTalliesWithCovariance<E extends Tally>
       extends ListOfTallies<E>\begin{hide} {
   private double[] tempArray;
   private double[][] sxy;

   // Determines if we use a numerically stable covariance
   // formula.
   private boolean isStable = true;

   // The average of the first observations, for each tally
   private double[] curAverages;

   // The sum (xi - average)(yi - average) of the first observations
   private double[][] curSum2;
   private Logger log = Logger.getLogger ("umontreal.iro.lecuyer.stat.list");
\end{hide}
\end{code}
\subsubsection*{Constructors}
\begin{code}

   public ListOfTalliesWithCovariance()\begin{hide} {
      super();
   }\end{hide}
\end{code}
\begin{tabb}   Creates an empty list of tallies with covariance support.
   One must fill the list with tallies, and call \method{init}{} before
  adding any observation.
\end{tabb}
\begin{code}

   public ListOfTalliesWithCovariance (String name)\begin{hide} {
      super (name);
   }\end{hide}
\end{code}
\begin{tabb}   Creates an empty list of tallies with covariance
  support and name \texttt{name}.
   One must fill the list with tallies, and call \method{init}{} before
  adding any observation.
\end{tabb}
\begin{htmlonly}
   \param{name}{the name of the new list.}
\end{htmlonly}
\subsubsection*{Methods}
\begin{code}

   public static ListOfTalliesWithCovariance<Tally> createWithTally (int size)\begin{hide} {
      ListOfTalliesWithCovariance<Tally> list = new ListOfTalliesWithCovariance<Tally>();
      for (int i = 0; i < size; i++)
         list.add (new Tally());
      list.init();
      return list;
   }\end{hide}
\end{code}
\begin{tabb}  This factory method constructs and returns a list of tallies with \texttt{size} instances of
   \class{Tally}.
\end{tabb}
\begin{htmlonly}
   \param{size}{the size of the list.}
   \return{the created list.}
\end{htmlonly}
\begin{code}

   public static ListOfTalliesWithCovariance<TallyStore> createWithTallyStore
                                             (int size)\begin{hide} {
      ListOfTalliesWithCovariance<TallyStore> list = new ListOfTalliesWithCovariance<TallyStore>();
      for (int i = 0; i < size; i++)
         list.add (new TallyStore());
      list.init();
      return list;
   }\end{hide}
\end{code}
\begin{tabb}  This factory method constructs and returns a list of tallies with \texttt{size} instances of
   \class{TallyStore}.
\end{tabb}
\begin{htmlonly}
   \param{size}{the size of the list.}
   \return{the created list.}
\end{htmlonly}
\begin{code}\begin{hide}

   private void createSxy() {
      int l = size();
      if (isStable) {
         curAverages = new double[l];
         curSum2 = new double[l-1][];
         for (int i = 0; i < l - 1; i++)
            curSum2[i] = new double[l - 1 - i];
      }
      else {
         sxy = new double[l - 1][];
         for (int i = 0; i < l - 1; i++)
            sxy[i] = new double[l - 1 - i];
      }
      tempArray = new double[l];
   }

   public void init() {
      super.init();

      if (isModifiable()) {
         setUnmodifiable();
         createSxy();
      }
      if (isStable) {
         for (int i = 0; i < curAverages.length; i++)
            curAverages[i] = 0;
         for (int i = 0; i < curSum2.length; i++)
            for (int j = 0; j < curSum2[i].length; j++)
               curSum2[i][j] = 0;
      }
      else
         for (int i = 0; i < sxy.length; i++)
            for (int j = 0; j < sxy[i].length; j++)
               sxy[i][j] = 0;
   }\end{hide}

   public void add (double[] x)\begin{hide} {
      int l = size();

      int structSize = 0;
      structSize = (isStable) ? curSum2.length : sxy.length;
      if (structSize != l - 1)
            throw new IllegalArgumentException ("The structure's size mismatches the list's size");

      super.add (x);
      if (isStable) {
         int numObs = get (0).numberObs();
         // get (i1).average() would return the average over the n
         // observations, but we need the average over the last n-1 observations.
         for (int i1 = 0; i1 < l - 1; i1++)
            for (int i2 = i1 + 1; i2 < l; i2++)
               curSum2[i1][i2 - i1 - 1] += (numObs - 1)*
                  (x[i1] - curAverages[i1])*(x[i2] - curAverages[i2])/numObs;
         for (int i = 0; i < l; i++)
            curAverages[i] += (x[i] - curAverages[i])/numObs;
         // Now, curAverages[i] == get (i).average()
      }
      else
         for (int i1 = 0; i1 < l - 1; i1++)
            for (int i2 = i1 + 1; i2 < l; i2++)
               sxy[i1][i2 - i1 - 1] += x[i1]*x[i2];
   }\end{hide}
\end{code}
\begin{tabb}   Adds a new vector of observations \texttt{x} to this
  list of tallies, and updates the internal data structures computing
  averages, and sums of products.
  One must use this method instead of adding observations to
  individual tallies to get a covariance estimate.
\end{tabb}
\begin{htmlonly}
    \param{x}{the new vector of observations.}
\end{htmlonly}
\begin{code}\begin{hide}

   public void add (DoubleMatrix1D x) {
      x.toArray (tempArray);
      add (tempArray);
   }

   public double covariance (int i, int j) {
      if (i == j)
         return get (i).variance();
      if (i > j) {
         // Make sure that i1 < i2, to have a single case
         int tmp = i;
         i = j;
         j = tmp;
      }

      Tally tallyi = get (i);
      Tally tallyj = get (j);
      if (tallyi == null || tallyj == null)
         return Double.NaN;
      int n = tallyi.numberObs();
      if (n != tallyj.numberObs()) {
         log.logp (Level.WARNING, "ListOfTalliesWithCovariance", "covariance",
            "Tally " + i  + ", with name " + tallyi.getName() + ", contains " 
            + n + " observations while " +
              "tally " + j + ", with name " + tallyj.getName() + ", contains " + tallyj.numberObs() + "observations");
         return Double.NaN;
      }

      if (n < 2) {
         log.logp (Level.WARNING, "ListOfTalliesWithCovariance", "covariance",
            "Tally " + i + ", with name " + tallyi.getName() + ", contains " + n + " observation");
         return Double.NaN;
      }
      if (tallyi instanceof TallyStore && tallyj instanceof TallyStore)
         return ((TallyStore) tallyi).covariance ((TallyStore) tallyj);
      else if (isStable)
         return curSum2[i][j - i - 1]/(n-1);
      else {
         double sum1 = tallyi.sum();
         double sum2 = tallyj.sum();
         double sum12 = sxy[i][j - i - 1];
         return (sum12 - sum1*sum2/n)/(n-1);
      }
   }\end{hide}

   public ListOfTalliesWithCovariance<E> clone()\begin{hide} {
      ListOfTalliesWithCovariance<E> ta = (ListOfTalliesWithCovariance<E>)super.clone();
      ta.tempArray = new double[size()];
      if (curAverages != null)
         ta.curAverages = curAverages.clone();
      if (sxy != null) {
         ta.sxy = new double[sxy.length][];
         for (int i = 0; i < sxy.length; i++)
            ta.sxy[i] = sxy[i].clone();
      }
      if (curSum2 != null) {
         ta.curSum2 = new double[curSum2.length][];
         for (int i = 0; i < curSum2.length; i++)
            ta.curSum2[i] = curSum2[i].clone();
      }
      return ta;
   }
}\end{hide}
\end{code}
\begin{tabb}   Clones this object.
   This clones the list of tallies and the data structures holding the sums of products but
   not the tallies comprising the list.
  The created clone is modifiable, even though the original list is unmodifiable.
\end{tabb}
