\defclass {DigitalNetFromFile}

This class allows us to read the parameters defining a digital net either
from a file, or from a URL address on the World Wide Web.
The parameters used in building the net are those defined in class
\externalclass{umontreal.iro.lecuyer.hups}{DigitalNet}.
The format of the data files must be the following:
\begin{htmlonly} (see the format in \texttt{guidehups.pdf})\end{htmlonly}

\begin{figure}
\begin{center}
\tt
\fbox {
\begin {tabular}{llll}
 \multicolumn{4}{l}{// Any number of comment lines starting with //} \\
     $b$      & &      & //  $\mbox{Base}$  \\
     $k$      & &      & //    Number of columns   \\
     $r$      & &      & //    Maximal number of rows  \\
     $n$      & &      & //    Number of points = $b^k$  \\
     $s$      & &      & //    Maximal dimension of points \\
\\
 \multicolumn{4}{l}{// dim = 1} \\
  $c_{11}$ & $c_{21}$ & $\cdots$ & $c_{r1}$ \\
  $c_{12}$ & $c_{22}$ & $\cdots$ & $c_{r2}$ \\
    &  $\vdots$ && \\
  $c_{1k}$ & $c_{2k}$ & $\cdots$ & $c_{rk}$ \\
\\
 \multicolumn{4}{l}{// dim = 2} \\
    &  $\vdots$ && \\
\\
 \multicolumn{4}{l}{// dim = $s$} \\
  $c_{11}$ & $c_{21}$ & $\cdots$ & $c_{r1}$ \\
  $c_{12}$ & $c_{22}$ & $\cdots$ & $c_{r2}$ \\
    &  $\vdots$ && \\
  $c_{1k}$ & $c_{2k}$ & $\cdots$ & $c_{rk}$ \\
\end {tabular}
}
\end{center}
%\caption { General format of the parameter file for
%  \externalclass{umontreal.iro.lecuyer.hups}{DigitalNetFromFile}.
%\label{formatdon}}
\end{figure}

The figure above gives the general format of the data file
needed by \texttt{DigitalNetFromFile}.
The values of the parameters on the left must appear in the file
as integers. On the right of each parameter, there is an optional
 comment that is disregarded by the reader program. In general, the
 Java line comments  \texttt{//} are accepted anywhere and will
ensure that the rest of the line is dropped by the reader. Blank lines
are also disregarded by the reader program. For each dimension, there must
 be a $k\times r$ matrix of integers in $\{0, 1, \ldots, b-1\}$ (note that
the matrices must appear in transposed form).

The predefined files of parameters are kept in different directories,
depending on the criteria used in the searches for the parameters defining
the digital net. These files have all been stored at the address
 \url{http://www.iro.umontreal.ca/~simardr/ssj/data}.
 Each file contains the parameters for a specific digital net.
% One may get a list of all available files in a directory by using
% method \method{listDir}{} below.
The name of the files gives information about the main parameters of
the digital net. For example, the file named \texttt{Edel/OOA2/B3S13R9C9St6}
 contains the parameters for a digital net proposed by Yves Edel
(see \url{http://www.mathi.uni-heidelberg.de/~yves/index.html}) based
on ordered orthogonal arrays; the digital net has base \texttt{B = 3},
dimension \texttt{S = 13}, the generating matrices have \texttt{R = 9} rows
and \texttt{C = 9} columns, and the strength of the net is \texttt{St = 6}.
%At the moment, there are no existing subdirectories of predefined files in SSJ.
\iffalse
At the moment, the existing subdirectories of predefined files in SSJ
are the following
 (for details on the available files,
 see \url{http://www.iro.umontreal.ca/~simardr/ssj/data})
(in OOA, O is the letter O, not the number 0):
\begin {table}[htb]
\begin{center}
% \caption {\label {tab:datadir1}}
\begin {tabular}{|l|l|l|}
\hline
  $\mbox{Directory}$  &   $\mbox{Remark}$  & $\mbox{Reference}$  \\
\hline
 \texttt{Edel/OOA2/} & Based on orthogonal ordered arrays & Yves Edel  \\
 \texttt{Edel/OOA3/} & Based on orthogonal ordered arrays & Yves Edel  \\
 \texttt{Edel/OOA4/} & Based on orthogonal ordered arrays & Yves Edel  \\
 \texttt{Edel/RSNet/} & Maximally equidistributed-collision free
   \cite{rLEC99a} & Yves Edel  \\
\hline
\end {tabular}
\label {tab:datadir1}
\end{center}
\end {table}
\fi


\bigskip\hrule\bigskip
%%%%%%%%%%%%%%%%%%%%%%%%%%%%%%%%%%%%%%%%%%%%%%%%%%%%%%%%%%%%%%%%%%%%%%%%%%%

\begin{code}
\begin{hide}
/*
 * Class:        DigitalNetFromFile
 * Description:  read the parameters defining a digital net from a file
                 or from a URL address
 * Environment:  Java
 * Software:     SSJ 
 * Copyright (C) 2001  Pierre L'Ecuyer and Université de Montréal
 * Organization: DIRO, Université de Montréal
 * @author       
 * @since

 * SSJ is free software: you can redistribute it and/or modify it under
 * the terms of the GNU General Public License (GPL) as published by the
 * Free Software Foundation, either version 3 of the License, or
 * any later version.

 * SSJ is distributed in the hope that it will be useful,
 * but WITHOUT ANY WARRANTY; without even the implied warranty of
 * MERCHANTABILITY or FITNESS FOR A PARTICULAR PURPOSE.  See the
 * GNU General Public License for more details.

 * A copy of the GNU General Public License is available at
   <a href="http://www.gnu.org/licenses">GPL licence site</a>.
 */
\end{hide}
package umontreal.iro.lecuyer.hups;\begin{hide}

import java.io.*;
import java.util.*;
import java.net.URL;
import java.net.MalformedURLException;
import umontreal.iro.lecuyer.util.PrintfFormat;
\end{hide}

public class DigitalNetFromFile extends DigitalNet \begin{hide} {
   private String filename;

   private void readMatrices (StreamTokenizer st,
                              int r, int k, int dim)
      throws IOException, NumberFormatException {
      // Read dim matrices with r rows and k columns.
      // dim is the dimension of the digital net.
      genMat = new int[dim * k][r];
      for (int i = 0; i < dim; i++)
         for (int c = 0; c < k; c++) {
             for (int j = 0; j < r; j++) {
                 st.nextToken ();
                 genMat[i*numCols + c][j]  = (int) st.nval;
             }
             // If we do not use all the rows, drop the unused ones.
             for (int j = r; j < numRows; j++) {
                 st.nextToken ();
             }
         }
   }


   void readData (StreamTokenizer st) throws
                                      IOException, NumberFormatException
   {
      // Read beginning of data file, but do not read the matrices
      st.eolIsSignificant (false);
      st.slashSlashComments (true);
      int i = st.nextToken ();
      if (i != StreamTokenizer.TT_NUMBER)
         throw new NumberFormatException(" readData: cannot read base");
      b = (int) st.nval;
      st.nextToken ();   numCols = (int) st.nval;
      st.nextToken ();   numRows = (int) st.nval;
      st.nextToken ();   numPoints = (int) st.nval;
      st.nextToken ();   dim = (int) st.nval;
      if (dim < 1)
         throw new IllegalArgumentException (" dimension dim <= 0");
   }


   static BufferedReader openURL (String filename)
                                  throws MalformedURLException, IOException {
      try {
         URL url = new URL (filename);
         BufferedReader input = new BufferedReader (
                                    new InputStreamReader (
                                        url.openStream()));
         return input;

      } catch (MalformedURLException e) {
         System.err.println (e + "   Invalid URL address:   " + filename);
         throw e;

      }  catch (IOException e) {
          // This can receive a FileNotFoundException
         System.err.println (e + " in openURL with " + filename);
         throw e;
      }
   }

   static BufferedReader openFile (String filename) throws
            IOException {
      try {
         BufferedReader input;
         File f = new File (filename);

         // If file with relative path name exists, read it
         if (f.exists()) {
            if (f.isDirectory())
               throw new IOException (filename + " is a directory");
            input = new BufferedReader (new FileReader (filename));
         } else {              // else read it from ssj.jar
            String pseudo = "umontreal/iro/lecuyer/hups/data/";
            StringBuffer pathname = new StringBuffer (pseudo);
            for (int ci = 0; ci < filename.length(); ci++) {
               char ch = filename.charAt (ci);
               if (ch == File.separatorChar)
                  pathname.append ('/');
               else
                  pathname.append (ch);
            }
            InputStream dataInput =
                DigitalNetFromFile.class.getClassLoader().getResourceAsStream (
                  pathname.toString());
            if (dataInput == null)
               throw new FileNotFoundException();
            input = new BufferedReader (new InputStreamReader (dataInput));
         }
         return input;

       } catch (FileNotFoundException e) {
         System.err.println (e + " *** cannot find  " + filename);
         throw e;

      } catch (IOException e) {
         // This will never catch FileNotFoundException since there
         // is a catch clause above.
         System.err.println (e + " cannot read from  " + filename);
         throw e;
      }
   }

\end{hide}
\end{code}
%%%%%%%%%%%%%%%%%%%%%%%%%%%%%%%%%
\subsubsection* {Constructors}
\begin{code}

   public DigitalNetFromFile (String filename, int r1, int w, int s1)
          throws MalformedURLException, IOException \begin{hide}
   {
      super ();
      BufferedReader input = null;
      StreamTokenizer st = null;
      try {
         if (filename.startsWith("http:") || filename.startsWith("ftp:"))
            input = openURL(filename);
         else
            input = openFile(filename);
         st = new StreamTokenizer (input);
         readData (st);

      } catch (MalformedURLException e) {
         System.err.println ("   Invalid URL address:   " + filename);
         throw e;
      } catch (FileNotFoundException e) {
         System.err.println ("   Cannot find  " + filename);
         throw e;
      } catch (NumberFormatException e) {
         System.err.println ("   Cannot read number from " + filename);
         throw e;
      }  catch (IOException e) {
         System.err.println ("   IOException:   " + filename);
         throw e;
      }

      if (b == 2) {
         System.err.println ("   base = 2, use DigitalNetBase2FromFile");
         throw new IllegalArgumentException
             ("base = 2, use DigitalNetBase2FromFile");
      }
      if ((double)numCols * Math.log ((double)b) > (31.0 * Math.log (2.0)))
         throw new IllegalArgumentException
            ("DigitalNetFromFile:   too many points" + PrintfFormat.NEWLINE);
      if (r1 > numRows)
         throw new IllegalArgumentException
            ("DigitalNetFromFile:   One must have   r1 <= Max num rows" +
                PrintfFormat.NEWLINE);
      if (s1 > dim)
         throw new IllegalArgumentException
            ("DigitalNetFromFile:   One must have   s1 <= Max dimension" +
                 PrintfFormat.NEWLINE);
      if (w < 0) {
         r1 = w = numRows;
         s1 = dim;
      }
      if (w < numRows)
         throw new IllegalArgumentException
            ("DigitalNetFromFile:   One must have   w >= numRows" +
              PrintfFormat.NEWLINE);

      try {
         readMatrices (st, r1, numCols, s1);
      } catch (NumberFormatException e) {
         System.err.println (e + "   cannot read matrices from " + filename);
         throw e;
      }  catch (IOException e) {
         System.err.println (e + "   cannot read matrices from  " + filename);
         throw e;
      }
      input.close();

      this.filename = filename;
      numRows = r1;
      dim = s1;
      outDigits = w;
      int x = b;
      for (int i=1; i<numCols; i++) x *= b;
      if (x != numPoints) {
         System.out.println ("DigitalNetFromFile:   numPoints != b^k");
         throw new IllegalArgumentException (" numPoints != b^k");
      }

      // Compute the normalization factors.
      normFactor = 1.0 / Math.pow ((double) b, (double) outDigits);
      double invb = 1.0 / b;
      factor = new double[outDigits];
      factor[0] = invb;
      for (int j = 1; j < outDigits; j++)
         factor[j] = factor[j-1] * invb;
  }\end{hide}
\end{code}
\begin{tabb}
    Constructs a digital net after reading its parameters from file
    {\texttt{filename}}. If a file named \texttt{filename}
   can be found relative to the program's directory, then the parameters
   will be read from this file; otherwise, they will be read from the file
   named  \texttt{filename} in the \texttt{ssj.jar} archive.
   If {\texttt{filename}} is a URL string, it will be read on
   the World Wide Web.
   For example, to construct a digital net from the parameters in file
   \texttt{B3S13R9C9St6} in the current directory,  one must give the string
   \texttt{"B3S13R9C9St6"} as argument to the constructor.
   As an example of a file read from the WWW, one may give
   as argument to the constructor the string
   \texttt{
  "http://www.iro.umontreal.ca/\~{}simardr/ssj/data/Edel/OOA3/B3S13R6C6St4"}.
   Parameter \texttt{w} gives the number of digits of resolution, \texttt{r1} is
   the number of rows, and \texttt{s1} is the dimension.
   Restrictions: \texttt{s1} must be less than the maximal dimension, and
   \texttt{r1} less than the maximal number of rows in the data file.
   Also \texttt{w} $\ge$ \texttt{r1}.
\end{tabb}
\begin{htmlonly}
   \param{filename}{Name of the file to be read}
   \param{r1}{Number of rows for the generating matrices}
   \param{w}{Number of digits of resolution}
   \param{s1}{Number of dimensions}
\end{htmlonly}
\begin{code}

   public DigitalNetFromFile (String filename, int s)
          throws MalformedURLException, IOException \begin{hide}
   {
       this (filename, -1, -1, s);
   }

   DigitalNetFromFile ()
   {
       super ();
   }\end{hide}
\end{code}
\begin{tabb}
   Same as \method{DigitalNetFromFile}{}\texttt{(filename, r, r, s)} where
   \texttt{s} is the dimension and  \texttt{r} is given in data file \texttt{filename}.
\end{tabb}
\begin{htmlonly}
   \param{filename}{Name of the file to be read}
   \param{s}{Number of dimensions}
\end{htmlonly}



%%%%%%%%%%%%%%%%%%%%%%%%%%%%%%%%%
\subsubsection*{Methods}

\begin{code}\begin{hide}

   public String toString() {
      StringBuffer sb = new StringBuffer ("File:   " + filename +
         PrintfFormat.NEWLINE);
      sb.append (super.toString());
      return sb.toString();
   }\end{hide}

   public String toStringDetailed() \begin{hide} {
      StringBuffer sb = new StringBuffer (toString());
      sb.append (PrintfFormat.NEWLINE + "n = " + numPoints  +
                 PrintfFormat.NEWLINE);
      sb.append ("dim = " + dim  + PrintfFormat.NEWLINE);
      for (int i = 0; i < dim; i++) {
         sb.append (PrintfFormat.NEWLINE + " // dim = " + (1 + i) +
                    PrintfFormat.NEWLINE);
         for (int c = 0; c < numCols; c++) {
            for (int r = 0; r < numRows; r++)
                sb.append (genMat[i*numCols + c][r] + " ");
            sb.append (PrintfFormat.NEWLINE);
         }
      }
      return sb.toString ();
   }\end{hide}
\end{code}
\begin{tabb}
    Writes the parameters and the generating matrices of this digital net
    to a string. This is useful to check that the file parameters have been
    read correctly.
\end{tabb}
\begin{code} \begin{hide}

   static class NetComparator implements Comparator {
      // Used to sort list of nets. Sort first by base, then by dimension,
      // then by the number of rows. Don't forget that base = 4 are in files
      // named B4_2* and that the computations are done in base 2.
      public int compare (Object o1, Object o2) {
         DigitalNetFromFile net1 = (DigitalNetFromFile) o1;
         DigitalNetFromFile net2 = (DigitalNetFromFile) o2;
         if (net1.b < net2.b)
            return -1;
         if (net1.b > net2.b)
            return 1;
         if (net1.filename.indexOf("_") >= 0 &&
             net2.filename.indexOf("_") < 0 )
            return 1;
         if (net2.filename.indexOf("_") >= 0 &&
             net1.filename.indexOf("_") < 0 )
            return -1;
         if (net1.dim < net2.dim)
            return -1;
         if (net1.dim > net2.dim)
            return 1;
         if (net1.numRows < net2.numRows)
            return -1;
         if (net1.numRows > net2.numRows)
            return 1;
         return 0;
      }
   }


   private static List getListDir (String dirname) throws IOException {
      try {
         String pseudo = "umontreal/iro/lecuyer/hups/data/";
         StringBuffer pathname = new StringBuffer (pseudo);
         for (int ci = 0; ci < dirname.length(); ci++) {
            char ch = dirname.charAt (ci);
            if (ch == File.separatorChar)
               pathname.append ('/');
            else
               pathname.append (ch);
         }
         URL url = DigitalNetFromFile.class.getClassLoader().getResource (
                      pathname.toString());
         File dir = new File (url.getPath());
         if (!dir.isDirectory())
            throw new IllegalArgumentException (
               dirname + " is not a directory");
         File[] files = dir.listFiles();
         List alist = new ArrayList (200);
         if (!dirname.endsWith (File.separator))
            dirname += File.separator;
         for (int i = 0; i < files.length; i++) {
            if (files[i].isDirectory())
               continue;
            if (files[i].getName().endsWith ("gz") ||
                files[i].getName().endsWith ("zip"))
               continue;
            DigitalNetFromFile net = new DigitalNetFromFile();
            BufferedReader input = net.openFile(dirname + files[i].getName());
            StreamTokenizer st = new StreamTokenizer (input);
            net.readData (st);
            net.filename = files[i].getName();
            alist.add (net);
         }
         if (alist != null && !files[0].isDirectory())
            Collections.sort (alist, new NetComparator ());
         return alist;

      } catch (NullPointerException e) {
         System.err.println ("getListDir: cannot find directory   " + dirname);
         throw e;

      } catch (NumberFormatException e) {
         System.err.println (e + "***   cannot read number ");
         throw e;

      }  catch (IOException e) {
         System.err.println (e);
         throw e;
      }
   }


   private static String listFiles (String dirname) {
      try {
         String pseudo = "umontreal/iro/lecuyer/hups/data/";
         StringBuffer pathname = new StringBuffer (pseudo);
         for (int ci = 0; ci < dirname.length(); ci++) {
            char ch = dirname.charAt (ci);
            if (ch == File.separatorChar)
               pathname.append ('/');
            else
               pathname.append (ch);
         }
         URL url = DigitalNetFromFile.class.getClassLoader().getResource (
                      pathname.toString());
         File dir = new File (url.getPath());
         File[] list = dir.listFiles();
         List alist = new ArrayList (200);
         final int NPRI = 3;
         StringBuffer sb = new StringBuffer(1000);
         for (int i = 0; i < list.length; i++) {
            if (list[i].isDirectory()) {
               sb.append (PrintfFormat.s(-2, list[i].getName()));
               sb.append (File.separator + PrintfFormat.NEWLINE);
            } else {
               sb.append (PrintfFormat.s(-25, list[i].getName()));
               if (i % NPRI == 2)
                  sb.append (PrintfFormat.NEWLINE);
            }
         }
         if (list.length % NPRI > 0)
            sb.append (PrintfFormat.NEWLINE);
         return sb.toString();

      } catch (NullPointerException e) {
         System.err.println ("listFiles: cannot find directory   " + dirname);
         throw e;
      }
   }\end{hide}

   public static String listDir (String dirname) throws IOException \begin{hide} {
      try {
         List list = getListDir (dirname);
         if (list == null || list.size() == 0)
            return listFiles (dirname);
         StringBuffer sb = new StringBuffer(1000);

         sb.append ("Directory:   " + dirname  + PrintfFormat.NEWLINE +
                    PrintfFormat.NEWLINE);
         sb.append (PrintfFormat.s(-25, "     File") +
                    PrintfFormat.s(-15, "       Base") +
                    PrintfFormat.s(-10, "Dimension") +
                    PrintfFormat.s(-10, " numRows") +
                    PrintfFormat.s(-10, "numColumns" +
                    PrintfFormat.NEWLINE));
         int base = 0;
         for (int i = 0; i < list.size(); i++) {
            DigitalNet net = (DigitalNet) list.get(i);
            int j = ((DigitalNetFromFile)net).filename.lastIndexOf
                (File.separator);
            if (net.b != base) {
               sb.append (
      "----------------------------------------------------------------------"
            + PrintfFormat.NEWLINE);
            base = net.b;
            }
            String name = ((DigitalNetFromFile)net).filename.substring(j+1);
            sb.append (PrintfFormat.s(-25, name) +
                       PrintfFormat.d(10, net.b) +
                       PrintfFormat.d(10, net.dim) +
                       PrintfFormat.d(10, net.numRows) +
                       PrintfFormat.d(10, net.numCols) +
                       PrintfFormat.NEWLINE);
         }
         return sb.toString();

      } catch (NullPointerException e) {
         System.err.println (
            "formatPlain: cannot find directory   " + dirname);
         throw e;
      }
   }\end{hide}
\end{code}
\begin{tabb}
  Lists all files (or directories) in directory \texttt{dirname}. Only relative
  pathnames should be used. The files are  parameter files used in defining
  digital nets.  For example, calling \texttt{listDir("")} will give the list
  of the main data directory in SSJ, while calling \texttt{listDir("Edel/OOA2")}
  will give the list of all files in directory \texttt{Edel/OOA2}.
 \end{tabb}
\begin{code}

   public static void listDirHTML (String dirname, String filename)
          throws IOException \begin{hide} {
      String list = listDir(dirname);
      StreamTokenizer st = new StreamTokenizer (new StringReader(list));
      st.eolIsSignificant(true);
      st.ordinaryChar('/');
      st.ordinaryChar('_');
      st.ordinaryChar('-');
      st.wordChars('-', '-');
      st.wordChars('_', '_');
      st.slashSlashComments(false);
      st.slashStarComments(false);
      PrintWriter out = new PrintWriter (
                            new BufferedWriter (
                               new FileWriter (filename)));
      out.println ("<html>" + PrintfFormat.NEWLINE +
          "<head>" + PrintfFormat.NEWLINE + "<title>");
      while (st.nextToken () != st.TT_EOL)
         ;
      out.println ( PrintfFormat.NEWLINE + "</title>" +
           PrintfFormat.NEWLINE + "</head>");
//      out.println ("<body background bgcolor=#e1eae8 vlink=#c00000>");
      out.println ("<body>");
      out.println ("<table border>");
      out.println ("<caption> Directory: " + dirname + "</caption>");
      st.nextToken(); st.nextToken();
      while (st.sval.compareTo ("File") != 0)
         st.nextToken();
      out.print ("<tr align=center><th>" + st.sval + "</th>");
      while (st.nextToken () != st.TT_EOL) {
         out.print ("<th>" + st.sval + "</th>" );
      }
      out.println ("</tr>" + PrintfFormat.NEWLINE);
      while (st.nextToken () != st.TT_EOF) {
          switch(st.ttype) {
          case StreamTokenizer.TT_EOL:
             out.println ("</tr>");
             break;
          case StreamTokenizer.TT_NUMBER:
             out.print ("<td>" + (int) st.nval + "</td>" );
             break;
          case StreamTokenizer.TT_WORD:
             if (st.sval.indexOf ("---") >= 0) {
                st.nextToken ();
                continue;
             }
             out.print ("<tr align=center><td>" + st.sval + "</td>");
             break;
          default:
             out.print (st.sval);
             break;
        }
      }

      out.println ("</table>");
      out.println ("</body>" + PrintfFormat.NEWLINE + "</html>");
      out.close();
}\end{hide}
\end{code}
\begin{tabb}
 Creates a list of all data files in directory \texttt{dirname} and writes
that list in format HTML in output file \texttt{filename}.
Each data file contains the parameters required to build a digital net.
The resulting list contains a line for each data file giving the
name of the file, the base, the dimension, the number of rows and
the number of columns of the corresponding digital net.
 \end{tabb}
\begin{code}
\begin{hide}
}
\end{hide}
\end{code}
