\defclass{HammersleyPointSet}

This class implements \emph{Hammersley point sets}, 
which are defined as follows.
Let $2 = b_1 < b_2 < \cdots$ denote the sequence of all prime 
numbers by increasing order.
The Hammersley point set with $n$ points in $s$ dimensions contains
the points
\eq
 \mathbf{u}_i = (i/n,\psi_{b_1}(i),\psi_{b_2}(i),\dots, \psi_{b_{s-1}}(i)),
                                         \eqlabel{eq:Hammersley-point2}
\endeq
for $i=0,\dots,n-1$, where $\psi_b$ is the radical inverse function
in base $b$, defined in \class{RadicalInverse}.
This class is not a subclass of \class{DigitalNet}, because the basis
is not the same for all coordinates.
We do obtain a net in a generalized sense if 
$n = b_1^{k_1} = b_2^{k_2} = \cdots = b_{s-1}^{k_{s-1}}$
for some integers $k_1,\dots,k_{s-1}$.

The points of a Hammersley point set can be ``scrambled'' by applying a 
permutation to the digits of $i$ before computing each coordinate\latex{
via (\ref{eq:Hammersley-point})}.  If 
\[
  i = a_0 + a_1 b_j + \dots + a_{k_j-1} b_j^{k_j-1},
\]
and $\pi_j$ is a permutation of the digits $\{0,\dots,b_j-1\}$, then
\[
 \psi_{b_j}(i) = \sum_{r=0}^{k_j-1} a_r b_j^{-r-1} 
\]
is replaced \latex{in (\ref{eq:Hammersley-point})} by
\[
 u_{i,j}= \sum_{r=0}^{k_j-1} \pi_j[a_r] b_j^{-r-1}.
\]
The permutations $\pi_j$ can be deterministic or random.
One (deterministic) possibility implemented here is to use
the Faure permutation of $\{0,\dots,b_j\}$ for $\pi_j$, for each 
coordinate $j > 0$.


\bigskip\hrule\bigskip
%%%%%%%%%%%%%%%%%%%%%%%%%%%%%%%%%%%%%%%%%%%%%%%%%%%%%%%%%%%%%%%%%%

\begin{code}
\begin{hide}
/*
 * Class:        HammersleyPointSet
 * Description:  
 * Environment:  Java
 * Software:     SSJ 
 * Copyright (C) 2001  Pierre L'Ecuyer and Université de Montréal
 * Organization: DIRO, Université de Montréal
 * @author       
 * @since

 * SSJ is free software: you can redistribute it and/or modify it under
 * the terms of the GNU General Public License (GPL) as published by the
 * Free Software Foundation, either version 3 of the License, or
 * any later version.

 * SSJ is distributed in the hope that it will be useful,
 * but WITHOUT ANY WARRANTY; without even the implied warranty of
 * MERCHANTABILITY or FITNESS FOR A PARTICULAR PURPOSE.  See the
 * GNU General Public License for more details.

 * A copy of the GNU General Public License is available at
   <a href="http://www.gnu.org/licenses">GPL licence site</a>.
 */
\end{hide}
package umontreal.iro.lecuyer.hups;


public class HammersleyPointSet extends PointSet\begin{hide} {
   private int[] base;           // Vector of prime bases.
   private int[][] permutation;  // Digits permutation, for each dimension.
   private boolean permuted;     // Permute digits?
\end{hide}
\end{code}

%%%%%%%%%%%%%%%%%%%%%%%%%%%%%%%%%
\subsubsection*{Constructor}

\begin{code}

   public HammersleyPointSet (int n, int dim) \begin{hide} {
      if (n < 0 || dim < 1)
         throw new IllegalArgumentException
            ("Hammersley point sets must have positive dimension and n >= 0");
      numPoints = n;
      this.dim  = dim;
      if (dim > 1)
         base = RadicalInverse.getPrimes (dim - 1);
   }\end{hide}
\end{code}
 \begin{tabb}
   Constructs a new Hammersley point set with \texttt{n} points in \texttt{dim}
   dimensions.
 \end{tabb}
\begin{htmlonly}
   \param{n}{number of points}
   \param{dim}{dimension of the point set}
\end{htmlonly}

%%%%%%%%%%%%%%%%%%%%%%%%%%%%%%%%%
\subsubsection*{Methods}

\begin{code}

   public void addFaurePermutations()\begin{hide} {
      if (dim > 1) {
         permutation = new int[dim][];
         for (int i = 0; i < dim - 1; i++) {
            permutation[i] = new int[base[i]];
            RadicalInverse.getFaurePermutation (base[i], permutation[i]);
         }
      }
      permuted = true;
   }\end{hide}
\end{code}
 \begin{tabb}
  Permutes the digits using Faure permutations for all coordinates.
  After the method is called, the coordinates $u_{i,j}$ are generated via
\[
  u_{i,j} = \sum_{r=0}^{k-1} \pi_j[a_r] b_j^{-r-1},
\]
 for $j=1,\dots,s-1$ and $u_{i,0}=i/n$,
 where $\pi_j$ is the Faure permutation of $\{0,\dots,b_j-1\}$.
 \end{tabb}
\begin{code}

   public void ErasePermutations()\begin{hide} {
      permuted = false;
      permutation = null;
   }
\end{hide}
\end{code}
 \begin{tabb}
  Erases the Faure permutations: from now on, the digits will not be
  Faure permuted.
 \end{tabb}
\begin{code}
\begin{hide}

   public double getCoordinate (int i, int j) {
      if (j == 0)
         return (double) i / (double) numPoints;
      if (permuted)
         return RadicalInverse.permutedRadicalInverse 
                   (base[j - 1], permutation[j - 1], i);
      else 
         return RadicalInverse.radicalInverse (base[j - 1], i);
   }
}
\end{hide}
\end{code}
