\defclass {HistogramSeriesCollection}

Stores data used in a \texttt{HistogramChart}. %This class extends
%\externalclass{umontreal.iro.lecuyer.charts}{SSJXYSeriesCollection}.
\texttt{HistogramSeriesCollection} provides complementary tools to draw histograms.
One may add observation sets, define histogram bins, set plot color and plot style,
enable/disable filled shapes, and set margin between shapes for each series.
%
This class is linked with class \texttt{CustomHistogramDataset} to store data plots,
and linked with JFreeChart \texttt{XYBarRenderer} to render the plot.
\texttt{CustomHistogramDataset} has been developed at the Universit\'
e de Montr\'eal to extend the JFreeChart API,
and is used to manage histogram datasets in a JFreeChart chart.

\bigskip\hrule
\begin{code}
\begin{hide}
/*
 * Class:        HistogramSeriesCollection
 * Description:  
 * Environment:  Java
 * Software:     SSJ 
 * Copyright (C) 2001  Pierre L'Ecuyer and Université de Montréal
 * Organization: DIRO, Université de Montréal
 * @author       
 * @since

 * SSJ is free software: you can redistribute it and/or modify it under
 * the terms of the GNU General Public License (GPL) as published by the
 * Free Software Foundation, either version 3 of the License, or
 * any later version.

 * SSJ is distributed in the hope that it will be useful,
 * but WITHOUT ANY WARRANTY; without even the implied warranty of
 * MERCHANTABILITY or FITNESS FOR A PARTICULAR PURPOSE.  See the
 * GNU General Public License for more details.

 * A copy of the GNU General Public License is available at
   <a href="http://www.gnu.org/licenses">GPL licence site</a>.
 */
\end{hide}
package umontreal.iro.lecuyer.charts;\begin{hide}

import   umontreal.iro.lecuyer.stat.TallyStore;
import   umontreal.iro.lecuyer.util.Num;

import   org.jfree.chart.renderer.xy.XYBarRenderer;
import   org.jfree.data.statistics.HistogramBin;

import   cern.colt.list.DoubleArrayList;

import   java.util.Locale;
import   java.util.Formatter;
import   java.util.List;
import   java.util.ListIterator;
import   java.awt.Color;\end{hide}

public class HistogramSeriesCollection extends SSJXYSeriesCollection \begin{hide} {
   protected boolean[] filled;   // fill flag for each series
   protected double[] lineWidth; // sets line width
   protected int numBin = 20;    // default number of bins

   private int calcNumBins (int n) {
      // set the number of bins
      int numbins = (int) Math.ceil (1.0 + Num.log2(n));
      if (n > 5000)
         numbins *= 2;
      return numbins;
   }\end{hide}
\end{code}

%%%%%%%%%%%%%%%%%%%%%%%%%%%
\subsubsection*{Constructors}

\begin{code}
   public HistogramSeriesCollection() \begin{hide} {
      renderer = new XYBarRenderer();
      seriesCollection = new CustomHistogramDataset();
   }\end{hide}
\end{code}
\begin{tabb}
   Creates a new \texttt{HistogramSeriesCollection} instance with empty dataset.
\end{tabb}
\begin{code}

   public HistogramSeriesCollection (double[]... data) \begin{hide} {
      seriesCollection = new CustomHistogramDataset();
      renderer = new XYBarRenderer();
      CustomHistogramDataset tempSeriesCollection =
             (CustomHistogramDataset)seriesCollection;
      for (int i = 0; i < data.length; i ++) {
         numBin = calcNumBins(data[i].length);
         tempSeriesCollection.addSeries(i, data[i], numBin);
      }

      // set default colors
      for (int i = 0; i < tempSeriesCollection.getSeriesCount(); i++) {
         renderer.setSeriesPaint(i, getDefaultColor(i));
      }

      // set default plot style
      filled = new boolean[seriesCollection.getSeriesCount()];
      lineWidth = new double[seriesCollection.getSeriesCount()];
      for (int i = 0; i < tempSeriesCollection.getSeriesCount(); i++) {
         filled[i] = false;
         lineWidth[i] = 0.5;
         setFilled(i, false);
      }
   }\end{hide}
\end{code}
\begin{tabb}
   Creates a new \texttt{HistogramSeriesCollection} instance with given data series.
   Bins the elements of data in equally spaced containers (the number of bins
   can be changed using the method \method{setBins}{}).
   Each input parameter % (or matrix column)
   represents a data series.
\end{tabb}
\begin{htmlonly}
   \param{data}{series of point sets.}
\end{htmlonly}
\begin{code}

   public HistogramSeriesCollection (double[] data, int numPoints) \begin{hide} {
      seriesCollection = new CustomHistogramDataset();
      renderer = new XYBarRenderer();
      CustomHistogramDataset tempSeriesCollection =
             (CustomHistogramDataset)seriesCollection;
      numBin = calcNumBins(numPoints);
      tempSeriesCollection.addSeries(0, data, numPoints, numBin);

      // set default colors
      for (int i = 0; i < tempSeriesCollection.getSeriesCount(); i++) {
         renderer.setSeriesPaint(i, getDefaultColor(i));
      }

      // set default plot style
      filled = new boolean[seriesCollection.getSeriesCount()];
      lineWidth = new double[seriesCollection.getSeriesCount()];
      for (int i = 0; i < tempSeriesCollection.getSeriesCount(); i++) {
         filled[i] = false;
         lineWidth[i] = 0.5;
         setFilled(i, false);
      }
   }\end{hide}
\end{code}
\begin{tabb}
   Creates a new \texttt{HistogramSeriesCollection} instance with the given data
   series \texttt{data}. Bins the elements of data in equally spaced containers
   (the number of bins can be changed using the method \method{setBins}{}).
   Only the first \texttt{numPoints} of \texttt{data} will be taken into account.
\end{tabb}
\begin{htmlonly}
   \param{data}{Point set}
   \param{numPoints}{Number of points to plot}
\end{htmlonly}
\begin{code}

   public HistogramSeriesCollection (DoubleArrayList... data) \begin{hide} {
      seriesCollection = new CustomHistogramDataset();
      renderer = new XYBarRenderer();
      CustomHistogramDataset tempSeriesCollection =
          (CustomHistogramDataset)seriesCollection;

      for (int i = 0; i < data.length; i++) {
         numBin = calcNumBins(data[i].size());
         tempSeriesCollection.addSeries(i, data[i].elements(), data[i].size(), numBin);
      }

      // set default colors
      for (int i = 0; i < tempSeriesCollection.getSeriesCount(); i++) {
         renderer.setSeriesPaint(i, getDefaultColor(i));
      }

      // set default plot style
      filled = new boolean[seriesCollection.getSeriesCount()];
      lineWidth = new double[seriesCollection.getSeriesCount()];
      for (int i = 0; i < tempSeriesCollection.getSeriesCount(); i++) {
         filled[i] = false;
         lineWidth[i] = 0.5;
         setFilled(i, false);
      }
   }\end{hide}
\end{code}
\begin{tabb}
   Creates a new \texttt{HistogramSeriesCollection}.
   Bins the elements of data in equally spaced containers (the number of bins
   can be changed using the method \method{setBins}{}).
   The input parameter represents a set of data plots. Each
  \externalclass{cern.colt.list}{DoubleArrayList} variable corresponds
   to a histogram on the chart.
\end{tabb}
\begin{htmlonly}
   \param{data}{series of observation sets.}
\end{htmlonly}
\begin{code}

   public HistogramSeriesCollection (TallyStore... tallies) \begin{hide} {
      seriesCollection = new CustomHistogramDataset();
      renderer = new XYBarRenderer();
      CustomHistogramDataset tempSeriesCollection = (CustomHistogramDataset)seriesCollection;

      double h;
      for (int i = 0; i < tallies.length; i++) {
         // Scott's formula
         h = 3.5*tallies[i].standardDeviation() /
                           Math.pow(tallies[i].numberObs(), 1.0/3.0);
         numBin = (int) ((tallies[i].max() - tallies[i].min()) / (1.5*h));
         tempSeriesCollection.addSeries (i, tallies[i].getArray(),
              tallies[i].numberObs(), numBin, tallies[i].min(), tallies[i].max());
      }

      // set default colors
      for (int i = 0; i < tempSeriesCollection.getSeriesCount(); i++) {
         renderer.setSeriesPaint(i, getDefaultColor(i));
      }

      // set default plot style
      filled = new boolean[seriesCollection.getSeriesCount()];
      lineWidth = new double[seriesCollection.getSeriesCount()];
      for (int i = 0; i < tempSeriesCollection.getSeriesCount(); i++) {
         filled[i] = false;
         lineWidth[i] = 0.5;
         setFilled(i, false);
      }
   }\end{hide}
\end{code}
\begin{tabb}
   Creates a new \texttt{HistogramSeriesCollection} instance with default
 parameters and given data. The input parameter represents a collection of
 data observation sets. Each \texttt{TallyStore} input parameter represents
 an observation set.
\end{tabb}
\begin{htmlonly}
   \param{tallies}{series of point sets.}
\end{htmlonly}
\begin{code}

   public HistogramSeriesCollection (CustomHistogramDataset data) \begin{hide} {
      renderer = new XYBarRenderer();
      seriesCollection = data;

      // set default colors
      for (int i = 0; i < data.getSeriesCount(); i++) {
         renderer.setSeriesPaint(i, getDefaultColor(i));
      }

      // set default plot style
      filled = new boolean[seriesCollection.getSeriesCount()];
      lineWidth = new double[seriesCollection.getSeriesCount()];
      for (int i = 0; i < data.getSeriesCount(); i++) {
         filled[i] = false;
         lineWidth[i] = 0.5;
         setFilled(i, false);
      }
   }\end{hide}
\end{code}
\begin{tabb}
   Creates a new \texttt{HistogramSeriesCollection} instance.
   The input parameter represents a set of plotting data.
   Each series of the given collection corresponds to a histogram.
\end{tabb}
\begin{htmlonly}
   \param{data}{series of point sets.}
\end{htmlonly}

%%%%%%%%%%%%%%%%%%%%%%%%%%%%%%%%%%%%%%%%%%%%%%%%%%%%%%%%%%%%%%%%%%%%%%%%%%%
\subsubsection*{Data control methods}

\begin{code}

   public int add (double[] data) \begin{hide} {
      return add (data, data.length);
   }\end{hide}
\end{code}
\begin{tabb}
   Adds a data series into the series collection.
\end{tabb}
\begin{htmlonly}
   \param{data}{new series.}
   \return{Integer that represent the new point set.}
\end{htmlonly}
\begin{code}

   public int add (double[] data, int numPoints) \begin{hide} {
      CustomHistogramDataset tempSeriesCollection = (CustomHistogramDataset)seriesCollection;
      tempSeriesCollection.addSeries(tempSeriesCollection.getSeriesCount(),
            data, numPoints, numBin);

      boolean[] newFilled = new boolean[seriesCollection.getSeriesCount()];
      double[] newLineWidth = new double[seriesCollection.getSeriesCount()];

      // color
      int j = seriesCollection.getSeriesCount() - 1;
      renderer.setSeriesPaint(j, getDefaultColor(j));
      newFilled[j] = false;
      newLineWidth[j] = 0.5;

      for (j = 0; j < seriesCollection.getSeriesCount() - 1; j++) {
         newFilled[j] = filled[j];
         newLineWidth[j] = lineWidth[j];
      }

      filled = newFilled;
      lineWidth = newLineWidth;

      return seriesCollection.getSeriesCount() - 1;
   }\end{hide}
\end{code}
\begin{tabb}
   Adds a data series into the series collection. Only \emph{the first}
  \texttt{numPoints} of \texttt{data} will be added to the new series.
\end{tabb}
\begin{htmlonly}
   \param{data}{new series.}
   \param{numPoints}{Number of points to add}
   \return{Integer that represent the new point set.}
\end{htmlonly}
\begin{code}

   public int add (DoubleArrayList observationSet) \begin{hide} {
      return add(observationSet.elements(), observationSet.size());
   }\end{hide}
\end{code}
\begin{tabb}
   Adds a data series into the series collection.
\end{tabb}
\begin{htmlonly}
   \param{observationSet}{new series values.}
   \return{Integer that represent the new point set's position in the \texttt{CustomHistogramDataset} object.}
\end{htmlonly}
\begin{code}

   public int add (TallyStore tally) \begin{hide} {
      return add(tally.getArray(), tally.numberObs());
   }\end{hide}
\end{code}
\begin{tabb}
   Adds a data series into the series collection.
\end{tabb}
\begin{htmlonly}
   \param{tally}{\externalclass{umontreal.iro.lecuyer.stat}{TallyStore} to add values.}
   \return{Integer that represent the new point set's position in the \texttt{CustomHistogramDataset} object.}
\end{htmlonly}
\begin{code}

   public CustomHistogramDataset getSeriesCollection() \begin{hide} {
      return (CustomHistogramDataset)seriesCollection;
   }\end{hide}
\end{code}
\begin{tabb}
   Returns the \texttt{CustomHistogramDataset} object associated with the current variable.
\end{tabb}
\begin{htmlonly}
   \return{\texttt{CustomHistogramDataset} object associated with the current variable.}
\end{htmlonly}
\begin{code}

   public List getBins (int series) \begin{hide} {
      return ((CustomHistogramDataset)seriesCollection).getBins(series);
   }\end{hide}
\end{code}
\begin{tabb}
   Returns the bins for a series.
\end{tabb}
\begin{htmlonly}
   \param{series}{the series index (in the range \texttt{0} to \texttt{getSeriesCount() - 1}).}
   \return{A list of bins.}
   \throws{IndexOutOfBoundsException}{if \texttt{series} is outside the specified range.}
\end{htmlonly}
\begin{code}

   public void setBins (int series, int bins) \begin{hide} {
      ((CustomHistogramDataset)seriesCollection).setBins(series, bins);
   }\end{hide}
\end{code}
\begin{tabb}
   Sets \texttt{bins} periodic bins from the observation minimum values to the observation maximum value for a series.
\end{tabb}
\begin{htmlonly}
   \param{series}{the series index (in the range \texttt{0} to \texttt{getSeriesCount() - 1}).}
   \param{bins}{the number of periodic bins.}
   \throws{IndexOutOfBoundsException}{if \texttt{series} is outside the specified range.}
\end{htmlonly}
\begin{code}

   public void setBins (int series, int bins, double minimum, double maximum) \begin{hide} {
      ((CustomHistogramDataset)seriesCollection).setBins(series, bins, minimum, maximum);
   }\end{hide}
\end{code}
\begin{tabb}
   Sets \texttt{bins} periodic bins from \texttt{minimum} to \texttt{maximum} for a series.
   Values falling on the boundary of adjacent bins will be assigned to the higher indexed bin.
\end{tabb}
\begin{htmlonly}
   \param{series}{the series index (in the range \texttt{0} to \texttt{getSeriesCount() - 1}).}
   \param{bins}{the number of periodic bins.}
   \param{minimum}{minimum value.}
   \param{maximum}{maximum value.}
   \throws{IndexOutOfBoundsException}{if \texttt{series} is outside the specified range.}
\end{htmlonly}
\begin{code}

   public void setBins (int series, HistogramBin[] binsTable) \begin{hide} {
      ((CustomHistogramDataset)seriesCollection).setBins(series, binsTable);
   }\end{hide}
\end{code}
\begin{tabb}
   Links bins given by table \texttt{binsTable} to a series.
   Values falling on the boundary of adjacent bins will be assigned to the higher indexed bin.
\end{tabb}
\begin{htmlonly}
   \param{series}{the series index (in the range \texttt{0} to \texttt{getSeriesCount() - 1}).}
   \param{binsTable}{new bins table.}
   \throws{IndexOutOfBoundsException}{if \texttt{series} is outside the specified range.}
\end{htmlonly}
\begin{code}

   public List getValuesList (int series) \begin{hide} {
      return ((CustomHistogramDataset)seriesCollection).getValuesList(series);
   }\end{hide}
\end{code}
\begin{tabb}
   Returns the values for a series.
\end{tabb}
\begin{htmlonly}
   \param{series}{the series index (in the range \texttt{0} to \texttt{getSeriesCount() - 1}).}
   \return{A list of values.}
   \throws(IndexOutOfBoundsException}{if <code>series</code> is outside the specified range.}
\end{htmlonly}
\begin{code}

   public double[] getValues (int series) \begin{hide} {
      return ((CustomHistogramDataset)seriesCollection).getValues(series);
   }\end{hide}
\end{code}
\begin{tabb}
   Returns the values for a series.
\end{tabb}
\begin{htmlonly}
   \param{series}{the series index (in the range \texttt{0} to \texttt{getSeriesCount() - 1}).}
   \return{A list of values.}
   \throws(IndexOutOfBoundsException}{if <code>series</code> is outside the specified range.}
\end{htmlonly}
\begin{code}

   public void setValues (int series, List valuesList) \begin{hide} {
      ((CustomHistogramDataset)seriesCollection).setValues(series, valuesList);
   }\end{hide}
\end{code}
\begin{tabb}
   Sets a new values set to a series from a \texttt{List} variable.
\end{tabb}
\begin{htmlonly}
   \param{series}{the series index (in the range \texttt{0} to \texttt{getSeriesCount() - 1}).}
   \param{valuesList}{new values list.}
\end{htmlonly}
\begin{code}

   public void setValues (int series, double[] values) \begin{hide} {
      ((CustomHistogramDataset)seriesCollection).setValues(series, values);
   }\end{hide}
\end{code}
\begin{tabb}
   Sets a new values set to a series from a table.
\end{tabb}
\begin{htmlonly}
   \param{series}{the series index (in the range \texttt{0} to \texttt{getSeriesCount() - 1}).}
   \param{values}{new values table.}
\end{htmlonly}


\newpage

%%%%%%%%%%%%%%%%%%%%%%%%%%%
\subsubsection*{Rendering methods}

\begin{code}

   public boolean getFilled (int series) \begin{hide} {
      return filled[series];
   }\end{hide}
\end{code}
\begin{tabb}
   Returns the \texttt{filled} flag associated with the \texttt{series}-th
  data series.
\end{tabb}
\begin{htmlonly}
   \param{series}{series index.}
   \return{fill flag.}
\end{htmlonly}
\begin{code}

   public void setFilled (int series, boolean filled) \begin{hide} {
      this.filled[series] = filled;
   }\end{hide}
\end{code}
\begin{tabb}
   Sets the \texttt{filled} flag. This option fills histogram rectangular.
   The fill color is the current series color, alpha's color parameter will
   be used to draw transparency.
\end{tabb}
\begin{htmlonly}
   \param{series}{series index.}
   \param{filled}{fill flag.}
\end{htmlonly}
\begin{code}

   public double getMargin() \begin{hide} {
      return ((XYBarRenderer)renderer).getMargin();
   }\end{hide}
\end{code}
\begin{tabb}
    Returns the margin which is a percentage amount by which the bars are trimmed.
\end{tabb}
\begin{code}

   public void setMargin (double margin) \begin{hide} {
      ((XYBarRenderer)renderer).setMargin(margin);
   }\end{hide}
\end{code}
\begin{tabb}
    Sets the margin which is a percentage amount by which the bars are trimmed for all series.
\end{tabb}
\begin{htmlonly}
   \param{margin}{margin percentage amount.}
\end{htmlonly}
\begin{code}

   public double getOutlineWidth (int series) \begin{hide} {
      return lineWidth[series];
   }\end{hide}
\end{code}
\begin{tabb}
    Returns the outline width in pt.
\end{tabb}
\begin{htmlonly}
   \param{series}{series index.}
\end{htmlonly}
\begin{code}

   public void setOutlineWidth (int series, double outline) \begin{hide} {
      this.lineWidth[series] = outline;
   }\end{hide}
\end{code}
\begin{tabb}
    Sets the outline width in pt.
\end{tabb}
\begin{htmlonly}
   \param{series}{series index.}
   \param{outline}{outline width.}
\end{htmlonly}
\begin{code}

   public String toLatex (double XScale, double YScale, double XShift,
                          double YShift, double xmin, double xmax,
                          double ymin, double ymax) \begin{hide} {

      // Calcule les bornes reelles du graphique, en prenant en compte la position des axes
      xmin = Math.min(XShift, xmin);
      xmax = Math.max(XShift, xmax);
      ymin = Math.min(YShift, ymin);
      ymax = Math.max(YShift, ymax);

      CustomHistogramDataset tempSeriesCollection = (CustomHistogramDataset)seriesCollection;
      Formatter formatter = new Formatter(Locale.US);
      double var;
      double margin = ((XYBarRenderer)renderer).getMargin();

      for (int i = tempSeriesCollection.getSeriesCount() - 1; i >= 0; i--) {
         List temp = tempSeriesCollection.getBins(i);
         ListIterator iter = temp.listIterator();

         Color color = (Color)renderer.getSeriesPaint(i);
         String colorString = detectXColorClassic(color);
         if (colorString == null) {
            colorString = "color"+i;
            formatter.format( "\\definecolor{%s}{rgb}{%.2f, %.2f, %.2f}%n",
                              colorString, color.getRed()/255.0, color.getGreen()/255.0,
                              color.getBlue()/255.0);
         }

         HistogramBin currentBin=null;
         while(iter.hasNext()) {
            double currentMargin;
            currentBin = (HistogramBin)iter.next();
            currentMargin = ((margin*(currentBin.getEndBoundary()-currentBin.getStartBoundary())))*XScale;
            if ((currentBin.getStartBoundary() >= xmin && currentBin.getStartBoundary() <= xmax)
               && (currentBin.getCount() >= ymin && currentBin.getCount() <= ymax) )
            {
               var = Math.min( currentBin.getEndBoundary(), xmax);
               if (filled[i]) {
                  formatter.format("\\filldraw [line width=%.2fpt, opacity=%.2f, color=%s] ([xshift=%.4f] %.4f, %.4f) rectangle ([xshift=-%.4f] %.4f, %.4f); %%%n",
                        lineWidth[i], (color.getAlpha()/255.0), colorString,
                        currentMargin, (currentBin.getStartBoundary()-XShift)*XScale, 0.0,
                        currentMargin, (var-XShift)*XScale, (currentBin.getCount()-YShift)*YScale);
              }
              else {
                  formatter.format("\\draw [line width=%.2fpt, color=%s] ([xshift=%.4f] %.4f, %.4f) rectangle ([xshift=-%.4f] %.4f, %.4f); %%%n",
                        lineWidth[i], colorString,
                        currentMargin, (currentBin.getStartBoundary()-XShift)*XScale, 0.0,
                        currentMargin, (var-XShift)*XScale, (currentBin.getCount()-YShift)*YScale);
              }
            }
            else if (   (currentBin.getStartBoundary() >= xmin && currentBin.getStartBoundary() <= xmax)
                        && (currentBin.getCount() >= ymin && currentBin.getCount() > ymax) )
            { // Cas ou notre rectangle ne peut pas etre affiche en entier (trop haut)
               var = Math.min( currentBin.getEndBoundary(), xmax);
               if (filled[i]) {
                  formatter.format("\\filldraw [line width=%.2fpt,  opacity=%.2f, color=%s] ([xshift=%.4f] %.4f, %.4f) rectangle ([xshift=-%.4f] %.4f, %.4f); %%%n",
                        lineWidth[i], (color.getAlpha()/255.0), colorString,
                        currentMargin, (currentBin.getStartBoundary()-XShift)*XScale, 0.0,
                        currentMargin, (var-XShift)*XScale, (ymax-YShift)*YScale);
                  formatter.format("\\draw [line width=%.2fpt, color=%s, style=dotted] ([xshift=%.4f] %.4f, %.4f) rectangle ([yshift=3mm, xshift=-%.4f] %.4f, %.4f); %%%n",
                        lineWidth[i], colorString,
                        currentMargin, (currentBin.getStartBoundary()-XShift)*XScale, (ymax-YShift)*YScale,
                        currentMargin, (var-XShift)*XScale, (ymax-YShift)*YScale);
               }
               else {
                  formatter.format("\\draw [line width=%.2fpt, color=%s] ([xshift=%.4f] %.4f, %.4f) rectangle ([xshift=-%.4f] %.4f, %.4f); %%%n",
                        lineWidth[i], colorString,
                        currentMargin, (currentBin.getStartBoundary()-XShift)*XScale, 0.0,
                        currentMargin, (var-XShift)*XScale, (ymax-YShift)*YScale);

                  formatter.format("\\draw [line width=%.2fpt, color=%s, style=dotted] ([xshift=%.4f] %.4f, %.4f) rectangle ([yshift=3mm, xshift=-%.4f] %.4f, %.4f); %%%n",
                        lineWidth[i], colorString,
                        currentMargin, (currentBin.getStartBoundary()-XShift)*XScale, (ymax-YShift)*YScale,
                        currentMargin, (var-XShift)*XScale, (ymax-YShift)*YScale);
               }
            }
         }
      }
      return formatter.toString();
   }
}\end{hide}
\end{code}
