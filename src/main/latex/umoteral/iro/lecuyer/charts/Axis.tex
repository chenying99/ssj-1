\defclass {Axis}

Represents an axis of a chart encapsulated by
an instance of \texttt{XYChart}.
\texttt{Axis} uses the JFreeChart class \texttt{NumberAxis}
to store some axis properties.
This class represents the $x$-axis or the $y$-axis of a \texttt{XYChart}
and, consequently,  is drawn when calling the \method{toLatex}{} method.
It provides tools to customize the axis in modifying labels and description.

\bigskip\hrule
\begin{code}
\begin{hide}
/*
 * Class:        Axis
 * Description:  An axis of a chart and its properties
 * Environment:  Java
 * Software:     SSJ 
 * Copyright (C) 2001  Pierre L'Ecuyer and Université de Montréal
 * Organization: DIRO, Université de Montréal
 * @author       
 * @since

 * SSJ is free software: you can redistribute it and/or modify it under
 * the terms of the GNU General Public License (GPL) as published by the
 * Free Software Foundation, either version 3 of the License, or
 * any later version.

 * SSJ is distributed in the hope that it will be useful,
 * but WITHOUT ANY WARRANTY; without even the implied warranty of
 * MERCHANTABILITY or FITNESS FOR A PARTICULAR PURPOSE.  See the
 * GNU General Public License for more details.

 * A copy of the GNU General Public License is available at
   <a href="http://www.gnu.org/licenses">GPL licence site</a>.
 */
\end{hide}
package umontreal.iro.lecuyer.charts;\begin{hide}

import org.jfree.chart.axis.NumberAxis;
import org.jfree.chart.axis.NumberTickUnit;
import java.util.Locale;
import java.util.Formatter;
import java.lang.Math;\end{hide}


public class Axis \begin{hide} {
   \end{hide}

   public static final boolean ORIENTATION_VERTICAL = false;
   public static final boolean ORIENTATION_HORIZONTAL = true;\begin{hide}

   private NumberAxis axis;        // Uses JFreeChart API
   private boolean orientation;    // Vertical or horizontal
   private double twinAxisPosition = 0;

   private boolean tick0Flag; // if true, label number at origin is shifted
   // right or above a little to prevent the label from being on the
   // perpendicular axis.

   private boolean  labelsFlag;    // true : enable manual labels
   private String[] labelsName;    // Sets manual labels
   private double[] labelsValue;   // Where putting the labels on the axes

   private double fixStep (double minA, double maxA, double step) {
      // reset step for labels on axes if it is too large or too small
      final double H = 10;
      final double inter = maxA - minA;
      while (step > inter)   // step is too large, make it smaller
         step /= H;
      while (step < inter / H)   // step is too small, make it larger
         step *= H;
      if (step > inter/2)
         step /= 2;
      return step;
   }

\end{hide}
\end{code}

%%%%%%%%%%%%%%%%%%%%%%%%%%%
\subsubsection*{Constructor}

\begin{code}
   public Axis (NumberAxis inAxis, boolean orientation) \begin{hide} {
      this.axis = inAxis;
      this.orientation = orientation;
      this.labelsFlag = false;
      this.labelsValue = null;
      this.labelsName = null;
      inAxis.setAutoRangeIncludesZero(false);
      inAxis.setLowerMargin(0);
      inAxis.setUpperMargin(0);
      axis.setTickUnit(new NumberTickUnit(computeAutoTickValue()));
   }\end{hide}
\end{code}
\begin{tabb}
   Create a new \texttt{Axis} instance from an existing \texttt{NumberAxis}
   instance with vertical ($y$-axis) or horizontal ($x$-axis) orientation.
\end{tabb}
\begin{htmlonly}
   \param{inAxis}{NumberAxis instance associated to the new variable.}
   \param{orientation}{axis direction, horizontal or vertical}
\end{htmlonly}

%%%%%%%%%%%%%%%%%%%%%%%%%%%
\subsubsection*{Methods}

\begin{code}

   protected NumberAxis getAxis() \begin{hide} {
      return axis;
   }\end{hide}
\end{code}
\begin{tabb}
   Returns the \texttt{NumberAxis} instance (from JFreeChart) linked with the current variable.
\end{tabb}
\begin{htmlonly}
   \return{the \texttt{NumberAxis} instance (from JFreeChart) linked with the current variable.}
\end{htmlonly}
\begin{code}

   public String getLabel() \begin{hide} {
      return axis.getLabel();
   }\end{hide}
\end{code}
\begin{tabb}
   Returns the axis description.
\end{tabb}
\begin{htmlonly}
   \return{the axis description.}
\end{htmlonly}
\begin{code}

   public void setLabel (String label) \begin{hide} {
      axis.setLabel(label);
   }\end{hide}
\end{code}
\begin{tabb}
   Sets the axis description. This description will be displayed on the chart, near the axis.
\end{tabb}
\begin{htmlonly}
   \param{label}{axis label.}
\end{htmlonly}
\begin{code}

   public void setLabels (double tick) \begin{hide} {
      axis.setTickUnit(new NumberTickUnit(tick));
      labelsFlag = false;
   }\end{hide}
\end{code}
\begin{tabb}
Sets a periodic label display. Labels will be shown every \texttt{tick} unit.
This tick unit replaces the default unit.
\end{tabb}
\begin{htmlonly}
   \param{tick}{tick unit.}
\end{htmlonly}
\begin{code}

   public void setLabelsAuto() \begin{hide} {
      axis.setTickUnit(new NumberTickUnit(computeAutoTickValue()));
      labelsFlag = false;
   }\end{hide}
\end{code}
\begin{tabb}
   Calculates and sets an automatic tick unit.
\end{tabb}

%%%%%%%%%%%%%%%%%%%%%%%%%%%
\subsubsection*{LaTex-specific methods}

\begin{code}

   @Deprecated
   public void enableCustomLabels() \begin{hide} {
      labelsFlag = true;
   }\end{hide}
\end{code}
\begin{tabb}
Not used anymore.
\end{tabb}
\begin{code}

   @Deprecated
   public void disableCustomLabels() \begin{hide} {
      labelsFlag = false;
   }\end{hide}
\end{code}
\begin{tabb}
Not used anymore.
\end{tabb}
\begin{code}

   public void setLabels (double[] position) \begin{hide} {
      this.labelsName = null;
      this.labelsValue = position.clone();
      double max = max(position);
      if (axis.getUpperBound() < max)
         axis.setUpperBound(max);
      double min = min(position);
      if (axis.getLowerBound() > min)
         axis.setLowerBound(min);
      labelsFlag = true;
   }\end{hide}
\end{code}
\begin{tabb}
Sets the position of each label on this axis. This method requires an array
containing an increasing sequence of numbers representing the positions at
 which labels will appear on the axis. It is designed to export the
 axis to a LaTeX source code; it has no effect on the chart appearance
 displayed with \texttt{XYChart.view()}.
\end{tabb}
\begin{htmlonly}
   \param{position}{new label positions.}
\end{htmlonly}
\begin{code}

   public void setLabels (double[] position, String[] label) \begin{hide} {
      if (label.length != position.length)
         throw new IllegalArgumentException
            ("A label is required for each given position");
      this.labelsName = label.clone();
      this.labelsValue = position.clone();
      double max = max(position);
      if (axis.getUpperBound() < max)
         axis.setUpperBound(max);
      double min = min(position);
      if (axis.getLowerBound() > min)
         axis.setLowerBound(min);
      labelsFlag = true;
   }\end{hide}
\end{code}
\begin{tabb}
  Assigns custom labels to user-defined positions on the axis.
  This method requires an array of positions as well as an array of labels.
  The label \texttt{label[i]} will be used at position \texttt{position[i]}.
   It is designed to export the axis to a LaTeX source code,
  and to use \LaTeX/TikZ commands to write prettier characters; it has no
  effect on the chart appearance displayed with \texttt{XYChart.view()}.
\end{tabb}
\begin{htmlonly}
   \param{position}{label series position on the axis.}
   \param{label}{label series name on the axis.}
\end{htmlonly}
\begin{code}

   public double getTwinAxisPosition() \begin{hide} {
      return this.twinAxisPosition;
   }\end{hide}
\end{code}
\begin{tabb}
   Returns the drawing position parameter (default equals 0).
\end{tabb}
\begin{htmlonly}
   \return{drawing position parameter.}
\end{htmlonly}
\begin{code}

   public void setTwinAxisPosition (double position) \begin{hide} {
      this.twinAxisPosition = position;
   }\end{hide}
\end{code}
\begin{tabb}
   Defines where the opposite axis must be drawn on the current axis,
   where it should appear, and on which label.
\end{tabb}
\begin{htmlonly}
   \param{position}{new drawing position.}
\end{htmlonly}
\begin{code}

   public String toLatex (double scale) \begin{hide} {
      Formatter formatter = new Formatter(Locale.US);
      double maxAxis = Math.max(axis.getRange().getUpperBound(), twinAxisPosition);  //valeur du label le plus grand
      double minAxis = Math.min(axis.getRange().getLowerBound(), twinAxisPosition);  //valeur du label le plus petit

      String precisionAffichageLabel;  //determine combien de decimales seront affichees pour les labels
      double pas = axis.getTickUnit().getSize();   //step d'affichage des labels
      pas = fixStep (minAxis, maxAxis, pas);

      int puissDix;                                //echelle des valeurs sur l'axe
      if (Math.log10(pas) < 0)
         puissDix = (int)Math.log10(pas) - 1;
      else
         puissDix = (int)Math.log10(pas);

      //Placement des fleches, on pourrait facilement les rendre personnalisables...
      String arrowLeftType = "latex";
      String arrowRightType = "latex";
      String arrowLeftMargin = "3mm";
      String arrowRightMargin = "3mm";
      if (twinAxisPosition == minAxis) {
         arrowLeftType = "";
         arrowLeftMargin = "0mm";
      } else if (twinAxisPosition == maxAxis) {
         arrowRightType = "";
         arrowRightMargin = "0mm";
      }

      // Label pour l'axe, avec eventuellement l'echelle
      String label = axis.getLabel();
      if (label == null)
         label = " ";

      if (puissDix < -2 || puissDix > 2)
         label = label + " $(10^{" + puissDix + "})$";
      else
         puissDix = 0;

      if (orientation) { //on est sur l'axe des abscisses

         //affichage de l'axe
         formatter.format("\\draw [%s-%s] ([xshift=-%s] %s,0) -- ([xshift=%s] %s,0) node[right] {%s};%n", arrowLeftType, arrowRightType, arrowLeftMargin, (minAxis - twinAxisPosition)*scale, arrowRightMargin, (maxAxis - twinAxisPosition)*scale, label);

         if (labelsFlag) {   //labels manuels
            String name;
            double value;
            double labelTemp;
            for (int i = 0; i < labelsValue.length; i++) {
               value = labelsValue[i];
               if (labelsName == null) {
                  labelTemp = (value * 100.0 / Math.pow(10, puissDix));
                  //                   if(labelTemp == (int)(labelTemp))
                  name = Integer.toString((int)(labelTemp / 100.0));
                  //                   else
                  //                      name = Double.toString(Math.round(labelTemp)/100.0);
               } else
                  name = labelsName[i];
               if (value == twinAxisPosition && tick0Flag)
                  formatter.format("\\draw (%s,00) -- +(0mm,1mm) -- +(0mm,-1mm) node[below right] {%s};%n", (value - twinAxisPosition)*scale, name);
               else
                  formatter.format("\\draw (%s,0) -- +(0mm,1mm) -- +(0mm,-1mm) node[below] {%s};%n", (value - twinAxisPosition)*scale, name);
            }
         } else {   //labels automatiques
            double labelTemp;
            double k = twinAxisPosition;
            labelTemp = (Math.round(k * 100 / Math.pow(10, puissDix))) / 100.0;
            if (labelTemp == (int)(labelTemp))
               precisionAffichageLabel = "0";
            else if (labelTemp*10 == (int)(labelTemp*10))
               precisionAffichageLabel = "1";
            else
               precisionAffichageLabel = "2";
            if (tick0Flag)
               formatter.format("\\draw (%s,0) -- +(0mm,1mm) -- +(0mm,-1mm) node[below right] {%." + precisionAffichageLabel + "f};%n", (k - twinAxisPosition)*scale, labelTemp);
            else
               formatter.format("\\draw (%s,0) -- +(0mm,1mm) -- +(0mm,-1mm) node[below] {%." + precisionAffichageLabel + "f};%n", (k - twinAxisPosition)*scale, labelTemp);
            k += pas;
            while (k <= maxAxis) { //cote positif
               labelTemp = (Math.round(k * 100 / Math.pow(10, puissDix))) / 100.0;
               if (labelTemp == (int)(labelTemp))
                  precisionAffichageLabel = "0";
               else if (labelTemp*10 == (int)(labelTemp*10))
                  precisionAffichageLabel = "1";
               else
                  precisionAffichageLabel = "2";
               formatter.format("\\draw (%s,0) -- +(0mm,1mm) -- +(0mm,-1mm) node[below] {%." + precisionAffichageLabel + "f};%n", (k - twinAxisPosition)*scale, labelTemp);
               k += pas;
            }

            k = twinAxisPosition - pas;
            while (k >= minAxis) { //cote negatif
               labelTemp = (Math.round(k * 100 / Math.pow(10, puissDix))) / 100.0;
               if (labelTemp == (int)(labelTemp))
                  precisionAffichageLabel = "0";
               else if (labelTemp*10 == (int)(labelTemp*10))
                  precisionAffichageLabel = "1";
               else
                  precisionAffichageLabel = "2";
               formatter.format("\\draw (%s,0) -- +(0mm,1mm) -- +(0mm,-1mm) node[below] {%." + precisionAffichageLabel + "f};%n", (k - twinAxisPosition)*scale, labelTemp);
               k -= pas;
            }
         }

      } else { //On est sur l'axe des ordonnees

         //affichage de l'axe
         formatter.format("\\draw [%s-%s] ([yshift=-%s] 0,%s) -- ([yshift=%s] 0, %s) node[above] {%s};%n", arrowLeftType, arrowRightType, arrowLeftMargin, (minAxis - twinAxisPosition)*scale, arrowRightMargin, (maxAxis - twinAxisPosition)*scale, label);

         if (labelsFlag) {
            //labels manuels
            String name;
            double value;
            double labelTemp;
            for (int i = 0; i < labelsValue.length; i++) {
               value = labelsValue[i];
               if (labelsName == null) {
                  labelTemp = (value * scale * 100 / Math.pow(10, puissDix));
                  if (labelTemp == (int)(labelTemp))
                     name = Integer.toString((int)(labelTemp / 100.0));
                  else
                     name = Double.toString(Math.round(labelTemp) / 100.0);
               } else
                  name = labelsName[i];
               if (value == twinAxisPosition && tick0Flag)
                  formatter.format("\\draw (%s,%s) -- +(1mm,0mm) -- +(-1mm,0mm) node[above left] {%s};%n", 0, (value - twinAxisPosition)*scale, name);
               else
                  formatter.format("\\draw (%s,%s) -- +(1mm,0mm) -- +(-1mm,0mm) node[left] {%s};%n", 0, (value - twinAxisPosition)*scale, name);
            }
         } else {
            //Les labels automatiques
            double k = twinAxisPosition;
            double labelTemp;
            labelTemp = (Math.round(k * 100 / Math.pow(10, puissDix))) / 100.0;
            if (labelTemp == (int)(labelTemp))
               precisionAffichageLabel = "0";
            else if (labelTemp*10 == (int)(labelTemp*10))
               precisionAffichageLabel = "1";
            else
               precisionAffichageLabel = "2";
            if (tick0Flag)
               formatter.format("\\draw (0,%s) -- +(1mm,0mm) -- +(-1mm,0mm) node[above left] {%." + precisionAffichageLabel + "f};%n", (k - twinAxisPosition)*scale, labelTemp);
            else
               formatter.format("\\draw (0,%s) -- +(1mm,0mm) -- +(-1mm,0mm) node[left] {%." + precisionAffichageLabel + "f};%n", (k - twinAxisPosition)*scale, labelTemp);
            k += pas;
            while (k <= maxAxis) { //cote positif de l'axe
               labelTemp = (Math.round(k * 100 / Math.pow(10, puissDix))) / 100.0;
               if (labelTemp == (int)(labelTemp))
                  precisionAffichageLabel = "0";
               else if (labelTemp*10 == (int)(labelTemp*10))
                  precisionAffichageLabel = "1";
               else
                  precisionAffichageLabel = "2";
               formatter.format("\\draw (0,%s) -- +(1mm,0mm) -- +(-1mm,0mm) node[left] {%." + precisionAffichageLabel + "f};%n", (k - twinAxisPosition)*scale, labelTemp);
               k += pas;
            }
            k = twinAxisPosition - pas;
            while (k >= minAxis) { //cote negatif de l'axe
               labelTemp = (Math.round(k * 100 / Math.pow(10, puissDix))) / 100.0;
               if (labelTemp == (int)(labelTemp))
                  precisionAffichageLabel = "0";
               else if (labelTemp*10 == (int)(labelTemp*10))
                  precisionAffichageLabel = "1";
               else
                  precisionAffichageLabel = "2";
               formatter.format("\\draw (0,%s) -- +(1mm,0mm) -- +(-1mm,0mm) node[left] {%." + precisionAffichageLabel + "f};%n", (k - twinAxisPosition)*scale, labelTemp);
               k -= pas;
            }
         }
      }
      return formatter.toString();
   }\end{hide}
\end{code}
\begin{tabb}
   Formats and returns a string containing a \LaTeX-compatible
   source code which represents this axis and its parameters.
\end{tabb}
\begin{htmlonly}
   \return{LaTeX source code in a String.}
   \param{scale}{current axis wished scale.}
\end{htmlonly}
\begin{code}
\begin{hide}
   private double computeAutoTickValue() {
      // Calcule le pas d'affichage par defaut des labels
      double pas;
      double[] bounds = new double[2];
      bounds[0] = Math.min(axis.getRange().getLowerBound(), twinAxisPosition);
      bounds[1] = Math.max(axis.getRange().getUpperBound(), twinAxisPosition);
      if ( bounds[1] - bounds[0] >= 1) {
         int nbMaxDeFois = (int)(bounds[1] - bounds[0]);
         int puissDix = (int)Math.log10(nbMaxDeFois);
         nbMaxDeFois = (int)(nbMaxDeFois / Math.pow(10, puissDix)) + 1;
         if (nbMaxDeFois > 5)
            pas = Math.pow(10, puissDix);
         else if (nbMaxDeFois > 3)
            pas = 0.5 * Math.pow(10, puissDix);
         else if (nbMaxDeFois > 1)
            pas = 0.25 * Math.pow(10, puissDix);
         else //nbMaxDeFois==1
            pas = 0.1 * Math.pow(10, puissDix);
      } else {
         double nbMaxDeFois = bounds[1] - bounds[0];
         int puissDix = 0;
         while (nbMaxDeFois < 1) {
            nbMaxDeFois *= 10;
            puissDix++;
         }
         if (nbMaxDeFois > 5)
            pas = 1 / Math.pow(10, puissDix);
         else if (nbMaxDeFois > 3)
            pas = 0.5 / Math.pow(10, puissDix);
         else if (nbMaxDeFois > 1)
            pas = 0.3 / Math.pow(10, puissDix);
         else //nbMaxDeFois==1
            pas = 0.1 / Math.pow(10, puissDix);
      }
      return pas;
   }

   private static double max (double[] t) {
      if (t == null)
         throw new IllegalArgumentException ("max:   null argument.");
      double aux = t[0];
      for (int i = 1; i < t.length; i++)
         if (t[i] > aux)
            aux = t[i];
      return aux;
   }

   private static double min (double[] t) {
      if (t == null)
         throw new IllegalArgumentException ("min:   null argument.");
      double aux = t[0];
      for (int i = 1; i < t.length; i++)
         if (t[i] < aux)
            aux = t[i];
      return aux;
   }


   void setTick0Flag(boolean flag) {
        tick0Flag = flag;
   }
}\end{hide}
\end{code}
