\defclass {LognormalSpecialGen}

Implements methods for generating random variates from the 
{\em lognormal\/} distribution using an arbitrary normal random 
variate generator.
The (non-static) \texttt{nextDouble} method calls the \texttt{nextDouble} 
method of the normal generator and takes the exponential of the result.


\bigskip\hrule

\begin{code}
\begin{hide}
/*
 * Class:        LognormalSpecialGen
 * Description:  random variates from the lognormal distribution
 * Environment:  Java
 * Software:     SSJ 
 * Copyright (C) 2001  Pierre L'Ecuyer and Université de Montréal
 * Organization: DIRO, Université de Montréal
 * @author       
 * @since

 * SSJ is free software: you can redistribute it and/or modify it under
 * the terms of the GNU General Public License (GPL) as published by the
 * Free Software Foundation, either version 3 of the License, or
 * any later version.

 * SSJ is distributed in the hope that it will be useful,
 * but WITHOUT ANY WARRANTY; without even the implied warranty of
 * MERCHANTABILITY or FITNESS FOR A PARTICULAR PURPOSE.  See the
 * GNU General Public License for more details.

 * A copy of the GNU General Public License is available at
   <a href="http://www.gnu.org/licenses">GPL licence site</a>.
 */
\end{hide}
package umontreal.iro.lecuyer.randvar;\begin{hide}
\end{hide}

public class LognormalSpecialGen extends RandomVariateGen \begin{hide} {

   NormalGen myGen;
\end{hide}\end{code}

\subsubsection* {Constructors}

\begin{code}
   public LognormalSpecialGen (NormalGen g) \begin{hide} {
      // Necessary to compile, but we do not want to use stream and dist
      super (g.stream, null);
      stream = null;
      myGen = g;
   }\end{hide}
\end{code}
 \begin{tabb}  Create a lognormal random variate generator 
   using the normal generator \texttt{g} and with the same parameters.
 \end{tabb}

%%%%%%%%%%%%%%%%%%%%%%%%%%%%%%%%%%%%%%%%%%%%%%%%5
% \subsubsection* {Methods}
\begin{code}\begin{hide} 

   public double nextDouble() {
      return Math.exp (myGen.nextDouble());
   }\end{hide}
\begin{hide}
}\end{hide}
\end{code}
