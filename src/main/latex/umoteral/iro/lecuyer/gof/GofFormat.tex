\defclass {GofFormat}

% \textbf{THIS CLASS WILL BE REWRITTEN AND SIMPLIFIED IN THE FUTURE.}
This class contains methods used to format results of GOF
test statistics, or to apply a series of tests
simultaneously and format the results.
It is in fact a translation from C to Java of a set of functions that
were specially written for the implementation of TestU01, a software
package for testing uniform random number generators \cite{iLEC01t}.

Strictly speaking, applying several tests simultaneously makes the
$p$-values ``invalid'' in the sense that the probability of having
{\em at least one\/} $p$-value less than 0.01, say, is larger than 0.01.
One must therefore be careful with the interpretation of these
$p$-values (one could use, e.g., the Bonferroni inequality \cite{sLAW00a}).
Applying simultaneous tests is convenient in some situations, such as in
screening experiments for detecting statistical deficiencies
in random number generators.  In that context, rejection of the null
hypothesis typically occurs with extremely small $p$-values (e.g., less
than $10^{-15}$), and the interpretation is quite obvious in this case.

The class also provides tools to plot an empirical or
theoretical distribution function, by creating a data file that
contains a graphic plot in a format compatible with the software
specified by the environment variable \method{graphSoft}{}.
  NOTE: see also the more recent package
  \externalclass{umontreal.iro.lecuyer}{charts}.

Note: This class uses the Colt library.

\bigskip\hrule
\begin{code}
\begin{hide}
/*
 * Class:        GofFormat
 * Description:  
 * Environment:  Java
 * Software:     SSJ 
 * Copyright (C) 2001  Pierre L'Ecuyer and Université de Montréal
 * Organization: DIRO, Université de Montréal
 * @author       
 * @since

 * SSJ is free software: you can redistribute it and/or modify it under
 * the terms of the GNU General Public License (GPL) as published by the
 * Free Software Foundation, either version 3 of the License, or
 * any later version.

 * SSJ is distributed in the hope that it will be useful,
 * but WITHOUT ANY WARRANTY; without even the implied warranty of
 * MERCHANTABILITY or FITNESS FOR A PARTICULAR PURPOSE.  See the
 * GNU General Public License for more details.

 * A copy of the GNU General Public License is available at
   <a href="http://www.gnu.org/licenses">GPL licence site</a>.
 */
\end{hide}
package umontreal.iro.lecuyer.gof;
   import cern.colt.list.*;
\begin{hide}
import umontreal.iro.lecuyer.util.PrintfFormat;
import umontreal.iro.lecuyer.probdist.*;
import java.io.PrintWriter;\end{hide}

public class GofFormat\begin{hide} {
   private GofFormat() {} \end{hide}
\end{code}

%%%%%%%%%%%%%%%%%%%%%%%%%%%
\subsubsection*{Plotting distribution functions}

\begin{code}

   public static final int GNUPLOT\begin{hide} = 0; \end{hide}
\end{code}
 \begin{tabb}  Data file format used for plotting functions with Gnuplot.
 \end{tabb}
\begin{code}

   public static final int MATHEMATICA\begin{hide} = 1; \end{hide}
\end{code}
 \begin{tabb}  Data file format used for creating graphics with Mathematica.
 \end{tabb}
\begin{code}

   public static int graphSoft = GNUPLOT;
\end{code}
 \begin{tabb} Environment variable that selects the type of software to be
   used for plotting the graphs of functions.
   The data files produced by \method{graphFunc}{} and
   \method{graphDistUnif}{} will be in a format suitable
   for this selected software.
   The default value is \texttt{GNUPLOT}.
   To display a graphic in file \texttt{f} using \texttt{gnuplot}, for example,
   one can use the command ``\texttt{plot f with steps, x with lines}''
   in \texttt{gnuplot}.
\begin{htmlonly}
  \texttt{graphSoft} can take the values \method{GNUPLOT}{} or \method{MATHEMATICA}{}.
\end{htmlonly}
 \end{tabb}
\begin{code}
\begin{hide}
   private static String formatMath2 (double x, double y)    {
      // Writes the pair (x, y) in file f, in a format understood
      // by Mathematica
      StringBuffer sb = new StringBuffer();
      String S;

      sb.append ("   { ");
      if ((x != 0.0) && (x < 0.1 || x > 1.0)) {
         S = PrintfFormat.E (16, 7, x);
         int exppos = S.indexOf ('E');
         if (exppos != -1)
            S = S.substring (0, exppos) + "*10^(" +
                             S.substring (exppos+1) + ")";
      }
      else
         S = PrintfFormat.g (16, 8, x);

      sb.append (S + ",     ");

      if (y != 0.0 && (y < 0.1 || y > 1.0)) {
         S = PrintfFormat.E (16, 7, y);
         int exppos = S.indexOf ('E');
         if (exppos != -1)
            S = S.substring (0, exppos) + "*10^(" +
                             S.substring (exppos+1) + ")";
      }
      else
        S = PrintfFormat.g (16, 8, y);

      sb.append (S + " }");
      return sb.toString();
   }


   private static String graphFunc (ContinuousDistribution dist, double a,
                                    double b, int m, int mono, String desc) {
// Renommer drawCDF en fixant mono = 1 et éliminant mono.
      int i;
      double yprec, y, x, h;
      StringBuffer sb = new StringBuffer();
      String openComment = "";
      String closeComment = "";
      String openGraph = "";
      String closeGraph = "";
      if (mono != 1 && mono != -1)
         throw new IllegalArgumentException ("mono must be 1 or -1");
      switch (graphSoft) {
      case GNUPLOT:
        openComment = "# ";
        closeComment = "";
        openGraph = "";
        closeGraph = PrintfFormat.NEWLINE;
        break;
      case MATHEMATICA:
        openComment = "(* ";
        closeComment = " *)";
        openGraph = "points = { " + PrintfFormat.NEWLINE;
        closeGraph = "}" + PrintfFormat.NEWLINE;
        break;
      }

      sb.append (openComment + "----------------------------------" +
                   closeComment  + PrintfFormat.NEWLINE);
      sb.append (openComment + PrintfFormat.s (-70, desc)
                 + closeComment  + PrintfFormat.NEWLINE +
                   PrintfFormat.NEWLINE);

      sb.append (openGraph);
      h = (b - a) / m;
      if (mono == 1)
         yprec = -Double.MAX_VALUE;
      else if (mono == -1)
         yprec = Double.MAX_VALUE;
      else
         yprec = 0.0;

      for (i = 0; i <= m; i++) {
         x = a + i*h;
         y = mono == 1 ? dist.cdf (x) : dist.barF (x);
         switch (graphSoft) {
         case MATHEMATICA:
            sb.append (formatMath2 (x, y));
            if (i < m)
               sb.append (',');
            break;
         default: // Default and GNUPLOT
            sb.append (PrintfFormat.g (20, 14, x) +  "      " +
                       PrintfFormat.g (20, 14, y));
         }

         switch (mono) {
         case 1:
            if (y < yprec)
               sb.append ("    " + openComment +
                    "  DECREASING" + closeComment);
            break;
         case -1:
            if (y > yprec)
               sb.append ("    " + openComment +
                    "  INCREASING" + closeComment);
            break;
         default:
            break;
         }
         sb.append (PrintfFormat.NEWLINE);
         yprec = y;
      }
      sb.append (closeGraph);
      return sb.toString();
   }\end{hide}
\end{code}
%  \begin{tabb}
%   Use \method{drawCdf}{} instead.
%   Formats data to plot the graph of the distribution function $F$ (or $\bar F$)
%   over the interval $[a,b]$, and returns the result as a \class{String}.
%   \texttt{dist.cdf(x)} (or \texttt{dist.barF(x)}) returns the value of $F$
%   (or $\bar F$) at $x$,
%   and that $F$ is either non-decreasing or non-increasing.
%   If \texttt{mono} = 1, the method will verify that $F$ is non-decreasing;
%   if \texttt{mono} = $-1$, it will verify that $\bar{F}$ is non-increasing.
%   (This is useful to verify if $F$ is effectively a sensible
%   approximation to a distribution function or its complementary
%   in the given interval.)
%   The \class{String} \texttt{desc} gives a short caption for the graphic plot.
%   The method computes the $m+1$ points $(x_i,\, F (x_i))$,
%   where $x_i = a + i (b-a)/m$ for $i=0,1,\ldots,m$, and formats these points
%   into a \texttt{String} in a format suitable for the
%   software specified by \method{graphSoft}{}.
%  \end{tabb}
% \begin{htmlonly}
%    \param{dist}{continuous distribution function to plot}
%    \param{a}{lower bound of the interval to plot}
%    \param{b}{upper bound of the interval to plot}
%    \param{m}{number of points in the plot minus one}
%    \param{mono}{1 for plotting a distribution function, -1 if for a complementary
%        distribution function}
%    \param{desc}{short caption describing the plot}
%    \return{a string representation of the plot data}
% \end{htmlonly}
\begin{code}

   public static String drawCdf (ContinuousDistribution dist, double a,
                                 double b, int m, String desc)\begin{hide} {
      return graphFunc (dist, a, b, m, 1, desc);
   }\end{hide}
\end{code}
 \begin{tabb}
  Formats data to plot the graph of the distribution function $F$ over the
  interval $[a,b]$, and returns the result as a \class{String}.
  The method \texttt{dist.cdf(x)} returns the value of $F$ at $x$.
% Checks that $F$ is non-decreasing.
%   If \texttt{mono} = 1, the method will verify that $F$ is non-decreasing;
%   (This is useful to verify if $F$ is effectively a sensible
%   approximation to a distribution function in the given interval.)
  The \class{String} \texttt{desc} gives a short caption for the graphic plot.
  The method computes the $m+1$ points $(x_i,\, F (x_i))$,
  where $x_i = a + i (b-a)/m$ for $i=0,1,\ldots,m$, and formats these points
  into a \texttt{String} in a format suitable for the
  software specified by \method{graphSoft}{}.
  NOTE: see also the more recent class
  \externalclass{umontreal.iro.lecuyer.charts}{ContinuousDistChart}.
 \end{tabb}
\begin{htmlonly}
   \param{dist}{continuous distribution function to plot}
   \param{a}{lower bound of the interval to plot}
   \param{b}{upper bound of the interval to plot}
   \param{m}{number of points in the plot minus one}
   \param{desc}{short caption describing the plot}
   \return{a string representation of the plot data}
\end{htmlonly}
\begin{code}

   public static String drawDensity (ContinuousDistribution dist, double a,
                                     double b, int m, String desc)\begin{hide} {
      int i;
      double y, x, h;
      StringBuffer sb = new StringBuffer();
      String openComment = "";
      String closeComment = "";
      String openGraph = "";
      String closeGraph = "";

      switch (graphSoft) {
      case GNUPLOT:
        openComment = "# ";
        closeComment = "";
        openGraph = "";
        closeGraph = PrintfFormat.NEWLINE;
        break;
      case MATHEMATICA:
        openComment = "(* ";
        closeComment = " *)";
        openGraph = "points = { " + PrintfFormat.NEWLINE;
        closeGraph = "}" + PrintfFormat.NEWLINE;
        break;
      }

      sb.append (openComment + "----------------------------------" +
                   closeComment  + PrintfFormat.NEWLINE);
      sb.append (openComment + PrintfFormat.s (-70, desc)
                    + closeComment  + PrintfFormat.NEWLINE +
                      PrintfFormat.NEWLINE);

      sb.append (openGraph);
      h = (b - a) / m;

      for (i = 0; i <= m; i++) {
         x = a + i*h;
         y = dist.density (x);

         switch (graphSoft) {
         case MATHEMATICA:
            sb.append (formatMath2 (x, y));
            if (i < m)
               sb.append (',');
            break;
         default: // Default and GNUPLOT
            sb.append (PrintfFormat.g (16, 8, x) +  "      " +
                       PrintfFormat.g (16, 8, y));
         }
         sb.append (PrintfFormat.NEWLINE);
      }
      sb.append (closeGraph);
      return sb.toString();
   }\end{hide}
\end{code}
 \begin{tabb}
  Formats data to plot the graph of the density $f(x)$ over the interval $[a,b]$,
  and returns the result as a \class{String}. The method
  \texttt{dist.density(x)} returns the value of $f(x)$ at $x$.
  The \class{String} \texttt{desc} gives a short caption for the graphic
  plot. The method computes the $m+1$ points $(x_i,\, f(x_i))$,
  where $x_i = a + i (b-a)/m$ for $i=0,1,\ldots,m$, and formats these points
  into a \texttt{String} in a format suitable for the
  software specified by \method{graphSoft}{}.
  NOTE: see also the more recent class
  \externalclass{umontreal.iro.lecuyer.charts}{ContinuousDistChart}.
 \end{tabb}
\begin{htmlonly}
   \param{dist}{continuous density function to plot}
   \param{a}{lower bound of the interval to plot}
   \param{b}{upper bound of the interval to plot}
   \param{m}{number of points in the plot minus one}
   \param{desc}{short caption describing the plot}
   \return{a string representation of the plot data}
\end{htmlonly}
\begin{code}

   public static String graphDistUnif (DoubleArrayList data, String desc)\begin{hide} {
      double[] u = data.elements();
      int n = data.size();
      int i;
      double unSurN = 1.0/n;
      StringBuffer sb = new StringBuffer();

      switch (graphSoft) {
      case GNUPLOT:
         sb.append ("#----------------------------------" +
                     PrintfFormat.NEWLINE);
         sb.append ("# " + PrintfFormat.s (-70, desc) +
                     PrintfFormat.NEWLINE + PrintfFormat.NEWLINE);
         sb.append (PrintfFormat.g (16, 8, 0.0) + "  " +
                    PrintfFormat.g (16, 8, 0.0) + PrintfFormat.NEWLINE);
         for (i = 0; i < n; i++)
            sb.append (PrintfFormat.g (16, 8, u[i]) + "  " +
                       PrintfFormat.g (16, 8, (i + 1)*unSurN) +
                       PrintfFormat.NEWLINE);

         sb.append (PrintfFormat.g (16, 8, 1.0) + "  " +
                    PrintfFormat.g (16, 8, 1.0) + PrintfFormat.NEWLINE +
                    PrintfFormat.NEWLINE);
         break;
      case MATHEMATICA:
         sb.append ("(*----------------------------------*)" +
                     PrintfFormat.NEWLINE);
         sb.append ("(* " + PrintfFormat.s (-70, desc)  +
                     PrintfFormat.NEWLINE + " *)" +
                     PrintfFormat.NEWLINE + PrintfFormat.NEWLINE +
                     "points = { " + PrintfFormat.NEWLINE);

         sb.append (formatMath2 (0.0, 0.0) + "," + PrintfFormat.NEWLINE);
         for (i = 0; i < n; i++)
            sb.append (formatMath2 (u[i], (i + 1)*unSurN) + "," +
                       PrintfFormat.NEWLINE);
         sb.append (formatMath2 (1.0, 1.0) + PrintfFormat.NEWLINE);
         break;
      default:
         throw new IllegalArgumentException ("graphSoft unknown");
      }
      return sb.toString();
   }\end{hide}
\end{code}
 \begin{tabb}  Formats data to plot the empirical distribution of
  $U_{(1)},\dots,U_{(N)}$, which are assumed to be in \texttt{data[0...N-1]},
  and to compare it with the uniform distribution. The $U_{(i)}$ must be sorted.
  The two endpoints $(0, 0)$ and $(1, 1)$ are always included in the plot.
  The string \texttt{desc} gives a short caption for the graphic plot.
  The data is printed in a format suitable for the
  software specified by \method{graphSoft}{}.
   NOTE: see also the more recent class
  \externalclass{umontreal.iro.lecuyer.charts}{EmpiricalChart}.
 \end{tabb}
\begin{htmlonly}
   \param{data}{array of observations to plot}
   \param{desc}{short caption describing the plot}
   \return{a string representation of the plot data}
\end{htmlonly}
\begin{code}
\end{code}

%%%%%%%%%%%%%%%%%%%%%%%%%%%%%%%%%%%%%%%%%%
\subsubsection*{Computing and printing $p$-values for EDF test statistics}

\begin{code}

   public static double EPSILONP = 1.0E-15;
\end{code}
 \begin{tabb}  Environment variable used in \method{formatp0}{} to determine
   which $p$-values are too close to 0 or 1 to be printed explicitly.
   If \texttt{EPSILONP} $= \epsilon$, then any $p$-value
   (or significance level) less than $\epsilon$ or larger than
   $1-\epsilon$ is {\em not\/} written explicitly;
   the program simply writes ``\texttt{eps}'' or ``\texttt{1-eps}''.
   The default value is $10^{-15}$.
 \hpierre{Nouveau.  Utilis\'e seulement dans \texttt{formatp0}.}
 \end{tabb}
\begin{code}

   public static double SUSPECTP = 0.01;
\end{code}
 \begin{tabb}  Environment variable used in \method{formatp1}{} to determine
   which $p$-values should be marked as suspect when printing test results.
   If \texttt{SUSPECTP} $= \alpha$, then any $p$-value
   (or significance level) less than $\alpha$ or larger than
   $1-\alpha$ is considered suspect and is
   ``singled out'' by \texttt{formatp1}.
   The default value is 0.01.
 \hpierre{Utilis\'e seulement dans \texttt{formatp1}, mais
   on ne veut pas la mettre en parametre car utilise souvent et ne
   change presque jamais. }
 \end{tabb}
\begin{code}

   public static String formatp0 (double p)\begin{hide} {
      // Formats the significance level of a test, without a descriptor
      if ((p >= 0.01) && (p <= 0.99))
         return PrintfFormat.format (8, 2, 1, p);
      else if (p < EPSILONP)
         return "   eps  ";
      else if (p < 0.01)
         return PrintfFormat.format (8, 2, 2, p);
      else if (p >= 1.0 - EPSILONP)
         return " 1 - eps ";
      else
         return " 1 - " + PrintfFormat.g (8, 2, 1.0 - p);
   }\end{hide}
\end{code}
\begin{tabb}  Returns the significance level (or $p$-value) $p$ of a test,
   in the format ``$1-p$'' if $p$ is close to 1, and $p$ otherwise.
   Uses the environment variable  \method{EPSILONP}{} and replaces $p$
   by $\epsilon$ when it is too small.
 \hrichard {Pourquoi ne pas la cacher dans \tt gofw.c?}
 \hpierre {Quelqu'un pourrait vouloir utiliser un autre format que
   celui de \texttt{formatp1}. }
\end{tabb}
\begin{htmlonly}
   \param{p}{the $p$-value or significance level to be formated}
   \return{the string representation of the $p$-value}
\end{htmlonly}
\begin{code}

   public static String formatp1 (double p)\begin{hide} {
      // Prints the significance level of a test, with a descriptor.
      StringBuffer sb = new StringBuffer();
      sb.append ("Significance level of test            :" + formatp0 (p));
      if (p < SUSPECTP || p > 1.0 - SUSPECTP)
         sb.append ("    *****");

      sb.append (PrintfFormat.NEWLINE + PrintfFormat.NEWLINE);
      return sb.toString();
   }\end{hide}
\end{code}
\begin{tabb}  Returns the string ``\texttt{Significance level of test : }'',
  then calls \method{formatp0}{} to print $p$, and adds
  the marker ``\texttt{****}'' if $p$ is considered suspect
  (uses the environment variable \texttt{RSUSPECTP} for this).
\end{tabb}
\begin{htmlonly}
   \param{p}{the $p$-value or significance level to be formated}
   \return{the string representation of the significance level of test}
\end{htmlonly}
\begin{code}

   public static String formatp2 (double x, double p)\begin{hide} {
      // Prints the statistic x and its significance level p.
      return PrintfFormat.format (8, 2, 1, x) + PrintfFormat.NEWLINE +
             formatp1 (p);
   }\end{hide}
\end{code}
\begin{tabb}  Returns \texttt{x} on a single line, then go to the next line
   and calls \method{formatp1}{}.
  \hpierre {Peut \^etre \`a cacher?}
\end{tabb}
\begin{htmlonly}
   \param{x}{value of the statistic for which the significance level is formated}
   \param{p}{the $p$-value or significance level to be formated}
   \return{the string representation of the significance level of test}
\end{htmlonly}
\begin{code}

   public static String formatp3 (String testName, double x, double p)\begin{hide} {
      final String SLT = "Significance level of test";
      int l = Math.max (SLT.length(), testName.length());
      PrintfFormat pf = new PrintfFormat();
      pf.append (-l, testName).append (" : ").append (8, 2, 1, x).append
               (PrintfFormat.NEWLINE);
      pf.append (-l, SLT).append (" : ").append (formatp0 (p));
      if (p < SUSPECTP || p > 1.0 - SUSPECTP)
         pf.append ("    *****");
      pf.append (PrintfFormat.NEWLINE + PrintfFormat.NEWLINE);
      return pf.toString();
   }\end{hide}
\end{code}
\begin{tabb}   Formats the test statistic \texttt{x} for a test named \texttt{testName}
   with $p$-value \texttt{p}.  The first line of the returned string contains
   the name of the test and the statistic whereas the second line contains
   its significance level.  The formated values of \texttt{x} and \texttt{p} are
   aligned.
\end{tabb}
\begin{htmlonly}
   \param{testName}{name of the test that was performed}
   \param{x}{value of the test statistic}
   \param{p}{significance level (or $p$-value) of the test}
   \return{the string representation of the test result}
\end{htmlonly}
\begin{code}

   public static String formatChi2 (int k, int d, double chi2)\begin{hide} {
      StringBuffer sb = new StringBuffer();
      sb.append ("Chi2 statistic                        : " +
                  PrintfFormat.format (8, 2, 1, chi2));
      sb.append (PrintfFormat.NEWLINE +
                 "p-value                               : " +
                 formatp0 (GofStat.pDisc
                          (ChiSquareDist.cdf (k - 1, d, chi2),
                           ChiSquareDist.barF (k - 1, d, chi2))));
      sb.append (PrintfFormat.NEWLINE + PrintfFormat.NEWLINE);
      return sb.toString();
   }\end{hide}
\end{code}
\begin{tabb}   Computes the $p$-value of the chi-square statistic
  \texttt{chi2} for a test with \texttt{k} intervals.  Uses $d$ decimal digits
  of precision in the calculations. The result of the
  test is returned as a string.  The $p$-value is computed using
  \externalmethod{}{GofStat}{pDisc}{}.
\end{tabb}
\begin{htmlonly}
   \param{k}{number of subintervals for the chi-square test}
   \param{chi2}{chi-square statistic}
   \return{the string representation of the test result and $p$-value}
\end{htmlonly}
\begin{code}

   public static String formatKS (int n, double dp,
                                  double dm, double d)\begin{hide} {
      // Prints the results of a Kolmogorov-Smirnov test
      return "Kolmogorov-Smirnov+ statistic = D+    :" +
             formatp2 (dp, KolmogorovSmirnovPlusDist.barF (n, dp)) +
             "Kolmogorov-Smirnov- statistic = D-    :" +
             formatp2 (dm, KolmogorovSmirnovPlusDist.barF (n, dm)) +
             "Kolmogorov-Smirnov statistic = D      :" +
             formatp2 (d, KolmogorovSmirnovDistQuick.barF (n, d)) +
                       PrintfFormat.NEWLINE + PrintfFormat.NEWLINE;
   }\end{hide}
\end{code}
\begin{tabb}  Computes the $p$-values of the three Kolmogorov-Smirnov statistics
  $D_N^+$, $D_N^-$, and $D_N$, whose values are in \texttt{dp, dm, d},
  respectively, assuming a sample of size \texttt{n}.
  Then formats these statistics and their $p$-values
  using \method{formatp2}{} for each one.
 \hrichard {Devrait \^etre \'elimin\'e. Cas particulier de EDF avec les
     autres CM, AD, Wat, .... Pourquoi traitement de faveur.}
 \hpierre {Je pr\'ef\`ere laisser ceci pour les gens qui ne veulent pas
   prendre le temps de comprendre l'id\'ee des ``active tests'', que
   j'ai d'ailleurs repouss\'es compl\`etement \`a la fin.
   Les tests de KS sont de loin les plus utilis\'es par les gens
   pour les lois continues.}
\end{tabb}
\begin{htmlonly}
   \param{n}{sample size}
   \param{dp}{value of the $D_N^+$ statistic}
   \param{dm}{value of the $D_N^-$ statistic}
   \param{d}{value of the $D_N$ statistic}
   \return{the string representation of the Kolmogorov-Smirnov statistics and their
     significance levels}
\end{htmlonly}
\begin{code}

   public static String formatKS (DoubleArrayList data,
                                  ContinuousDistribution dist)\begin{hide} {

      double[] v = data.elements();
      int n = data.size();

      DoubleArrayList dataUnif = GofStat.unifTransform (data, dist);
      dataUnif.quickSortFromTo (0, dataUnif.size() - 1);
      double[] ret = GofStat.kolmogorovSmirnov (dataUnif);
      return formatKS (n, ret[0], ret[1], ret[2]);
   }\end{hide}
\end{code}
\begin{tabb} Computes the KS test statistics to compare the
 empirical distribution of the observations in \texttt{data}
 with the theoretical distribution \texttt{dist} and
 formats the results.
\end{tabb}
\begin{htmlonly}
   \param{data}{array of observations to be tested}
   \param{dist}{assumed distribution of the observations}
   \return{the string representation of the Kolmogorov-Smirnov statistics and their
     significance levels}
\end{htmlonly}
\begin{code}

   public static String formatKSJumpOne (int n, double a, double dp)\begin{hide} {
      double d = 1.0 - FDist.kolmogorovSmirnovPlusJumpOne (n, a, dp);

      return PrintfFormat.NEWLINE +
             "Kolmogorov-Smirnov+ statistic = D+    : " +
             PrintfFormat.g (8, 2, dp) + PrintfFormat.NEWLINE +
             formatp1 (d) + PrintfFormat.NEWLINE;
   }\end{hide}
\end{code}
\begin{tabb}  Similar to \method{formatKS}{int,double,double,double},
   but for the KS statistic $D_N^+(a)$\latex{ defined in (\ref{eq:KSPlusJumpOne})}.
   Writes a header,
  computes the $p$-value and calls \method{formatp2}{}.
\end{tabb}
\begin{htmlonly}
   \param{n}{sample size}
   \param{a}{size of the jump}
   \param{dp}{value of $D_N^+(a)$}
   \return{the string representation of the Kolmogorov-Smirnov statistic and its significance level}
\end{htmlonly}
\begin{code}

   public static String formatKSJumpOne (DoubleArrayList data,
                                         ContinuousDistribution dist,
                                         double a)\begin{hide} {

      double[] v = data.elements();
      int n = data.size();
      DoubleArrayList dataUnif = GofStat.unifTransform (data, dist);
      dataUnif.quickSortFromTo (0, dataUnif.size() - 1);
      double[] ret =  GofStat.kolmogorovSmirnovJumpOne (dataUnif, a);
      return formatKSJumpOne (n, a, ret[0]);
   }\end{hide}
\end{code}
\begin{tabb} Similar to \method{formatKS}{DoubleArrayList,ContinuousDistribution},
  but for $D_N^+(a)$\latex{  defined in (\ref{eq:KSPlusJumpOne})}.
\end{tabb}
\begin{htmlonly}
   \param{data}{array of observations to be tested}
   \param{dist}{assumed distribution of the data}
   \param{a}{size of the jump}
   \return{string representation of the Kolmogorov-Smirnov statistic and its significance level}
\end{htmlonly}

%%%%%%%%%%%%%%%%%%%%%%%%%%%%%%%%%%%%%
\subsubsection* {Applying several tests at once and printing results}

Higher-level tools for applying several EDF goodness-of-fit tests
simultaneously are offered here.
The environment variable \texttt{activeTests} specifies which
tests in this list are to be performed when asking for several
simultaneous tests via the functions \texttt{activeTests},
\texttt{formatActiveTests}, etc.

\begin{code}

   public  static final int KSP = 0;
\end{code}
\begin{tabb} Kolmogorov-Smirnov+ test \end{tabb}
\begin{code}

   public static final int KSM = 1;
\end{code}
\begin{tabb}   Kolmogorov-Smirnov$-$ test \end{tabb}
\begin{code}

   public static final int KS = 2;
\end{code}
\begin{tabb}   Kolmogorov-Smirnov test  \end{tabb}
\begin{code}

   public static final int AD = 3;
\end{code}
\begin{tabb}   Anderson-Darling test  \end{tabb}
\begin{code}

   public static final int CM = 4;
\end{code}
\begin{tabb}   Cram\'er-von Mises test  \end{tabb}
\begin{code}

   public static final int WG = 5;
\end{code}
\begin{tabb}   Watson G test  \end{tabb}
\begin{code}

   public static final int WU = 6;
\end{code}
\begin{tabb}   Watson U test   \end{tabb}
\begin{code}

   public static final int MEAN = 7;
\end{code}
\begin{tabb}   Mean  \end{tabb}
\begin{code}

   public static final int COR = 8;
\end{code}
\begin{tabb}   Correlation  \end{tabb}
\begin{code}

   public static final int NTESTTYPES = 9;
\end{code}
\begin{tabb}   Total number of test types  \end{tabb}

\iffalse
 \begin{tabb}  Types of EDF tests supported by the present modules.
  Here, \texttt{MEAN} and \texttt{COR} usually
  represent tests based on the mean of the observations and on the
  correlation between pairs of successive observations.
 \hpierre {Est-ce utile que garder ces deux-l\`a dans ce type enum?}
 \hrichard {N\'ecessaire si l'on veut que les tableaux sVal, pVal,
  r\'eservent aussi une place pour ces 2. Sinon, il faudrait introduire
  au moins 4 nouvelles variables globales pour chaque module.}
%  {\bf Remark:} if \texttt{testType} is modified in any way, then the
%  variable \texttt{TESTNAMES} must be updated correspondingly.
 \end{tabb}
\fi
\begin{code}

   public static final String[] TESTNAMES\begin{hide} = {
    "KolmogorovSmirnovPlus", "KolmogorovSmirnovMinus",
    "KolmogorovSmirnov", "Anderson-Darling",
    "CramerVon-Mises", "Watson G", "Watson U",
    "Mean", "Correlation"
   };\end{hide}
\end{code}
 \begin{tabb}  Name of each \texttt{testType} test.
  Could be used for printing the test results, for example.
%  {\bf Remark:} If the type \texttt{testType}
%  is modified in any way, then \texttt{TESTNAMES} must be updated
%  correspondingly.
 \hpierre{Il me semble que cette liste des noms devrait \^etre juste \`a
   cot\'e de la liste \texttt{testType}.  Est-ce OK comme ceci? }
 \hrichard {On ne peut pas initialiser les variables extern dans *.h.}
 \end{tabb}
\begin{code}

   public static boolean[] activeTests\begin{hide} = null;
   private static void initActiveTests() {
      activeTests = new boolean[NTESTTYPES];
      for (int i = 0; i < activeTests.length; i++)
        activeTests[i] = false;
      activeTests[KSP] = activeTests[KSM] = true;
      activeTests[AD] = activeTests[MEAN] = activeTests[COR] = true;
   }
   static {
      initActiveTests();
   }\end{hide}
\end{code}
 \begin{tabb} The set of EDF tests that are to be performed when calling
  the methods \method{activeTests}{}, \method{formatActiveTests}{}, etc.
  By default, this set contains \texttt{KSP}, \texttt{KSM},
  and \texttt{AD}.  %, but it can be changed.
  Note: \texttt{MEAN} and \texttt{COR} are {\em always excluded\/}
  from this set of active tests.
\begin{htmlonly}
% These constants are not regrouped under an identifying label in Javadoc
   The valid indices for this array are \method{KSP}{}, \method{KSM}{},
      \method{KS}{}, \method{AD}{}, \method{CM}{}, \method{WG}{},
      \method{WU}{}, \method{MEAN}{}, and \method{COR}{}.
\end{htmlonly}
 \end{tabb}
\begin{code}

   public static void tests (DoubleArrayList sortedData, double[] sVal)\begin{hide} {
      double[] u = sortedData.elements();
      int n = sortedData.size();
      int i;
      double a2 = 0.0, w2, dm = 0.0, dp = 0.0, w;
      double u1, ui, d2, d1;
      double sumZ;
      double unSurN;

      if (n <= 0)
        throw new IllegalArgumentException ("n <= 0");
      if (sVal.length != NTESTTYPES)
        throw new IllegalArgumentException ("sVal must " +
                              "be of size NTESTTYPES.");

      // We assume that u is already sorted.
      if (n == 1) {
         sVal[KSP] = 1.0 - u[0];
         sVal[MEAN] = u[0];
         return;
      }
      unSurN = 1.0 / n;
      w2 = unSurN / 12.0;
      sumZ = 0.0;
      for (i = 0; i < n; i++) {
         // Statistics KS
         d1 = u[i] - i*unSurN;
         d2 = (i + 1)*unSurN - u[i];
         if (d1 > dm)
            dm = d1;
         if (d2 > dp)
            dp = d2;
         // Watson U and G
         sumZ += u[i];
         w = u[i] - (i + 0.5)*unSurN;
         w2 += w*w;
         // Anderson-Darling
         ui = u[i];
         u1 = 1.0 - ui;
         if (ui < GofStat.EPSILONAD)
            ui = GofStat.EPSILONAD;
         else if (u1 < GofStat.EPSILONAD)
            u1 = GofStat.EPSILONAD;
         a2 += (2*i + 1) * Math.log (ui) + (1 + 2*(n - i - 1))*Math.log (u1);
      }
      if (dm > dp)
         sVal[KS] = dm;
      else
         sVal[KS] = dp;
      sVal[KSM] = dm;
      sVal[KSP] = dp;
      sumZ = sumZ * unSurN - 0.5;
      sVal[CM] = w2;
      sVal[WG] = Math.sqrt ((double) n) * (dp + sumZ);
      sVal[WU] = w2 - sumZ * sumZ * n;
      sVal[AD] = -n - a2 * unSurN;
      sVal[MEAN] = sumZ + 0.5;  // Nouveau ...
   }\end{hide}
\end{code}
\begin{tabb} Computes all EDF test statistics\latex{ enumerated above
 (except \texttt{COR})}
  to compare the empirical
  distribution of  $U_{(0)},\dots,U_{(N-1)}$ with the uniform distribution,
  assuming that these sorted observations are in \texttt{sortedData}.
  If $N > 1$, returns \texttt{sVal} with the values of the KS
  statistics $D_N^+$, $D_N^-$ and $D_N$, of the Cram\'er-von Mises
  statistic $W_N^2$, Watson's $G_N$ and $U_N^2$, Anderson-Darling's
  $A_N^2$, and the average of the $U_i$'s, respectively.
  If $N = 1$, only puts $1 - {}$\texttt{sortedData.get (0)} in \texttt{sVal[KSP]}.
  Calling this method is more efficient than computing these statistics
  separately by calling the corresponding methods in \class{GofStat}.
 \hrichard{\'Eliminer, garder seulement Active. On devrait \'eliminer
  les fonctions qui ont une version active, et enlever le mot Active dans
  leur nom. Les actives sont d\'etermin\'ees
  par une variable d'environnement; donc toujours sous-entendu actives.
  Les non-actives ne sont jamais utilis\'ees.}
 \hpierre {On ne peut pas les \'eliminer car les \texttt{Active...} appellent
  celles-ci pour calculer toutes les stats d'un seul coup.
  Si on trouve qu'elles genent, on peut les cacher, mais je ne crois pas.
  Devrait peut-etre aller dans gofs, mais je l'ai mis
  ici pour eviter de definir \texttt{testArray}  dans GofStat. }
\end{tabb}
\begin{htmlonly}
   \param{sortedData}{array of sorted observations}
   \param{sVal}{array that will be filled with the results of the tests}
\end{htmlonly}
\begin{code}

   public static void tests (DoubleArrayList data,
                             ContinuousDistribution dist, double[] sVal)\begin{hide} {

      double[] v = data.elements();
      int n = data.size();

      if (n <= 0)
        throw new IllegalArgumentException ("n <= 0");

      DoubleArrayList sortedData = GofStat.unifTransform (data, dist);
      sortedData.quickSortFromTo (0, sortedData.size()-1);
      tests (sortedData, sVal);
      if (n == 1)
         sVal[MEAN] = v[0];     // On veut v[0], pas u[0].
   }\end{hide}
\end{code}
\begin{tabb} The observations $V$ are in \texttt{data},
  not necessarily sorted, and their empirical
  distribution is compared with the continuous distribution \texttt{dist}.
 \hpierre{ Note: if $N > 1$, the value returned in \texttt{sVal[MEAN]}
    is the average of the \texttt{U[i] = dist.cdf (V[i])}, not
    the average of the \texttt{V[i]}. }
% The EDF tests (p-values) are valid only if \texttt{F} is continuous.
 If $N = 1$, only puts \texttt{data.get (0)} in \texttt{sVal[MEAN]},
 and $1 - {}$\texttt{dist.cdf (data.get (0))} in \texttt{sVal[KSP]}.
\end{tabb}
\begin{htmlonly}
   \param{data}{array of observations to test}
   \param{dist}{assumed distribution of the observations}
   \param{sVal}{array that will be filled with the results of the tests}
\end{htmlonly}
\begin{code}

   public static void activeTests (DoubleArrayList sortedData,
                                   double[] sVal, double[] pVal)\begin{hide} {

      double[] u = sortedData.elements();
      int n = sortedData.size();

      if (n <= 0)
        throw new IllegalArgumentException ("n <= 0");

      if (sVal.length != NTESTTYPES || pVal.length != NTESTTYPES)
        throw new IllegalArgumentException ("sVal and pVal must " +
              "be of length NTESTTYPES.");

      if (n == 1) {
         sVal[KSP] = 1.0 - u[0];
         pVal[KSP] = 1.0 - u[0];
         pVal[MEAN] = pVal[KSP];
         return;
      }
      // We assume that u is already sorted.
      tests (sortedData, sVal);

      if (activeTests.length != NTESTTYPES) {
        initActiveTests();
        System.err.println ("activeTests was invalid, it was reinitialized.");
      }

      if (activeTests[KSP])
         pVal[KSP] = KolmogorovSmirnovPlusDist.barF (n, sVal[KSP]);

      if (activeTests[KSM])
         pVal[KSM] = KolmogorovSmirnovPlusDist.barF (n, sVal[KSM]);

      if (activeTests[KS])
         pVal[KS] = KolmogorovSmirnovDistQuick.barF (n, sVal[KS]);

      if (activeTests[AD])
         pVal[AD] = AndersonDarlingDistQuick.barF (n, sVal[AD]);

      if (activeTests[CM])
         pVal[CM] = CramerVonMisesDist.barF (n, sVal[CM]);

      if (activeTests[WG])
         pVal[WG] = WatsonGDist.barF (n, sVal[WG]);

      if (activeTests[WU])
         pVal[WU] = WatsonUDist.barF (n, sVal[WU]);
   }\end{hide}
\end{code}
\begin{tabb} Computes the EDF test statistics by calling
  \method{tests}{DoubleArrayList,double[]}, then computes the $p$-values of those
  that currently belong to \texttt{activeTests},
%   \{\texttt{KSP, KSM, KS0, CM,
%  WG, WU, AD}\}.
  and return these quantities in \texttt{sVal} and \texttt{pVal}, respectively.
  Assumes that $U_{(0)},\dots,U_{(N-1)}$ are in \texttt{sortedData}
  and that we want to compare their empirical distribution
  with the uniform distribution.
  If $N = 1$, only puts $1 - {}$\texttt{sortedData.get (0)} in
  \texttt{sVal[KSP], pVal[KSP]}, and \texttt{pVal[MEAN]}.
\end{tabb}
\begin{htmlonly}
   \param{sortedData}{array of sorted observations}
   \param{sVal}{array that will be filled with the results of the tests}
   \param{pVal}{array that will be filled with the $p$-values}
\end{htmlonly}
\begin{code}

   public static void activeTests (DoubleArrayList data,
                                   ContinuousDistribution dist,
                                   double[] sVal, double[] pVal)\begin{hide} {
      double[] v = data.elements();
      int n = data.size();

      if (n <= 0)
        throw new IllegalArgumentException ("n <= 0");

      DoubleArrayList sortedData = GofStat.unifTransform (data, dist);
      sortedData.quickSortFromTo (0, sortedData.size() - 1);

      activeTests (sortedData, sVal, pVal);
      if (n == 1)
         sVal[MEAN] = v[0];
   }\end{hide}
\end{code}
\begin{tabb} The observations are in \texttt{data},
 not necessarily sorted, and we want to compare their empirical
 distribution with the distribution \texttt{dist}.
 If $N = 1$, only puts \texttt{data.get(0)} in \texttt{sVal[MEAN]},
 and $1 - {}$\texttt{dist.cdf (data.get (0))} in \texttt{sVal[KSP], pVal[KSP]},
 and \texttt{pVal[MEAN]}.
 \hrichard {Utilis\'e dans tous les tests de testu01.}
 \end{tabb}
\begin{htmlonly}
   \param{data}{array of observations to test}
   \param{dist}{assumed distribution of the observations}
   \param{sVal}{array that will be filled with the results of the tests}
   \param{pVal}{array that will be filled with the $p$-values}
\end{htmlonly}
\begin{code}

   public static String formatActiveTests (int n, double[] sVal,
                                           double[] pVal)\begin{hide} {

      if (activeTests.length != NTESTTYPES) {
        initActiveTests();
        System.err.println ("activeTests was invalid, it was reinitialized.");
      }
      if (sVal.length != NTESTTYPES || pVal.length != NTESTTYPES)
        throw new IllegalArgumentException ("The length of " +
           "sVal and pVal must be NTESTTYPES.");
      if (n == 1)
         return formatp1 (pVal[KSP]);;

      StringBuffer sb = new StringBuffer (PrintfFormat.NEWLINE);
      if (activeTests[KSP])
         sb.append ("Kolmogorov-Smirnov+ statistic = D+    :" +
           formatp2 (sVal[KSP], pVal[KSP]));
      if (activeTests[KSM])
         sb.append ("Kolmogorov-Smirnov- statistic = D-    :" +
           formatp2 (sVal[KSM], pVal[KSM]));
      if (activeTests[KS])
         sb.append ("Kolmogorov-Smirnov statistic  = D     :" +
           formatp2 (sVal[KS], pVal[KS]));
      if (activeTests[AD])
         sb.append ("Anderson-Darling statistic = A2       :" +
           formatp2 (sVal[AD], pVal[AD]));
      if (activeTests[CM])
         sb.append ("Cramer-von Mises statistic = W2       :" +
           formatp2 (sVal[CM], pVal[CM]));
      if (activeTests[WG])
         sb.append ("Watson statistic = G                  :" +
           formatp2 (sVal[WG], pVal[WG]));
      if (activeTests[WU])
         sb.append ("Watson statistic = U2                 :" +
           formatp2 (sVal[WU], pVal[WU]));
      sb.append (PrintfFormat.NEWLINE);
      return sb.toString();
   }\end{hide}
\end{code}
\begin{tabb} Gets the $p$-values of the {\em active\/} EDF test statistics,
  which are in \texttt{activeTests}.  It is assumed that the values
  of these statistics and their $p$-values are {\em already computed},
  in \texttt{sVal} and \texttt{pVal}, and that the sample size is \texttt{n}.
  These statistics and $p$-values are formated
  using \method{formatp2}{} for each one.
  If \texttt{n=1}, prints only \texttt{pVal[KSP]} using \method{formatp1}{}.
 \hrichard{Utilis\'e souvent dans testu01. ..Test1 jamais utilis\'e.}
 \hpierre{Je pr\'ef\`ere conserver ...Tests quand meme.}
% \{\texttt{StatCalc.KSP, StatCalc.KSM, StatCalc.KS,
%  StatCalc.watsonG, StatCalc.watsonU, Statcalc.cramerVonMises,
%  StatCalc.andDar}\} and \texttt{StatCalc.testActives}
\end{tabb}
\begin{htmlonly}
   \param{n}{sample size}
   \param{sVal}{array containing the results of the tests}
   \param{pVal}{array containing the $p$-values}
   \return{the results formated as a string}
\end{htmlonly}
\begin{code}

   public static String iterSpacingsTests (DoubleArrayList sortedData, int k,
                                           boolean printval, boolean graph,
                                           PrintWriter f)\begin{hide} {

      int n = sortedData.size();

      DoubleArrayList sortedDataCopy = (DoubleArrayList)sortedData.clone();
      DoubleArrayList diffArrayList = new DoubleArrayList(sortedData.size()+2);

      int j;
      int i;
      double[] sVal = new double[NTESTTYPES], pVal = new double[NTESTTYPES];

      StringBuffer sb = new StringBuffer (PrintfFormat.NEWLINE);

      for (j = 1; j <= k; j++) {
         sb.append ("-----------------------------------" +
                     PrintfFormat.NEWLINE +
                     "EDF Tests after \"iterateSpacings\", level : " +
                     PrintfFormat.d (2, j) + PrintfFormat.NEWLINE);

         GofStat.diff (sortedDataCopy, diffArrayList, 0, n - 1, 0.0, 1.0);
         GofStat.iterateSpacings (sortedDataCopy, diffArrayList);
         sortedDataCopy.quickSortFromTo (0, sortedDataCopy.size() - 1);
         activeTests (sortedDataCopy, sVal, pVal);

         sb.append (formatActiveTests (n, sVal, pVal));
         String desc = "Values of Uniforms after iterateSpacings, level " +
             PrintfFormat.d (2, j);
         if (printval) {
          sb.append (desc + PrintfFormat.NEWLINE +
                     "------------------------" + PrintfFormat.NEWLINE);
          sb.append (sortedDataCopy + PrintfFormat.NEWLINE);
         }
         if (graph && f != null)
          f.print (graphDistUnif (sortedDataCopy, desc));
         else if (graph && f == null)
          sb.append (graphDistUnif (sortedDataCopy, desc));
       }
       return sb.toString();
   }\end{hide}
\end{code}
\begin{tabb} Repeats the following \texttt{k} times:
  Applies the \clsexternalmethod{}{GofStat}{iterateSpacings}{}
  transformation to the
  $U_{(0)},\dots,U_{(N-1)}$, assuming that these observations are in
  \texttt{sortedData}, then computes the EDF test statistics and calls
  \method{activeTests}{DoubleArrayList,double[],double[]} after each transformation.
  The function returns the {\em original\/} array \texttt{sortedData} (the
  transformations are applied on a copy of \texttt{sortedData}).
%  If \texttt{statsDetail = true}, prints all the values.
%  If \texttt{graphicsDetail = true}, prints  to \texttt{PrintWriter f}
%  the values appropriate for drawing graphics.
  If \texttt{printval = true}, stores all the values into the returned
  \class{String} after each iteration.
  If \texttt{graph = true}, calls \method{graphDistUnif}{} after each iteration
  to print to stream \texttt{f} the data for plotting the distribution
  function of the $U_i$.
\end{tabb}
\begin{htmlonly}
   \param{sortedData}{array containing the sorted observations}
   \param{k}{number of times the tests are applied}
   \param{printval}{if \texttt{true}, stores all the values of the observations at each iteration}
   \param{graph}{if \texttt{true}, the distribution of the $U_i$ will be plotted after each
     iteration}
   \param{f}{stream where the plots are written to}
   \return{a string representation of the test results}
\end{htmlonly}
\begin{code}

   public static String iterPowRatioTests (DoubleArrayList sortedData, int k,
                                           boolean printval, boolean graph,
                                           PrintWriter f)\begin{hide} {

      int n = sortedData.size();
      DoubleArrayList sortedDataCopy = (DoubleArrayList)sortedData.clone();

      int i;
      int j;
      double[] sVal = new double[NTESTTYPES], pVal = new double[NTESTTYPES];

      StringBuffer sb = new StringBuffer (PrintfFormat.NEWLINE);

      for (i = 1; i <= k; i++) {
         GofStat.powerRatios (sortedDataCopy);
         sb.append ("-----------------------------------" +
                    PrintfFormat.NEWLINE +
                    "EDF Tests after \"powerRatios\", level : " +
                    PrintfFormat.d (2, i) + PrintfFormat.NEWLINE);

         sortedDataCopy.quickSortFromTo (0, sortedDataCopy.size() - 1);

         activeTests (sortedDataCopy, sVal, pVal);
         sb.append (formatActiveTests (n, sVal, pVal));
         String desc = "Values of Uniforms after PowerRatios, level " +
             PrintfFormat.d (2, i);
         if (printval) {
           sb.append (desc + PrintfFormat.NEWLINE +
                      "--------------------------" + PrintfFormat.NEWLINE);
           sb.append (sortedDataCopy + PrintfFormat.NEWLINE);
         }
         if (graph && f != null)
            f.print (graphDistUnif (sortedDataCopy, desc));
         else if (graph && f == null)
            sb.append (graphDistUnif (sortedDataCopy, desc));
      }
      return sb.toString();
   }
}\end{hide}
\end{code}
\begin{tabb}  Similar to \method{iterSpacingsTests}{}, but with the
  \clsexternalmethod{}{GofStat}{powerRatios}{} transformation.
\end{tabb}
\begin{htmlonly}
   \param{sortedData}{array containing the sorted observations}
   \param{k}{number of times the tests are applied}
   \param{printval}{if \texttt{true}, stores all the values of the observations at each iteration}
   \param{graph}{if \texttt{true}, the distribution of the $U_i$ will be plotted after each
     iteration}
   \param{f}{stream where the plots are written to}
   \return{a string representation of the test results}
\end{htmlonly}
