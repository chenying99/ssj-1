\defclass{MathFunctionUtil}

Provides utility methods for computing
derivatives and integrals of functions.

\bigskip\hrule

\begin{code}
\begin{hide}
/*
 * Class:        MathFunctionUtil
 * Description:  
 * Environment:  Java
 * Software:     SSJ 
 * Copyright (C) 2001  Pierre L'Ecuyer and Université de Montréal
 * Organization: DIRO, Université de Montréal
 * @author       Éric Buist
 * @since

 * SSJ is free software: you can redistribute it and/or modify it under
 * the terms of the GNU General Public License (GPL) as published by the
 * Free Software Foundation, either version 3 of the License, or
 * any later version.

 * SSJ is distributed in the hope that it will be useful,
 * but WITHOUT ANY WARRANTY; without even the implied warranty of
 * MERCHANTABILITY or FITNESS FOR A PARTICULAR PURPOSE.  See the
 * GNU General Public License for more details.

 * A copy of the GNU General Public License is available at
   <a href="http://www.gnu.org/licenses">GPL licence site</a>.
 */
\end{hide}
package umontreal.iro.lecuyer.functions;\begin{hide}

import umontreal.iro.lecuyer.util.Misc;
\end{hide}

public class MathFunctionUtil\begin{hide} {
\end{hide}

   public static double H = 1e-6;
\end{code}
\begin{tabb} Step length in $x$ to compute derivatives. Default: $10^{-6}$.
\end{tabb}
\begin{code}\begin{hide}
   private MathFunctionUtil() {}

   // For Gauss-Lobatto: nodes Cg and weights Wg
   private static final double[] Cg = { 0, 0.17267316464601142812, 0.5,
                                           0.82732683535398857188, 1 };
   private static final double[] Wg = { 0.05, 0.27222222222222222222,
                  0.35555555555555555555, 0.27222222222222222222, 0.05 };


   private static double[] fixBounds (MathFunction func, double a,
                                      double b, int numIntervals) {
   // For functions which are 0 on parts of [a, b], shorten the interval
   // [a, b] to the non-zero part of f(x). Returns the shortened interval.

       final double h = (b - a)/numIntervals;
       double x = b;
       while ((0 == func.evaluate (x)) && (x > a))
           x -= h;
       if (x < b)
           b = x + h;

       x = a;
       while ((0 == func.evaluate (x)) && (x < b))
           x += h;
       if (x > a)
          a = x - h;
       double[] D = {a, b};
       return D;
   }\end{hide}

   public static int NUMINTERVALS = 1024;
\end{code}
\begin{tabb} Default number of intervals for Simpson's integral.
\end{tabb}
\begin{code}

   public static double derivative (MathFunction func, double x)\begin{hide} {
      if (func instanceof MathFunctionWithFirstDerivative)
         return ((MathFunctionWithFirstDerivative)func).derivative (x);
      else if (func instanceof MathFunctionWithDerivative)
         return ((MathFunctionWithDerivative)func).derivative (x, 1);
      else
         return finiteCenteredDifferenceDerivative (func, x, H);
   }\end{hide}
\end{code}
\begin{tabb}   Returns the first derivative of the function \texttt{func}
 evaluated at \texttt{x}. If the given function implements
 \class{MathFunctionWithFirstDerivative}, this method calls
\clsexternalmethod{}{MathFunctionWithFirst\-Derivative}{derivative}{}~\texttt{(double)}.
 Otherwise, if the function implements \class{MathFunction\-WithDerivative},
 this method calls
\clsexternalmethod{}{MathFunctionWithDerivative}{derivative}{}~\texttt{(double, int)}.
If the function does not implement any of these two interfaces, the method
 uses \method{finiteCenteredDifferenceDerivative}{}~\texttt{(MathFunction,
 double, double)} to obtain an estimate of the derivative.
\end{tabb}
\begin{htmlonly}
   \param{func}{the function to derivate.}
   \param{x}{the evaluation point.}
   \return{the first derivative.}
\end{htmlonly}
\begin{code}

   public static double derivative (MathFunction func, double x, int n)\begin{hide} {
      if (n == 0)
         return func.evaluate (x);
      else if (n == 1)
         return derivative (func, x);
      else if (func instanceof MathFunctionWithDerivative)
         return ((MathFunctionWithDerivative)func).derivative (x, n);
      else if (n % 2 == 0)
         return finiteCenteredDifferenceDerivative (func, x, n, H);
      else
         return finiteDifferenceDerivative (func, x, n, H);
   }\end{hide}
\end{code}
\begin{tabb}   Returns the $n$th derivative of function
 \texttt{func} evaluated at \texttt{x}.
 If $n=0$, this returns $f(x)$.
 If $n=1$, this calls
 \method{derivative}{}~\texttt{(MathFunction, double)}
 and returns the resulting first derivative.
 Otherwise,
 if the function implements
 \class{MathFunctionWithDerivative},
 this method calls
\clsexternalmethod{}{MathFunctionWithDerivative}{derivative}{}~\texttt{(double, int)}.
 If the function does not implement this
 interface,
 the method uses
 \method{finiteCenteredDifferenceDerivative}{}~\texttt{(MathFunction, double, int, double)}
 if $n$ is even, or
 \method{finiteDifferenceDerivative}{}~\texttt{(MathFunction, double, int, double)}
 if $n$ is odd,
 to obtain a numerical approximation of the derivative.
\end{tabb}
\begin{htmlonly}
   \param{func}{the function to derivate.}
   \param{x}{the evaluation point.}
   \param{n}{the order of the derivative.}
   \return{the $n$th derivative.}
\end{htmlonly}
\begin{code}

   public static double finiteDifferenceDerivative (
                 MathFunction func, double x, int n, double h)\begin{hide} {
      if (n < 0)
         throw new IllegalArgumentException
         ("n must not be negative");
      if (n == 0)
         return func.evaluate (x);
      final double err = Math.pow (h, 1.0 / n);
      final double[] values = new double[n+1];
      for (int i = 0; i < values.length; i++)
         values[i] = func.evaluate (x + i*err);
      for (int j = 0; j < n; j++) {
         for (int i = 0; i < n - j; i++)
            values[i] = values[i + 1] - values[i];
      }
      return values[0] / h;
   }\end{hide}
\end{code}
\begin{tabb}   Computes and returns an estimate
 of the $n$th derivative of the
 function $f(x)$.
 This method estimates
 \[\frac{d^nf(x)}{dx^n},\]
 the $n$th derivative of $f(x)$
 evaluated at $x$.
 This method first computes
 $f_i=f(x+i\epsilon)$, for $i=0,\ldots,n$, with
 $\epsilon=h^{1/n}$.
 The estimate is then given by
 $\Delta^nf_0/h$, where
 $\Delta^nf_i=\Delta^{n-1}f_{i+1} - \Delta^{n-1}f_i$, and
 $\Delta f_i = f_{i+1} - f_i$.
\end{tabb}
\begin{htmlonly}
   \param{func}{the function to derivate.}
   \param{x}{the evaluation point.}
   \param{n}{the order of the derivative.}
   \param{h}{the error.}
   \return{the estimate of the derivative.}
\end{htmlonly}
\begin{code}

   public static double finiteCenteredDifferenceDerivative (
                 MathFunction func, double x, double h)\begin{hide} {
      final double fplus = func.evaluate (x + h);
      final double fminus = func.evaluate (x - h);
      return (fplus - fminus) / (2*h);
   }\end{hide}
\end{code}
\begin{tabb}   Returns $(f(x + h) - f(x - h))/(2h)$,
 an estimate of the first derivative of $f(x)$
 using centered differences.
\end{tabb}
\begin{htmlonly}
   \param{func}{the function to derivate.}
   \param{x}{the evaluation point.}
   \param{h}{the error.}
   \return{the estimate of the first derivative.}
\end{htmlonly}
\begin{code}

   public static double finiteCenteredDifferenceDerivative (
                 MathFunction func, double x, int n, double h)\begin{hide} {
      if (n < 0)
         throw new IllegalArgumentException
         ("n must not be negative");
      if (n == 0)
         return func.evaluate (x);
      if (n % 2 == 1)
         throw new IllegalArgumentException ("n must be even");
      final double err = Math.pow (h, 1.0 / n);
      return finiteDifferenceDerivative (func, x - n*err / 2, n, h);
   }\end{hide}
\end{code}
\begin{tabb}   Computes and returns an estimate of the $n$th derivative of the
 function $f(x)$ using finite centered differences.
 If $n$ is even, this method returns
\method{finiteDifferenceDerivative}{MathFunction, double, int, double}~\texttt{(func, x - $\epsilon$*n/2, n, h)}, with $h=\epsilon^n$.
\end{tabb}
\begin{htmlonly}
   \param{func}{the function to derivate.}
   \param{x}{the evaluation point.}
   \param{n}{the order of the derivative.}
   \param{h}{the error.}
   \return{the estimate of the derivative.}
\end{htmlonly}
\begin{code}

   public static double[][] removeNaNs (double[] x, double[] y)\begin{hide} {
      if (x.length != y.length)
         throw new IllegalArgumentException();
      int numNaNs = 0;
      for (int i = 0; i < x.length; i++)
         if (Double.isNaN (x[i]) || Double.isNaN (y[i]))
            ++numNaNs;
      if (numNaNs == 0)
         return new double[][] { x, y };
      final double[] nx = new double[x.length - numNaNs];
      final double[] ny = new double[y.length - numNaNs];
      int j = 0;
      for (int i = 0; i < x.length; i++)
         if (!Double.isNaN (x[i]) && !Double.isNaN (y[i])) {
            nx[j] = x[i];
            ny[j++] = y[i];
         }
      return new double[][] { nx, ny };
   }\end{hide}
\end{code}
\begin{tabb}   Removes any point \texttt{(NaN, y)} or \texttt{(x, NaN)} from
 \texttt{x} and \texttt{y}, and returns a 2D array containing the filtered
 points. This method filters each pair (\texttt{x[i]}, \texttt{y[i]})
 containing at least one NaN element. It constructs a 2D array containing
 the two filtered arrays, whose size is smaller than or equal to
 \texttt{x.length}.
\end{tabb}
\begin{htmlonly}
   \param{x}{the $X$ coordinates.}
   \param{y}{the $Y$ coordinates.}
   \return{the filtered $X$ and $Y$ arrays.}
\end{htmlonly}
\begin{code}

   public static double integral (MathFunction func, double a, double b) \begin{hide} {
      if (func instanceof MathFunctionWithIntegral)
         return ((MathFunctionWithIntegral)func).integral (a, b);
      else
         return simpsonIntegral (func, a, b, NUMINTERVALS);
   }\end{hide}
\end{code}
\begin{tabb}   Returns the integral of the function \texttt{func} over $[a, b]$.
 If the given function implements \class{MathFunctionWithIntegral},
 this returns
 \clsexternalmethod{umontreal.iro.lecuyer.functions}{MathFunctionWithIntegral}{integral}{}~\texttt{(double, double)}.
 Otherwise, this calls
 \method{simpsonIntegral}{}~\texttt{(MathFunction, double, double, int)}
 with \method{NUMINTERVALS}{} intervals.
\end{tabb}
\begin{htmlonly}
   \param{func}{the function to integrate.}
   \param{a}{the lower bound.}
   \param{b}{the upper bound.}
   \return{the value of the integral.}
\end{htmlonly}
\begin{code}

   public static double simpsonIntegral (MathFunction func, double a,
                                         double b, int numIntervals) \begin{hide} {
      if (numIntervals % 2 != 0)
         throw new IllegalArgumentException
         ("numIntervals must be an even number");
      if (Double.isInfinite (a) || Double.isInfinite (b) ||
         Double.isNaN (a) || Double.isNaN (b))
         throw new IllegalArgumentException
             ("a and b must not be infinite or NaN");
      if (b < a)
         throw new IllegalArgumentException ("b < a");
      if (a == b)
         return 0;
      double[] D = fixBounds (func, a, b, numIntervals);
      a = D[0];
      b = D[1];
      final double h = (b - a) / numIntervals;
      final double h2 = 2*h;
      final int m = numIntervals / 2;
      double sum = 0;
      for (int i = 0; i < m - 1; i++) {
         final double x = a + h + h2*i;
         sum += 4*func.evaluate (x) + 2*func.evaluate (x + h);
      }
      sum += func.evaluate (a) + func.evaluate (b) + 4*func.evaluate (b - h);
      return sum * h / 3;
   }\end{hide}
\end{code}
\begin{tabb}
 Computes and returns an approximation of the integral of \texttt{func} over
 $[a, b]$, using the Simpson's $1/3$ method with \texttt{numIntervals}
 intervals. This method estimates
\[\int_a^b f(x)dx,\]
 where $f(x)$ is the function defined by \texttt{func} evaluated at $x$,
 by dividing $[a, b]$ in $n=$~\texttt{numIntervals} intervals of length
 $h=(b - a)/n$. The integral is estimated by
\[\frac{h}{3}(f(a)+4f(a+h)+2f(a+2h)+4f(a+3h)+\cdots+f(b))\]
This method assumes that $a\le b<\infty$, and $n$ is even.
\end{tabb}
\begin{htmlonly}
\param{func}{the function being integrated.}
\param{a}{the left bound}
\param{b}{the right bound.}
\param{numIntervals}{the number of intervals.}
\return{the approximate value of the integral.}
\end{htmlonly}
\begin{code}

   public static double gaussLobatto (MathFunction func, double a, double b,
                                      double tol)\begin{hide} {
      if (b < a)
         throw new IllegalArgumentException ("b < a");
      if (Double.isInfinite (a) || Double.isInfinite (b) ||
          Double.isNaN (a) || Double.isNaN (b))
         throw new IllegalArgumentException ("a or b is infinite or NaN");
      if (a == b)
         return 0;
      double r0 = simpleGaussLob (func, a, b);
      final double h = (b - a)/2;
      double r1 = simpleGaussLob (func, a, a + h) +
                  simpleGaussLob (func, a + h, b);
      if (Math.abs(r0 - r1) <= tol)
         return r1;
      return gaussLobatto (func, a, a + h, tol) +
             gaussLobatto (func, a + h, b, tol);
   }


   private static double simpleGaussLob (MathFunction func, double a, double b) {
      // Gauss-Lobatto integral over [a, b] with 5 nodes
      if (a == b)
         return 0;
      final double h = b - a;
      double sum = 0;
      for (int i = 0; i < 5; i++) {
         sum += Wg[i] * func.evaluate(a + h*Cg[i]);
      }
      return h*sum;
   }\end{hide}
\end{code}
\begin{tabb}
 Computes and returns a numerical approximation of the integral of
  $f(x)$ over $[a, b]$, using Gauss-Lobatto adaptive quadrature with
   5 nodes, with tolerance \texttt{tol}. This method estimates
\[
\int_a^b f(x)dx,
\]
 where $f(x)$ is the function defined by \texttt{func}.
Whenever the estimated error is larger than \texttt{tol}, the interval
$[a, b]$ will be halved in two smaller intervals, and  the method will
 recursively call itself on the two smaller intervals until the estimated
 error is smaller than \texttt{tol}.
\end{tabb}
\begin{htmlonly}
\param{func}{the function being integrated.}
\param{a}{the left bound}
\param{b}{the right bound.}
\param{tol}{error.}
\return{the approximate value of the integral.}
\end{htmlonly}
\begin{code}

   public static double gaussLobatto (MathFunction func, double a, double b,
                                      double tol, double[][] T) \begin{hide} {
      if (b < a)
         throw new IllegalArgumentException ("b < a");
      if (a == b) {
         T[0] = new double [1];
         T[1] = new double [1];
         T[0][0] = a;
         T[1][0] = 0;
         return 0;
      }

      int n = 1;         // initial capacity
      T[0] = new double [n];
      T[1] = new double [n];
      T[0][0] = a;
      T[1][0] = 0;
      int[] s = {0};    // actual number of intervals
      double res = innerGaussLob (func, a, b, tol, T, s);
      n = s[0] + 1;
      double[] temp = new double[n];
      System.arraycopy (T[0], 0, temp, 0, n);
      T[0] = temp;
      temp = new double[n];
      System.arraycopy (T[1], 0, temp, 0, n);
      T[1] = temp;
      return res;
   }


   private static double innerGaussLob (MathFunction func, double a, double b,
                                        double tol, double[][] T, int[] s) {
      double r0 = simpleGaussLob (func, a, b);
      final double h = (b - a) / 2;
      double r1 = simpleGaussLob (func, a, a + h) +
                  simpleGaussLob (func, a + h, b);
      if (Math.abs(r0 - r1) <= tol) {
         ++s[0];
         int len = s[0];
         if (len >= T[0].length) {
            double[] temp = new double[2 * len];
            System.arraycopy (T[0], 0, temp, 0, len);
            T[0] = temp;
            temp = new double[2 * len];
            System.arraycopy (T[1], 0, temp, 0, len);
            T[1] = temp;
         }
         T[0][len] = b;
         T[1][len] = r1;
         return r1;
      }

      return innerGaussLob (func, a, a + h, tol, T, s) +
             innerGaussLob (func, a + h, b, tol, T, s);
   }\end{hide}
\end{code}
\begin{tabb}
Similar to method
\method{gaussLobatto}{}\texttt{(MathFunction, double, double, double)}, but
also returns in \texttt{T[0]} the subintervals of integration, and in
\texttt{T[1]}, the partial values of the integral over the corresponding
 subintervals. Thus \texttt{T[0][0]} $= x_0 = a$ and \texttt{T[0][n]}
  $=x_n =b$; \texttt{T[1][i]} contains the value of the integral over
  the subinterval $[x_{i-1}, x_i]$; we also have \texttt{T[1][0]} $=0$.
 The sum over all \texttt{T[1][i]}, for $i=1, \ldots, n$ gives the
 value of the integral over $[a,b]$, which is the value returned by this
 method. \emph{WARNING:} The user \emph{must reserve}  the 2 elements
 of the first dimension (\texttt{T[0]} and \texttt{T[1]})
 before calling this method.
 % The user program must call
 %    double[][] T = new double[2][1];
 % before calling gaussLobatto.
\end{tabb}
\begin{htmlonly}
\param{func}{function being integrated}
\param{a}{left bound of interval}
\param{b}{right bound of interval}
\param{tol}{error}
\param{T}{$(x,y)$ = (values of partial intervals,partial values of integral)}
\return{value of the integral}
\end{htmlonly}

\begin{code}\begin{hide}
}\end{hide}
\end{code}
