\defclass{Polynomial}

Represents a polynomial of degree $n$ in power form. Such a polynomial is of
the form
\begin{equation}
p(x) = c_0 + c_1x + \cdots + c_nx^n,
\end{equation}
where $c_0, \ldots, c_n$ are the coefficients of the polynomial.

\bigskip\hrule

\begin{code}
\begin{hide}
/*
 * Class:        Polynomial
 * Description:  
 * Environment:  Java
 * Software:     SSJ 
 * Copyright (C) 2001  Pierre L'Ecuyer and Université de Montréal
 * Organization: DIRO, Université de Montréal
 * @author       Éric Buist
 * @since

 * SSJ is free software: you can redistribute it and/or modify it under
 * the terms of the GNU General Public License (GPL) as published by the
 * Free Software Foundation, either version 3 of the License, or
 * any later version.

 * SSJ is distributed in the hope that it will be useful,
 * but WITHOUT ANY WARRANTY; without even the implied warranty of
 * MERCHANTABILITY or FITNESS FOR A PARTICULAR PURPOSE.  See the
 * GNU General Public License for more details.

 * A copy of the GNU General Public License is available at
   <a href="http://www.gnu.org/licenses">GPL licence site</a>.
 */
\end{hide}
package umontreal.iro.lecuyer.functions;\begin{hide}

import java.io.Serializable;
import umontreal.iro.lecuyer.util.PrintfFormat;
\end{hide}

public class Polynomial implements MathFunction\begin{hide}
,
      MathFunctionWithFirstDerivative, MathFunctionWithDerivative,
      MathFunctionWithIntegral, Serializable, Cloneable {
   private static final long serialVersionUID = -2911550952861456470L;
   private double[] coeff;
\end{hide}

   public Polynomial (double... coeff)\begin{hide} {
      if (coeff == null)
         throw new NullPointerException ();
      if (coeff.length == 0)
         throw new IllegalArgumentException (
               "At least one coefficient is needed");
      this.coeff = coeff.clone ();
   }\end{hide}
\end{code}
\begin{tabb}   Constructs a new polynomial with coefficients \texttt{coeff}. The value of
 \texttt{coeff[i]} in this array corresponds to $c_i$.
\end{tabb}
\begin{htmlonly}
   \param{coeff}{the coefficients of the polynomial.}
   \exception{NullPointerException}{if \texttt{coeff} is \texttt{null}.}
   \exception{IllegalArgumentException}{if the length of \texttt{coeff} is 0.}
\end{htmlonly}
\begin{code}

   public int getDegree ()\begin{hide} {
      return coeff.length - 1;
   }\end{hide}
\end{code}
\begin{tabb}   Returns the degree of this polynomial.
\end{tabb}
\begin{htmlonly}
   \return{the degree of this polynomial.}
\end{htmlonly}
\begin{code}

   public double[] getCoefficients ()\begin{hide} {
      return coeff.clone ();
   }\end{hide}
\end{code}
\begin{tabb}   Returns an array containing the coefficients of the polynomial.
\end{tabb}
\begin{htmlonly}
   \return{the array of coefficients.}
\end{htmlonly}
\begin{code}

   public double getCoefficient (int i)\begin{hide} {
      return coeff[i];
   }\end{hide}
\end{code}
\begin{tabb}   Returns the $i$th coefficient of the polynomial.
\end{tabb}
\begin{htmlonly}
   \return{the array of coefficients.}
\end{htmlonly}
\begin{code}

   public void setCoefficients (double... coeff)\begin{hide} {
      if (coeff == null)
         throw new NullPointerException ();
      if (coeff.length == 0)
         throw new IllegalArgumentException (
               "At least one coefficient is needed");
      this.coeff = coeff.clone ();
   }\end{hide}
\end{code}
\begin{tabb}   Sets the array of coefficients of this polynomial to \texttt{coeff}.
\end{tabb}
\begin{htmlonly}
   \param{coeff}{the new array of coefficients.}
   \exception{NullPointerException}{if \texttt{coeff} is \texttt{null}.}
   \exception{IllegalArgumentException}{if the length of \texttt{coeff} is 0.}
\end{htmlonly}
\begin{code}\begin{hide}

   public double evaluate (double x) {
      double res = coeff[coeff.length - 1];
      for (int i = coeff.length - 2; i >= 0; i--)
         res = coeff[i] + x * res;
      return res;
   }

   public double derivative (double x) {
      return derivative (x, 1);
   }

   public double derivative (double x, int n) {
      if (n < 0)
         throw new IllegalArgumentException ("n < 0");
      if (n == 0)
         return evaluate (x);
      if (n >= coeff.length)
         return 0;
//      double res = coeff[coeff.length - 1]*(coeff.length - 1);
//      for (int i = coeff.length - 2; i >= n; i--)
//         res = i*(coeff[i] + x * res);
      double res = getCoeffDer (coeff.length - 1, n);
      for (int i = coeff.length - 2; i >= n; i--)
         res = getCoeffDer (i, n) + x * res;
      return res;
   }\end{hide}

   public Polynomial derivativePolynomial (int n)\begin{hide} {
      if (n < 0)
         throw new IllegalArgumentException ("n < 0");
      if (n == 0)
         return this;
      if (n >= coeff.length)
         return new Polynomial (0);
      final double[] coeffDer = new double[coeff.length - n];
      for (int i = coeff.length - 1; i >= n; i--)
         coeffDer[i - n] = getCoeffDer (i, n);
      return new Polynomial (coeffDer);
   }\end{hide}
\end{code}
\begin{tabb} Returns a polynomial corresponding to the $n$th derivative of
this polynomial.
\end{tabb}
\begin{htmlonly}
   \param{n}{the degree of the derivative.}
   \return{the derivative.}
\end{htmlonly}
\begin{code}\begin{hide}

   private double getCoeffDer (int i, int n) {
      double coeffDer = coeff[i];
      for (int j = i; j > i - n; j--)
         coeffDer *= j;
      return coeffDer;
   }

   public double integral (double a, double b) {
      return integralA0 (b) - integralA0 (a);
   }

   private double integralA0 (double u) {
      final int n = coeff.length - 1;
      double res = u * coeff[n] / (n + 1);
      for (int i = coeff.length - 2; i >= 0; i--)
         res = coeff[i] * u / (i + 1) + u * res;
      return res;
   }\end{hide}

   public Polynomial integralPolynomial (double c)\begin{hide} {
      final double[] coeffInt = new double[coeff.length + 1];
      coeffInt[0] = c;
      for (int i = 0; i < coeff.length; i++)
         coeffInt[i + 1] = coeff[i] / (i + 1);
      return new Polynomial (coeffInt);
   }

   @Override
\end{hide}
\end{code}
\begin{tabb} Returns a polynomial representing the integral of this polynomial.
 This integral is of the form
\[\int p(x)dx = c + c_0x + \frac{c_1 x^2}2 + \cdots + \frac{c_n x^{n+1}}{n+1},
 \]
where $c$ is a user-defined constant.
\end{tabb}
\begin{htmlonly}
   \param{c}{the constant for the integral.}
   \return{the polynomial representing the integral.}
\end{htmlonly}
\begin{code}\begin{hide}

   public String toString () {
      final StringBuilder sb = new StringBuilder ();
      for (int i = 0; i < coeff.length; i++) {
         if (i > 0) {
            if (coeff[i] == 0)
               continue;
            else if (coeff[i] > 0)
               sb.append (" + ");
            else
               sb.append (" - ");
            sb.append (PrintfFormat.format (8, 3, 3, Math.abs (coeff[i])));
         }
         else
            sb.append (PrintfFormat.format (8, 3, 3, coeff[i]));
         if (i > 0) {
            sb.append ("*X");
            if (i > 1)
               sb.append ("^").append (i);
         }
      }
      return sb.toString ();
   }

   @Override
   public Polynomial clone () {
      Polynomial pol;
      try {
         pol = (Polynomial) super.clone ();
      }
      catch (final CloneNotSupportedException cne) {
         throw new IllegalStateException ("Clone not supported");
      }
      pol.coeff = coeff.clone ();
      return pol;
   }
}\end{hide}
\end{code}
