\defclass{DirichletGen}

Extends \class{RandomMultivariateGen} for a
\emph{Dirichlet} \cite{tJOH72a} distribution.  This distribution uses the
parameters
\hrichard{Doit-on utiliser $d$ pour la dimension comme dans
  \texttt{MultiNormalGen et RandomMultivariateGen}?}
$\alpha_1,\dots,\alpha_k$, and has density
\begin{latexonly}%
\[
  f(x_1,\ldots,x_k) = \frac{\Gamma(\alpha_0)\prod_{i=1}^k x_i^{\alpha_i - 1}}
     {\prod_{i=1}^k \Gamma(\alpha_i)}
\]
\end{latexonly}%
\begin{htmlonly}
\[
  f(x_1,\ldots,x_k) = \Gamma(\alpha_0)\prod_{i=1}^k x_i^{\alpha_i - 1}\  / \ \
     ({\prod_{i=1}^k \Gamma(\alpha_i)})
\]
\end{htmlonly}%
where $\alpha_0=\sum_{i=1}^k\alpha_i$.

Here, the successive coordinates of the Dirichlet vector are generated
\pierre{How? }
via the class
\externalclass{umontreal.iro.lecuyer.randvar}{GammaAcceptance\-Rejec\-tion\-Gen}{}
in package \texttt{randvar}, using the same stream for all the uniforms.
% with its particular value of $\alpha$.


\bigskip\hrule

\begin{code}
\begin{hide}
/*
 * Class:        DirichletGen
 * Description:  multivariate generator for a Dirichlet} distribution
 * Environment:  Java
 * Software:     SSJ 
 * Copyright (C) 2001  Pierre L'Ecuyer and Université de Montréal
 * Organization: DIRO, Université de Montréal
 * @author       
 * @since

 * SSJ is free software: you can redistribute it and/or modify it under
 * the terms of the GNU General Public License (GPL) as published by the
 * Free Software Foundation, either version 3 of the License, or
 * any later version.

 * SSJ is distributed in the hope that it will be useful,
 * but WITHOUT ANY WARRANTY; without even the implied warranty of
 * MERCHANTABILITY or FITNESS FOR A PARTICULAR PURPOSE.  See the
 * GNU General Public License for more details.

 * A copy of the GNU General Public License is available at
   <a href="http://www.gnu.org/licenses">GPL licence site</a>.
 */
\end{hide}
package umontreal.iro.lecuyer.randvarmulti;\begin{hide}

import umontreal.iro.lecuyer.probdist.GammaDist;
import umontreal.iro.lecuyer.randvar.RandomVariateGen;
import umontreal.iro.lecuyer.randvar.GammaAcceptanceRejectionGen;
import umontreal.iro.lecuyer.rng.RandomStream;
\end{hide}


public class DirichletGen extends RandomMultivariateGen\begin{hide} {
   private GammaAcceptanceRejectionGen[] ggens;
\end{hide}\end{code}

\subsubsection* {Constructor}
\begin{code}

   public DirichletGen (RandomStream stream, double[] alphas)\begin{hide} {
      if (stream == null)
         throw new NullPointerException ("stream is null");
      this.stream = stream;
      dimension = alphas.length;
      ggens = new GammaAcceptanceRejectionGen[alphas.length];
      for (int k = 0; k < alphas.length; k++)
         ggens[k] = new GammaAcceptanceRejectionGen
            (stream, new GammaDist (alphas[k], 1.0/2.0));
   }\end{hide}
\end{code}
\begin{tabb}   Constructs a new Dirichlet
 generator with parameters $\alpha_{i+1}=$~\texttt{alphas[i]},
 for $i=0,\ldots,k-1$, and the stream \texttt{stream}.
\end{tabb}
\begin{htmlonly}
   \param{stream}{the random number stream used to generate uniforms.}
   \param{alphas}{the $\alpha_i$ parameters of the generated distribution.}
   \exception{IllegalArgumentException}{if one $\alpha_k$ is negative or 0.}
   \exception{NullPointerException}{if any argument is \texttt{null}.}
\end{htmlonly}

%
%%%%%%%%%%%%%%%%%%%%%%%%%%%%%%%%%%%%%%%%%%%%%%%%5
\subsubsection* {Methods}
\begin{code}

   public double getAlpha (int i)\begin{hide} {
      return ((GammaDist)ggens[i].getDistribution()).getAlpha();
   }\end{hide}
\end{code}
\begin{tabb}   Returns the $\alpha_{i+1}$ parameter for this
 Dirichlet generator.
\end{tabb}
\begin{htmlonly}
   \param{i}{the index of the parameter.}
   \return{the value of the parameter.}
   \exception{ArrayIndexOutOfBoundsException}{if \texttt{i} is
    negative or greater than or equal to \method{getDimension}{()}.}
\end{htmlonly}
\begin{code}

   public static void nextPoint (RandomStream stream, double[] alphas,
                                 double[] p)\begin{hide} {
      double total = 0;
      for (int i = 0; i < alphas.length; i++) {
         p[i] = GammaAcceptanceRejectionGen.nextDouble
            (stream, stream, alphas[i], 1.0/2.0);
         total += p[i];
      }
      for (int i = 0; i < alphas.length; i++)
         p[i] /= total;
   }\end{hide}
\end{code}
\begin{tabb}   Generates a new point from the Dirichlet distribution with
 parameters \texttt{alphas}, using the stream \texttt{stream}.
 The generated values are placed into \texttt{p}.
\end{tabb}
\begin{htmlonly}
   \param{stream}{the random number stream used to generate the uniforms.}
   \param{alphas}{the $\alpha_i$ parameters of the distribution, for
    $i=1,\ldots,k$.}
   \param{p}{the array to be filled with the generated point.}
\end{htmlonly}
\begin{code}

   public void nextPoint (double[] p)\begin{hide} {
      int n = ggens.length;
      double total = 0;
      for (int i = 0; i < n; i++) {
         p[i] = ggens[i].nextDouble();
         total += p[i];
      }
      for (int i = 0; i < n; i++)
         p[i] /= total;
   }
}\end{hide}
\end{code}
\begin{tabb}  Generates a point from the Dirichlet distribution.
\end{tabb}
\begin{htmlonly}
   \param{p}{the array to be filled with the generated point.}
\end{htmlonly}
