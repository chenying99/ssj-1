\defmodule {RandMrg}

\textit{USE} \class{MRG32k3a} \textit{INSTEAD of this class}.
This class implements the interface \class{RandomStream} directly, with a few
additional tools.  It uses the same backbone (or main) generator as
\class{MRG32k3a}, but it is an older
 implementation that does not extend \class{RandomStreamBase},
 and it is about 10\%{} slower.

\iffalse  %%%%%%%%
The method \texttt{getState} returns the state of a stream.
One can change the state of a given stream, without modifying
the state of the other streams, by calling \texttt{setSeed} 
or \texttt{advanceState}.
However, after calling \texttt{setSeed} for a given stream,
the initial states of the different streams are no longer spaced
$Z$ values apart.  Therefore, this method should be called
very sparingly, if at all.  The methods \texttt{reset...}
suffices for almost all applications.
\fi  %%%%%%%%%%


%%%%%%%%%%%%%%%%%%%%%%%
\bigskip\hrule

\begin{code}
\begin{hide}
/*
 * Class:        RandMrg
 * Description:  Old class replaced by MRG32k3a
 * Environment:  Java
 * Software:     SSJ 
 * Copyright (C) 2001  Pierre L'Ecuyer and Université de Montréal
 * Organization: DIRO, Université de Montréal
 * @author       
 * @since

 * SSJ is free software: you can redistribute it and/or modify it under
 * the terms of the GNU General Public License (GPL) as published by the
 * Free Software Foundation, either version 3 of the License, or
 * any later version.

 * SSJ is distributed in the hope that it will be useful,
 * but WITHOUT ANY WARRANTY; without even the implied warranty of
 * MERCHANTABILITY or FITNESS FOR A PARTICULAR PURPOSE.  See the
 * GNU General Public License for more details.

 * A copy of the GNU General Public License is available at
   <a href="http://www.gnu.org/licenses">GPL licence site</a>.
 */
\end{hide}
package umontreal.iro.lecuyer.rng; \begin{hide}

import umontreal.iro.lecuyer.util.PrintfFormat;
import java.io.Serializable;
\end{hide}

@Deprecated
public class RandMrg implements CloneableRandomStream, Serializable\begin{hide} {

   private static final long serialVersionUID = 70510L;
   //La date de modification a l'envers, lire 10/05/2007

// %%%%%%%%%%%%%%%%%%%%%%%%%%%%%%%%%
// Private constants.

   private static final double m1     = 4294967087.0;
   private static final double m2     = 4294944443.0;
   private static final double a12    =  1403580.0;
   private static final double a13n   =   810728.0;
   private static final double a21    =   527612.0;
   private static final double a23n   =   1370589.0;
   private static final double two17    =  131072.0;
   private static final double two53    =  9007199254740992.0;
   private static final double invtwo24 = 5.9604644775390625e-8;
//   private static final double norm   = 2.328306549295727688e-10;
   private static final double norm   = 1.0 / (m1 + 1.0);

   private static final double InvA1[][] = {   // Inverse of A1p0
      { 184888585.0, 0.0, 1945170933.0 },
      {         1.0, 0.0,          0.0 },
      {         0.0, 1.0,          0.0 }
      };
   private static final double InvA2[][] = {   // Inverse of A2p0
      { 0.0, 360363334.0, 4225571728.0 },
      { 1.0,         0.0,          0.0 },
      { 0.0,         1.0,          0.0 }
      };
   private static final double A1p0[][]  =  {
      {       0.0,       1.0,      0.0 },
      {       0.0,       0.0,      1.0 },
      { -810728.0, 1403580.0,      0.0 }
      };
   private static final double A2p0[][]  =  {
      {        0.0,   1.0,         0.0 },
      {        0.0,   0.0,         1.0 },
      { -1370589.0,   0.0,    527612.0 }
      };
   private static final double A1p76[][] = {
      {      82758667.0, 1871391091.0, 4127413238.0 },
      {    3672831523.0,   69195019.0, 1871391091.0 },
      {    3672091415.0, 3528743235.0,   69195019.0 }
      };
   private static final double A2p76[][] = {
      {    1511326704.0, 3759209742.0, 1610795712.0 },
      {    4292754251.0, 1511326704.0, 3889917532.0 },
      {    3859662829.0, 4292754251.0, 3708466080.0 }
      };
   private static final double A1p127[][] = {
      {    2427906178.0, 3580155704.0,  949770784.0 },
      {     226153695.0, 1230515664.0, 3580155704.0 },
      {    1988835001.0,  986791581.0, 1230515664.0 }
      };
   private static final double A2p127[][] = {
      {    1464411153.0,  277697599.0, 1610723613.0 },
      {      32183930.0, 1464411153.0, 1022607788.0 },
      {    2824425944.0,   32183930.0, 2093834863.0 }
      };


// %%%%%%%%%%%%%%%%%%%%%%%%%%%%%%%%%
// Private variables (fields) for each stream.

   private static double nextSeed[] = {12345, 12345, 12345, 12345, 12345, 12345};
   // Default seed of the package and seed for the next stream to be created.

   private double Cg[] = new double[6];
   private double Bg[] = new double[6];
   private double Ig[] = new double[6];
   // The arrays \texttt{Cg}, \texttt{Bg}, and \texttt{Ig} contain the current state, 
   // the starting point of the current substream,
   // and the starting point of the stream, respectively.

   private boolean anti;
   // This stream generates antithetic variates if 
   // and only if \texttt{anti = true}.

   private boolean prec53;
   // The precision of the output numbers is ``increased'' (see
   // \texttt{increasedPrecis}) if and only if \texttt{prec53 = true}.

   private String descriptor;
   // Describes the stream (for writing the state, error messages, etc.).

// %%%%%%%%%%%%%%%%%%%%%%%%%%%%%%%%%
// Private methods

    //--------------------------------------------------------------
    /* Compute (a*s + c) MOD m ; m must be < 2^35 */
    /* Works also for s, c < 0.                   */
    private static double multModM 
           (double a, double s, double c, double m) {
        double v;
        int a1;
        v = a * s + c;
        if (v >= two53 || v <= -two53 ) {
                a1 = (int)(a / two17);  a -= a1 * two17;
                v  = a1 * s;
                a1 = (int)(v / m);    v -= a1 * m;
                v  = v * two17 + a * s + c;
        }
        a1 = (int)(v / m);
        if ((v -= a1 * m) < 0.0) return v += m; else return v;
    }


    //-----------------------------------------------------------
    /* Returns v = A*s MOD m.  Assumes that -m < s[i] < m. */
    /* Works even if v = s.                                */
    private static void matVecModM (double A[][], double s[], 
                                    double v[], double m) {
            int i;
            double x[] = new double[3];
            for (i = 0; i < 3;  ++i) {
                x[i] = multModM (A[i][0], s[0], 0.0, m);
                x[i] = multModM (A[i][1], s[1], x[i], m);
                x[i] = multModM (A[i][2], s[2], x[i], m);
            }
            for (i = 0; i < 3;  ++i)  v[i] = x[i];
    }

    //------------------------------------------------------------
    /* Returns C = A*B MOD m */
    /* Note: work even if A = C or B = C or A = B = C.         */
    private static void matMatModM (double A[][], double B[][], 
                                    double C[][], double m){
             int i, j;
             double V[] = new double[3], W[][] = new double[3][3];
             for (i = 0; i < 3;  ++i) {
                    for (j = 0; j < 3;  ++j)  V[j] = B[j][i];
                            matVecModM (A, V, V, m);
                    for (j = 0; j < 3;  ++j)  W[j][i] = V[j];
             }
             for (i = 0; i < 3;  ++i) {
                    for (j = 0; j < 3;  ++j)
                        C[i][j] = W[i][j];
             }
    }

    //-------------------------------------------------------------
    /* Compute matrix B = (A^(2^e) Mod m);  works even if A = B */
    private static void matTwoPowModM (double A[][], double B[][], 
                                       double m, int e) {
            int i, j;
            /* initialize: B = A */
            if (A != B) {
                for (i = 0; i < 3; i++) {
                    for (j = 0; j < 3;  ++j)  B[i][j] = A[i][j];
                }
            }
            /* Compute B = A^{2^e} */
            for (i = 0; i < e; i++) matMatModM (B, B, B, m);
    }

    //-------------------------------------------------------------
    /* Compute matrix D = A^c Mod m ;  works even if A = B */
    private static void matPowModM (double A[][], double B[][], 
                                    double m, int c){
            int i, j;
            int n = c;
            double W[][] = new double[3][3];

            /* initialize: W = A; B = I */
            for (i = 0; i < 3; i++) {
                for (j = 0; j < 3;  ++j)  {
                    W[i][j] = A[i][j];
                    B[i][j] = 0.0;
                }
            }
            for (j = 0; j < 3;  ++j)   B[j][j] = 1.0;

            /* Compute B = A^c mod m using the binary decomp. of c */
            while (n > 0) {
                if ((n % 2)==1) matMatModM (W, B, B, m);
                matMatModM (W, W, W, m);
                n /= 2;
            }
    } 

   //-------------------------------------------------------------
   // Generate a uniform random number, with 32 bits of resolution.
   private double U01() {
        int k;
        double p1, p2, u;
        /* Component 1 */
        p1 = a12 * Cg[1] - a13n * Cg[0];
        k = (int)(p1 / m1);
        p1 -= k * m1;
        if (p1 < 0.0) p1 += m1;
        Cg[0] = Cg[1];   Cg[1] = Cg[2];   Cg[2] = p1;
        /* Component 2 */
        p2 = a21 * Cg[5] - a23n * Cg[3];
        k  = (int)(p2 / m2);
        p2 -= k * m2;
        if (p2 < 0.0) p2 += m2;
        Cg[3] = Cg[4];   Cg[4] = Cg[5];   Cg[5] = p2;
        /* Combination */
        u = ((p1 > p2) ? (p1 - p2) * norm : (p1 - p2 + m1) * norm);
        return (anti) ? (1 - u) : u;
   }

   //-------------------------------------------------------------
   // Generate a uniform random number, with 52 bits of resolution.
   private double U01d() {
        double u = U01();
        if (anti) {
            // Antithetic case: note that U01 already returns 1-u.
            u += (U01() - 1.0) * invtwo24;
            return (u < 0.0) ? u + 1.0 : u;
        } else {
            u += U01() * invtwo24;
            return (u < 1.0) ? u : (u - 1.0);
        }
   }
\end{hide}
\end{code}

%%%%%%%%%%%%%%%%%%%%%%%%%%%%%%%%%%%
\subsubsection* {Constructors}

\begin{code}

   public RandMrg() \begin{hide} {
      anti = false;
      prec53 = false;
      for (int i = 0; i < 6; ++i)  
         Bg[i] = Cg[i] = Ig[i] = nextSeed[i];
      matVecModM (A1p127, nextSeed, nextSeed, m1);
      double temp[] = new double[3];
      for (int i = 0; i < 3; ++i)  
         temp[i] = nextSeed[i + 3];
      matVecModM (A2p127, temp, temp, m2);
      for (int i = 0; i < 3; ++i)  
         nextSeed[i + 3] = temp[i];
   }\end{hide}
\end{code}
 \begin{tabb} Constructs a new stream, initializes its seed $I_g$,
   sets $B_g$ and $C_g$ equal to $I_g$, and sets its antithetic switch 
   to \texttt{false}.
   The seed $I_g$ is equal to the initial seed of the package given by 
   \method{setPackageSeed}{long[]} if this is the first stream created,
   otherwise it is $Z$ steps ahead of that of the stream most recently
   created in this class.
 \end{tabb}
\begin{code}

   public RandMrg (String name) \begin{hide} {
      this();
      descriptor = name;
   }\end{hide}
\end{code}
 \begin{tabb}  Constructs a new stream with an identifier \texttt{name}
   (can be used when printing the stream state, in error messages, etc.).
 \end{tabb}
\begin{htmlonly}
   \param{name}{name of the stream}
\end{htmlonly}
 
%%%%%%%%%%%%%%%%%%%%%%%%%%%%%%%%%%%
\subsubsection* {Methods}
\begin{code}

   public static void setPackageSeed (long seed[]) \begin{hide} {
      // Must use long because there is no unsigned int type.
      if (seed.length != 6)
         throw new IllegalArgumentException ("Seed must contain 6 values");
      if (seed[0] == 0 && seed[1] == 0 && seed[2] == 0)
         throw new IllegalArgumentException ("The first 3 values must not be 0");
      if (seed[5] == 0 && seed[3] == 0 && seed[4] == 0)
         throw new IllegalArgumentException ("The last 3 values must not be 0");
      final long m1 = 4294967087L;
      if (seed[0] >= m1 || seed[1] >= m1 || seed[2] >= m1)
         throw new IllegalArgumentException ("The first 3 values must be less than " + m1);
      final long m2 = 4294944443L;
      if (seed[5] >= m2 || seed[3] >= m2 || seed[4] >= m2)
         throw new IllegalArgumentException ("The last 3 values must be less than " + m2);
      for (int i = 0; i < 6;  ++i)  nextSeed[i] = seed[i];
   }\end{hide}
\end{code}
  \begin{tabb}  Sets the initial seed for the class \texttt{RandMrg} to the 
   six integers in the vector \texttt{seed[0..5]}.
   This will be the seed (initial state) of the first stream.
   If this method is not called, the default initial seed
   is $(12345, 12345, 12345, 12345, 12345, 12345)$.
   If it is called, the first 3 values of the seed must all be
   less than $m_1 = 4294967087$, and not all 0;
   and the last 3 values 
   must all be less than $m_2 = 4294944443$, and not all 0.
 \end{tabb}
\begin{htmlonly}
   \param{seed}{array of 6 elements representing the seed}
\end{htmlonly}
\begin{hide}
\begin{code}

   public void resetStartStream() \begin{hide} {
      for (int i = 0; i < 6;  ++i)  Cg[i] = Bg[i] = Ig[i];
   } \end{hide}

   public void resetStartSubstream() \begin{hide} {
      for (int i = 0; i < 6;  ++i)  Cg[i] = Bg[i];
   } \end{hide}

   public void resetNextSubstream() \begin{hide} {
      int i;
      matVecModM (A1p76, Bg, Bg, m1);
      double temp[] = new double[3];
      for (i = 0; i < 3; ++i) temp[i] = Bg[i + 3];
      matVecModM (A2p76, temp, temp, m2);
      for (i = 0; i < 3; ++i) Bg[i + 3] = temp[i];
      for (i = 0; i < 6;  ++i) Cg[i] = Bg[i];
   } \end{hide}
\end{code}
\end{hide}
\begin{code}

   public void increasedPrecis (boolean incp) \begin{hide} {
      prec53 = incp;
   }\end{hide}
\end{code}
 \begin{tabb} After calling this method with \texttt{incp = true}, each call to 
  the generator (direct or indirect) for this stream 
  will return a uniform random number with (roughly) 53 bits of resolution 
  instead of 32 bits,
  and will advance the state of the stream by 2 steps instead of 1.  
  More precisely, if \texttt{s} is a stream of the class \texttt{RandMrg},
  in the non-antithetic case, the instruction
  ``\texttt{u = s.nextDouble()}'', when the resolution has been increased,
  is equivalent to
  ``\texttt{u = (s.nextDouble() + s.nextDouble()*fact) \%\ 1.0}'' 
  where the constant \texttt{fact} is equal to $2^{-24}$.
  This also applies when calling \texttt{nextDouble} indirectly
  (e.g., via \texttt{nextInt}, etc.).

  By default, or if this method is called again with \texttt{incp = false}, 
  each call to \texttt{nextDouble} for this stream advances the state by 1 step
  and returns a number with 32 bits of resolution.
 \end{tabb}
\begin{htmlonly}
   \param{incp}{\texttt{true} if increased precision is desired, \texttt{false} otherwise}
\end{htmlonly}
\begin{code}\begin{hide}

   public void setAntithetic (boolean anti)  {
      this.anti = anti;
   }\end{hide}

   public void advanceState (int e, int c) \begin{hide} {
      double B1[][]= new double[3][3], C1[][]=new double[3][3];
      double B2[][]= new double[3][3], C2[][]=new double[3][3];

      if (e > 0) {
          matTwoPowModM (A1p0, B1, m1, e);
          matTwoPowModM (A2p0, B2, m2, e);
      } else if (e < 0) {
          matTwoPowModM (InvA1, B1, m1, -e);
          matTwoPowModM (InvA2, B2, m2, -e);
      }

      if (c >= 0) {
          matPowModM (A1p0, C1, m1, c);
          matPowModM (A2p0, C2, m2, c);
      } else {
          matPowModM (InvA1, C1, m1, -c);
          matPowModM (InvA2, C2, m2, -c);
      }

      if (e != 0) {
          matMatModM (B1, C1, C1, m1);
          matMatModM (B2, C2, C2, m2);
      }

      matVecModM (C1, Cg, Cg, m1);
      double[] cg3 = new double[3];
      for (int i = 0; i < 3; i++)  cg3[i] = Cg[i+3];
      matVecModM (C2, cg3, cg3, m2);
      for (int i = 0; i < 3; i++)  Cg[i+3] = cg3[i];
   }\end{hide}
\end{code}
 \begin{tabb} Advances the state of this stream by $k$ values,
  without modifying the states of other streams (as in \texttt{setSeed}),
  nor the values of $B_g$ and $I_g$ associated with this stream.
  If $e > 0$, then $k=2^e + c$; 
  if $e < 0$,  then $k=-2^{-e} + c$; and if $e = 0$,  then $k=c$.
  Note: $c$ is allowed to take negative values.
  This method should be used only in very 
  exceptional cases; proper use of the \texttt{reset...} methods 
  and of the stream constructor cover most reasonable situations.
 \end{tabb}
\begin{htmlonly}
   \param{e}{an exponent}
   \param{c}{a constant}
\end{htmlonly}
\begin{code}

   public void setSeed (long seed[]) \begin{hide} {
      // Must use long because there is no unsigned int type.
      if (seed.length != 6)
         throw new IllegalArgumentException ("Seed must contain 6 values");
      if (seed[0] == 0 && seed[1] == 0 && seed[2] == 0)
         throw new IllegalArgumentException ("The first 3 values must not be 0");
      if (seed[3] == 0 && seed[4] == 0 && seed[5] == 0)
         throw new IllegalArgumentException ("The last 3 values must not be 0");
      final long m1 = 4294967087L;
      if (seed[0] >= m1 || seed[1] >= m1 || seed[2] >= m1)
         throw new IllegalArgumentException ("The first 3 values must be less than " + m1);
      final long m2 = 4294944443L;
      if (seed[3] >= m2 || seed[4] >= m2 || seed[5] >= m2)
         throw new IllegalArgumentException ("The last 3 values must be less than " + m2);
      for (int i = 0; i < 6;  ++i)
         Cg[i] = Bg[i] = Ig[i] = seed[i];
   }\end{hide}
\end{code}
 \begin{tabb}  Sets the initial seed $I_g$ of this stream 
  to the vector \texttt{seed[0..5]}.  This vector must satisfy the same 
  conditions as in \texttt{setPackageSeed}.
  The stream is then reset to this initial seed.
  The states and seeds of the other streams are not modified.
  As a result, after calling this method, the initial seeds
  of the streams are no longer spaced $Z$ values apart.
  For this reason, this method should be used only in very 
  exceptional situations; proper use of \texttt{reset...} 
  and of the stream constructor is preferable.
 \end{tabb}
\begin{htmlonly}
   \param{seed}{array of 6 integers representing the new seed}
\end{htmlonly}
\begin{code}

   public double[] getState() \begin{hide} {
      return Cg;
   }\end{hide}
\end{code}
 \begin{tabb} Returns the current state $C_g$ of this stream.
  This is a vector of 6 integers represented in floating-point format.
  This method is convenient if we want to save the state for 
  subsequent use.  
 \end{tabb}
\begin{htmlonly}
   \return{the current state of the generator}
\end{htmlonly}

\begin{code}\begin{hide}
   public String toString() {
       PrintfFormat str = new PrintfFormat();
       
       str.append ("The current state of the RandMrg");
       if (descriptor != null && descriptor.length() > 0)
          str.append (" " + descriptor);
       str.append (":" + PrintfFormat.NEWLINE + "   Cg = { ");
       for (int i = 0; i < 5; i++)
          str.append ((long) Cg[i] + ", ");
       str.append ((long) Cg[5] + " }" + PrintfFormat.NEWLINE +
              PrintfFormat.NEWLINE);
     
       return str.toString();
   }\end{hide} 

   public String toStringFull() \begin{hide} {
       PrintfFormat str = new PrintfFormat();
       str.append ("The RandMrg");
       if (descriptor != null && descriptor.length() > 0)
          str.append (" " + descriptor);
       str.append (":" + PrintfFormat.NEWLINE + "   anti = " +
          (anti ? "true" : "false")).append(PrintfFormat.NEWLINE);

       str.append ("   Ig = { ");
       for (int i = 0; i < 5; i++)
          str.append ((long) Ig[i] + ", ");
       str.append ((long) Ig[5] + " }" + PrintfFormat.NEWLINE);

       str.append ("   Bg = { ");
       for (int i = 0; i < 5; i++)
          str.append ((long) Bg[i] + ", ");
       str.append ((long) Bg[5] + " }" + PrintfFormat.NEWLINE);

       str.append ("   Cg = { ");
       for (int i = 0; i < 5; i++)
          str.append ((long) Cg[i] + ", ");
       str.append ((long) Cg[5] + " }" + PrintfFormat.NEWLINE +
           PrintfFormat.NEWLINE);
 
       return str.toString();
   }\end{hide}
\end{code}
 \begin{tabb} Returns a string containing the name of this stream and the 
   values of all its internal variables.
 \end{tabb}
\begin{htmlonly}
   \return{the detailed state of the generator, formatted as a string}
\end{htmlonly}
\begin{code}

   public double nextDouble() \begin{hide} {
      if (prec53) return this.U01d();
      else return this.U01();
   }\end{hide}
\end{code}
 \begin{tabb} Returns a (pseudo)random number from the uniform distribution
   over the interval $(0,1)$, using this stream,
   after advancing its state by one step.  
   Normally, the returned number has 32 bits of resolution,
   in the sense that it is always a multiple of $1/(2^{32}-208)$.
   However, if the precision has been increased by calling 
   \texttt{increasedPrecis} for this stream, the resolution is higher
   and the stream state advances by two steps.
 \end{tabb}
\begin{hide}
\begin{code}

   public void nextArrayOfDouble (double[] u, int start, int n)\begin{hide} {
      if (n <= 0)
         throw new IllegalArgumentException ("n must be positive.");
      for (int i = start; i < start+n; i++)
         u[i] = nextDouble();
   }\end{hide}

   public int nextInt (int i, int j) \begin{hide} {
      // This works even for an interval [0, 2^31 - 1].
      // It would not with u*(j - i + 1)
      return (i + (int)(nextDouble() * (j - i + 1.0)));
   }\end{hide}

   public void nextArrayOfInt (int i, int j, int[] u, int start, int n)\begin{hide} {
      if (n <= 0)
         throw new IllegalArgumentException ("n must be positive.");
      for (int k = start; k < start+n; k++)
         u[k] = nextInt (i, j);
   }\end{hide}
\end{code}
\end{hide}
\begin{code}

   public RandMrg clone() \begin{hide} {
      RandMrg retour = null;
      try {
         retour = (RandMrg)super.clone();
         retour.Cg = new double[6];
         retour.Bg = new double[6];
         retour.Ig = new double[6];
         for (int i = 0; i<6; i++) {
            retour.Cg[i] = Cg[i];
            retour.Bg[i] = Bg[i];
            retour.Ig[i] = Ig[i];
         }
      }
      catch (CloneNotSupportedException cnse) {
         cnse.printStackTrace(System.err);
      }
      return retour;
   }\end{hide}
\end{code}
 \begin{tabb} Clones the current generator and return its copy.
 \end{tabb}
 \begin{htmlonly}
   \return{A deep copy of the current generator}
\end{htmlonly}

\begin{code}\begin{hide}

}\end{hide}
\end{code}
