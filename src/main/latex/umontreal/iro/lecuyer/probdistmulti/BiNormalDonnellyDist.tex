\defclass {BiNormalDonnellyDist}
 
  Extends the class \class{BiNormalDist} for the {\em bivariate 
  normal\/} distribution\latex{ \cite[page 84]{tJOH72a}}
  using a translation of Donnelly's {\sc Fortran} code\latex{
 in \cite{tDON73a}}.

\bigskip\hrule

%%%%%%%%%%%%%%%%%%%%%%%%%%%%%%%%%%%%%%%%
\begin{code}
\begin{hide}
/*
 * Class:        BiNormalDonnellyDist
 * Description:  bivariate normal distribution using Donnelly's code
 * Environment:  Java
 * Software:     SSJ 
 * Organization: DIRO, Université de Montréal
 * @author       
 * @since
 */
\end{hide}
package umontreal.iro.lecuyer.probdistmulti;
\begin{hide}
import umontreal.iro.lecuyer.probdist.NormalDist;
import umontreal.iro.lecuyer.util.Num;
\end{hide}

public class BiNormalDonnellyDist extends BiNormalDist \begin{hide} {

private static final double TWOPI = 2.0*Math.PI;
private static final double SQRTPI = Math.sqrt(Math.PI);
private static final int KMAX = 6;

private static double BB[] = new double[KMAX + 1];
private static final double CB[] = {
    0.9999936, -0.9992989, 0.9872976,
   -0.9109973, 0.6829098, -0.3360210, 0.07612251 };


private static double BorthT (double h, double a)
{
   int k;
   final double w = a * h / Num.RAC2;
   final double w2 = w * w;
   double z = w * Math.exp(-w2);

   BB[0] = SQRTPI * (Gauss (Num.RAC2 * w) - 0.5);
   for (k = 1; k <= KMAX; ++k) {
      BB[k] = ((2 * k - 1) * BB[k - 1] - z) / 2.0;
      z *= w2;
   }

   final double h2 = h * h / 2.0;
   z = h / Num.RAC2;
   double sum = 0;
   for (k = 0; k <= KMAX; ++k) {
      sum += CB[k] * BB[k] / z;
      z *= h2;
   }
   return sum * Math.exp (-h2) / TWOPI;
}


\end{hide}
\end{code}
%%%%%%%%%%%%%%%%%%%%%%%%%%%%%%%%%%%%%%%%%
\subsubsection* {Constructors}

\begin{code}

   public BiNormalDonnellyDist (double rho, int ndig) \begin{hide} {
       super (rho);
       if (ndig > 15)
          throw new IllegalArgumentException ("ndig > 15");
       decPrec = ndigit = ndig;
   }\end{hide}
\end{code}
\begin{tabb}
 Constructor with default  parameters
 $\mu_1 = \mu_2 = 0$, $\sigma_1 = \sigma_2 = 1$, correlation
 $\rho = $\texttt{ rho}, and $d = $ \texttt{ndig} digits of accuracy 
 (the absolute error is smaller than $10^{-d}$). Restriction: $d \le 15$.
  \end{tabb}
\begin{code}

   public BiNormalDonnellyDist (double rho) \begin{hide} {
       this (rho, 15);
   }\end{hide}
\end{code}
\begin{tabb}
  Same as \method{BiNormalDonnellyDist}{double,int}~\texttt{(rho, 15)}.
  \end{tabb}
\begin{code}

   public BiNormalDonnellyDist (double mu1, double sigma1, double mu2,
                                double sigma2, double rho, int ndig) \begin{hide} {
      super (mu1, sigma1, mu2, sigma2, rho);
      if (ndig > 15)
         throw new IllegalArgumentException ("ndig > 15");
      decPrec = ndigit = ndig;
   }\end{hide}
\end{code}
\begin{tabb}
  Constructor with parameters
  $\mu_1$ = \texttt{mu1}, $\mu_2$ = \texttt{mu2}, $\sigma_1$ = \texttt{sigma1}, 
 $\sigma_2$ = \texttt{sigma2}, $\rho$ = \texttt{rho}, and $d = $ \texttt{ndig}
  digits of accuracy. Restriction: $d \le 15$.
  \end{tabb}
\begin{code}

   public BiNormalDonnellyDist (double mu1, double sigma1, double mu2,
                                double sigma2, double rho) \begin{hide} {
      this (mu1, sigma1, mu2, sigma2, rho, 15);
   }\end{hide}
\end{code}
\begin{tabb}
  Same as \method{BiNormalDonnellyDist}{double,double,double,double,double,int}~\texttt{(mu1, sigma1, mu2, sigma2, rho, 15)}.
\end{tabb}


%%%%%%%%%%%%%%%%%%%%%%%%%%%%%%%%%%%
\subsubsection* {Methods}
  The following methods use the parameter \texttt{ndig} for the number of digits of
  absolute accuracy. If the same methods are called without the  \texttt{ndig}
  parameter, a default value of \texttt{ndig} = 15 will be used.

\begin{code}

   public static double cdf (double x, double y, double rho, int ndig) \begin{hide} {
   /* 
    * This is a translation of the FORTRAN code published by Thomas G. Donnelly
    * in the CACM, Vol. 16, Number 10, p. 638, (1973)
    */

      if (ndig > 15)
         throw new IllegalArgumentException ("ndig > 15");
      double b = specialCDF (x, y, rho, 13.0);
      if (b >= 0.0)
         return b;
      b = 0;

      final boolean SINGLE_FLAG = ndig <= 7 ? true : false;
      final double TWO_PI = 2.0 * Math.PI;
      final double r = rho;
      final double ah = -x;
      final double ak = -y;

      double a2, ap, cn, conex, ex, g2, gh, gk, gw = 0, h2, h4, rr, s1, s2,
         sgn, sn, sp, sqr, t, temp, w2, wh = 0, wk = 0;
      int is = -1;

      if (SINGLE_FLAG) {
         gh = Gauss (x) / 2.0;
         gk = Gauss (y) / 2.0;
      } else {
         gh = NormalDist.cdf01 (x) / 2.0;
         gk = NormalDist.cdf01 (y) / 2.0;
      }
      boolean flag = true;    // Easiest way to translate a Fortran goto

      rr = (1 - r) * (1 + r);
      if (rr < 0)
         throw new IllegalArgumentException ("|rho| > 1");
      sqr = Math.sqrt(rr);
      final double con = Math.PI * Num.TEN_NEG_POW[ndig];
      final double EPSILON = 0.5*Num.TEN_NEG_POW[ndig];

      if (ah != 0) {
         b = gh;
         if (ah * ak < 0)
            b -= 0.5;
         else if (ah * ak == 0) {
            flag = false;
         }
      } else if (ak == 0) {
         return Math.asin (r) / TWO_PI + 0.25;
      }

      if (flag)
         b += gk;
      if (ah != 0) {
         flag = false;
         wh = -ah;
         wk = (ak / ah - r) / sqr;
         gw = 2 * gh;
         is = -1;
      }

      do {
         if (flag) {
            wh = -ak;
            wk = (ah / ak - r) / sqr;
            gw = 2 * gk;
            is = 1;
         }
         flag = true;
         sgn = -1;
         t = 0;
         if (wk != 0) {
            if (Math.abs (wk) >= 1)
               if (Math.abs (wk) == 1) {
                  t = wk * gw * (1 - gw) / 2;
                  b += sgn * t;
                  if (is >= 0)        // Another Fortran goto
                     break;
                  else
                     continue;
               } else {
                  sgn = -sgn;
                  wh = wh * wk;
                  if (SINGLE_FLAG)
                     g2 = Gauss(wh);
                  else
                     g2 = NormalDist.cdf01(wh);
                  wk = 1 / wk;
                  if (wk < 0)
                     b = b + .5;
                  b = b - (gw + g2) / 2 + gw * g2;
               }
        /*****
            // Cette m'ethode de Borth est plus lente que simple Donnelly 
            if (ndig <= 7 && Math.abs (wh) > 1.6 && Math.abs (wk) > 0.3) {
               b += sgn * BorthT (wh, wk);
               if (is >= 0)
                  break;
               else
                  continue;
            }
        *****/
            h2 = wh * wh;
            a2 = wk * wk;
            h4 = h2 * .5;
            ex = 0;
            if (h4 < 300.0)
               ex = Math.exp (-h4);
            w2 = h4 * ex;
            ap = 1;
            s2 = ap - ex;
            sp = ap;
            s1 = 0;
            sn = s1;
            conex = Math.abs (con / wk);
            do {
               cn = ap * s2 / (sn + sp);
               s1 += cn;
               if (Math.abs (cn) <= conex)
                  break;
               sn = sp;
               sp += 1;
               s2 -= w2;
               w2 *= h4 / sp;
               ap = -ap * a2;
            } while (true);
            t = (Math.atan (wk) - wk * s1) / TWO_PI;
            b += sgn * t;
         }
         if (is >= 0)
            break;
      } while (ak != 0);

      if (b < EPSILON)
         b = 0;
      if (b > 1)
         b = 1;
      return b;
}\end{hide}
\end{code}
\begin{tabb}
  Computes the standard {\em binormal\/} distribution (\ref{eq:cdf2binormal})
  with the method described in \cite{tDON73a}, where \texttt{ndig} is the
  number of decimal digits of accuracy provided (\texttt{ndig} $\le 15$).
  The code was translated from the Fortran program written by T. G. Donnelly
  and copyrighted by the ACM (see 
  \url{http://www.acm.org/pubs/copyright_policy/#Notice}). The absolute error
  is expected to be smaller than $10^{-d}$, where $d=$ \texttt{ndig}.
\end{tabb}
\begin{code}

   public static double cdf (double mu1, double sigma1, double x, 
                             double mu2, double sigma2, double y,
                             double rho, int ndig) \begin{hide} {
      if (sigma1 <= 0)
         throw new IllegalArgumentException ("sigma1 <= 0");
      if (sigma2 <= 0)
         throw new IllegalArgumentException ("sigma2 <= 0");
      double X = (x - mu1)/sigma1;
      double Y = (y - mu2)/sigma2;
      return cdf (X, Y, rho, ndig);
   }\end{hide}
\end{code}
\begin{tabb} Computes the {\em binormal\/} distribution function 
   (\ref{eq:cdf1binormal}) with parameters $\mu_1$ = \texttt{mu1},
 $\mu_2$ = \texttt{mu2}, $\sigma_1$ = \texttt{sigma1},  $\sigma_2$ = \texttt{sigma2},
 correlation $\rho$ = \texttt{rho} and \texttt{ndig} decimal digits of accuracy.
\end{tabb}
\begin{code}

   public static double barF (double mu1, double sigma1, double x, 
                              double mu2, double sigma2, double y,
                              double rho, int ndig) \begin{hide} {
      if (sigma1 <= 0)
         throw new IllegalArgumentException ("sigma1 <= 0");
      if (sigma2 <= 0)
         throw new IllegalArgumentException ("sigma2 <= 0");
      double X = (x - mu1)/sigma1;
      double Y = (y - mu2)/sigma2;
      return barF (X, Y, rho, ndig);
   }\end{hide}
\end{code}
\begin{tabb} Computes the upper {\em binormal\/} distribution function
    (\ref{eq:cdf3binormal})  with parameters $\mu_1$ = \texttt{mu1},
 $\mu_2$ = \texttt{mu2}, $\sigma_1$ = \texttt{sigma1},  $\sigma_2$ = \texttt{sigma2},
 $\rho$ = \texttt{rho} and \texttt{ndig} decimal digits of accuracy.
\end{tabb}
\begin{code}

   public static double barF (double x, double y, double rho, int ndig) \begin{hide}  {
      return cdf (-x, -y, rho, ndig);
   }\end{hide}
\end{code}
\begin{tabb} Computes the upper {\em standard binormal\/} distribution function
    (\ref{eq:cdf3binormal})  with parameters $\rho$ = \texttt{rho} and 
   \texttt{ndig} decimal digits of accuracy.
\end{tabb}
\begin{code} \begin{hide} 

   public double cdf (double x, double y) {
      return cdf ((x-mu1)/sigma1, (y-mu2)/sigma2, rho, ndigit);
   }

   public static double cdf (double x, double y, double rho) {
       return cdf (x, y, rho, 15);
   }
 
   public static double cdf (double mu1, double sigma1, double x, 
                             double mu2, double sigma2, double y,
                             double rho) {
      return cdf (mu1, sigma1, x, mu2, sigma2, y, rho, 15);
   }

   public double barF (double x, double y) {
      return barF ((x-mu1)/sigma1, (y-mu2)/sigma2, rho, ndigit);
   }

   public static double barF (double mu1, double sigma1, double x, 
                              double mu2, double sigma2, double y,
                              double rho) {
      return barF (mu1, sigma1, x, mu2, sigma2, y, rho, 15);
   }

   public static double barF (double x, double y, double rho) {
      return barF (x, y, rho, 15);
   }
}\end{hide}
\end{code}
