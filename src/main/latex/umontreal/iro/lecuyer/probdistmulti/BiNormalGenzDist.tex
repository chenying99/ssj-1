\defclass {BiNormalGenzDist}
 
Extends the class \class{BiNormalDist} for the {\em bivariate 
normal\/} distribution\latex{ \cite[page 84]{tJOH72a}}
   using Genz's algorithm as described in \cite{tGEN04a}. 
  
\bigskip\hrule

%%%%%%%%%%%%%%%%%%%%%%%%%%%%%%%%%%%%%%%%
\begin{code}
\begin{hide}
/*
 * Class:        BiNormalGenzDist
 * Description:  bivariate normal distribution using Genz's algorithm
 * Environment:  Java
 * Software:     SSJ 
 * Organization: DIRO, Université de Montréal
 * @author       
 * @since
 */
\end{hide}
package umontreal.iro.lecuyer.probdistmulti;
\begin{hide}
import umontreal.iro.lecuyer.probdist.NormalDist;
\end{hide}

public class BiNormalGenzDist extends BiNormalDist \begin{hide} {

   private static final double[][] W = {
   //       Gauss Legendre points and weights, n =  6
   { 0.1713244923791705, 0.3607615730481384, 0.4679139345726904},

   //       Gauss Legendre points and weights, n = 12
   { 0.4717533638651177e-1, 0.1069393259953183, 0.1600783285433464,
     0.2031674267230659, 0.2334925365383547, 0.2491470458134029 },

   //       Gauss Legendre points and weights, n = 20
   { 0.1761400713915212e-1, 0.4060142980038694e-1, 0.6267204833410906e-1,
     0.8327674157670475e-1, 0.1019301198172404, 0.1181945319615184,   
     0.1316886384491766, 0.1420961093183821, 0.1491729864726037,   
     0.1527533871307259 }
   };

   private static final double[][] X = {
   //       Gauss Legendre points and weights, n =  6
   { 0.9324695142031522, 0.6612093864662647, 0.2386191860831970},

   //       Gauss Legendre points and weights, n = 12
   { 0.9815606342467191, 0.9041172563704750, 0.7699026741943050,
     0.5873179542866171, 0.3678314989981802, 0.1252334085114692 },

   //       Gauss Legendre points and weights, n = 20
   { 0.9931285991850949, 0.9639719272779138, 0.9122344282513259,    
     0.8391169718222188, 0.7463319064601508, 0.6360536807265150,    
     0.5108670019508271, 0.3737060887154196, 0.2277858511416451,    
     0.7652652113349733e-1 }
   };\end{hide}
\end{code}
%%%%%%%%%%%%%%%%%%%%%%%%%%%%%%%%%%%%%%%%%
\subsubsection* {Constructors}

\begin{code}

   public BiNormalGenzDist (double rho) \begin{hide} {
       super (rho);
   }\end{hide}
\end{code}
\begin{tabb}
 Constructs a \texttt{BiNormalGenzDist} object with default  parameters
 $\mu_1 = \mu_2 = 0$, $\sigma_1 = \sigma_2 = 1$ and correlation
 $\rho = $\texttt{ rho}.
  \end{tabb}
\begin{code}

   public BiNormalGenzDist (double mu1, double sigma1,
                            double mu2, double sigma2, double rho) \begin{hide} {
      super (mu1, sigma1, mu2, sigma2, rho);
   }\end{hide}
\end{code}
\begin{tabb}
 Constructs a \texttt{BiNormalGenzDist} object with parameters $\mu_1$ = \texttt{mu1},
 $\mu_2$ = \texttt{mu2}, $\sigma_1$ = \texttt{sigma1},  $\sigma_2$ = \texttt{sigma2}
 and $\rho$ = \texttt{rho}.   
  \end{tabb}

%%%%%%%%%%%%%%%%%%%%%%%%%%%%%%%%%%%
\subsubsection* {Methods}
\begin{code}

   public static double cdf (double x, double y, double rho) \begin{hide} {
      double bvn = specialCDF (x, y, rho, 40.0);
      if (bvn >= 0.0)
         return bvn;

/*
//   Copyright (C) 2005, Alan Genz,  All rights reserved.               
//
//   Redistribution and use in source and binary forms, with or without
//   modification, are permitted provided the following conditions are met:
//     1. Redistributions of source code must retain the above copyright
//        notice, this list of conditions and the following disclaimer.
//     2. Redistributions in binary form must reproduce the above copyright
//        notice, this list of conditions and the following disclaimer in the
//        documentation and/or other materials provided with the distribution.
//     3. The contributor name(s) may not be used to endorse or promote 
//        products derived from this software without specific prior written 
//        permission.
//   THIS SOFTWARE IS PROVIDED BY THE COPYRIGHT HOLDERS AND CONTRIBUTORS
//   "AS IS" AND ANY EXPRESS OR IMPLIED WARRANTIES, INCLUDING, BUT NOT 
//   LIMITED TO, THE IMPLIED WARRANTIES OF MERCHANTABILITY AND FITNESS 
//   FOR A PARTICULAR PURPOSE ARE DISCLAIMED. IN NO EVENT SHALL THE 
//   COPYRIGHT OWNER OR CONTRIBUTORS BE LIABLE FOR ANY DIRECT, INDIRECT, 
//   INCIDENTAL, SPECIAL, EXEMPLARY, OR CONSEQUENTIAL DAMAGES (INCLUDING, 
//   BUT NOT LIMITED TO, PROCUREMENT OF SUBSTITUTE GOODS OR SERVICES; LOSS 
//   OF USE, DATA, OR PROFITS; OR BUSINESS INTERRUPTION) HOWEVER CAUSED AND 
//   ON ANY THEORY OF LIABILITY, WHETHER IN CONTRACT, STRICT LIABILITY, OR 
//   TORT (INCLUDING NEGLIGENCE OR OTHERWISE) ARISING IN ANY WAY OUT OF THE 
//   USE OF THIS SOFTWARE, EVEN IF ADVISED OF THE POSSIBILITY OF SUCH DAMAGE.
//
//   function p = bvnl( dh, dk, r )
//
//  A function for computing bivariate normal probabilities.
//  bvnl calculates the probability that x < dh and y < dk. 
//    parameters  
//      dh 1st upper integration limit
//      dk 2nd upper integration limit
//      r   correlation coefficient
//
//   Author
//       Alan Genz
//       Department of Mathematics
//       Washington State University
//       Pullman, Wa 99164-3113
//       Email : alangenz@wsu.edu
//   This function is based on the method described by 
//        Drezner, Z and G.O. Wesolowsky, (1989),
//        On the computation of the bivariate normal inegral,
//        Journal of Statist. Comput. Simul. 35, pp. 101-107,
//    with major modifications for double precision, for |r| close to 1,
//    and for matlab by Alan Genz - last modifications 7/98.
//
//      p = bvnu( -dh, -dk, r );
//      return
//
//   end bvnl
//
//      function p = bvnu( dh, dk, r )
//
//  A function for computing bivariate normal probabilities.
//  bvnu calculates the probability that x > dh and y > dk. 
//    parameters  
//      dh 1st lower integration limit
//      dk 2nd lower integration limit
//      r   correlation coefficient
//
//   Author
//       Alan Genz
//       Department of Mathematics
//       Washington State University
//       Pullman, Wa 99164-3113
//       Email : alangenz@wsu.edu
//
//    This function is based on the method described by 
//        Drezner, Z and G.O. Wesolowsky, (1989),
//        On the computation of the bivariate normal inegral,
//        Journal of Statist. Comput. Simul. 35, pp. 101-107,
//    with major modifications for double precision, for |r| close to 1,
//    and for matlab by Alan Genz - last modifications 7/98.
//        Note: to compute the probability that x < dh and y < dk, use 
//              bvnu( -dh, -dk, r ). 
//
*/

      final double TWOPI = 2.0 * Math.PI;
      final double sqrt2pi = 2.50662827463100050241; // sqrt(TWOPI)
      double h, k, hk, hs, asr, sn, as, a, b, c, d, sp, rs, ep, bs, xs;
      int i, lg, ng, is;

      if (Math.abs (rho) < 0.3) {
         ng = 0;
         lg = 3;

      } else if (Math.abs (rho) < 0.75) {
         ng = 1;
         lg = 6;

      } else {
         ng = 2;
         lg = 10;
      }

      h = -x;
      k = -y;
      hk = h * k;
      bvn = 0;
      if (Math.abs (rho) < 0.925) {
         hs = (h * h + k * k) / 2.0;
         asr = Math.asin (rho);
         for (i = 0; i < lg; ++i) {
            sn = Math.sin (asr * (1.0 - X[ng][i]) / 2.0);
            bvn += W[ng][i] * Math.exp ((sn * hk - hs) / (1.0 - sn * sn));
            sn = Math.sin (asr * (1.0 + X[ng][i]) / 2.0);
            bvn += W[ng][i] * Math.exp ((sn * hk - hs) / (1.0 - sn * sn));
         }
         bvn =  bvn * asr /(4.0*Math.PI) + 
                NormalDist.cdf01 (-h) * NormalDist.cdf01 (-k);

      } else {
         if (rho < 0.0) {
            k = -k;
            hk = -hk;
         }
         if (Math.abs (rho) < 1.0) {
            as = (1.0 - rho) * (1.0 + rho);
            a = Math.sqrt (as);
            bs = (h - k) * (h - k);
            c = (4.0 - hk) / 8.0;
            d = (12.0 - hk) / 16.0;
            asr = -(bs / as + hk) / 2.0;
            if (asr > -100.0)
               bvn = a * Math.exp (asr) * (1.0 - c * (bs - as) * (1.0 -
                     d * bs / 5.0) / 3.0 + c * d * as * as / 5.0);

            if (-hk < 100.0) {
               b = Math.sqrt (bs);
               sp = sqrt2pi * NormalDist.cdf01 (-b / a);
               bvn = bvn - Math.exp (-hk / 2.0) * sp * b * (1.0 - c * bs * (1.0 -
                     d * bs / 5.0) / 3.0);
            }
            a = a / 2.0;
            for (i = 0; i < lg; ++i) {
               for (is = -1; is <= 1; is += 2) {
                  xs = (a * (is * X[ng][i] + 1.0));
                  xs = xs * xs;
                  rs = Math.sqrt (1.0 - xs);
                  asr = -(bs / xs + hk) / 2.0;
                  if (asr > -100.0) {
                     sp = (1.0 + c * xs * (1.0 + d * xs));
                     ep = Math.exp (-hk * (1.0 - rs) / (2.0 * (1.0 + rs))) / rs;
                     bvn += a * W[ng][i] * Math.exp (asr) * (ep - sp);
                  }
               }
            }
            bvn = -bvn / TWOPI;
         }
         if (rho > 0.0) {
            if (k > h)
               h = k;
            bvn += NormalDist.cdf01 (-h);
         }
         if (rho < 0.0) {
            xs = NormalDist.cdf01(-h) - NormalDist.cdf01(-k);
            if (xs < 0.0)
               xs = 0.0;
            bvn = -bvn + xs;
         }
      }
   if (bvn <= 0.0)
      return 0.0;
   if (bvn >= 1.0)
      return 1.0;
   return bvn;

   }\end{hide}
\end{code}
\begin{tabb}
% Same as 
%   \method{cdf}{double,double,double,double,double,double,double}~{\tt
%       (0, 1, x, 0, 1, y, rho)}. 
     Computes the standard {\em binormal\/} distribution (\ref{eq:cdf2binormal})
   with the method described in \cite{tGEN04a}. The code for the \texttt{cdf} 
  was translated directly from the Matlab code written by Alan Genz
  and available from his web page at
  \url{http://www.math.wsu.edu/faculty/genz/homepage}
   (the code is copyrighted by Alan Genz 
  and is included in this package with the kind permission of the author).
   The absolute error is expected to be smaller  than $0.5 \cdot 10^{-15}$.  
 \end{tabb}
\begin{code} \begin{hide} 
 
   public static double cdf (double mu1, double sigma1, double x, 
                             double mu2, double sigma2, double y,
                             double rho) {
      if (sigma1 <= 0)
         throw new IllegalArgumentException ("sigma1 <= 0");
      if (sigma2 <= 0)
         throw new IllegalArgumentException ("sigma2 <= 0");
      double X = (x - mu1)/sigma1;
      double Y = (y - mu2)/sigma2;
      return cdf (X, Y, rho);
   }

   public double cdf (double x, double y) {
      return cdf ((x-mu1)/sigma1, (y-mu2)/sigma2, rho);
   }

   public double barF (double x, double y) {
      return barF ((x-mu1)/sigma1, (y-mu2)/sigma2, rho);
   }

   public static double barF (double mu1, double sigma1, double x, 
                              double mu2, double sigma2, double y,
                              double rho) {
      if (sigma1 <= 0)
         throw new IllegalArgumentException ("sigma1 <= 0");
      if (sigma2 <= 0)
         throw new IllegalArgumentException ("sigma2 <= 0");
      double X = (x - mu1)/sigma1;
      double Y = (y - mu2)/sigma2;
      return barF (X, Y, rho);
   }

   public static double barF (double x, double y, double rho) {
      return cdf (-x, -y, rho);
   }
}\end{hide}
\end{code}
