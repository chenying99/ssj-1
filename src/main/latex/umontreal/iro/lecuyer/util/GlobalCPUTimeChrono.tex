\defclass {GlobalCPUTimeChrono}

Extends the \class{AbstractChrono}  class to compute the global CPU time used
by the Java Virtual Machine. This includes CPU time taken by any thread,
including the garbage collector, class loader, etc.

Part of this class is implemented in the C language and the 
implementation is unfortunately operating system-dependent.
The C functions for the current class have been compiled on a 32-bit machine
running Linux. % and will not work on 64-bit machines.
For a {\em platform-independent} CPU timer (valid only with Java--1.5 or later),
 one should use the class \class{ThreadCPUTimeChrono} which is programmed 
directly in Java.

%%%%%%%%%%%%%%%%%%%%%%%%%%%%%%
\bigskip\hrule
\begin{code}
\begin{hide}
/*
 * Class:        GlobalCPUTimeChrono
 * Description:  Compute the global CPU time used by the Java Virtual Machine
 * Environment:  Java
 * Software:     SSJ 
 * Copyright (C) 2001  Pierre L'Ecuyer and Université de Montréal
 * Organization: DIRO, Université de Montréal
 * @author       
 * @since

 * SSJ is free software: you can redistribute it and/or modify it under
 * the terms of the GNU General Public License (GPL) as published by the
 * Free Software Foundation, either version 3 of the License, or
 * any later version.

 * SSJ is distributed in the hope that it will be useful,
 * but WITHOUT ANY WARRANTY; without even the implied warranty of
 * MERCHANTABILITY or FITNESS FOR A PARTICULAR PURPOSE.  See the
 * GNU General Public License for more details.

 * A copy of the GNU General Public License is available at
   <a href="http://www.gnu.org/licenses">GPL licence site</a>.
 */
\end{hide}
package umontreal.iro.lecuyer.util;


public class GlobalCPUTimeChrono extends AbstractChrono\begin{hide} {
   
   private static boolean ssjUtilLoaded = false;
   private static native void Heure (long[] tab);

   protected void getTime (long[] tab) {
      if (!ssjUtilLoaded) {
         // We cannot load the library in a static block
         // since the subclass ChronoSingleThread does not
         // need the native method.
         System.loadLibrary ("ssjutil");

         ssjUtilLoaded = true;
      }
      
      Heure (tab);
   }
\end{hide}
\end{code}

%%%%%%%%%%%%%%%%%%%%%%%
\subsubsection*{Constructor}

\begin{code}

   public GlobalCPUTimeChrono()\begin{hide} {
      init();
   }\end{hide}
\end{code}
  \begin{tabb} Constructs a \texttt{Chrono} object and initializes it to zero.
  \end{tabb}

\begin{code}\begin{hide}
}\end{hide}
\end{code}
