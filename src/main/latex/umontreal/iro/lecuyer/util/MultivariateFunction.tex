\defclass{MultivariateFunction}

Represents a function of multiple variables.
This interface specifies a method \texttt{evaluate} that computes
a $g(\mathbf{x})$ function, where $\mathbf{x}=(x_0,\ldots,x_{d-1})\in\RR^d$.  It also specifies
a method \texttt{evaluateGradient} for computing
its gradient $\nabla g(\mathbf{x})$.

The dimension $d$ can be fixed or variable.  When $d$ is fixed, the
methods specified by this interface always take the same number of
arguments.  This is the case, for example, with a ratio of two
variables.
When $d$ is variable, the implementation can compute the
function for a vector $\mathbf{x}$ of any length.  This can happen for a
product or sum of variables.

The methods of this interface take a variable number of arguments to
accomodate the common case of fixed dimension with more convenience;
the programmer can call the method without creating an array.
For the generic case, however, one can replace the arguments with an
array.

\bigskip\hrule

\begin{code}\begin{hide}
/*
 * Class:        MultivariateFunction
 * Description:  Represents a function of multiple variables.
 * Environment:  Java
 * Software:     SSJ 
 * Copyright (C) 2001  Pierre L'Ecuyer and Université de Montréal
 * Organization: DIRO, Université de Montréal
 * @author       
 * @since

 * SSJ is free software: you can redistribute it and/or modify it under
 * the terms of the GNU General Public License (GPL) as published by the
 * Free Software Foundation, either version 3 of the License, or
 * any later version.

 * SSJ is distributed in the hope that it will be useful,
 * but WITHOUT ANY WARRANTY; without even the implied warranty of
 * MERCHANTABILITY or FITNESS FOR A PARTICULAR PURPOSE.  See the
 * GNU General Public License for more details.

 * A copy of the GNU General Public License is available at
   <a href="http://www.gnu.org/licenses">GPL licence site</a>.
 */
\end{hide}
package umontreal.iro.lecuyer.util;


public interface MultivariateFunction\begin{hide} {\end{hide}

   public int getDimension();
\end{code}
\begin{tabb} Returns $d$, the dimension of the function computed
   by this implementation.  If the dimension is not fixed,
   this method must return a negative value.
\end{tabb}
\begin{htmlonly}
   \return{the dimension.}
\end{htmlonly}
\begin{code}

   public double evaluate (double... x);
\end{code}
\begin{tabb}   Computes the function $g(\mathbf{x})$
 for the vector \texttt{x}.  The length of the
 given array must correspond to the dimension of
 this function.  The method must compute and return the result
 of the function without modifying the elements
 in \texttt{x} since the array can be reused for
 further computation.
\end{tabb}
\begin{htmlonly}
   \param{x}{a vector $\mathbf{x}$.}
   \return{the value of $g(\mathbf{x})$.}
   \exception{NullPointerException}{if \texttt{x} is \texttt{null}.}
   \exception{IllegalArgumentException}{if \texttt{x.length}
    does not correspond to the dimension of this function.}
\end{htmlonly}
\begin{code}

   public double evaluateGradient (int i, double... x);\end{code}
\begin{tabb}   Computes $\partial g(\mathbf{x})/\partial x_i$,
 the derivative of $g(\mathbf{x})$
 with respect to $x_i$.    The length of the
 given array must correspond to the dimension of
 this function.  The method must compute and return the result
 of the derivative without modifying the elements
 in \texttt{x} since the array can be reused for
 further computations, e.g., the gradient $\nabla g(\mathbf{x})$.
\end{tabb}
\begin{htmlonly}
   \param{i}{the variable to derive with respect to.}
   \param{x}{a vector $\mathbf{x}$.}
   \return{the value of the partial derivative.}
   \exception{NullPointerException}{if \texttt{x} is \texttt{null}.}
   \exception{IllegalArgumentException}{if \texttt{x.length}
    does not correspond to the dimension of this function.}
   \exception{IndexOutOfBoundsException}{if \texttt{i} is negative
    or greater than or equal to the dimension of this function.}
\end{htmlonly}
\begin{code}\begin{hide}
}\end{hide}
\end{code}
