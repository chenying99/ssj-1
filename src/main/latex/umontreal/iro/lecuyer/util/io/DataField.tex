\defclass{DataField}

This class represents a data field from a file read by an instance of a class
 implementing \class{DataReader}.

\bigskip\hrule
%%%%%%%%%%%%%%%%%%%%%%%%%%%%%%%%%%%%%%%%%%%%%%%%%%%%%%%%%%%%%%%%%%

\begin{code}
\begin{hide}
/*
 * Class:        DataField
 * Description:  Represents a data field
 * Environment:  Java
 * Software:     SSJ 
 * Copyright (C) 2001  Pierre L'Ecuyer and Université de Montréal
 * Organization: DIRO, Université de Montréal
 * @author       David Munger 
 * @since        August 2009

 * SSJ is free software: you can redistribute it and/or modify it under
 * the terms of the GNU General Public License (GPL) as published by the
 * Free Software Foundation, either version 3 of the License, or
 * any later version.

 * SSJ is distributed in the hope that it will be useful,
 * but WITHOUT ANY WARRANTY; without even the implied warranty of
 * MERCHANTABILITY or FITNESS FOR A PARTICULAR PURPOSE.  See the
 * GNU General Public License for more details.

 * A copy of the GNU General Public License is available at
   <a href="http://www.gnu.org/licenses">GPL licence site</a>.
 */
\end{hide}
package umontreal.iro.lecuyer.util.io;
\begin{hide}
import java.lang.reflect.Array;
\end{hide}

public class DataField \begin{hide} {
   protected String label;
   protected Object data;
   protected int effectiveLength;
\end{hide}
\end{code}

%%%%%%%%%%%%%%%%%%%%%%%%%%%%
\subsubsection*{Constructors}

\begin{code}
   public DataField (String label, Object data) \begin{hide} {
      this(label, data, -1);
   }
   \end{hide}
\end{code}
\begin{tabb}
Constructor. Creates a field named \texttt{label} of value \texttt{data}.
\end{tabb}
\begin{htmlonly}
\param{label}{name of the field}
\param{data}{value of the field}
\end{htmlonly}
\begin{code}
   
   public DataField (String label, Object data, int effectiveLength) \begin{hide} {
      this.label = label;
      this.data = data;
      this.effectiveLength = effectiveLength;
   }
   \end{hide}
\end{code}
\begin{tabb}
Constructor. Creates a field named \texttt{label} of value \texttt{data}.
 \texttt{effectiveLength} is the number of significant elements contained in
 \texttt{data} if it is an array.
\end{tabb}
\begin{htmlonly}
\param{label}{name of the field}
\param{data}{value of the field}
\param{effectiveLength}{number of significant elements contained in \texttt{data}}
\end{htmlonly}


%%%%%%%%%%%%%%%%%%%%%%%%%%%%
\subsubsection*{Information on the field}

\begin{code}
   public String getLabel() \begin{hide} {
      return label;
   }
   \end{hide}
\end{code}
\begin{tabb}
Returns the field label (or name).
\end{tabb}
\begin{code}

   public Class getType() \begin{hide} {
      return data.getClass();
   }
   \end{hide}
\end{code}
\begin{tabb}
Returns the type of the field.
\end{tabb}
\begin{code}
   
   public boolean isAtomic() \begin{hide} {
      return !isArray();
   }
   \end{hide}
\end{code}
\begin{tabb}
Returns \texttt{true} if the field value is atomic data.
\end{tabb}
\begin{code}
   
   public boolean isArray() \begin{hide} {
      return data.getClass().isArray();
   }
   \end{hide}
\end{code}
\begin{tabb}
Returns \texttt{true} if the field contains an array.
\end{tabb}
\begin{code}

   public boolean isArray2D() \begin{hide} {
      return isArray() && Array.get(data, 0).getClass().isArray();
   }
   \end{hide}
\end{code}
\begin{tabb}
Returns \texttt{true} if the field contains a two-dimensional array.
\end{tabb}
\begin{code}
   
   public int getArrayLength() \begin{hide} {
      if (!isArray()) return -1;
      if (effectiveLength < 0) return Array.getLength(data);
      return effectiveLength;
   }
   \end{hide}
\end{code}
\begin{tabb}
Returns the length of the array contained by the field, or \texttt{-1}
  if it is not an array.
\end{tabb}
\begin{code}

   public boolean isString() \begin{hide} {
      return (data instanceof String);
   }
   \end{hide}
\end{code}
\begin{tabb}
Returns \texttt{true} if the field value is an atomic \texttt{String}.
\end{tabb}
\begin{code}

   public boolean isInt() \begin{hide} {
      return (data instanceof Integer);
   }
   \end{hide}
\end{code}
\begin{tabb}
Returns \texttt{true} if the field value is an atomic \texttt{int}.
\end{tabb}
\begin{code}
   
   public boolean isFloat() \begin{hide} {
      return (data instanceof Float);
   }
   \end{hide}
\end{code}
\begin{tabb}
Returns \texttt{true} if the field value is an atomic \texttt{float}.
\end{tabb}
\begin{code}

   public boolean isDouble() \begin{hide} {
      return (data instanceof Double);
   }
   \end{hide}
\end{code}
\begin{tabb}
Returns \texttt{true} if the field value is an atomic \texttt{double}.
\end{tabb}
      

%%%%%%%%%%%%%%%%%%%%%%%%%%%%
\subsubsection*{Obtaining the value as atomic data}

\begin{code}

   public String asString() \begin{hide} {
      return (data instanceof String) ? (String)data : null;
   }
   \end{hide}
\end{code}
\begin{tabb}
Returns the value as \texttt{String}, or \texttt{null} if it is not
  of type \texttt{String}.
See \method{isString}{}.
\end{tabb}
\begin{code}
   
   public int asInt() \begin{hide} {
      return (data instanceof Integer) ? ((Integer)data).intValue() : 0;
   }
   \end{hide}
\end{code}
\begin{tabb}
Returns the value as \texttt{int} or \texttt{0} if it is not of type \texttt{int}
See \method{isInt}{}.
\end{tabb}
\begin{code}
   
   public float asFloat() \begin{hide} {
      return (data instanceof Float) ? ((Float)data).floatValue() : 0;
   }
   \end{hide}
\end{code}
\begin{tabb}
Returns the value as \texttt{float} or \texttt{0} if it is not of type \texttt{float}
See \method{isFloat}{}.
\end{tabb}
\begin{code}
   
   public double asDouble() \begin{hide} {
      return (data instanceof Double) ? ((Double)data).doubleValue() : 0;
   }
   \end{hide}
\end{code}
\begin{tabb}
Returns the value as \texttt{double} or \texttt{0} if it is not of type \texttt{double}
See \method{isDouble}{}.
\end{tabb}

   
%%%%%%%%%%%%%%%%%%%%%%%%%%%%
\subsubsection*{Obtaining the value as a one-dimensional array}

\begin{code}

   public String[] asStringArray() \begin{hide} {
      return (data instanceof String[]) ? (String[])data : null;
   }
   \end{hide}
\end{code}
\begin{tabb}
Returns the value as one-dimensional \texttt{String} array or \texttt{null} if it is not of type \texttt{String[]}.
\end{tabb}
\begin{code}
      
   public int[] asIntArray() \begin{hide} {
      return (data instanceof int[]) ? (int[])data : null;
   }
   \end{hide}
\end{code}
\begin{tabb}
Returns the value as one-dimensional \texttt{int} array or \texttt{null} if it is not of type \texttt{int[]}.
\end{tabb}
\begin{code}
   
   public float[] asFloatArray() \begin{hide} {
      return (data instanceof float[]) ? (float[])data : null;
   }
   \end{hide}
\end{code}
\begin{tabb}
Returns the value as one-dimensional \texttt{float} array or \texttt{null} if it is not of type \texttt{float[]}.
\end{tabb}
\begin{code}
   
   public double[] asDoubleArray() \begin{hide} {
      return (data instanceof double[]) ? (double[])data : null;
   }
   \end{hide}
\end{code}
\begin{tabb}
Returns the value as one-dimensional \texttt{double} array or \texttt{null} if it is not of type \texttt{double[]}.
\end{tabb}
   

%%%%%%%%%%%%%%%%%%%%%%%%%%%%
\subsubsection*{Obtaining the value as a two-dimensional array}

\begin{code}
   
   public String[][] asStringArray2D() \begin{hide} {
      return (data instanceof String[][]) ? (String[][])data : null;
   }
   \end{hide}
\end{code}
\begin{tabb}
Returns the value as two-dimensional \texttt{String} array or \texttt{null} if it is not of type \texttt{String[][]}.
\end{tabb}
\begin{code}

   public int[][] asIntArray2D() \begin{hide} {
      return (data instanceof int[][]) ? (int[][])data : null;
   }
   \end{hide}
\end{code}
\begin{tabb}
Returns the value as two-dimensional \texttt{int} array or \texttt{null} if it is not of type \texttt{int[][]}.
\end{tabb}
\begin{code}
   
   public float[][] asFloatArray2D() \begin{hide} {
      return (data instanceof float[][]) ? (float[][])data : null;
   }
   \end{hide}
\end{code}
\begin{tabb}
Returns the value as two-dimensional \texttt{float} array or \texttt{null} if it is not of type \texttt{float[][]}.
\end{tabb}
\begin{code}

   public double[][] asDoubleArray2D() \begin{hide} {
      return (data instanceof double[][]) ? (double[][])data : null;
   }
   \end{hide}
\end{code}
\begin{tabb}
Returns the value as two-dimensional \texttt{double} array or \texttt{null} if it is not of type \texttt{double[][]}.
\end{tabb}

%%%%%%%%%%%%%%%%%%%%%%%%%%%%
\subsubsection*{Obtaining the value as an \texttt{Object}}

\begin{code}
   public Object asObject() \begin{hide} {
      return data;
   }
   \end{hide}
\end{code}
\begin{tabb}
Returns the value of the field as an \texttt{Object}.
\end{tabb}
   
\begin{code}\begin{hide}
}
\end{hide}\end{code}
