\defclass{SystemTimeChrono}

Extends the \class{AbstractChrono} class to compute
the total system time using Java's builtin \texttt{System.nanoTime}.
The system can be used as a rough approximation of the CPU time taken
by a program if no
other tasks are executed on the host while the program is running.

%%%%%%%%%%%%%%%%%%%%%%%%%%%%%%%%%%%%%%%%%%%%%%%%%%%%%%%%%%
\bigskip\hrule

\begin{code}
\begin{hide}
/*
 * Class:        SystemTimeChrono
 * Description:  
 * Environment:  Java
 * Software:     SSJ 
 * Copyright (C) 2001  Pierre L'Ecuyer and Université de Montréal
 * Organization: DIRO, Université de Montréal
 * @author       Éric Buist
 * @since

 * SSJ is free software: you can redistribute it and/or modify it under
 * the terms of the GNU General Public License (GPL) as published by the
 * Free Software Foundation, either version 3 of the License, or
 * any later version.

 * SSJ is distributed in the hope that it will be useful,
 * but WITHOUT ANY WARRANTY; without even the implied warranty of
 * MERCHANTABILITY or FITNESS FOR A PARTICULAR PURPOSE.  See the
 * GNU General Public License for more details.

 * A copy of the GNU General Public License is available at
   <a href="http://www.gnu.org/licenses">GPL licence site</a>.
 */
\end{hide}
package umontreal.iro.lecuyer.util;


public class SystemTimeChrono extends AbstractChrono\begin{hide} {

   protected void getTime (long[] tab) {
      long rawTime = System.nanoTime();
      final long DIV = 1000000000L;
      long seconds = rawTime/DIV;
      long micros = (rawTime % DIV)/1000L;
      tab[0] = seconds;
      tab[1] = micros;
   }\end{hide}
\end{code}

%%%%%%%%%%%%%%%%%%%%%%%
\subsubsection*{Constructor}

\begin{code}

   public SystemTimeChrono()\begin{hide} {
      super();
      init();
   }\end{hide}
\end{code}
  \begin{tabb} Constructs a new chrono object and
    initializes it to zero.
  \end{tabb}
\begin{code}
\begin{hide}
}\end{hide}
\end{code}
