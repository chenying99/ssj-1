\defclass {FBar}

This class is similar to \class{FDist}, except that it provides static methods
to compute or approximate the complementary distribution function of $X$,
which we define as $\bar F (x) = P[X\ge x]$, instead of $F (x)=P[X\le x]$.
Note that with our definition of $\bar F$, one has
$\bar F (x) = 1 - F (x)$ for continuous distributions and
$\bar F (x) = 1 - F (x-1)$ for discrete distributions over the integers.

\bigskip\hrule

\begin{code}
\begin{hide}
/*
 * Class:        FBar
 * Description:  Complementary empirical distributions
 * Environment:  Java
 * Software:     SSJ 
 * Copyright (C) 2001  Pierre L'Ecuyer and Université de Montréal
 * Organization: DIRO, Université de Montréal
 * @author       
 * @since

 * SSJ is free software: you can redistribute it and/or modify it under
 * the terms of the GNU General Public License (GPL) as published by the
 * Free Software Foundation, either version 3 of the License, or
 * any later version.

 * SSJ is distributed in the hope that it will be useful,
 * but WITHOUT ANY WARRANTY; without even the implied warranty of
 * MERCHANTABILITY or FITNESS FOR A PARTICULAR PURPOSE.  See the
 * GNU General Public License for more details.

 * A copy of the GNU General Public License is available at
   <a href="http://www.gnu.org/licenses">GPL licence site</a>.
 */
\end{hide}
package umontreal.iro.lecuyer.gof;
\begin{hide}
import umontreal.iro.lecuyer.probdist.*;
\end{hide}

public class FBar\begin{hide} {
   private FBar() {}

   private static final double EPSILONSCAN = 1.0E-7;

   private static double scanGlaz (int n, double d, int m) {
      int j, jmoy;
      double temp;
      double jr, jm1r, nr = n;
      int signe;
      double q = 1.0 - d;
      double q4, q3, q2, q1;
      double bin, binMoy;

      jmoy = (int)((n + 1)*d);              // max term of the Binomial
      if (jmoy < m - 1)
         jmoy = m - 1;

      /*---------------------------------------------------------
       * Compute q1: formula (2.5) in Glaz (1989)
       * Compute q2: formula (A.6) in Berman and Eagleson (1985)
       * Compute q3, q4 : Theorem (3.2) in Glaz (1989)
       *---------------------------------------------------------*/

      // compute the probability of term j = jmoy
      q1 = 0.0;
      for (j = 1; j <= jmoy; j++) {
         jr = j;
         q1 += Math.log (nr - jr + 1.0) - Math.log (jr);
      }
      q1 += jmoy*Math.log (d) + (nr - jmoy)*Math.log (q);
      binMoy = Math.exp (q1);
      q1 = binMoy;
      jm1r = jmoy - m + 1;
      if (((jmoy - m + 1) & 1) != 0)
         signe = -1;
      else
         signe = 1;
      q2 = signe*binMoy;
      q3 = signe*binMoy*(2.0 - jm1r*jm1r + jm1r);
      q4 = signe*binMoy*(jm1r + 1.0)*(jm1r + 2.0)*(6.0 + jm1r*jm1r -
         5.0*jm1r);

      // compute the probability of terms j > jmoy
      if (((jmoy - m + 1) & 1) != 0)
         signe = -1;
      else
         signe = 1;

      jm1r = jmoy - m + 1;
      bin = binMoy;
      for (j = jmoy + 1; j <= n; j++) {
         jr = j;
         jm1r += 1.0;
         signe = -signe;
         bin = (bin*(nr - jr + 1.0)*d)/(jr*q);
         if (bin < EPSILONSCAN)
            break;
         q1 += bin;
         q2 += signe*bin;
         q3 += signe*bin*(2.0 - jm1r*jm1r + jm1r);
         q4 += signe*bin*(jm1r + 1.0)*(jm1r + 2.0)*(6.0 + jm1r*jm1r -
            5.0*jm1r);
      }

      q1 = 1.0 - q1;
      q3 /= 2.0;
      q4 /= 12.0;
      if (m == 3) {
        // Problem with this formula; I do not get the same results as Glaz
         q4 = ((nr*(nr - 1.0)*d*d*Math.pow (q, nr - 2.0))/8.0
            + nr*d*2.0*Math.pow (1.0 - 2.0*d, nr - 1.0))
            - 4.0*Math.pow (1.0 - 2.0*d, nr);
         if (d < 1.0/3.0) {
            q4 += nr*d*2.0*Math.pow (1.0 - 3.0*d, nr - 1.0)
                  + 4.0*Math.pow (1.0 - 3.0*d, nr);
         }
      }
      // compute probability: Glaz, equations (3.2) and (3.3)
      q3 = q1 - q2 - q3;
      q4 = q3 - q4;
      //when the approximation is bad, avoid overflow
      temp = Math.log (q3) + (nr - m - 2.0)*Math.log (q4/q3);
      if (temp >= 0.0)
         return 0.0;
      if (temp < -30.0)
         return 1.0;
      q4 = Math.exp (temp);
      return 1.0 - q4;
     }

     //-----------------------------------------------------------------

     private static double scanWNeff (int n, double d, int m) {
      double q = 1.0 - d;
      double temp;
      double bin;
      double sum;
      int j;

      /*--------------------------------------
       * Anderson-Titterington: equation (4)
       *--------------------------------------*/

      // compute the probability of term j = m
      sum = 0.0;
      for (j = 1; j <= m; j++)
         sum += Math.log ((double)(n - j + 1)) - Math.log ((double)j);

      sum += m*Math.log (d) + (n - m)*Math.log (q);
      bin = Math.exp (sum);
      temp = (m/d - n - 1.0)*bin;
      sum = bin;

      // compute the probability of terms j > m
      for (j = m + 1; j <= n; j++) {
         bin *= (n - j + 1)*d/(j*q);
         if (bin < EPSILONSCAN)
            break;
         sum += bin;
      }
      sum = 2.0*sum + temp;
      return sum;
     }

     //----------------------------------------------------------------

     private static double scanAsympt (int n, double d, int m) {
      double kappa;
      double temp;
      double theta;
      double sum;

      /*--------------------------------------------------------------
       * Anderson-Titterington: asymptotic formula after equation (4)
       *--------------------------------------------------------------*/

      theta = Math.sqrt (d/(1.0 - d));
      temp = Math.sqrt ((double)n);
      kappa = m/(d*temp) - temp;
      temp = theta*kappa;
      temp = temp*temp/2.0;
      sum = 2.0*(1.0 - NormalDist.cdf01 (theta*kappa)) +
         (kappa*theta*Math.exp (-temp))/(d*Math.sqrt (2.0*Math.PI));
      return sum;
   }\end{hide}

   public static double scan (int n, double d, int m)\begin{hide} {
      double mu;
      double prob;

      if (n < 2)
        throw new IllegalArgumentException ("Calling scan with n < 2");
      if (d <= 0.0 || d >= 1.0)
         throw new IllegalArgumentException ("Calling scan with "+
                    "d outside (0,1)");

      if (m > n)
         return 0.0;
      if (m <= 1)
         return 1.0;
      if (m <= 2) {
         if ((n - 1)*d >= 1.0)
            return 1.0;
         return 1.0 - Math.pow (1.0 - (n - 1)*d, (double)n);
      }
      if (d >= 0.5 && m <= (n + 1)/2.0)
         return 1.0;
      if (d > 0.5)
        return -1.0;              // Error
      // util_Assert (d <= 0.5, "Calling fbar_Scan with d > 1/2");

      mu = n*d;                    // mean of a binomial
      if (m <= mu + d)
         return 1.0;
      if (mu <= 10.0)
         return scanGlaz (n, d, m);
      prob = scanAsympt (n, d, m);
      if ((d >= 0.3 && n >= 50.0) || (n*d*d >= 250.0 && d < 0.3)) {
         if (prob <= 0.4)
            return prob;
      }
      prob = scanWNeff (n, d, m);
      if (prob <= 0.4)
         return prob;
      prob = scanGlaz (n, d, m);
      if (prob > 0.4 && prob <= 1.0)
         return prob;
      return 1.0;
   }
}\end{hide}
\end{code}
 \begin{tabb} Return $P[S_N (d) \ge m]$, where $S_N (d)$ is the scan
 statistic\latex{(see \cite{tGLA89a,tGLA01a} and
   \externalmethod{}{GofStat}{scan}{int,double,int}),}\html{.  It is} defined as
  \eq
    S_N (d) = \latex{\sup}\html{sup}_{0\le y\le 1-d} \eta[y,\,y+d],    \latex{\eqlabel{eq:scan}}
  \endeq
  where $d$ is a constant in $(0, 1)$,
  $\eta[y,\,y+d]$ is the number of observations falling inside
  the interval $[y, y+d]$, from a sample of $N$ i.i.d.\ $U (0,1)$
  random variables.
\begin{latexonly}
  One has (see \cite {tAND95b}),
  \begin{eqnarray}
   P[S_N (d) \ge m]
    &\approx& \left (\frac{m}{d}-N-1\right) b (m)
              + 2 \sum_{i=m}^{N} b (i)            \eqlabel{DistScan1} \\[6pt]
    &\approx& 2 (1-\Phi (\theta\kappa)) + \theta\kappa
              \frac{\exp(-\theta^2\kappa^2 /2)}{d \sqrt{2\pi}}
                                                 \eqlabel {DistScan2}
  \end{eqnarray}
   where $\Phi$ is the standard normal distribution function.
%  (\ref{eq:cdfnormal}),
  \begin{eqnarray*}
   b (i)    &=& \binom{N}{i} d^i (1-d)^{N-i}, \\[4pt]
   \theta  &=& \sqrt{\frac d{1-d}}, \\[4pt]
   \kappa  &=& \frac m{d \sqrt{N}} - \sqrt{N}.
  \end{eqnarray*}
  For $d \le 1/2$, (\ref{DistScan1}) is exact for $m > N/2$,
  but only an approximation otherwise.
  The approximation (\ref{DistScan2}) is good when
  $N d^2$ is large or when $d > 0.3$ and $N>50$.
  In other cases, this implementation sometimes use the approximation
  proposed by Glaz \cite{tGLA89a}.
  For more information, see \cite {tAND95b,tGLA89a,tWAL87a}.
\end{latexonly}
  The approximation returned by this function is generally good when
  it is close to 0, but is not very reliable when it exceeds, say, 0.4.
\begin{detailed}  %%%%%%
  If $m \le (N + 1)d$, the method returns 1.
  Else, if $Nd \le 10$, it returns the approximation given by
  Glaz \cite{tGLA89a}.
  If $Nd > 10$, it computes (\ref{DistScan2}) or (\ref{DistScan1})
  and returns the result if it does not exceed 0.4, otherwise it computes
  the approximation from \cite{tGLA89a}, returns it if it is less than 1.0,
% inside the interval $(0.4, 1.0)$,
  and returns 1.0 otherwise.
 \hpierre{Even if it 0.40001 in the first approximation, and 0.3999
   in the second one?}
 \hrichard{C'est ce qui est programm\'e. Probablement que dans ce cas,
   l'approximation est tellement mauvaise que nous avons d\'ecid\'e de
   retourner 1. On pourrait retourner 0 au lieu??}
 \hpierre {\c Ca n'a pas de bon sens.  Je ne vois pas pourquoi elle serait
   ``tellement mauvaise'' dans le cas cit\'e avec 0.3999 et subitement
   bonne si la seconde approximation retourne 0.4.
   Il me semble que dans un tel cas on devrait retourner quelque chose
   aux alentours de 0.4.
   Et ce qui est g\^enant c'est qu'en retournant 0 ou 1 comme p-valeur,
   on croira que l'hypoth\`ese nulle est rejet\'ee!  }
  The relative error can
  reach 10\%\ when $Nd \le 10$ or when the returned value
  is less than 0.4.  For $m > Nd$ and $Nd > 10$, a returned value
  that exceeds $0.4$ should be regarded as unreliable.
  For $m = 3$, the returned values are totally unreliable.
  (There may be an error in the original formulae in \cite{tGLA89a}).
\end{detailed}  %%%%
  Restrictions: $N \ge 2$  and $d \le 1/2$.\\
\end{tabb}
\begin{htmlonly}
   \param{n}{sample size ($\ge 2$)}
   \param{d}{length of the test interval ($\in(0,1)$)}
   \param{m}{scan statistic}
   \return{the complementary distribution function of the statistic evaluated at \texttt{m}}
\end{htmlonly}
