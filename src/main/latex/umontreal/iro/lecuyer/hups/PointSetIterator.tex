\defclass{PointSetIterator}

Objects of classes that implement this interface are \emph{iterators} 
that permit one to enumerate
(or observe) the successive points of a point set and the successive 
coordinates of these points.
Each \class{PointSetIterator}{} is associated with a given point set
and maintains a \emph{current point} index $i$ and a \emph{current coordinate}
index $j$, which are both initialized to zero.

Successive coordinates can be accessed one or many at a time by the methods
\method{nextCoor\-di\-nate}{} and \method{nextCoordi\-nates}{}, respectively.
The current coordinate index $j$ can be set explicitely by
\method{setCurCoordIndex}{} and \method{resetCurCoordIndex}{}. 
Similar methods are available for resetting and accessing the current point.
The method \method{nextPoint}{} permits one to 
enumerate the successive points in natural order. 

This class also implements the \class{RandomStream} interface.
This permits one to replace random numbers by the coordinates of
(randomized) quasi-Monte Carlo points without changing the code that calls
the generators in a simulation program.
That is, the same simulation program can be used for both Monte Carlo 
and quasi-Monte Carlo simulations.
The method \method{nextDouble}{} does exactly the same as 
\method{nextCoordinate}{}, it returns the current coordinate of the 
current point and advances the current coordinate by one.  
The substreams correspond to the points, so  
\method{resetStartSubstream}{} resets the current point coordinate to zero, 
\method{resetNextSubstream}{} resets the iterator to the next point, and
\method{resetStartStream}{} resets the iterator to the first point of 
the point set.

There can be several iterators over the same point set.
These iterators are independent from each other.  
Classes that implement this interface must maintain enough information 
so that each iterator is unaffected by other iterator's operations.
However, the iterator does not need to be independent of the underlying 
point set.  If the point set is modified (e.g., randomized), the iterator
may continue to work as usual.

Point set iterators are implemented as inner classes because
this gives a direct access to the private members (or variables) 
of the class.  This is important for efficiency.
They are quite similar to the iterators in Java \emph{collections}.

%%%%%%%%%%%%%%%%%%%%%%%%%%%%%%%%%%%%%%%%%%%%%%%%%%%%%%%%%%%%%%%%%%%%
\bigskip\hrule\bigskip

\begin{code}
\begin{hide}
/*
 * Interface:    PointSetIterator
 * Description:  Iterator over point sets
 * Environment:  Java
 * Software:     SSJ 
 * Copyright (C) 2001  Pierre L'Ecuyer and Université de Montréal
 * Organization: DIRO, Université de Montréal
 * @author       
 * @since

 * SSJ is free software: you can redistribute it and/or modify it under
 * the terms of the GNU General Public License (GPL) as published by the
 * Free Software Foundation, either version 3 of the License, or
 * any later version.

 * SSJ is distributed in the hope that it will be useful,
 * but WITHOUT ANY WARRANTY; without even the implied warranty of
 * MERCHANTABILITY or FITNESS FOR A PARTICULAR PURPOSE.  See the
 * GNU General Public License for more details.

 * A copy of the GNU General Public License is available at
   <a href="http://www.gnu.org/licenses">GPL licence site</a>.
 */
\end{hide}
package umontreal.iro.lecuyer.hups;\begin{hide}

import umontreal.iro.lecuyer.rng.RandomStream;\end{hide}

public interface PointSetIterator extends RandomStream\begin{hide} {\end{hide}

   public void setCurCoordIndex (int j);
\end{code}
 \begin{tabb}
   Sets the current coordinate index to $j$, so that 
   the next calls to \method{nextCoordinate}{} or  \method{nextCoordinates}{}
   will return the values $u_{i,j}, u_{i,j+1}, \dots$, where $i$ is the
   index of the current point.
 \end{tabb}
\begin{htmlonly}
   \param{j}{index of the new current coordinate}
\end{htmlonly}
\begin{code}

   public void resetCurCoordIndex();
\end{code}
 \begin{tabb}
  Equivalent to \method{setCurCoordIndex}{}~\texttt{(0)}.
 \end{tabb}
\begin{code}

   public int getCurCoordIndex();
\end{code}
 \begin{tabb}
   Returns the index $j$ of the current coordinate.  This may be useful,
   e.g., for testing if enough coordinates are still available.
 \end{tabb}
\begin{htmlonly}
   \return{index of the current coordinate}
\end{htmlonly}
\begin{code}

   public boolean hasNextCoordinate();
\end{code}
 \begin{tabb}
   Returns \texttt{true} if the current point has another coordinate.
   This can be useful for testing if coordinates are still available.
 \end{tabb}
\begin{htmlonly}
   \return{\texttt{true} if the current point has another coordinate}
\end{htmlonly}
\begin{code}

   public double nextCoordinate();
\end{code}
 \begin{tabb}
   Returns the current coordinate $u_{i,j}$ and advances to the next one.
   If no current coordinate is available (either because the current
   point index has reached the number of points or because the current
   coordinate index has reached the number of dimensions), it throws a 
   \externalclass{java.util}{NoSuchElementException}.
 \end{tabb}
\begin{htmlonly}
   \return{value of the current coordinate}
   \exception{NoSuchElementException}{if no such coordinate is available}
\end{htmlonly}
\begin{code}

   public void nextCoordinates (double[] p, int d);
\end{code}
 \begin{tabb}
   Returns the next \texttt{d} coordinates of the current point in \texttt{p}
   and advances the current coordinate index by \texttt{d}.
   If the remaining number of coordinates is too small, a
   \texttt{NoSuchElementException} is thrown, as in \method{nextCoordinate}{}.
%   This method does not necessarily check if \texttt{p} is large enough.
 \end{tabb}
\begin{htmlonly}
   \param{p}{array to be filled with the coordinates, starting at index 0}
   \param{d}{number of coordinates to get}
   \exception{NoSuchElementException}{if there are not enough 
             remaining coordinates in the current point}
\end{htmlonly}
\begin{code}

   public void setCurPointIndex (int i);
\end{code}
 \begin{tabb}
   Resets the current point index to $i$ and the current coordinate 
   index to zero.  If \texttt{i} is larger or equal to the number of points,
   an exception will \emph{not} be raised here, but only later if we 
   ask for a new coordinate or point. 
% The next call to either \method{nextCoordinates}{}
% or \method{nextCoordinate}{} will return coordinates starting from $u_{i,0}$.
 \end{tabb}
\begin{htmlonly}
   \param{i}{new index of the current point}
\end{htmlonly}
\begin{code}

   public void resetCurPointIndex();
\end{code}
 \begin{tabb}
   Equivalent to \method{setCurPointIndex}{}~\texttt{(0)}.
 \end{tabb}
\begin{code}

   public int resetToNextPoint();
\end{code}
 \begin{tabb}
   Increases the current point index by 1 and returns its new value.
   If there is no more point, an exception will be raised only if we 
   ask for a new coordinate or point later on. 
%    Equivalent to 
%   \method{setCurPointIndex}{}~\texttt{(}\method{getCurPointIndex()}{}~\texttt{+ 1)}
 \end{tabb}
\begin{htmlonly}
   \return{index of the new current point}
\end{htmlonly}
\begin{code}

   public int getCurPointIndex();
\end{code}
 \begin{tabb}
   Returns the index $i$ of the current point.  
   This can be useful, e.g., for caching point sets.
%   and efficient implementations.
% (see examples in the code).
 \end{tabb}
\begin{htmlonly}
   \return{index of the current point}
\end{htmlonly}
\begin{code}

   public boolean hasNextPoint();
\end{code}
 \begin{tabb}
   Returns \texttt{true} if there is a next point.
   This can be useful for testing if points are still available.
 \end{tabb}
\begin{htmlonly}
   \return{\texttt{true} if a next point is available from the iterated point set}
\end{htmlonly}
\begin{code}

   public int nextPoint (double[] p, int d);\begin{hide}
}\end{hide}
\end{code}
 \begin{tabb}
   Returns the \emph{first} \texttt{d} coordinates of the \emph{current} 
   point in \texttt{p}, advances to the next point, and
   returns the index of the \emph{new} current point.
   Even if the current coordinate index is 0, the point returned
   starts from coordinate 0.
   After obtaining the last point via this method, the current point
   index (returned by the method) is equal to the number of points,
   so it is no longer a valid point index.
   An exception will then be raised if we attempt to generate additional
   points or coordinates.

  Specialized implementations of this method often allow for increased 
  efficiency, e.g., for cycle-based point sets where the cycles 
  (but not the points)
  are stored explicitly or for digital nets  
  by allowing non-incremental point enumerations via Gray-code counters.
 \end{tabb}
\begin{htmlonly}
   \param{p}{array to be filled with the coordinates, 
             starting from array index 0}
   \param{d}{number of coordinates to return}
   \return{index of the new current point}
   \exception{NoSuchElementException}{if there are not enough coordinates 
     available in the current point for filling \texttt{p}}
\end{htmlonly}
