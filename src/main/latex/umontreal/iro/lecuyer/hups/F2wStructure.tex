\defclass{F2wStructure}

This class implements methods and fields needed by the classes
 \externalclass{umontreal.iro.lecuyer.hups}{F2wNetLFSR},
 \externalclass{umontreal.iro.lecuyer.hups}{F2wNetPolyLCG},
 \externalclass{umontreal.iro.lecuyer.hups}{F2wCycleBasedLFSR} and
 \externalclass{umontreal.iro.lecuyer.hups}{F2wCycleBasedPolyLCG}.
It also stores the parameters of these point sets which will contain
$2^{rw}$ points (see the meaning of $r$ and $w$ below).
The parameters can be stored as a polynomial $P(z)$ over
 $\latex{\mathbb{F}}\html{\mathbf{F}}_{2^w}[z]$
$$
P(z) = z^{r} + \sum_{i=1}^{r} b_i z^{r-i}
$$
where $b_i\in \latex{\mathbb{F}}\html{\mathbf{F}}_{2^w}$ for $i=1,\ldots,r$.
 Let $\zeta$ be the root of an irreducible polynomial
 $Q(z)\in \latex{\mathbb{F}}\html{\mathbf{F}}_2[z]$.  It is well known
that $\zeta$ is a generator of the finite field
 $\latex{\mathbb{F}}\html{\mathbf{F}}_{2^w}$.
The elements of $\latex{\mathbb{F}}\html{\mathbf{F}}_{2^w}$ are
 represented using the polynomial ordered
 basis $(1,\zeta,\ldots,\zeta^{w-1})$.

In this class, only the non-zero coefficients of $P(z)$ are stored.
 It is stored as
$$
P(z) = z^{\mathtt{r}} + \sum_{i=0}^{\mathtt{nbcoeff}} {\mathtt{coeff[}}i{\mathtt{]}}
   z^{{\mathtt{nocoeff[}}i{\mathtt{]}}}
$$
where the coefficients in \texttt{coeff[]} represent the non-zero
 coefficients $b_i$ of $P(z)$ using the polynomial basis.
The finite field $\latex{\mathbb{F}}\html{\mathbf{F}}_{2^w}$ used is
 defined by the polynomial
$$
Q(z) = z^{w} +  \sum_{i=1}^{w} a_i z^{w-i}
$$
where $a_i\in \latex{\mathbb{F}}\html{\mathbf{F}}_{2}$,
 for $i=1,\ldots,w$. Polynomial $Q$ is
 stored as the bit vector {\texttt{modQ}} = $(a_w,\ldots,a_1)$.

The class also stores the parameter \texttt{step} that is used by the classes
\externalclass{umontreal.iro.lecuyer.hups}{F2wNetLFSR},
 \externalclass{umontreal.iro.lecuyer.hups}{F2wNetPolyLCG},
 \externalclass{umontreal.iro.lecuyer.hups}{F2wCycleBasedLFSR} and
 \externalclass{umontreal.iro.lecuyer.hups}{F2wCycleBasedPolyLCG}.
This parameter is such that the implementation of the recurrence
 will output a value  at every {\texttt{step}} iterations.

\bigskip\hrule\bigskip
%%%%%%%%%%%%%%%%%%%%%%%%%%%%%%%%%%%%%%%%%%%%%%%%%%%%%%%%%%%%%%%%%%%%%%%%%%%%

\begin{code}
\begin{hide}
/*
 * Class:        F2wStructure
 * Description:  Tools for point sets and sequences based on field F_{2^w}
 * Environment:  Java
 * Software:     SSJ 
 * Copyright (C) 2001  Pierre L'Ecuyer and Université de Montréal
 * Organization: DIRO, Université de Montréal
 * @author       
 * @since

 * SSJ is free software: you can redistribute it and/or modify it under
 * the terms of the GNU General Public License (GPL) as published by the
 * Free Software Foundation, either version 3 of the License, or
 * any later version.

 * SSJ is distributed in the hope that it will be useful,
 * but WITHOUT ANY WARRANTY; without even the implied warranty of
 * MERCHANTABILITY or FITNESS FOR A PARTICULAR PURPOSE.  See the
 * GNU General Public License for more details.

 * A copy of the GNU General Public License is available at
   <a href="http://www.gnu.org/licenses">GPL licence site</a>.
 */
\end{hide}
package umontreal.iro.lecuyer.hups;\begin{hide}
import java.io.*;
import java.util.*;
\end{hide}

public class F2wStructure \begin{hide} {

   private final int ALLONES = 2147483647; // 2^31-1 --> 01111...1
   int w;
   int r;
   int numBits;
   private int modQ;
   private int step;
   private int[] coeff;
   private int[] nocoeff;
   private int nbcoeff;
   int S;
   private int maskw;
   private int maskrw;
   private int maskZrm1;
   private int mask31;
   private int t;
   private int masktrw;
   private int[] maskv;
   int state;
   int output;            // augmented state
   double normFactor;
   double EpsilonHalf;
   final static int MBL = 140; //maximum of bytes in 1 line
   //92 bytes for a number of coeff = 15


   private void init (int w, int r, int modQ, int step,
      int nbcoeff, int coeff[], int nocoeff[])
   {
      normFactor = 1.0 / (1L << 31); // 4.65661287307739258e-10;
      EpsilonHalf = 0.5*normFactor;
      numBits = 31;
      this.step = step;
      this.w = w;
      this.r = r;
      S = 31 - r * w;
      mask31 = ~(1 << 31);
      maskw = (1 << w) - 1;
      maskrw = ((1 << (r * w)) - 1) << S;
      maskZrm1 = (ALLONES >> (r * w)) ^ (ALLONES >> ((r - 1) * w));
      this.modQ = modQ;
      this.nbcoeff = nbcoeff;
      this.nocoeff = new int[nbcoeff];
      this.coeff = new int[nbcoeff];
      for (int j = 0; j < nbcoeff; j++) {
         this.nocoeff[j] = nocoeff[j];
         this.coeff[j] = coeff[j];
      }
   }

   void initParamLFSR ()
   {
      t = (31 - r * w) / w;
      masktrw = (~0) << (31 - (t + r) * w) & mask31;
      maskv = new int[r];
      for (int j = 0; j < r; j++) {
         maskv[j] = maskw << (S + ((r - 1 - j) * w));
      }
   }
\end{hide}
\end{code}

%%%%%%%%%%%%%%%%%%%%%%%%%%%%%%%%%
\subsubsection* {Constructors}
\begin{code}


   F2wStructure (int w, int r, int modQ, int step, int nbcoeff,
                 int coeff[], int nocoeff[]) \begin{hide}
   {
      init (w, r, modQ, step, nbcoeff, coeff, nocoeff);
   }
\end{hide}
\end{code}
\begin{tabb}
  Constructs a \texttt{F2wStructure} object that contains  the parameters of a
  polynomial in $\latex{\mathbb{F}}\html{\mathbf{F}}_{2^w}[z]$,
 as well as a stepping parameter.
\end{tabb}
\begin{code}

   F2wStructure (String filename, int no)\begin{hide}
   {
     // If filename can be found starting from the program's directory,
     // it will be used; otherwise, the filename in the Jar archive will
     // be used.
     BufferedReader input;
     try {
       if ((new File (filename)).exists()) {
          input = new BufferedReader (new FileReader (filename));
       } else {
          DataInputStream dataInput;
          dataInput = new DataInputStream (
             F2wStructure.class.getClassLoader().getResourceAsStream (
                 "umontreal/iro/lecuyer/hups/dataF2w/" + filename));
          input = new BufferedReader (new InputStreamReader (dataInput));
       }
       initFromReader (filename, input, no);
       input.close ();

     } catch (Exception e) {
       System.out.println ("IO Error: problems finding file " + filename);
       System.exit (1);
     }
   }\end{hide}
\end{code}
 \begin{tabb}
   Constructs a polynomial in $\latex{\mathbb{F}}\html{\mathbf{F}}_{2^w}[z]$
   after reading its parameters from file {\texttt{filename}};
   the parameters of this polynomial are stored  at line number
   {\texttt{no}} of {\texttt{filename}}.
   The files are kept in different
   directories depending on the criteria used in the searches for the
   parameters defining the polynomials. The different criteria for the
   searches and the theory behind it are described in \cite{rPAN04d,rPAN04t}.
   The existing files and the number of polynomials they contain are
   given in the following tables.
   The first table below contains files in subdirectory
    \texttt{LFSR\_equid\_max}. The name of each
   file indicates the value of $r$ and $w$ for the polynomials.
   For example, file \texttt{f2wR2\_W5.dat} in directory
   \texttt{LFSR\_equid\_max} contains the parameters of 2358
   polynomials with $r=2$ and $w=5$. For example, to use the 5\textit{-th}
    polynomial of file \texttt{f2wR2\_W5.dat}, one may call
   \texttt{F2wStructure("f2wR2\_W5.dat", 5)}.
   The files of parameters have been stored at the address
   \url{http://www.iro.umontreal.ca/~simardr/ssj-1/dataF2w/}.

 \end{tabb}
\begin{code}
\begin{hide}

   private int multiplyz (int a, int k)
   {
      int i;
      if (k == 0)
         return a & maskw;
      else {
         for (i = 0; i < k; i++) {
            if ((1 & a) == 1) {
               a = (a >> 1) ^ modQ;
            } else
               a = a >> 1;
         }
         return a & maskw;
      }
   }\end{hide}
\end{code}
%%%%%%%%%%%%%%%%%%%%%%%%%%%%%%%%%
\subsubsection* {Methods}
\begin{code}

   int getLog2N ()\begin{hide}
   {
      return r * w;
   }\end{hide}
\end{code}
 \begin{tabb}
  This method returns the product $rw$.
 \end{tabb}
\begin{code}

   int multiply (int a, int b)\begin{hide}

   {
      int i;
      int res = 0, verif = 1;
      for (i = 0; i < w; i++) {
         if ((b & verif) == verif)
            res ^= multiplyz (a, w - 1 - i);
         verif <<= 1;
      }
      return res & maskw;
   }\end{hide}
\end{code}
 \begin{tabb}
  Method that multiplies two elements in
 $\latex{\mathbb{F}}\html{\mathbf{F}}_{2^w}$.
 \end{tabb}
\begin{code} \begin{hide}

   void initF2wLFSR ()     // Initialisation de l'etat d'un LFSR
   {
      int v, d = 0;
      int tempState;

      tempState = state << S;
      output = tempState;
      for (int i = 1; i <= t; i++) {
         d = 0;
         for (int j = 0; j < nbcoeff; j++) {
            v = (tempState & maskv[nocoeff[j]]) >>
                 (S + (r - 1 - nocoeff[j]) * w);
            d ^= multiply (coeff[j], v);
         }
         output |= d << (S - i * w);
         tempState = (output << (i * w)) & maskrw;
      }
   }


   void F2wLFSR ()       // Une iteration d'un LFSR
   {
      int v, d = 0;
      int tempState;
      for (int i = 0; i < step; i++) {
         tempState = (output << (t * w)) & maskrw;
         d = 0;
         for (int j = 0; j < nbcoeff; j++) {
            v = (tempState & maskv[nocoeff[j]]) >>
                (S + (r - 1 - nocoeff[j]) * w);
            d ^= multiply (coeff[j], v);
         }
         output = ((output << w) & masktrw) |
                  (d << (31 - (r + t) * w));
      }
      state = (output & maskrw) >> S;
   }


   int F2wPolyLCG ()    // Une iteration d'un PolyLCG
   {
      int Zrm1, d;
      for (int i = 0; i < step; i++) {
         Zrm1 = (state & maskZrm1) >> S;
         state = (state >> w) & maskrw;
         for (int j = 0; j < nbcoeff; j++)
            state ^=
               multiply (coeff[j], Zrm1) << (S + (r - 1 - nocoeff[j]) * w);
      }
      return state;
   }\end{hide}

   public static void print (String filename)\begin{hide}
   {
     BufferedReader input;
     try {
       if ((new File (filename)).exists()) {
          input = new BufferedReader (new FileReader (filename));
       } else {
          DataInputStream dataInput;
          dataInput = new DataInputStream (
             F2wStructure.class.getClassLoader().getResourceAsStream (
                 "umontreal/iro/lecuyer/hups/dataF2w/" + filename));
          input = new BufferedReader (new InputStreamReader (dataInput));
       }

     String s;
     for (int i = 0; i < 4; i++)
        input.readLine ();
     while ((s = input.readLine ()) != null)
        System.out.println (s);
     input.close ();

     } catch (Exception e) {
       System.out.println ("IO Error: problems reading file " + filename);
       System.exit (1);
     }
   }\end{hide}
\end{code}
 \begin{tabb}
    Prints the content of file \texttt{filename}. See the constructor
    above for the conditions on \texttt{filename}.
 \end{tabb}
\begin{code}

   public String toString ()\begin{hide}
   {
      StringBuffer sb = new StringBuffer ("z^");
      sb.append (r);
      for (int j = nbcoeff - 1; j >= 0; j--)
         sb.append (" + (" + coeff[j] + ") z^" + nocoeff[j]);
      sb.append ("   modQ = " + modQ + "    w = " + w + "   step = " + step);
      return sb.toString ();
   }
\end{hide}
\end{code}
\begin{tabb}
  This method returns a string containing the polynomial $P(z)$ and
 the stepping parameter.
\end{tabb}

\begin{code} \begin{hide}
    private void initFromReader (String filename, BufferedReader input, int no)
    {
      int w, r, modQ, step, nbcoeff;
      int coeff[], nocoeff[];
      StringTokenizer line;
      int nl = no + 4;

      try {
        for (int j = 1; j < nl ; j++)
          input.readLine ();

        line = new StringTokenizer (input.readLine ());
        w = Integer.parseInt (line.nextToken ());
        r = Integer.parseInt (line.nextToken ());
        modQ = Integer.parseInt (line.nextToken ());
        step = Integer.parseInt (line.nextToken ());
        nbcoeff = Integer.parseInt (line.nextToken ());
        nocoeff = new int[nbcoeff];
        coeff = new int[nbcoeff];
        for (int i = 0; i < nbcoeff; i++) {
          coeff[i] = Integer.parseInt (line.nextToken ());
          nocoeff[i] = Integer.parseInt (line.nextToken ());
        }
        init (w, r, modQ, step, nbcoeff, coeff, nocoeff);
        input.close ();

      } catch (Exception e) {
        System.out.println ("Input Error: problems reading file " + filename);
        System.exit (1);
      }
    }
  }
  \end{hide}
\end{code}

\tt
\begin{minipage}{7cm}
   \begin {tabular}{|c|c|}
   \multicolumn{2}{c} {{\rm Directory} LFSR\_equid\_max} \\
\hline
  Filename     &  Num of poly.  \\
\hline
  f2wR2\_W5.dat  & 2358   \\
  f2wR2\_W6.dat  & 1618   \\
  f2wR2\_W7.dat  & 507    \\
  f2wR2\_W8.dat  & 26     \\
  f2wR2\_W9.dat  & 3      \\
  f2wR3\_W4.dat  & 369    \\
  f2wR3\_W5.dat  & 26     \\
  f2wR3\_W6.dat  & 1      \\
  f2wR4\_W3.dat  & 117    \\
  f2wR4\_W4.dat  & 1      \\
  f2wR5\_W2.dat  & 165    \\
  f2wR5\_W3.dat  & 1      \\
  f2wR6\_W2.dat  & 36     \\
  f2wR6\_W3.dat  & 1      \\
  f2wR7\_W2.dat  & 10     \\
  f2wR8\_W2.dat  & 11     \\
  f2wR9\_W2.dat  & 1      \\
\hline
\end {tabular}
\end{minipage}
\hfill
\begin{minipage}{7cm}
\begin {tabular}{|c|c|}
 \multicolumn{2}{c} {{\rm  Directory} LFSR\_equid\_sum} \\
\hline
  Filename     &  Num of poly.  \\
\hline
 f2wR2\_W5.dat  & 2276     \\
 f2wR2\_W6.dat  & 1121     \\
 f2wR2\_W7.dat  & 474      \\
 f2wR2\_W8.dat  & 37       \\
 f2wR2\_W9.dat  & 6        \\
 f2wR3\_W4.dat  & 381      \\
 f2wR3\_W5.dat  & 65       \\
 f2wR3\_W6.dat  & 7        \\
 f2wR4\_W3.dat  & 154      \\
 f2wR4\_W4.dat  & 2        \\
 f2wR5\_W2.dat  & 688      \\
 f2wR5\_W3.dat  & 5        \\
 f2wR6\_W2.dat  & 70       \\
 f2wR6\_W3.dat  & 1        \\
 f2wR7\_W2.dat  & 9        \\
 f2wR8\_W2.dat  & 3        \\
 f2wR9\_W2.dat  & 3        \\
\hline
\end {tabular}
\end{minipage}
\bigskip

\begin{minipage}{7cm}
\begin {tabular}{|c|c|}
 \multicolumn{2}{c} {{\rm Directory} LFSR\_mindist\_max} \\
\hline
  Filename     &  Num of poly.  \\
\hline
 f2wR2\_W5.dat  & 1    \\
 f2wR2\_W6.dat  & 1    \\
 f2wR2\_W7.dat  & 2    \\
 f2wR2\_W8.dat  & 2    \\
 f2wR2\_W9.dat  & 1    \\
 f2wR3\_W4.dat  & 2    \\
 f2wR3\_W5.dat  & 2    \\
 f2wR3\_W6.dat  & 1    \\
 f2wR4\_W3.dat  & 1    \\
 f2wR4\_W4.dat  & 1    \\
 f2wR5\_W2.dat  & 2    \\
 f2wR5\_W3.dat  & 1    \\
 f2wR6\_W2.dat  & 4    \\
 f2wR6\_W3.dat  & 1    \\
 f2wR7\_W2.dat  & 1    \\
 f2wR8\_W2.dat  & 1    \\
 f2wR9\_W2.dat  & 1    \\
\hline
\end {tabular}
\end{minipage}
\hfill
\begin{minipage}{7cm}
\begin {tabular}{|c|c|}
\multicolumn{2}{c} {{\rm Directory} LFSR\_mindist\_sum} \\
\hline
  Filename     &  Num of poly.  \\
\hline
  f2wR2\_W5.dat  & 1   \\
  f2wR2\_W6.dat  & 1   \\
  f2wR2\_W7.dat  & 1   \\
  f2wR2\_W8.dat  & 1   \\
  f2wR2\_W9.dat  & 1   \\
  f2wR3\_W4.dat  & 1   \\
  f2wR3\_W5.dat  & 1   \\
  f2wR3\_W6.dat  & 1   \\
  f2wR4\_W3.dat  & 1   \\
  f2wR4\_W4.dat  & 2   \\
  f2wR5\_W2.dat  & 2   \\
  f2wR5\_W3.dat  & 2   \\
  f2wR6\_W2.dat  & 1   \\
  f2wR6\_W3.dat  & 1   \\
  f2wR7\_W2.dat  & 2   \\
  f2wR8\_W2.dat  & 1   \\
  f2wR9\_W2.dat  & 2   \\
\hline
\end {tabular}
\end{minipage}

\bigskip
\begin{minipage}{7cm}
\begin {tabular}{|c|c|}
   \multicolumn{2}{c} {{\rm  Directory} LFSR\_tvalue\_max} \\
\hline
  Filename     &  Num of poly.  \\
\hline
 f2wR2\_W5.dat  & 7     \\
 f2wR2\_W6.dat  & 1     \\
 f2wR2\_W7.dat  & 1     \\
 f2wR2\_W8.dat  & 1     \\
 f2wR2\_W9.dat  & 1     \\
 f2wR3\_W4.dat  & 1     \\
 f2wR3\_W5.dat  & 1     \\
 f2wR3\_W6.dat  & 1     \\
 f2wR4\_W3.dat  & 2     \\
 f2wR4\_W4.dat  & 1     \\
 f2wR5\_W2.dat  & 14    \\
 f2wR5\_W3.dat  & 1     \\
 f2wR6\_W2.dat  & 2     \\
 f2wR6\_W3.dat  & 1     \\
 f2wR7\_W2.dat  & 1     \\
 f2wR8\_W2.dat  & 1     \\
 f2wR9\_W2.dat  & 1     \\
\hline
\end {tabular}
\end{minipage}
\hfill
\begin{minipage}{7cm}
\begin {tabular}{|c|c|}
   \multicolumn{2}{c} {{\rm  Directory} LFSR\_tvalue\_sum} \\
\hline
  Filename     &  Num of poly.  \\
\hline
 f2wR2\_W5.dat  & 15    \\
 f2wR2\_W6.dat  & 1     \\
 f2wR2\_W7.dat  & 1     \\
 f2wR2\_W8.dat  & 2     \\
 f2wR2\_W9.dat  & 1     \\
 f2wR3\_W4.dat  & 1     \\
 f2wR3\_W5.dat  & 1     \\
 f2wR3\_W6.dat  & 1     \\
 f2wR4\_W3.dat  & 2     \\
 f2wR4\_W4.dat  & 1     \\
 f2wR5\_W2.dat  & 13    \\
 f2wR5\_W3.dat  & 2     \\
 f2wR6\_W2.dat  & 12    \\
 f2wR6\_W3.dat  & 1     \\
 f2wR7\_W2.dat  & 1     \\
 f2wR8\_W2.dat  & 1     \\
 f2wR9\_W2.dat  & 1     \\
\hline
\end {tabular}
\end{minipage}
\rm
