\defclass {LCGPointSet}

Implements a recurrence-based point set defined via a linear 
congruential recurrence of the form $x_i = a x_{i-1} \mod n$
and $u_i = x_i / n$.  The implementation is done by storing the values
of $u_i$ over the set of all cycles of the recurrence.
 

\bigskip\hrule\bigskip

%%%%%%%%%%%%%%%%%%%%%%%%%%%%%%%%%%%%%%%%%%%%%%%%%%%%%%%%%%%%%%%%%%%%%%%%%%%%%%%
\begin{code}
\begin{hide}
/*
 * Class:        LCGPointSet
 * Description:  point set defined via a linear congruential recurrence
 * Environment:  Java
 * Software:     SSJ 
 * Copyright (C) 2001  Pierre L'Ecuyer and Université de Montréal
 * Organization: DIRO, Université de Montréal
 * @author       
 * @since

 * SSJ is free software: you can redistribute it and/or modify it under
 * the terms of the GNU General Public License (GPL) as published by the
 * Free Software Foundation, either version 3 of the License, or
 * any later version.

 * SSJ is distributed in the hope that it will be useful,
 * but WITHOUT ANY WARRANTY; without even the implied warranty of
 * MERCHANTABILITY or FITNESS FOR A PARTICULAR PURPOSE.  See the
 * GNU General Public License for more details.

 * A copy of the GNU General Public License is available at
   <a href="http://www.gnu.org/licenses">GPL licence site</a>.
 */
\end{hide}
package umontreal.iro.lecuyer.hups;\begin{hide}
import umontreal.iro.lecuyer.util.PrintfFormat;
import cern.colt.list.*;
\end{hide}

public class LCGPointSet extends CycleBasedPointSet \begin{hide} {

      private int a;                      // Multiplier.

\end{hide}
\end{code}

%%%%%%%%%%%%%%%%%%%%%%%%%%%%
\subsubsection* {Constructors}
\begin{code}

   public LCGPointSet (int n, int a) \begin{hide} {
      this.a = a;
      double invn = 1.0 / (double)n;   // 1/n
      DoubleArrayList c;  // Array used to store the current cycle.
      long currentState;  // The state currently visited. 
      int i;
      boolean stateVisited[] = new boolean[n];  
         // Indicates which states have been visited so far.
      for (i = 0; i < n; i++)
         stateVisited[i] = false;
      int startState = 0;    // First state of the cycle currently considered.
      numPoints = 0;
      while (startState < n) {
         stateVisited[startState] = true;
         c = new DoubleArrayList();
         c.add (startState * invn);
         // We use the fact that a "long" has 64 bits in Java.
         currentState = (startState * (long)a) % (long)n;
         while (currentState != startState) {
            stateVisited[(int)currentState] = true;
            c.add (currentState * invn);
            currentState = (currentState * (long)a) % (long)n;
            }
         addCycle (c);
         for (i = startState+1; i < n; i++)
            if (stateVisited[i] == false)
                break;
         startState = i;
         }
      }\end{hide}
\end{code}
 \begin{tabb} Constructs and stores the set of cycles for an LCG with
   modulus $n$ and multiplier $a$.
  If the LCG has full period length $n-1$,
% pgcd$(a,n)=1$, 
  there are two cycles, the first one containing only 0 
  and the second one of period length $n-1$.
 \end{tabb}
\begin{htmlonly}
   \param{n}{required number of points and modulus of the LCG}
   \param{a}{generator $a$ of the LCG}
\end{htmlonly}
\begin{code}

   public LCGPointSet (int b, int e, int c, int a) \begin{hide} {
      this (computeModulus (b, e, c), a);
   }\end{hide}
\end{code}
 \begin{tabb} Constructs and stores the set of cycles for an LCG with
   modulus $n = b^e + c$ and multiplier $a$.
 \end{tabb}
\begin{code}\begin{hide}
   private static int computeModulus (int b, int e, int c) {
      int n;
      int i;
      if (b == 2) 
         n = (1 << e);
      else {
         for (i = 1, n = b;  i < e;  i++)  n *= b;
         }
      n += c;
      return n;
   }


   public String toString() {
      StringBuffer sb = new StringBuffer ("LCGPointSet:" +
                                           PrintfFormat.NEWLINE);
      sb.append (super.toString());
      sb.append (PrintfFormat.NEWLINE + "Multiplier a: ");
      sb.append (a);
      return sb.toString();
   }
\end{hide}

   public int geta () \begin{hide} {
      return a;
   }
}\end{hide}
\end{code}
 \begin{tabb} Returns the value of the multiplier $a$.
 \end{tabb}
