\defclass{CachedPointSet}

This container class caches a point set by precomputing
and storing its points locally in an array.
This can be used to speed up computations when using
a small low-dimensional point set more than once.

\begin{detailed} %%
After the points are stored in the array, this class uses
the default methods and the default iterator type provided by
the base class \class{PointSet}{}.
This is one of the rare cases where direct use of the
\method{getCoordinate}{} method is efficient.
\pierre {We could also implement an iterator that directly returns
  \texttt{x[i][j]} instead of calling \texttt{getCoordinate}, for slightly
  better efficiency.  On the other hand, even better efficiency can
  be achieved by getting an entire point at a time in an array.
} However, it might require too much memory for a large point set.
\end{detailed} %%

% No array index range check is performed in this class,
% neither for the dimension nor for the number of points.
% However, this is done by Java.

\bigskip\hrule\bigskip
%%%%%%%%%%%%%%%%%%%%%%%%%%%%%%%%%%%%%%%%%%%%%%%%%%%%%%%%%%%%%%%%%%

\begin{code}
\begin{hide}
/*
 * Class:        CachedPointSet
 * Description:  
 * Environment:  Java
 * Software:     SSJ 
 * Copyright (C) 2001  Pierre L'Ecuyer and Université de Montréal
 * Organization: DIRO, Université de Montréal
 * @author       
 * @since

 * SSJ is free software: you can redistribute it and/or modify it under
 * the terms of the GNU General Public License (GPL) as published by the
 * Free Software Foundation, either version 3 of the License, or
 * any later version.

 * SSJ is distributed in the hope that it will be useful,
 * but WITHOUT ANY WARRANTY; without even the implied warranty of
 * MERCHANTABILITY or FITNESS FOR A PARTICULAR PURPOSE.  See the
 * GNU General Public License for more details.

 * A copy of the GNU General Public License is available at
   <a href="http://www.gnu.org/licenses">GPL licence site</a>.
 */
\end{hide}
package umontreal.iro.lecuyer.hups;\begin{hide}

import umontreal.iro.lecuyer.util.PrintfFormat;
\end{hide}
    import umontreal.iro.lecuyer.rng.RandomStream;


public class CachedPointSet extends PointSet \begin{hide} {
   protected PointSet P;        // Original PointSet which is cached here.
   protected double x[][];      // Cached points.
   protected CachedPointSet() {}
\end{hide}
\end{code}

%%%%%%%%%%%%%%%%%%%%%%%%%%%%
\subsubsection*{Constructors}
\begin{code}

   public CachedPointSet (PointSet P, int n, int dim) \begin{hide} {
      if (P.getNumPoints() < n)
         throw new IllegalArgumentException(
            "Cannot cache more points than in point set P.");
      if (P.getDimension() < dim)
         throw new IllegalArgumentException(
            "Cannot cache points with more coordinates than the dimension.");
      numPoints = n;
      this.dim = dim;
      this.P = P;
      init ();
   }

   protected void init () {
      PointSetIterator itr = P.iterator();
      x = new double[numPoints][dim];
      for (int i = 0; i < numPoints; i++)
         itr.nextPoint (x[i], dim);
   }\end{hide}
\end{code}
 \begin{tabb}
   Creates a new \texttt{PointSet} object that contains an array storing
   the first \texttt{dim} coordinates of the first \texttt{n} points of \texttt{P}.
   The original point set \texttt{P} itself is not modified.
%, except for its point and coordinate iterators.
 \end{tabb}
\begin{htmlonly}
   \param{P}{point set to be cached}
   \param{n}{number of points}
   \param{dim}{number of dimensions of the points}
\end{htmlonly}
\begin{code}

   public CachedPointSet (PointSet P) \begin{hide} {
      numPoints = P.getNumPoints();
      dim = P.getDimension();
      if (numPoints == Integer.MAX_VALUE)
         throw new IllegalArgumentException(
            "Cannot cache infinite number of points");
      if (dim == Integer.MAX_VALUE)
         throw new IllegalArgumentException(
            "Cannot cache infinite dimensional points");
      this.P = P;
      init ();
   }\end{hide}
\end{code}
 \begin{tabb}
   Creates a new \texttt{PointSet} object that contains an array storing
   the points of \texttt{P}.
   The number of points and their dimension are the same as in the
   original point set.  Both must be finite.
%  The point set \texttt{P} itself is not modified.
 \end{tabb}
\begin{htmlonly}
   \param{P}{point set to be cached}
\end{htmlonly}


%%%%%%%%%%%%%%%%%%%%%%%%%%%%
\subsubsection*{Methods}
\begin{code}

   public void addRandomShift(int d1, int d2, RandomStream stream)\begin{hide} {
        P.addRandomShift(d1, d2, stream);
        init();
   }\end{hide}
\end{code}
\begin{tabb}
Add the shift to the contained point set and recaches the points. See the doc of the
overridden method \externalmethod{umontreal.iro.lecuyer.hups}{PointSet}{addRandomShift}{(int, int, RandomStream)}\texttt{(d1, d2, stream)} in \class{PointSet}.
\end{tabb}
\begin{code}

   public void randomize (PointSetRandomization rand)\begin{hide} {
      P.randomize(rand);
      init();
   }\end{hide}
\end{code}
\begin{tabb}
Randomizes the underlying point set using \texttt{rand} and
recaches the points.
\end{tabb}
\begin{code}\begin{hide}

   public String toString() {
     StringBuffer sb = new StringBuffer ("Cached point set" +
          PrintfFormat.NEWLINE);
     sb.append (super.toString());
     sb.append (PrintfFormat.NEWLINE + "Cached point set information {"
                + PrintfFormat.NEWLINE);
     sb.append (P.toString());
     sb.append (PrintfFormat.NEWLINE + "}");
     return sb.toString();
   }

   public double getCoordinate (int i, int j) {
      return x[i][j];
   }

}
\end{hide}
\end{code}
