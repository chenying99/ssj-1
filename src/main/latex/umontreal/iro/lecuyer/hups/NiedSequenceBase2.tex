\defclass {NiedSequenceBase2}

This class implements digital sequences constructed from the
Niederreiter sequence in base 2.
\latex{For details on these point sets, see \cite{rBRA92a}.}
\hpierre{The generator matrices do not seem to be computed correctly
        in this implementation!}
\hrichard{Corrig\'e: elles le sont maintenant.}

%%%%%%%%%%%%%%%%%%%%%%%

\bigskip\hrule
\begin{code}
\begin{hide}
/*
 * Class:        NiedSequenceBase2
 * Description:  digital Niederreiter sequences in base 2.
 * Environment:  Java
 * Software:     SSJ 
 * Copyright (C) 2001  Pierre L'Ecuyer and Université de Montréal
 * Organization: DIRO, Université de Montréal
 * @author       
 * @since

 * SSJ is free software: you can redistribute it and/or modify it under
 * the terms of the GNU General Public License (GPL) as published by the
 * Free Software Foundation, either version 3 of the License, or
 * any later version.

 * SSJ is distributed in the hope that it will be useful,
 * but WITHOUT ANY WARRANTY; without even the implied warranty of
 * MERCHANTABILITY or FITNESS FOR A PARTICULAR PURPOSE.  See the
 * GNU General Public License for more details.

 * A copy of the GNU General Public License is available at
   <a href="http://www.gnu.org/licenses">GPL licence site</a>.
 */
\end{hide}
package umontreal.iro.lecuyer.hups;\begin{hide}

import java.io.Serializable;
import java.io.ObjectInputStream;
import java.io.InputStream;
import java.io.FileNotFoundException;
import java.io.IOException;
import umontreal.iro.lecuyer.util.PrintfFormat;
\end{hide}

public class NiedSequenceBase2 extends DigitalSequenceBase2\begin{hide} { 

   private static final int MAXDIM = 318;  // Maximum dimension.
   private static final int NUMCOLS = 30;  // Maximum number of columns.
\end{hide} 
\end{code}
%%%%%%%%%%%%%%%%%%%%%%%%%%%%%%%%%
\subsubsection* {Constructor}
\begin{code} 

   public NiedSequenceBase2 (int k, int w, int dim) \begin{hide} {
      init (k, w, w, dim);
   } 
\end{hide}
\end{code} 
\begin{tabb}
    Constructs a new digital sequence in base 2 from the first $n=2^k$ points 
    of the Niederreiter sequence,
    with $w$ output digits, in \texttt{dim} dimensions.
    The generator matrices $\mathbf{C}_j$ are $w\times k$.
%    Unless, one plans to apply a randomization on more than $k$ digits
%    (e.g., a random digital shift for $w > k$ digits, or a linear
%    scramble yielding $r > k$ digits), one should 
%    take $w = r = k$ for better computational efficiency.
    Restrictions: $0\le k\le 30$, $k\le w$, and \texttt{dim} $\le 318$.
\end{tabb}
\begin{htmlonly}
   \param{k}{there will be 2^k points}
   \param{w}{number of output digits}
   \param{dim}{dimension of the point set}
\end{htmlonly}
\begin{code}\begin{hide}

   public String toString() {
      StringBuffer sb = new StringBuffer ("Niederreiter sequence:" +
                                           PrintfFormat.NEWLINE);
      sb.append (super.toString());
      return sb.toString();
   }

   private void init (int k, int r, int w, int dim) {
      if ((dim < 1) || (dim > MAXDIM))
         throw new IllegalArgumentException 
            ("Dimension for NiedSequenceBase2 must be > 1 and <= " + MAXDIM);
      if (r < k || w < r || w > MAXBITS || k >= MAXBITS) 
         throw new IllegalArgumentException
            ("One must have k < 31 and k <= r <= w <= 31 for NiedSequenceBase2");
      numCols   = k;
      numRows   = r;   // Unused!
      outDigits = w;
      numPoints = (1 << k);
      this.dim  = dim;
      normFactor = 1.0 / ((double) (1L << (outDigits)));
      genMat = new int[dim * numCols];
      initGenMat();
   }


   public void extendSequence (int k) {
      init (k, numRows, outDigits, dim);
   }


   // Initializes the generator matrices for a sequence. 
   /* I multiply by 2 because the relevant columns are in the 30 least 
      significant bits of NiedMat, but if I understand correctly,
      SSJ assumes that they are in bits [31, ..., 1]. 
      Then I shift right if w < 31. */
   private void initGenMat ()  {
      for (int j = 0; j < dim; j++)
         for (int c = 0; c < numCols; c++) {
            genMat[j*numCols + c] = NiedMat[j*NUMCOLS + c] << 1;
            genMat[j*numCols + c] >>= MAXBITS - outDigits;
         }
   }

/*
   // Initializes the generator matrices for a net. 
   protected void initGenMatNet()  {
      int j, c;

      // the first dimension, j = 0.
      for (c = 0; c < numCols; c++)
         genMat[c] = (1 << (outDigits-numCols+c));

      for (j = 1; j < dim; j++)
         for (c = 0; c < numCols; c++)
            genMat[j*numCols + c] = 2 * NiedMat[(j - 1)*NUMCOLS + c];
   }
*/

   // ****************************************************************** 
   // Generator matrices of Niederreiter sequence. 
   // This array stores explicitly NUMCOLS columns in 318 dimensions.

   private static int[] NiedMat;
   private static final int MAXN = 9540;

   static {
      NiedMat = new int[MAXN];

      try {
         InputStream is = 
            NiedSequenceBase2.class.getClassLoader().getResourceAsStream (
            "umontreal/iro/lecuyer/hups/dataSer/Nieder/NiedSequenceBase2.ser");
         if (is == null)
            throw new FileNotFoundException (
               "Cannot find NiedSequenceBase2.ser");
         ObjectInputStream ois = new ObjectInputStream(is);
         NiedMat = (int[]) ois.readObject();
         ois.close();

      } catch(FileNotFoundException e) {
         e.printStackTrace();
         System.exit(1);

      } catch(IOException e) {
         e.printStackTrace();
         System.exit(1);

      } catch(ClassNotFoundException e) {
         e.printStackTrace();
         System.exit(1);
      }
   }

}
\end{hide}
\end{code}





