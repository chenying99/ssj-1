\defclass {Continuous}

Represents a variable in a continuous-time simulation.
This abstract class provides the basic structures and tools
for continuous-time simulation, where certain variables evolve 
continuously in time, according to differential equations.
Such continuous variables can be mixed with events and processes.

Each type of continuous-time variable should be defined as a 
subclass of \texttt{Continuous}.
The instances of these subclasses are the actual continuous-time
variables.  Each subclass must implement the method
\method{derivative}{} which returns its derivative with respect to time.
The trajectory of this variable is determined by integrating this
derivative.
The subclass may also reimplement the method \method{afterEachStep}{}, 
which is executed immediately after each integration step.
\hpierre {It would probably be better if we could change this method
  when calling \texttt{StartInteg}, as in SIMOD....}
By default (in the class \texttt{Continuous}), this method does nothing.
This method could, for example, verify if the variable has reached
a given threshold, or update a graphical illustration of the variable
trajectory.

When creating a class representing a continuous variable,
the \method{toString}{} method can be overridden to display
information about the continuous variable.  This information will
be displayed when formating the event list as a string.

Each continuous variable has a linked simulator represented by an instance of the
\class{Simulator} class.
If no simulator is provided explicitly when constructing a variable,
the default simulator returned by
\texttt{Simulator.getDefaultSimulator} is used.

\bigskip\hrule

%%%%%%%%%%%%%%%%%%%%%%%%%%%%%%%%%%%%%%%%%%%%%%%%%%%%%%%%%%%%%%%%%%
\begin{code}
\begin{hide}
/*
 * Class:        Continuous
 * Description:  provides the basic structures and tools for 
                 continuous-time simulation
 * Environment:  Java
 * Software:     SSJ 
 * Copyright (C) 2001  Pierre L'Ecuyer and Université de Montréal
 * Organization: DIRO, Université de Montréal
 * @author       
 * @since

 * SSJ is free software: you can redistribute it and/or modify it under
 * the terms of the GNU General Public License (GPL) as published by the
 * Free Software Foundation, either version 3 of the License, or
 * any later version.

 * SSJ is distributed in the hope that it will be useful,
 * but WITHOUT ANY WARRANTY; without even the implied warranty of
 * MERCHANTABILITY or FITNESS FOR A PARTICULAR PURPOSE.  See the
 * GNU General Public License for more details.

 * A copy of the GNU General Public License is available at
   <a href="http://www.gnu.org/licenses">GPL licence site</a>.
 */
\end{hide}
package umontreal.iro.lecuyer.simevents;

public abstract class Continuous \begin{hide} {

   // Private variables:

   boolean active; // This variable is currently being integrated.
   double  value;  // Current value of the variable.
   Event  ev;      // Event to be executed after each integ. step,

   //String name;
   double phi;
   double pi;
   double buffer;
   double sum;

   private Simulator sim;
  
\end{hide}
\end{code}

%%%%%%%%%%%%%%%%%%%%%%%%%%%%%%%%%%%
\subsubsection* {Constructors}
\begin{code}

   public Continuous() \begin{hide} {
      active = false;
      this.sim = Simulator.getDefaultSimulator();
   } \end{hide}
\end{code}
   \begin{tabb} Constructs a new continuous-time variable 
   linked to the default simulator, {\em without\/} initializing it.
   \end{tabb}
\begin{code}

   public Continuous (Simulator sim) \begin{hide} {
       if (sim == null)
          throw new NullPointerException();
      active = false;
      this.sim = sim;
   } \end{hide}
\end{code}
   \begin{tabb} Constructs a new continuous-time variable linked to
     the given simulator, {\em without\/}
    initializing it.
   \end{tabb}
\begin{htmlonly}
   \param{sim}{the simulator associated to this variable.}
\end{htmlonly}
%%%%%%%%%%%%
\begin{comment}
\begin{vcode}

   public Continuous (String name) \begin{hide} {
      this();
      this.name = name;
   } \end{hide}
\end{vcode}
   \begin{tabb} Constructs a new continuous-time variable 
    (same as \texttt{Continuous()}) with name  \texttt{name} 
    and linked to the default simulator.
    This name can be used to identify the \texttt{Continuous}
    variable  in traces and reports.
   \end{tabb}
%\begin{htmlonly}
%   \param{name}{name associated to this variable}
%\end{htmlonly}
\begin{vcode}

   public Continuous (Simulator sim, String name) \begin{hide} {
      this(sim);
      this.name = name;
   } \end{hide}
\end{vcode}
   \begin{tabb} Constructs a new continuous-time variable 
    (same as \texttt{Continuous(sim)}) with name  \texttt{name}.
    This name can be used to identify the \texttt{Continuous}
    variable  in traces and reports.
   \end{tabb}
%\begin{htmlonly}
%   \param{name}{name associated to this variable}
%   \param{sim}{simulator associated to this variable}
%\end{htmlonly}
\end{comment}
%%%%%%%%%%


%%%%%%%%%%%%%%%%%%%%%%%%%%%%%%%%%%%
\subsubsection* {Methods}
\begin{code}

   public void init (double val) \begin{hide} {
      value = val;
   } \end{hide}
\end{code}
 \begin{tabb}   Initializes or reinitializes the continuous-time variable
   to \texttt{val}.
 \end{tabb}
\begin{htmlonly}
   \param{val}{initial value of the variable}
\end{htmlonly}
\begin{code}

   public double value() \begin{hide} {
      return value;
   } \end{hide}
\end{code}
 \begin{tabb}   Returns the current value of this continuous-time variable.
 \end{tabb}
\begin{htmlonly}
   \return{the current value of the variable}
\end{htmlonly}
\begin{code}

   public Simulator simulator() \begin{hide} {
      return sim;
   } \end{hide}
\end{code}
 \begin{tabb}   Returns the simulator linked to this continuous-time variable.
 \end{tabb}
\begin{htmlonly}
   \return{the current simulator of the variable}
\end{htmlonly}
\begin{code}

   public void setSimulator(Simulator sim) \begin{hide} {
       if (sim == null)
          throw new NullPointerException();
      this.sim = sim;
   } \end{hide}
\end{code}
 \begin{tabb}   Sets the simulator linked to this continuous-time variable.
   This method should not be called while this variable is active.
 \end{tabb}
\begin{htmlonly}
   \param{sim}{the simulator of the current variable}
\end{htmlonly}
\begin{code}

   public void startInteg() \begin{hide} {
      sim.continuousState().startInteg(this);
   } \end{hide}
\end{code}
 \begin{tabb}  Starts the integration process that will change the state of
  this variable at each integration step.
 \end{tabb}
\begin{code}

   public void startInteg (double val) \begin{hide} {
      init (val);   startInteg();
   } \end{hide}
\end{code}
 \begin{tabb}  Same as \method{startInteg}{}, after initializing the variable 
   to \texttt{val}.
 \end{tabb}
\begin{htmlonly}
   \param{val}{initial value to start integration from}
\end{htmlonly}
\begin{code}

   public void stopInteg() \begin{hide} {
      sim.continuousState().stopInteg(this);
   } \end{hide}
\end{code}
 \begin{tabb}  Stops the integration process for this continuous variable.
  The variable keeps the value it took at the last integration step
  before calling \texttt{stopInteg}.
 \end{tabb}
\begin{code}

   public abstract double derivative (double t);
\end{code}
  \begin{tabb}  This method should return the derivative of this variable
   with respect to time, at time $t$.
   Every subclass of \texttt{Continuous} that is to be instantiated 
   must implement it.
   If the derivative does not depend explicitly on time, $t$ becomes
   a dummy parameter.  Internally, the method is used with $t$ not
   necessarily equal to the current simulation time.
  \end{tabb}
\begin{htmlonly}
   \param{t}{time at which the derivative must be computed}
\end{htmlonly}
\begin{code}

   public void afterEachStep() \begin{hide} {
   } \end{hide}
\end{code}
  \begin{tabb}  This method is executed after each integration step
   for this \texttt{Continuous} variable.
   Here, it does nothing, but every subclass of \texttt{Continuous} may
   reimplement it.
  \end{tabb}
\begin{code}

   public static void selectEuler(double h) \begin{hide} {
      Simulator.getDefaultSimulator().continuousState().selectEuler(h);
   } \end{hide}
\end{code}
  \begin{tabb}  Selects the Euler method as the integration method,
  with the integration step size \texttt{h}, in time units, for the default simulator.
  The non-static method \method{selectEuler}{} in \class{ContinuousState}
  can be used to set the integration method for any given simulator.
  This method appears here only to keep compatibility with older versions of SSJ; using
  a non-static \class{Simulator} instance rather than the default simulator is recommended.
  \end{tabb}
\begin{htmlonly}
   \param{h}{integration step, in simulation time units}
\end{htmlonly}
\begin{code}

   public static void selectRungeKutta4(double h) \begin{hide} {
      Simulator.getDefaultSimulator().continuousState().selectRungeKutta4(h);
   } \end{hide}
\end{code}
  \begin{tabb}  Selects a Runge-Kutta method of order~4 as the integration
  method to be used, with step size \texttt{h}.
  The non-static method \method{selectRungeKutta4}{} in \class{ContinuousState}
  can be used to set the integration method for any given simulator.
  This method appears here only to keep compatibility with older versions of SSJ; using
  a non-static \class{Simulator} instance rather than the default simulator is recommended.
  \end{tabb}
\begin{code}

   public static void selectRungeKutta2(double h) \begin{hide} {
      Simulator.getDefaultSimulator().continuousState().selectRungeKutta2(h);
   } \end{hide}
\end{code}
  \begin{tabb}  Selects a Runge-Kutta method of order~2 as the integration
  method to be used, with step size \texttt{h}.
  The non-static method \method{selectRungeKutta2}{} in \class{ContinuousState}
  can be used to set the integration method for any given simulator.
  This method appears here only to keep compatibility with older versions of SSJ; using
  a non-static \class{Simulator} instance rather than the default simulator is recommended.
  \end{tabb}
\begin{code}\begin{hide}
}\end{hide}
\end{code}
