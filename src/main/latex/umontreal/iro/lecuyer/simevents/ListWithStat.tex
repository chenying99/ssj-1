\defclass {ListWithStat}

Implements a list with integrated statistical
probes to provide automatic collection of
statistics on the sojourn times of objects in the list and on the
size of the list as a function of time given by a simulator.
The automatic statistical collection can be
enabled or disabled for each list, to reduce overhead.
This class extends \externalclass{umontreal.iro.lecuyer.util}{TransformingList}
and transforms elements into nodes associating insertion times with elements.

\bigskip\hrule

%%%%%%%%%%%%%%%%%%%%%%%%%%%%%%%%%%%%%%%%%%%%%%%%%%%%%%%%%%%%%%%%%%

\begin{code}
\begin{hide}
/*
 * Class:        ListWithStat
 * Description:  Implements a list with integrated statistical probes to
                 provide automatic collection of statistics on the sojourn
                 times of objects in the list and on the size of the list
 * Environment:  Java
 * Software:     SSJ 
 * Copyright (C) 2001  Pierre L'Ecuyer and Université de Montréal
 * Organization: DIRO, Université de Montréal
 * @author       
 * @since

 * SSJ is free software: you can redistribute it and/or modify it under
 * the terms of the GNU General Public License (GPL) as published by the
 * Free Software Foundation, either version 3 of the License, or
 * any later version.

 * SSJ is distributed in the hope that it will be useful,
 * but WITHOUT ANY WARRANTY; without even the implied warranty of
 * MERCHANTABILITY or FITNESS FOR A PARTICULAR PURPOSE.  See the
 * GNU General Public License for more details.

 * A copy of the GNU General Public License is available at
   <a href="http://www.gnu.org/licenses">GPL licence site</a>.
 */
\end{hide}
package umontreal.iro.lecuyer.simevents;\begin{hide}
import java.util.Collection;
import java.util.List;
import java.util.Iterator;
import java.util.ListIterator;
import java.util.NoSuchElementException;
import umontreal.iro.lecuyer.stat.Tally;
import umontreal.iro.lecuyer.util.TransformingList;
import umontreal.iro.lecuyer.util.PrintfFormat;
\end{hide}

public class ListWithStat<E>
             extends TransformingList<E, ListWithStat.Node<E>>\begin{hide} {
   private boolean stats; // true si on a appele setStatCollecting
   private double initTime; // temps de la derniere initialisation
   private Accumulate blockSize; //block stat. sur la longueur de la liste
   private Tally blockSojourn; // block stat. sur les durees de sejour
   private String name;
   private Simulator sim;
\end{hide}
\end{code}

%%%%%%%%%%%%%%%%%%%%%%%%%%%%%%%%%%%
\subsubsection* {Constructors}
\begin{code}

   public ListWithStat (List<Node<E>> nodeList)\begin{hide} {
      super (nodeList);
      nodeList.clear();
      sim = Simulator.getDefaultSimulator();
      stats = false;
   }\end{hide}
\end{code}
\begin{tabb}
  Constructs a new list with internal data structure using the default simulator and implemented by \texttt{nodeList}.
  The given list is cleared for the constructed list to be initially empty.
\end{tabb}
\begin{code}

   public ListWithStat (Simulator inSim, List<Node<E>> nodeList)\begin{hide} {
      super (nodeList);
      if (inSim == null || nodeList == null)
          throw new NullPointerException();
      nodeList.clear();
      sim = inSim;
      stats = false;
   }\end{hide}
\end{code}
\begin{tabb}
  Constructs a new list with internal data structure implemented by
  \texttt{nodeList}.
  The given list is cleared for the constructed list to be initially empty.
\end{tabb}
\begin{htmlonly}
    \param{nodeList}{the list containing the nodes}
\end{htmlonly}
\begin{code}

   public ListWithStat (List<Node<E>> nodeList, Collection<? extends E> c)\begin{hide} {
      this (Simulator.getDefaultSimulator(), nodeList);
      addAll (c);
   }\end{hide}
\end{code}
\begin{tabb} Constructs a list containing the elements of the specified
   collection, whose elements are stored into \texttt{nodeList} and using the default simulator.
\end{tabb}
\begin{htmlonly}
    \param{nodeList}{the list containing the nodes}
    \param{c}{collection containing elements to fill in this list with}
\end{htmlonly}
\begin{code}

   public ListWithStat (Simulator inSim, List<Node<E>> nodeList,
                        Collection<? extends E> c)\begin{hide} {
      this (inSim, nodeList);
      addAll (c);
   }\end{hide}
\end{code}
\begin{tabb} Constructs a list containing the elements of the specified
   collection, whose elements are stored into \texttt{nodeList}.
\end{tabb}
\begin{htmlonly}
    \param{inSim}{simulator associate to the current variable}
    \param{nodeList}{the list containing the nodes}
    \param{c}{collection containing elements to fill in this list with}
\end{htmlonly}
\begin{code}

   public ListWithStat (List<Node<E>> nodeList, String name)\begin{hide} {
      this (Simulator.getDefaultSimulator(), nodeList);
      this.name = name;
   }\end{hide}
\end{code}
\begin{tabb} Constructs a new list with name \texttt{name}, internal
   list \texttt{nodeList}, and using the default simulator.
   This name can be used to identify the list in traces and reports.
   The given list is cleared for the constructed list to be initially empty.
\end{tabb}
\begin{htmlonly}
    \param{nodeList}{the list containing the nodes}
   \param{name}{name for the list object}
\end{htmlonly}
\begin{code}

   public ListWithStat (Simulator inSim, List<Node<E>> nodeList,
                        String name)\begin{hide} {
      this (inSim, nodeList);
      this.name = name;
   }\end{hide}
\end{code}
\begin{tabb} Constructs a new list with name \texttt{name}, and
  internal list \texttt{nodeList}.
   This name can be used to identify the list in traces and reports.
  The given list is cleared for the constructed list to be initially empty.
\end{tabb}
\begin{htmlonly}
    \param{inSim}{simulator associate to the current variable}
    \param{nodeList}{the list containing the nodes}
   \param{name}{name for the list object}
\end{htmlonly}
\begin{code}

   public ListWithStat (List<Node<E>> nodeList, Collection<? extends E> c,
                        String name)\begin{hide} {
      this (Simulator.getDefaultSimulator(), nodeList);
      this.name = name;
      addAll (c);
   }\end{hide}
\end{code}
\begin{tabb} Constructs a new list containing the elements of the
   specified collection \texttt{c}, with name \texttt{name},
   internal list \texttt{nodeList}, and using the default simulator.
   This name can be used to identify the list in traces and reports.
\end{tabb}
\begin{htmlonly}
   \param{nodeList}{the list containing the nodes}
   \param{c}{collection containing elements to fill in this list with}
   \param{name}{name for the list object}
\end{htmlonly}
\begin{code}

   public ListWithStat (Simulator inSim, List<Node<E>> nodeList,
                        Collection<? extends E> c, String name)\begin{hide} {
      this (inSim, nodeList);
      this.name = name;
      addAll (c);
   }\end{hide}
\end{code}
\begin{tabb} Constructs a new list containing the elements of the
   specified collection \texttt{c}, with name \texttt{name}, and
   internal list \texttt{nodeList}.
   This name can be used to identify the list in traces and reports.
\end{tabb}
\begin{htmlonly}
    \param{inSim}{simulator associate to the current variable}
    \param{nodeList}{the list containing the nodes}
   \param{c}{collection containing elements to fill in this list with}
   \param{name}{name for the list object}
\end{htmlonly}

%%%%%%%%%%%%%%%%%%%%%%%%%%%%%%%%%%%
\subsubsection* {Methods}
\begin{code}

   public E convertFromInnerType (Node<E> node)\begin{hide} {
      return node.getElement();
   }\end{hide}

   public Node<E> convertToInnerType (E element)\begin{hide} {
      return new Node<E> (element, sim.time());
   }\end{hide}

   public Simulator simulator() \begin{hide} {
      return sim;
   } \end{hide}
\end{code}
 \begin{tabb}   Returns the simulator associated with this list.
 \end{tabb}
\begin{htmlonly}
   \return{the simulator associated with this list}
\end{htmlonly}
\begin{code}

   public void setSimulator(Simulator sim) \begin{hide} {
       if (sim == null)
         throw new NullPointerException();
      this.sim = sim;
      if (blockSize != null)
         blockSize.setSimulator (sim);
   } \end{hide}
\end{code}
 \begin{tabb}   Sets the simulator associated with this list.
   This list should be cleared after this method is called.
 \end{tabb}
\begin{htmlonly}
   \param{sim}{the simulator of this list}
\end{htmlonly}
\begin{code}\begin{hide}

   @Override
   public void clear() {
      if (stats)
         initStat();
      super.clear();
   }

   @Override
   public void add (int index, E obj)  {
      super.add (index, obj);
      if (stats)
         blockSize.update (size());
   }

   @Override
   public E remove (int index) {
      Node<E> node = getInnerList().get (index);
      if (stats)
         blockSojourn.add (sim.time() - node.getInsertionTime());
      E e = super.remove (index);
      if (stats)
         blockSize.update (size());
      return e;
   }

   @Override
   public Iterator<E> iterator() {
      return new IteratorWithStat (getInnerList().iterator());
   }

   @Override
   public ListIterator<E> listIterator() {
      return new ListIteratorWithStat (getInnerList().listIterator());
   }

   @Override
   public ListIterator<E> listIterator (int index) {
      return new ListIteratorWithStat (getInnerList().listIterator (index));
   }

   @Override
   public E set (int index, E element) {
      Node<E> oldNode = getInnerList().get (index);
      E oldElement = oldNode.getElement();
      boolean equal;
      if (oldElement == null || element == null)
         equal = oldElement == oldElement;
      else
         equal = oldElement.equals (element);
      if (equal) {
         getInnerList().set (index, new Node<E> (element, oldNode.getInsertionTime()));
         return oldElement;
      }
      else {
         if (stats)
            blockSojourn.add (sim.time() - oldNode.getInsertionTime());
         getInnerList().set (index, new Node<E> (element, sim.time()));
         return oldElement;
      }
   }

   private class IteratorWithStat implements Iterator<E> {
      private Iterator<Node<E>> itr;
      private Node<E> lastRet;

      public IteratorWithStat (Iterator<Node<E>> itr) {
         this.itr = itr;
      }

      public boolean hasNext() {
         return itr.hasNext();
      }

      public E next() {
         lastRet = itr.next();
         return lastRet.getElement();
      }

      public void remove() {
         itr.remove();
         if (stats) {
            blockSize.update (size());
            blockSojourn.add (sim.time() - lastRet.getInsertionTime());
         }
         lastRet = null;
      }
   }

   private class ListIteratorWithStat implements ListIterator<E> {
      private ListIterator<Node<E>> itr;
      private Node<E> lastRet;

      public ListIteratorWithStat (ListIterator<Node<E>> itr) {
         this.itr = itr;
      }

      public void add (E o) {
         itr.add (new Node<E> (o, sim.time()));
         lastRet = null;
         if (stats)
            blockSize.update (size());
      }

      public boolean hasNext() {
         return itr.hasNext();
      }

      public boolean hasPrevious() {
         return itr.hasPrevious();
      }

      public E next() {
         lastRet = itr.next();
         return lastRet.getElement();
      }

      public int nextIndex() {
         return itr.nextIndex();
      }

      public E previous() {
         lastRet = itr.previous();
         return lastRet.getElement();
      }

      public int previousIndex() {
         return itr.previousIndex();
      }

      public void remove() {
         itr.remove();
         if (stats) {
            blockSize.update (size());
            blockSojourn.add (sim.time() - lastRet.getInsertionTime());
         }
         lastRet = null;
      }

      public void set (E element) {
         if (lastRet == null)
            throw new NoSuchElementException();
         Node<E> oldNode = lastRet;
         E oldElement = oldNode.getElement();
         boolean equal;
         if (oldElement == null || element == null)
            equal = oldElement == oldElement;
         else
            equal = oldElement.equals (element);
         if (equal) {
            lastRet = new Node<E> (element, oldNode.getInsertionTime());
            itr.set (lastRet);
         }
         else {
            if (stats)
               blockSojourn.add (sim.time() - oldNode.getInsertionTime());
            lastRet = new Node<E> (element, sim.time());
            itr.set (lastRet);
         }
      }
   }
\end{hide}
\end{code}

%%%%%%%%%%%%%%%%%%%%%%%%%%%%%%%%%%
\subsubsection* {Statistic collection methods}

\begin{code}

  public boolean getStatCollecting()\begin{hide} {
     return stats;
  }\end{hide}
\end{code}
\begin{tabb}   Returns \texttt{true} if the list collects statistics
  about its size and sojourn times of elements, and
  \texttt{false} otherwise.
  By default, statistical collecting is turned off.
\end{tabb}
\begin{htmlonly}
   \return{the status of statistical collecting}
\end{htmlonly}
\begin{code}

  public void setStatCollecting (boolean b)\begin{hide} {
    if (b && !stats) {
      if (blockSize == null)
      blockSize = new Accumulate(sim, "List Size " + name);
      if (blockSojourn == null)
      blockSojourn = new Tally("List Sojourn " + name);
      blockSize.update (size());
      stats = true;
      initStat();
    } else
    stats = false;
  }\end{hide}
\end{code}
  \begin{tabb}  Starts or stops collecting statistics on this list.
   If the statistical collection is turned ON, the method
   creates two statistical probes if they do not exist yet.
   The first one, of the class \class{Accumulate}, measures the evolution
   of the size of the list as a function of time.
   It can be accessed by the method \method{statSize}{}.
   The second one, of the class \class{Tally}  and accessible via
   \method{statSojourn}{}, samples the sojourn times in the list of the
   objects removed during the observation period,
   i.e., between the last initialization time of this statistical
   probe and the current time.
   The method automatically calls \method{initStat}{} to
   initialize these two probes.
   When this method is used, it is normally invoked immediately after
   calling the constructor of the list.
  \end{tabb}
\begin{htmlonly}
   \exception{IllegalStateException}{if the statistical collection
     is in the same state as the caller requires}
\end{htmlonly}
\begin{code}

   public void initStat() \begin{hide} {
    if (!stats)
    throw new IllegalStateException("initStat for a list that did not call setStatCollecting (true).");
    blockSize.init();
    blockSojourn.init();
    blockSize.update (size());
    initTime = sim.time();
  }\end{hide}
\end{code}
  \begin{tabb} Reinitializes the two statistical probes created by
   \method{setStatCollecting}{}~\texttt{(true)} and makes an update for the
   probe on the list size.
  \end{tabb}
\begin{htmlonly}
   \exception{IllegalStateException}{if the statistical collection is disabled}
\end{htmlonly}
\begin{code}

   public double getInitTime()\begin{hide} {
      return initTime;
   }\end{hide}
\end{code}
\begin{tabb}   Returns the last simulation time \method{initStat}{} was called.
\end{tabb}
\begin{htmlonly}
   \return{the last simulation time \method{initStat}{} was called}
\end{htmlonly}
\begin{code}

   public Accumulate statSize() \begin{hide}   {
       return blockSize;
   }\end{hide}
\end{code}
  \begin{tabb}  Returns the statistical probe on the evolution of the size of
   the list as a function of the simulation time.  This probe
   exists only if \method{setStatCollecting}{}~\texttt{(true)}
   has been called for this list.
  \end{tabb}
\begin{htmlonly}
   \return{the statistical probe on the evolution of the size of the list}
\end{htmlonly}
\begin{code}

   public Tally statSojourn() \begin{hide}  {
      return blockSojourn;
   }\end{hide}
\end{code}
  \begin{tabb}  Returns the statistical probe on the sojourn times of the objects in
   the list.  This probe exists
   only if \method{setStatCollecting}{}~\texttt{(true)} has been called for this list.
  \end{tabb}
\begin{htmlonly}
   \return{the statistical probe for the sojourn times in the list}
\end{htmlonly}
\begin{code}

   public String report() \begin{hide}  {
        if (blockSojourn == null || blockSize == null)
            throw new IllegalStateException
                ("Calling report when no statistics were collected");

        PrintfFormat str = new PrintfFormat();
        str.append (PrintfFormat.NEWLINE +
            "REPORT ON LIST : ").append (name).append (PrintfFormat.NEWLINE);
        str.append ("   From time: ").append (7, 2, 2, initTime);
        str.append (" to time: ").append (10, 2, 2, sim.time());
        str.append ("                  min        max      average  ");
        str.append ("standard dev.  nb. Obs");

        str.append ("   Size    ");
        str.append (9, (int)(blockSize.min()+0.5));
        str.append (11, (int)(blockSize.max()+0.5));
        str.append (14, 3, 2, blockSize.average()).append(PrintfFormat.NEWLINE);

        str.append ("   Sojourn ");
        str.append ( 12, 3, 2, blockSojourn.min()).append (" ");
        str.append (10, 3, 2, blockSojourn.max()).append (" ");
        str.append (10, 3, 2, blockSojourn.average()).append (" ");
        str.append (10, 3, 2, blockSojourn.standardDeviation()).append (" ");
        str.append (11, blockSojourn.numberObs()).append (PrintfFormat.NEWLINE);

        return str.toString();
    }\end{hide}
\end{code}
  \begin{tabb}  Returns a string containing a statistical report on the list,
   provided that \method{setStatCollecting}{} \texttt{(true)} has been
   called before for this list.
   Even If \method{setStatCollecting}{} was called with \texttt{false}
   afterward, the report will be made for
   the collected observations.
   If the probes do not exist, i.e., \method{setStatCollecting}{} was never called
   for this object, an illegal state exception will be thrown.
  \end{tabb}
\begin{htmlonly}
   \return{a statistical report, represented as a string}
   \exception{IllegalStateException}{if no statistical probes exist}
\end{htmlonly}
\begin{code}

   public String getName()\begin{hide} {
      return name;
   }\end{hide}
\end{code}
\begin{tabb}   Returns the name associated to this list,
  or \texttt{null} if no name was assigned.
\end{tabb}
\begin{htmlonly}
   \return{the name associated to this list}
\end{htmlonly}

\subsubsection*{Inner class}

\begin{code}

   public static class Node<E>\begin{hide} {
      private E element;
      private double insertionTime;\end{hide}
\end{code}
\begin{tabb}   Represents a node that can be part of a list with
  statistical collecting.
\end{tabb}
\subsubsection*{Constructor}
\begin{code}

      public Node (E element, double insertionTime)\begin{hide} {
         this.element = element;
         this.insertionTime = insertionTime;
      }\end{hide}
\end{code}
\begin{tabbb}  Constructs a new node containing element
  \texttt{element}
  inserted into the list at time \texttt{insertionTime}.
\end{tabbb}
\begin{htmlonly}
   \param{element}{the element to add into this new node}
   \param{insertionTime}{the insertion time of the element}
\end{htmlonly}
\subsubsection*{Methods}
\begin{code}

      public E getElement()\begin{hide} { return element; }\end{hide}
\end{code}
\begin{tabbb}  Returns the element stored into this node.
\end{tabbb}
\begin{htmlonly}
   \return{the element into this node}
\end{htmlonly}
\begin{code}

      public double getInsertionTime()\begin{hide} { return insertionTime; }\end{hide}
\end{code}
\begin{tabbb} Returns the insertion time of the element in this node.
\end{tabbb}
\begin{htmlonly}
   \return{the insertion time of the element}
\end{htmlonly}
\begin{code}\begin{hide}

      public String toString() {
         String str = element == null ? "null" : element.toString();
         str += " (inserted at time " + insertionTime + ")";
         return str;
      }
   }\end{hide}
\end{code}
\begin{code}\begin{hide}
}\end{hide}
\end{code}
