
\section{Overview of SSJ}

SSJ is an organized set of packages whose purpose is to facilitate 
stochastic simulation programming in the Java language.
The facilities offered are grouped into different packages, 
each one having its own user's guide as a \texttt{.pdf} file.
This is the official documentation.
There is also a simplified on-line documentation in HTML 
format, produced via \texttt{javadoc}.
Early descriptions of SSJ are given in \cite{sLEC02a,sLEC05a}.
Some of the tools can also be used for modeling (e.g., selecting
and fitting distributions).
SSJ is still growing actively.
New packages, classes, and methods will be added in 
forthcoming years and others will be refined.




\begin{latexonly} %%%%%%%

The packages currently offered are the following:
%
\begin{verse}
\externalclass{umontreal.iro.lecuyer}{util}
  contains utility classes used in the implementation of SSJ,
  and which are often useful elsewhere.
  For example, there are timers (for CPU usage),
  utilities to read or format numbers and arrays from/to text,
  operations on binary vectors and matrices,
  some mathematical functions and constants,
  root-finding tools,
  facilities for SQL database interface, and so on.

\externalclass{umontreal.iro.lecuyer}{probdist}
  contains a set of Java classes providing methods to compute mass,
  density, distribution, complementary distribution,
  and inverse distribution functions for many discrete and continuous
  probability distributions, as well as estimating the parameters of
  these distributions.

\externalclass{umontreal.iro.lecuyer}{probdistmulti}
  contains a set of Java classes providing methods to compute mass,
  density, distribution, complementary distribution,
  for some multi-dimensionnal discrete and continuous
  probability distributions.

\externalclass{umontreal.iro.lecuyer}{rng}
  provides facilities for generating uniform random numbers over the
  interval $(0,1)$, or over a given range of integer values, and other
  types of simple random objects such as random permutations.

\externalclass{umontreal.iro.lecuyer}{hups}
 provides classes implementing highly uniform point sets and
 sequences (HUPS), also called low-discrepancy sets and sequences,
 and  tools for their randomization.

\externalclass{umontreal.iro.lecuyer}{randvar}
  provides a collection of classes for non-uniform random variate
  generation, primarily from standard distributions.

\externalclass{umontreal.iro.lecuyer}{randvarmulti}
  provides a collection of classes for
  random number generators for some multi-dimensional distributions.

\externalclass{umontreal.iro.lecuyer}{gof}
  contains tools for performing univariate goodness-of-fit
  (GOF) statistical tests.

\externalclass{umontreal.iro.lecuyer}{stat}
  provides elementary tools for collecting statistics
  and computing confidence intervals.

\externalclass{umontreal.iro.lecuyer}{stat.list}
  this subpackage of \texttt{stat}
  provides support to manage lists of statistical collectors.
% Statistical operations may be applied on all the collectors easily.

\externalclass{umontreal.iro.lecuyer}{simevents}
  provides and manages the event-driven simulation facilities as well
  as the simulation clock.  Can manage several simulations in parallel,
  in the same program.

\externalclass{umontreal.iro.lecuyer}{simevents.eventlist}
 this subpackage of \texttt{simevents} offers several kinds of event
 list implementations.

\externalclass{umontreal.iro.lecuyer}{simprocs}
  provides and manages the process-driven simulation facilities.
%  This package requires \emph{green threads}, which are available
%  only in JDK version 1.3.1 or earlier (unfortunately).
%
%  simprocs ne nécessite pas les green threads absolument.
%  Premièrement, si l'interpréteur D-SOL est employé, tout fonctionne #1, même
%  sans green threads. Avec l'implantation par défaut de SimProcess, qui utilise
%  les threads, c'est mieux avec les green threads, mais cela peut fonctionner
%  (quoique pas toujours stable) avec les native threads. Il faudrait donc dire,
%  dans le texte, qu'il est recommandé d'utiliser une JVM avec green threads,
%  comme Sun JRE <=1.3.1.

\externalclass{umontreal.iro.lecuyer}{functions}
 contains classes that allow one to pass an arbitrary function of one variable
 as argument to a method
 and to apply elementary mathematical operations on generic functions.

\externalclass{umontreal.iro.lecuyer}{functionfit}
provides basic facilities for curve fitting and interpolation
with polynomials. % as, for example, least square fit and spline interpolation.

\externalclass{umontreal.iro.lecuyer}{charts}
provides tools for easy construction, visualization, and customization
of $xy$ plots, histograms, and empirical styled charts
from a Java program.

\externalclass{umontreal.iro.lecuyer}{stochprocess}
implements different kinds of stochastic processes.
%%
\end{verse}
\end{latexonly}  %%%%%%%%%



%%%%%%%%%%%%%%%%%%%%%%%%%%%%%%%%%%%%%%%%%%%%
\subsection*{Dependence on other libraries}

SSJ uses some classes from other free Java libraries.

The \htmladdnormallink{Colt library}{http://acs.lbl.gov/software/colt/},
 developed at the Centre Europ\'een de Recherche
Nucl\'eaire (CERN) in Geneva \cite{iHOS04a},
is a large library that provides a wide range of facilities for
high performance scientific and technical computing in Java.
SSJ uses the class \externalclass{cern.colt.list}{DoubleArrayList}
from Colt in a few of its classes, namely in packages
\externalclass{umontreal.iro.lecuyer}{stat} and
\externalclass{umontreal.iro.lecuyer}{hups}.
The reason is that this class provides a very efficient and convenient
implementation of an (automatically) extensible array of {\tt double},
together with several methods for computing statistics for the observations
% Javadoc does not find cern.jet.stat.Descriptive and issues
% a warning; maybe a bug in Javadoc.
stored in the array (see, e.g., {\tt Descriptive}).
% The Colt library must be installed (see the installation instructions
% of SSJ) if one wishes to use a class relying on it.
% Otherwise, Colt is not needed.
The Colt library is distributed with the SSJ package as \textbf{colt.jar}.
It  must be added in the CLASSPATH environment variable.


The \textbf{linear\_algebra} library is based on public domain LINPACK routines.
They were translated from Fortran to Java by Steve Verrill at the
    USDA Forest Products Laboratory
    Madison, Wisconsin, USA.
This software is also in the public domain and is included in the
SSJ distribution as the \textbf{Blas.jar} archive. It is used only in the
\texttt{probdist} package to compute maximum likelihood estimators.


The optimization package of Steve Verrill includes Java translations of the
\htmladdnormallink{MINPACK}{http://www.iro.umontreal.ca/~simardr/Uncmin_f77/Minpack_f77.html}
 routines \cite{iMOR80a} for  nonlinear least squares problems as well as
\htmladdnormallink{UNCMIN}{http://www.iro.umontreal.ca/~simardr/Uncmin_f77/Uncmin_f77.html}
 routines \cite{iSCHa} for
unconstrained optimization. They were translated from Fortran to Java by
Steve Verrill and are in the public domain. They are included in the SSJ
distribution as the \textbf{optimization.jar} archive. It is used only in the \texttt{probdist}
 package to compute maximum likelihood estimators.


\htmladdnormallink{JFreeChart}{http://www.jfree.org/jfreechart/index.html} is a free
Java library that can generate a wide variety of charts and plots for use in
applications, applets and servlets. \textbf{JFreeChart}  currently supports, amongst
others, bar charts, pie charts, line charts, XY-plots, histograms, scatter plots and
time series plots. It is distributed with SSJ as \textbf{jfreechart-*.jar}.
\htmladdnormallink{JCommon}{http://www.jfree.org/jcommon/index.php} is a free
general purpose Java library containing many useful classes used by JFreeChart and
 other Java packages. It is distributed with SSJ as \textbf{jcommon-*.jar}.
JFreeChart (and JCommon) are used in the SSJ package \textbf{charts} to create
different kinds of charts.

SSJ also provides an interface to the
\htmladdnormallink{UNURAN}{http://statistik.wu-wien.ac.at/unuran/}
library for nonuniform random number generation \cite{iLEY02a}, in the
\externalclass{umontreal.iro.lecuyer}{randvar} package.
UNURAN does not have to be installed to be used with SSJ, because it is
linked statically with the appropriate SSJ native library.
However, the UNURAN documentation will be required
to take full advantage of the library.




%%%%%%%%%%%%%%%%%%%%%%%%%%%%%%%%%%%%%%%%%%%%%%%%%%%%%%%%%
\begin{comment}

% Three packages are used for uniform and non-uniform random number
% generation, one for the uniform random number generators, one for the
% probability distributions and one for the non-uniform random number
% generation.  It is also possible to replace the streams of uniform
% random numbers with highly uniform point sets for Quasi-Monte Carlo
% \cite{vLEC02a} simulation.
% As opposed to most random number generation libraries, it
% is also possible to compute densities, distribution functions and
% inverse distribution functions for all supported distributions.
% The \externalclass{umontreal.iro.lecuyer}{simevents}
% and \externalclass{umontreal.iro.lecuyer}{simprocs} packages make the
% heart of the SSJ library.
% They provide an efficient framework for stochastic simulation,
% supporting the event view, proces view, continuous simulation, and arbitrary 
% mixtures of these.
% The other packages allow one to perform miscellaneous tasks, such as
% statistical collection and goodness of fit tests.

\subsection*{random number generation}

Random numbers feed simulation models and allow one to compute
statistics.  To generate random numbers from
any probability distribution, uniform random numbers are required.
Such numbers are uniformly distributed in the $[0,1)$ interval, i.e.,
the probability of getting a given number $x$ in that interval is the same
for all values of $x\in[0,1)$.  Any generated number $x$ is also
independent from any previous or future generated numbers.  Although
the generated uniforms are not truly independent since one uniform is
obtained from the previous uniforms by a mathematical formula, one can
consider them independent for simulation purposes.  Selection of a
random number generator is based on several criteria such as
uniformity, performance, and portability \cite{rLEC01d}.
The package
\externalclass{umontreal.iro.lecuyer}{rng} contains the needed tools
to generate such numbers.  It defines an interface called
\externalclass{umontreal.iro.lecuyer.rng}{RandomStream} implemented by
any random number generator supported by SSJ.  This interface allows
one to easily interchange random number generators since they are
accessed through the same set of methods specified by the interface.
Only the random number generator setup depends on the type of
generator that was chosen.

If one wants to replace uniform random numbers with low-discrepancy
point sets for variance reduction, the package
\externalclass{umontreal.iro.lecuyer}{hups} contains all the necessary
facilities.  Such highly uniform point sets all inherit from the
\externalclass{umontreal.iro.lecuyer.hups}{PointSet} which provides a
\externalclass{umontreal.iro.lecuyer.hups}{PointSetIterator} extending
\externalclass{umontreal.iro.lecuyer.rng}{RandomStream}.  The
replacement can be easily done without modifying the model
implementation, except the setup-time code.

To generate non-uniform random numbers, one must select a probability
distribution based on the empirical data \cite{sLAW00a}.
SSJ does not provide
probability distribution estimation tools, but goodness of fit tests are
included to help in model validation.  The package
\externalclass{umontreal.iro.lecuyer}{probdist} contains several
standard, commonly-used, probability distributions.  It supports
discrete and continuous distributions through two different abstract
base classes:
\externalclass{umontreal.iro.lecuyer.probdist}{ContinuousDistribution}
and
\externalclass{umontreal.iro.lecuyer.probdist}{DiscreteDistribution},
respectively.  Again, since the distributions inherit from a common
class, their access can be independent from the selected distribution,
except for the setup case.  One can compute the density/mass, distribution,
complementary, and inverse distribution functions.  These facilities
are also accessible through static methods implemented in each
distribution class if one does not want to create objects or needs
distributions whose parameters vary in time.
However, setup-time operations must be performed for each operation,
which can be inefficient for certain distributions.

To generate non-uniform random numbers, the packages
\externalclass{umontreal.iro.lecuyer}{rng} (or
\externalclass{umontreal.iro.lecuyer}{hups}) and
\externalclass{umontreal.iro.lecuyer}{probdist} must be used
together.  The simplest generation method is to generate a uniform
random number using a generator implementing
\externalclass{umontreal.iro.lecuyer.rng}{RandomStream} (or get a
coordinate using a point set iterator) and to apply
inversion by using the selected
\externalclass{umontreal.iro.lecuyer}{probdist} distribution's
\externalmethod{umontreal.iro.lecuyer.probdist}{ContinuousDistribution}{inverseF}{} method.
However, inversion is not the only generation method and sometimes not
the most efficient.  For some distributions, closed-form inverse functions
or fast inversion algorithms exist.
For others, inversion is performed using
binary or even linear search.  In such cases, the performance
and precision depends on the complexity of the distribution function
which is calculated several times for one inverse.
The package
\externalclass{umontreal.iro.lecuyer}{randvar} acts as glue between
uniform random number generators and probability distributions.  Continuous
or discrete random number generators also inherits from common base
classes, namely
\externalclass{umontreal.iro.lecuyer.randvar}{RandomVariateGen} and
\externalclass{umontreal.iro.lecuyer.randvar}{RandomVariateGenInt}.
All generators use a random stream and a probability
distribution for their construction.  As opposed to
\externalclass{umontreal.iro.lecuyer}{probdist}, one can directly
instantiate
\externalclass{umontreal.iro.lecuyer.randvar}{RandomVariateGen} or
\externalclass{umontreal.iro.lecuyer.randvar}{RandomVariateGenInt}.
However, in such cases, only inversion generation method will be
available.  To use an alternate generation method, one must
instantiate a specialized generator class and switch to the given
generation algorithm using an object method.
Each specialized class also provides static method which perform
the same action.  Although they allow one to avoid object creation,
their signatures are specific to the used distribution and they have
to perform setup-time operations on each variate generation, which
can become inefficient.
The \externalclass{umontreal.iro.lecuyer}{randvar} package also
provides the class
\externalclass{umontreal.iro.lecuyer.randvar}{RandomVariateTrans} to
apply transformations to a generator.  This allows, for example, to
generate variates from a truncated distribution.

\subsection*{Performing simulation}

SSJ supports discrete-event, process-driven, continuous or mixed
simulation.  The discrete-event and continuous simulation are managed
by the package \externalclass{umontreal.iro.lecuyer}{simevents}.  This
package manages the simulation clock and the event list, two essential
components for all discrete-event simulations.  The simulation clock
tracks the simulation time whereas the event list stores the
scheduled events to execute them in the right order.
Events are user-defined subclasses of
\externalclass{umontreal.iro.lecuyer.simevents}{Event}.  When an event
occurs, any type of actions can then be taken.  The package
provides a class called
\externalclass{umontreal.iro.lecuyer.simevents}{List} which implements
a linked list supporting statistical collection.  Continuous
simulation can be performed using the class
\externalclass{umontreal.iro.lecuyer.simevents}{Continuous}.  It uses
the event framework to resolve differential equations numerically at
fixed steps in the simulation time.

Process-driven simulation requires a separate package called
\externalclass{umontreal.iro.lecuyer}{simprocs}.  This package
provides the base class
\externalclass{umontreal.iro.lecuyer.simprocs}{SimProcess} which must
be extended by any process.  A simulation process is an autonomous
object which executes tasks and interacts with other simulation
processes.  To allow such interactions, the package provides some
synchronization tools such as a resource, a bin, and a condition.
The process-driven simulation framework relies on the event-driven
framework.  Processes start, resume and wake up on scheduled events.
One can then easily mix event-driven and process-driven simulation.

\subsection*{Other tools}

The package \externalclass{umontreal.iro.lecuyer}{stat} provides basic
tools for statistical collection.  Statistics are collected using
statistical probes, i.e, objects implementing the abstract class
\externalclass{umontreal.iro.lecuyer.stat}{StatProbe}.  Two types of
probes are supported.  The
\externalclass{umontreal.iro.lecuyer.stat}{Tally} allows to collect
observations of the form $X_1,\dots,X_n$ whereas
\externalclass{umontreal.iro.lecuyer.simevents}{Accumulate} collects
statistics for a continuous variable evolving in simulation time.
During the simulation, one can add observations to such probes.  After
the simulation, measures can be obtained, such as sample average,
sample standard deviation or confidence interval.  A statistical
report can be obtained for all probes.  The package also provides
a way to detach statistical collection from the model implementation
by using bound properties.

To test a proposed model against empirical data, goodness of fit tests
are provided in the package
\externalclass{umontreal.iro.lecuyer}{gof}.  
Such tests, e.g.\ Kolmogorov-Smirnov
or Anderson-Darling, compute a statistic using the
empirical observations and the proposed distribution.  The empirical
observations are given as an array whereas the
distribution is given as a
\externalclass{umontreal.iro.lecuyer}{probdist} object.  From the computed
statistic, it is possible to compute the $p$-value which is useful to
evaluate the significance of the test.

\subsection*{Related documentation}

The \texttt{example.pdf} file, in the \texttt{doc/pdf} subdirectory of
the SSJ distribution, explains simulation examples implemented using
SSJ.  This may be the best starting point to learn SSJ.
\begin{htmlonly}
% This should be removed if we find a way to display blbliographical
% references in HTML
% LaTeX2HTML can do it when it is run for a LaTeX document, but it does not
% work for a single bibliography shared by several documents.
% Texjava processes each Java class as a separate, standalone, LaTeX document.
One can find additional information and references in the PDF version
of this documentation, available in the {\tt doc/pdf} subdirectory
of the SSJ distribution.
\end{htmlonly}
\begin{latexonly}
% In Javadoc, this is visible to the user as hyperlinks.  In LaTeX,
% this is separated filed.
Every package introduced here contains its own reference documentation
as a PDF file, in the \texttt{doc/pdf} subdirectory.
This documentation describes in more details how to
use the package and provides a description of each class and method.
\end{latexonly}


\end{comment}
%%%%%%%%%%%%%%%%%%%%%%
