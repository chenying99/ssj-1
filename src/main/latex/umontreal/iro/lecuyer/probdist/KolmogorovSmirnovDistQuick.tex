\defclass {KolmogorovSmirnovDistQuick}

Extends the class \class{KolmogorovSmirnovDist} for the \ks{} distribution.
The methods of this class %use a local program in our lab, and
are much faster than those of class
\class{KolmogorovSmirnovDist}.

\begin{htmlonly}
\end{htmlonly}%
\begin{latexonly}%
\end{latexonly}%


\bigskip\hrule

%%%%%%%%%%%%%%%%%%%%%%%%%%%%%%%%%%%%%%%%
\begin{code}\begin{hide}
/*
 * Class:        KolmogorovSmirnovDistQuick
 * Description:  Kolmogorov-Smirnov 2-sided 1-sample distribution
 * Environment:  Java
 * Software:     SSJ
 * Copyright (C) 2001  Pierre L'Ecuyer and Université de Montréal
 * Organization: DIRO, Université de Montréal
 * @author       Richard Simard
 * @since        January 2010

 * SSJ is free software: you can redistribute it and/or modify it under
 * the terms of the GNU General Public License (GPL) as published by the
 * Free Software Foundation, either version 3 of the License, or
 * any later version.

 * SSJ is distributed in the hope that it will be useful,
 * but WITHOUT ANY WARRANTY; without even the implied warranty of
 * MERCHANTABILITY or FITNESS FOR A PARTICULAR PURPOSE.  See the
 * GNU General Public License for more details.

 * A copy of the GNU General Public License is available at
   <a href="http://www.gnu.org/licenses">GPL licence site</a>.
 */
\end{hide}
package umontreal.iro.lecuyer.probdist;
\begin{hide}
import umontreal.iro.lecuyer.util.*;
import umontreal.iro.lecuyer.functions.MathFunction;
\end{hide}

public class KolmogorovSmirnovDistQuick extends KolmogorovSmirnovDist \begin{hide} {

   /*
      For n <= NEXACT, we use exact algorithms: the Durbin matrix and
      the Pomeranz algorithms. For n > NEXACT, we use asymptotic methods
      except for x close to 0 where we still use the method of Durbin
      for n <= NKOLMO. For n > NKOLMO, we use asymptotic methods only and
      so the precision is less for x close to 0.
      We could increase the limit NKOLMO to 10^6 to get better precision
      for x close to 0, but at the price of a slower speed.
   */
   private static final int NKOLMO = 100000;
\end{hide}
\end{code}
%%%%%%%%%%%%%%%%%%%%%%%%%%%%%%%%%%%%%%%%%
\subsubsection* {Constructor}

\begin{code}\begin{hide}

   private static class Function implements MathFunction {
      protected int n;
      protected double u;

      public Function (int n, double u) {
         this.n = n;
         this.u = u;
      }

      public double evaluate (double x) {
         return u - cdf(n,x);
      }
   }\end{hide}

   public KolmogorovSmirnovDistQuick (int n)\begin{hide} {
      super (n);
   }\end{hide}
\end{code}
\begin{tabb}
   Constructs a \ks{} distribution with parameter $n$.
\end{tabb}

%%%%%%%%%%%%%%%%%%%%%%%%%%%%%%%%%%%
\subsubsection* {Methods}

\begin{code}\begin{hide}

   public double density (double x) {
      return density (n, x);
   }

   public double cdf (double x) {
      return cdf (n, x);
   }

   public double barF (double x) {
      return barF (n, x);
   }

   public double inverseF (double u) {
      return inverseF (n, u);
   }
\end{hide}

   public static double density (int n, double x)\begin{hide} {

      double Res = densConnue(n,x);
      if (Res != -1.0)
         return Res;

      final double EPS = 1.0 / Num.TWOEXP[6];
      Res = (cdf(n, x + EPS) - cdf(n, x - EPS)) / (2.0 * EPS);
      if (Res <= 0.0)
         return 0.0;
      return Res;
   }\end{hide}
\end{code}
\begin{tabb} Computes the density for the \ks{} distribution with parameter $n$.
\end{tabb}
\begin{code} \begin{hide}

   private static double Pelz (int n, double x) {
   /* Approximating the Lower Tail-Areas of the Kolmogorov-Smirnov One-Sample
         Statistic,
         Wolfgang Pelz and I. J. Good,
         Journal of the Royal Statistical Society, Series B.
             Vol. 38, No. 2 (1976), pp. 152-156
    */

      final int JMAX = 20;
      final double EPS = 1.0e-10;
      final double RACN = Math.sqrt ((double) n);
      final double z = RACN * x;
      final double z2 = z * z;
      final double z4 = z2 * z2;
      final double z6 = z4 * z2;
      final double C2PI = 2.506628274631001;   // sqrt(2*Pi)
      final double DPI2 = 1.2533141373155001;  // sqrt(Pi/2)
      final double PI2 = Math.PI * Math.PI;
      final double PI4 = PI2 * PI2;
      final double w = PI2 / (2.0 * z * z);
      double ti, term, tom;
      double sum;
      int j;

      term = 1;
      j = 0;
      sum = 0;
      while (j <= JMAX && term > EPS * sum) {
         ti = j + 0.5;
         term = Math.exp (-ti * ti * w);
         sum += term;
         j++;
      }
      sum *= C2PI / z;

      term = 1;
      tom = 0;
      j = 0;
      while (j <= JMAX && Math.abs (term) > EPS * Math.abs (tom)) {
         ti = j + 0.5;
         term = (PI2 * ti * ti - z2) * Math.exp (-ti * ti * w);
         tom += term;
         j++;
      }
      sum += tom * DPI2 / (RACN * 3.0 * z4);

      term = 1;
      tom = 0;
      j = 0;
      while (j <= JMAX && Math.abs (term) > EPS * Math.abs (tom)) {
         ti = j + 0.5;
         term = 6 * z6 + 2 * z4 + PI2 * (2 * z4 - 5 * z2) * ti * ti +
                PI4 * (1 - 2 * z2) * ti * ti * ti * ti;
         term *= Math.exp (-ti * ti * w);
         tom += term;
         j++;
      }
      sum += tom * DPI2 / (n * 36.0 * z * z6);

      term = 1;
      tom = 0;
      j = 1;
      while (j <= JMAX && term > EPS * tom) {
         ti = j;
         term = PI2 * ti * ti * Math.exp (-ti * ti * w);
         tom += term;
         j++;
      }
      sum -= tom * DPI2 / (n * 18.0 * z * z2);

      term = 1;
      tom = 0;
      j = 0;
      while (j <= JMAX && Math.abs (term) > EPS * Math.abs (tom)) {
         ti = j + 0.5;
         ti = ti * ti;
         term = -30 * z6 - 90 * z6 * z2 + PI2 * (135 * z4 - 96 * z6) * ti +
                PI4 * (212 * z4 - 60 * z2) * ti * ti +
                PI2 * PI4 * ti * ti * ti * (5 - 30 * z2);
         term *= Math.exp (-ti * w);
         tom += term;
         j++;
      }
      sum += tom * DPI2 / (RACN * n * 3240.0 * z4 * z6);

      term = 1;
      tom = 0;
      j = 1;
      while (j <= JMAX && Math.abs (term) > EPS * Math.abs (tom)) {
         ti = j * j;
         term = (3 * PI2 * ti * z2 - PI4 * ti * ti) * Math.exp (-ti * w);
         tom += term;
         j++;
      }
      sum += tom * DPI2 / (RACN * n * 108.0 * z6);

      return sum;
   }


//========================================================================

   private static void CalcFloorCeil (
      int n,                       // sample size
      double t,                    // = nx
      double[] A,                  // A_i
      double[] Atflo,              // floor (A_i - t)
      double[] Atcei               // ceiling (A_i + t)
   )
   {
      // Precompute A_i, floors, and ceilings for limits of sums in the
      // Pomeranz algorithm
      int i;
      int ell = (int) t;           // floor (t)
      double z = t - ell;          // t - floor (t)
      double w = Math.ceil (t) - t;

      if (z > 0.5) {
         for (i = 2; i <= 2 * n + 2; i += 2)
            Atflo[i] = i / 2 - 2 - ell;
         for (i = 1; i <= 2 * n + 2; i += 2)
            Atflo[i] = i / 2 - 1 - ell;

         for (i = 2; i <= 2 * n + 2; i += 2)
            Atcei[i] = i / 2 + ell;
         for (i = 1; i <= 2 * n + 2; i += 2)
            Atcei[i] = i / 2 + 1 + ell;

      } else if (z > 0.0) {
         for (i = 1; i <= 2 * n + 2; i++)
            Atflo[i] = i / 2 - 1 - ell;

         for (i = 2; i <= 2 * n + 2; i++)
            Atcei[i] = i / 2 + ell;
         Atcei[1] = 1 + ell;

      } else {                       // z == 0
         for (i = 2; i <= 2 * n + 2; i += 2)
            Atflo[i] = i / 2 - 1 - ell;
         for (i = 1; i <= 2 * n + 2; i += 2)
            Atflo[i] = i / 2 - ell;

         for (i = 2; i <= 2 * n + 2; i += 2)
            Atcei[i] = i / 2 - 1 + ell;
         for (i = 1; i <= 2 * n + 2; i += 2)
            Atcei[i] = i / 2 + ell;
      }

      if (w < z)
         z = w;
      A[0] = A[1] = 0;
      A[2] = z;
      A[3] = 1 - A[2];
      for (i = 4; i <= 2 * n + 1; i++)
         A[i] = A[i - 2] + 1;
      A[2 * n + 2] = n;
   }


   //========================================================================

   private static double Pomeranz (int n, double x)
   {
      // The Pomeranz algorithm to compute the KS distribution
      final double EPS = 1.0e-15;
      final int ENO = 350;
      final double RENO = Math.scalb (1.0, ENO); // for renormalization of V
      int coreno;                    // counter: how many renormalizations
      final double t = n * x;
      double w, sum, minsum;
      int i, j, k, s;
      int r1, r2;                    // Indices i and i-1 for V[i][]
      int jlow, jup, klow, kup, kup0;
      double[] A = new double[2 * n + 3];
      double[] Atflo = new double[2 * n + 3];
      double[] Atcei = new double[2 * n + 3];
      double[][] V = new double[2][n + 2];
      double[][] H = new double[4][n + 2];     // = pow(w, j) / Factorial(j)

      CalcFloorCeil (n, t, A, Atflo, Atcei);

      for (j = 1; j <= n + 1; j++)
         V[0][j] = 0;
      for (j = 2; j <= n + 1; j++)
         V[1][j] = 0;
      V[1][1] = RENO;
      coreno = 1;

      // Precompute H[][] = (A[j] - A[j-1]^k / k!
      H[0][0] = 1;
      w = 2.0 * A[2] / n;
      for (j = 1; j <= n + 1; j++)
         H[0][j] = w * H[0][j - 1] / j;

      H[1][0] = 1;
      w = (1.0 - 2.0 * A[2]) / n;
      for (j = 1; j <= n + 1; j++)
         H[1][j] = w * H[1][j - 1] / j;

      H[2][0] = 1;
      w = A[2] / n;
      for (j = 1; j <= n + 1; j++)
         H[2][j] = w * H[2][j - 1] / j;

      H[3][0] = 1;
      for (j = 1; j <= n + 1; j++)
         H[3][j] = 0;

      r1 = 0;
      r2 = 1;
      for (i = 2; i <= 2 * n + 2; i++) {
         jlow = (int) (2 + Atflo[i]);
         if (jlow < 1)
            jlow = 1;
         jup = (int) (Atcei[i]);
         if (jup > n + 1)
            jup = n + 1;

         klow = (int) (2 + Atflo[i - 1]);
         if (klow < 1)
            klow = 1;
         kup0 = (int) (Atcei[i - 1]);

         // Find to which case it corresponds
         w = (A[i] - A[i - 1]) / n;
         s = -1;
         for (j = 0; j < 4; j++) {
            if (Math.abs (w - H[j][1]) <= EPS) {
               s = j;
               break;
            }
         }

         minsum = RENO;
         r1 = (r1 + 1) & 1;          // i - 1
         r2 = (r2 + 1) & 1;          // i

         for (j = jlow; j <= jup; j++) {
            kup = kup0;
            if (kup > j)
               kup = j;
            sum = 0;
            for (k = kup; k >= klow; k--)
               sum += V[r1][k] * H[s][j - k];
            V[r2][j] = sum;
            if (sum < minsum)
               minsum = sum;
         }

         if (minsum < 1.0e-280) {
            // V is too small: renormalize to avoid underflow of probabilities
            for (j = jlow; j <= jup; j++)
               V[r2][j] *= RENO;
            coreno++;              // keep track of log of RENO
         }
      }

      sum = V[r2][n + 1];
      w = Num.lnFactorial (n) - coreno * ENO * Num.LN2 + Math.log (sum);
      if (w >= 0.)
         return 1.;
      return Math.exp (w);
   }
   //========================================================================
\end{hide}

   public static double cdf (int n, double x)\begin{hide} {
      double u = cdfConnu (n, x);
      if (u >= 0.0)
         return u;

      final double w = n * x * x;
      if (n <= NEXACT) {
         if (w < 0.754693)
            return DurbinMatrix (n, x);
         if (w < 4.0)
            return Pomeranz (n, x);
         return 1.0 - barF (n, x);
      }

      if ((w * x * n <= 7.0) && (n <= NKOLMO))
         return DurbinMatrix(n, x);

      return Pelz (n, x);
   }\end{hide}
\end{code}
\begin{tabb}
  Computes the \ks{} distribution function $u = P[D_n \le x]$ with
 parameter $n$, using the program described in \cite{tSIM11a}.
 This method uses Pomeranz's recursion algorithm and the Durbin matrix algorithm
  \cite{tBRO08a,tPOM74a,tMAR03a} for
$n \le 500$, which returns at least 13 decimal digits of precision. It uses
the Pelz-Good asymptotic expansion \cite{tPEL76a} in the central part of
 the range for $n > 500$ and returns at least 7 decimal digits of precision
 everywhere for $500 < n \le 100000$. For $n > 100000$, it returns
 at least 5 decimal digits of precision for all $u > 10^{-16}$, and a
 few correct decimals when  $u \le 10^{-16}$.
% For a given $n> 500$, the precision increases as $x$ increases.
 This method is much faster than method \texttt{cdf} of
\class{KolmogorovSmirnovDist} for moderate or large $n$.
Restriction:  $n\ge 1$.
 \end{tabb}
\begin{code}

   public static double barF (int n, double x)\begin{hide} {
      double v = barFConnu (n, x);
      if (v >= 0.0)
         return v;

      final double w = n * x * x;
      if (n <= NEXACT) {
         if (w < 4.0)
            return 1.0 - cdf (n, x);
         else
            return 2.0 * KolmogorovSmirnovPlusDist.KSPlusbarUpper(n, x);
      }

      if (w >= 2.65)
         return 2.0 * KolmogorovSmirnovPlusDist.KSPlusbarUpper (n, x);

      return 1.0 - cdf (n, x);
   }\end{hide}
\end{code}
\begin{tabb}
 Computes the complementary \ks{} distribution
$P[D_n \ge x]$ with parameter $n$,
 in a form that is more precise in the upper tail,
using the program described in \cite{tSIM11a}.
It returns at least 10 decimal digits of precision everywhere for all
 $n \le 500$,
 at least 6 decimal digits of precision for $500 < n \le 200000$,
and a few correct decimal digits (1 to 5) for $n > 200000$.
This method is much faster and more precise for $x$ close to 1, than
 method \texttt{barF} of
\class{KolmogorovSmirnovDist} for moderate or large $n$.
 Restriction:  $n\ge 1$.
\end{tabb}
\begin{code}

   public static double inverseF (int n, double u)\begin{hide} {
      double Res = inverseConnue(n,u);
      if (Res != -1.0)
         return Res;
      Function f = new Function (n,u);
      return RootFinder.brentDekker (0.5/n, 1.0, f, 1e-5);
   }\end{hide}
\end{code}
\begin{tabb}
  Computes the inverse $x = F^{-1}(u)$ of the
  distribution $F(x)$ with parameter $n$.
\end{tabb}
\begin{code}\begin{hide}
}\end{hide}
\end{code}
