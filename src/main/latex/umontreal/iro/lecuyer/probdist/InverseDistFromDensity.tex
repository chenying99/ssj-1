\defclass {InverseDistFromDensity}

Implements a method for computing the  inverse of an \emph{arbitrary
continuous} distribution function when only the probability density
is known \cite{rDER09a}. The cumulative probabilities (cdf) are pre-computed by
 numerical quadrature  of the
density using Gauss-Lobatto integration over suitably small intervals to
satisfy the required precision, and these values are kept in tables. Then the
 algorithm uses polynomial interpolation  over the tabulated values to get
  the inverse cdf. The user can select the
  desired precision and the degree of the interpolating polynomials.

The algorithm may fail for some distributions for which the density
 becomes infinite at a point (for ex. the Gamma and the Beta distributions
 with $\alpha < 1$)  if one chooses too high a precision
(a  too small \texttt{eps}, for ex. $\epsilon \sim 10^{-15}$).
However, it should work also for continuous densities with finite discontinuities.


While the setup time for this class is relatively slow, the numerical inversion
 is extremely fast and practically independent of the required precision
and of the specific distribution. For comparisons between the times
of standard inversion and inversion from this class as well as 
comparisons between  setup times, see the introduction in class 
\externalclass{umontreal.iro.lecuyer.randvar}{InverseFromDensityGen}
from package \texttt{randvar}.

Thus if only a few inverses are needed, then using this class
 is not efficient because of the slow set-up. But if one wants to call
 \texttt{inverseF} thousands of times or more, then using this class will 
be very efficient.

\bigskip\hrule

\begin{code}
\begin{hide}
/*
 * Class:        InverseDistFromDensity
 * Description:  computing the inverse of an arbitrary continuous distribution
 * Environment:  Java
 * Software:     SSJ 
 * Copyright (C) 2001  Pierre L'Ecuyer and Université de Montréal
 * Organization: DIRO, Université de Montréal
 * @author       Richard Simard
 * @since        June 2009

 * SSJ is free software: you can redistribute it and/or modify it under
 * the terms of the GNU General Public License (GPL) as published by the
 * Free Software Foundation, either version 3 of the License, or
 * any later version.

 * SSJ is distributed in the hope that it will be useful,
 * but WITHOUT ANY WARRANTY; without even the implied warranty of
 * MERCHANTABILITY or FITNESS FOR A PARTICULAR PURPOSE.  See the
 * GNU General Public License for more details.

 * A copy of the GNU General Public License is available at
   <a href="http://www.gnu.org/licenses">GPL licence site</a>.
 */
\end{hide}
package umontreal.iro.lecuyer.probdist;
   import umontreal.iro.lecuyer.functions.MathFunction;\begin{hide}
import umontreal.iro.lecuyer.util.Misc;
import umontreal.iro.lecuyer.functions.MathFunctionUtil;
\end{hide}


public class InverseDistFromDensity extends ContinuousDistribution \begin{hide} {
   private final boolean DEBUG = false;
   protected static final double HALF_PI = Math.PI/2.0;
   private final double epsc = 1.0e-5; // for small values of lc in tails
   private final double HUGE = Double.MAX_VALUE / 2.0;
   private double epsu0;      // initial u-resolution
   private double xc;         // mode, mean, or median of distribution
                              // xc is an x where the density is high
   MathFunction m_dens;       // probability density
   private String name;       // Name of class

   private final int K0 = 128;  // initial size of tables A, F, ...
   private int Kmax;          // final size of tables A, F, ... is (Kmax + 1)
   private double[] A;        // x-values
   private double[] F;        // corresponding u-values (the CDF)
   private double[][] X;      // interpolation x-values in [A[k], A[k+1]]
   private double[][] U;      // interpolation u-values in [A[k], A[k+1]]
   private double[][] C;      // interpolation coefficients in [A[k], A[k+1]]
   private int order;  // order of interpolation polynomial in each [A[k], A[k+1]]
   private int[] Index;       // for indexed search in F[k]
   private int Imax;          // final size of Index is (Imax + 1)

   private double bleft;      // computational left limit of density
   private double bright;     // computational right limit of density
   private double bl;         // left border of the computational domain
   private double br;         // right border of the computational domain
   private double llc;        // absolute local concavity in left tail
   private double rlc;        // absolute local concavity in right tail
/*
   For NormalDist(0,1), local concavity c = 1/z^2
   For CauchyDist(0,1), local concavity c = -1/2 + 1/(2z^2)
   For GammaDist(a,lam), local concavity c = (a-1)/(1 - a +lam*x)^2
*/
   private double lc1;        // left, c1 = f(p)/f'(p) or f(p) / (c*f'(p))
   private double lc2;        // left, c2 = |f'(p)|(1 + c) / (f(p)^2)
   private double lc3;        // left, c3 = c/(1 + c)
   private double rc1;        // right, c1 = f(p) / f'(p) or f(p) / (c*f'(p))
   private double rc2;        // right, c2 = |f'(p)|(1 + c) / (f(p)^2)
   private double rc3;        // right, c3 = c/(1 + c)
   private double epstail;    // = 0.05*epsu*I0;
   private boolean lcutF = false;    // cut-off flag, left tail
   private boolean rcutF = false;   // cut-off flag, right tail


   protected void printArray (double[] U) {
      System.out.print("      Tableau = (");
      for (int j = 0; j < U.length; j++)
         System.out.printf("  %f", U[j]);
      System.out.println("  )");
   }


   private class MaDensite implements MathFunction {
      private ContinuousDistribution cdist;

      public MaDensite(ContinuousDistribution dist) {
         cdist = dist;
         supportA = cdist.getXinf();
         supportB = cdist.getXsup();
     }

      public double evaluate (double x) {
         return cdist.density (x);
      }
   }


   private void init (double xc, double epsu, int n) {
      double[] zs = new double[n + 1];
      double[] ys = new double[n + 1];   // ksi[]
      double[] xs = new double[n + 1];
      double[] vs = new double[n + 1];
      double[] us = new double[n + 1];
      double[] cs = new double[n + 1];
      epsu = 0.9*epsu;
      findSupport(xc);

      double I0 = MathFunctionUtil.gaussLobatto (m_dens, bleft, bright, 1.0e-6);
      if (I0 > 1.1 || I0 < 0.9)
         throw new IllegalStateException("NOT a probability density");
      epstail = 0.05*epsu*I0;
      epstail = Math.min(epstail, 1.e-10);
      epstail = Math.max(epstail, 1.e-15);
      double tol = epstail;
      findCutoff (bleft, epstail, false);    // left tail
      findCutoff (bright, epstail, true);    // right tail
      if (DEBUG)
         System.out.println("lcutF = " + lcutF + "\nrcutF = " + rcutF + "\n") ;

      reserve(0, n);
      A[0] = bl;
      if (lcutF)
         F[0] = epstail;
      else
         F[0] = 0;
      ys[0] = 0;
      final double HMIN = 1.0e-12;     // smallest integration step h
      double h = (br - bl) / K0;
      int j;
      int k = 0;
      calcChebyZ(zs, n);
      double eps = 0;
      if (DEBUG)
         System.out.println(
         "  k                 a_k                      F_k                h");

      while (A[k] < br) {
         while (h >= HMIN) {
            calcChebyX(zs, xs, n, h);
            calcU(m_dens, A[k], xs, us, n, tol);
            Misc.interpol(n, us, xs, cs);
            NTest(us, vs, n);
            // Evaluate Newton interpolating polynomial at vs[j].
            for (j = 1; j <= n; j++)
               ys[j] = Misc.evalPoly(n, us, cs, vs[j]);
            // NEval (cs, us, vs, ys, n);
            try {
               eps = calcEps(m_dens, A[k], ys, vs, n, tol);
            } catch (IllegalArgumentException e) {
               h = 0.5 * h;
               continue;
            }
            if (eps <= epsu)
               break;
            else
               h = 0.8 * h;
         }
         if (k + 1 >= A.length)
            reserve(k, n);
         copy (k, cs, us, xs, n);
         if (DEBUG)
            System.out.printf(
            " %d       %16.12f       %20.16g      %g%n", k, A[k], F[k], h);
         if (F[k] > 1.01)
            throw new IllegalStateException("Unable to compute CDF");
         k++;

         if (eps < epsu / 3.0)
            h = 1.3 * h;
         if (h < HMIN)
            h = HMIN;
         if (A[k] > br) {
            A[k] = br;
            F[k] = 1;
         }
      }

      if (DEBUG) {
         System.out.printf(
            " %d       %16.12f       %20.16g      %g%n",
            k, A[k], F[k], A[k] - A[k - 1]);
         System.out.println("\nFin du tableau");
      }
      Kmax = k;
      while (k > 0 && F[k] >= 1.) {
         F[k] = 1.;
         k--;
      }
      reserve(-Kmax, n);
      createIndex (Kmax);
   }\end{hide}\end{code}

\subsubsection* {Constructors}
\begin{code}

   public InverseDistFromDensity (ContinuousDistribution dist, double xc,
                                  double eps, int order) \begin{hide} {
      setParams (dist, null, xc, eps, order);
      init (xc, eps, order);
   } \end{hide}
\end{code}
\begin{tabb} Given a continuous distribution \texttt{dist} with a well-defined
density method, this class will compute tables for the numerical inverse of
the distribution. The user may wish to set the left and the right boundaries
 between which the density is non-zero by calling methods
\externalmethod{umontreal.iro.lecuyer.probdist}{ContinuousDistribution}{setXinf}{}
and
\externalmethod{umontreal.iro.lecuyer.probdist}{ContinuousDistribution}{setXsup}{}
of \texttt{dist}, for better efficiency.
Argument \texttt{xc} can be the mean,
the mode or any other $x$ for which the density is relatively large.
The $u$-resolution \texttt{eps} is the required absolute error in the cdf,
and \texttt{order} is the degree of the
Newton interpolating polynomial over each interval.
An \texttt{order} of 3 or 5, and an \texttt{eps} of $10^{-6}$ to $10^{-12}$
are usually good choices.
 Restrictions: $3 \le \texttt{order} \le 12$.
\end{tabb}
\begin{code}

   public InverseDistFromDensity (MathFunction dens, double xc, double eps,
                                  int order, double xleft, double xright) \begin{hide} {
      supportA = xleft;
      supportB = xright;
      setParams (null, dens, xc, eps, order);
      init (xc, eps, order);
   } \end{hide}
\end{code}
\begin{tabb} Given a continuous probability density \texttt{dens},
this class will compute tables for the numerical inverse of
the distribution. The left and the right boundaries of the density are
\texttt{xleft} and \texttt{xright} (the density is 0 outside the
interval \texttt{[xleft, xright]}).
See the description of the other constructor.
\end{tabb}


%%%%%%%%%%%%%%%%%%%%%%%%%%%%%%%%%%%%%%%%%%%%%%%%5
\subsubsection* {Methods}
\begin{code}

   public double density (double x) \begin{hide} {
      return m_dens.evaluate (x);
   }\end{hide}
\end{code}
\begin{tabb} Computes the probability density at $x$.
\end{tabb}
\begin{code}

   public double cdf (double x) \begin{hide} {
      throw new UnsupportedOperationException("cdf not implemented");
   }\end{hide}
\end{code}
\begin{tabb}
  Computes the  distribution function at $x$.
\end{tabb}
\begin{code}

   public double inverseF (double u)\begin{hide} {
      if (u < 0.0 || u > 1.0)
          throw new IllegalArgumentException ("u not in [0,1]");
      if (u >= 1.0)
          return supportB;
      if (u <= 0.0)
          return supportA;
      if ((u < epstail) && lcutF)
         return uinvLeftTail (u);
      if ((u > 1.0 - epstail) && rcutF)
         return uinvRightTail (u);

      int k = searchIndex(u);
      double x = A[k] + Misc.evalPoly(order, U[k], C[k], u - F[k]);
      if (x <= supportA)
         return supportA;
      if (x >= supportB)
         return supportB;
      return x;
   }\end{hide}
\end{code}
 \begin{tabb} Computes the inverse distribution function at $u$.
 \end{tabb}
\begin{code}

   public double getXc()\begin{hide} {
      return xc;
   }\end{hide}
\end{code}
\begin{tabb}
   Returns the \texttt{xc} given in the constructor.
\end{tabb}
\begin{code}

   public double getEpsilon()\begin{hide} {
      return epsu0;
   }\end{hide}
\end{code}
\begin{tabb}
   Returns the $u$-resolution \texttt{eps} associated with this object.
\end{tabb}
\begin{code}

   public int getOrder()\begin{hide} {
      return order;
   }\end{hide}
\end{code}
\begin{tabb}
   Returns the order associated with this object.
\end{tabb}
\begin{code}

   public double[] getParams()\begin{hide} {
      double[] retour = {xc, epsu0, order};
      return retour;
   }\end{hide}
\end{code}
\begin{tabb}
   Return a table containing the parameters of the current distribution.
   This table is returned as: [\texttt{xc}, \texttt{eps}, \texttt{order}].
\end{tabb}
\begin{code}

   public String toString()\begin{hide} {
      return name;
   }\end{hide}
\end{code}
\begin{tabb}
   Returns a \texttt{String} containing information about the current distribution.
\end{tabb}
\begin{hide}\begin{code}

   private void createIndex (int Kmax) {
      // create table for indexed search
      Imax = 2*Kmax;
      Index = new int [Imax + 1];
      Index[0] = 0;
      Index[Imax] = Kmax - 1;
      double u;
      int k = 1;
      for (int i = 1; i < Imax; i++) {
         u = (double) i / Imax;
         while (u >= F[k])
            k++;
         Index[i] = k-1;
      }
   }


   private int searchIndex (double u) {
      // search index of interval for interpolation in [F[k], F[k+1]]
      int i = (int) (Imax*u);
      int k = Index[i];
      while (u >= F[k]  && k < Kmax)
         k++;
      if (k <= 0)
         return 0;
      return k-1;
   }


   private void copy (int k, double[] cs, double[] us, double[] xs, int n) {
      // Copy parameters in interval [A[k], A[k+1]]
      for (int j = 0; j <= n; j++) {
         X[k][j] = xs[j];
         C[k][j] = cs[j];
         U[k][j] = us[j];
      }
      A[k+1] = A[k] + xs[n];
      F[k+1] = F[k] + us[n];
   }


   private double calcEps (MathFunction dens, double a, double[] Y,
                           double[] V, int n, double tol) {
      // Test if precision at test points Y is good enough
      // a is beginning of interval
      // Y are test points
      // V are values of CDF to compare with
      // n is order of interpolation
      // returns max eps
      // throw exception if Y[j] < Y[j-1] in gaussLobatto
      double eps = 0;
      double dif;
      double u = 0;
      for (int j = 1; j <= n; j++) {
         u += MathFunctionUtil.gaussLobatto (dens, a + Y[j-1], a + Y[j], tol);
         dif = Math.abs(u - V[j]);
         if (dif > eps)
            eps = dif;
      }
      return eps;
   }


   private void NEval (double[] C, double[] U, double[] T, double[] Y, int n) {
      // Evaluate Newton interpolating polynomial at T[j].
      // U are interpolation points
      // C are interpolation coefficients
      // Returns results in Y[j]
      int j;
      boolean fail = false;
      Y[0] = 0;
      for (j = 1; j <= n; j++) {
         Y[j] = Misc.evalPoly(n, U, C, T[j]);
         if (Y[j] < Y[j-1])
            fail = true;
      }

      if (fail) {
//       System.out.println("NEval");
         for (j = 1; j <= n; j++)
            Y[j] = Misc.evalPoly(1, U, C, T[j]);
      }
   }


   private void calcU (MathFunction dens, double a, double[] X, double[] U,
                       int n, double tol) {
      // compute CDF over n sub-intervals in [A[k], A[k+1]]
      // a is beginning of interval
      // X are x-values
      // U are values of CDF
      // precision is tol

      U[0] = 0;
      for (int j = 1; j <= n; j++)
         U[j] = U[j-1] +
            MathFunctionUtil.gaussLobatto (dens, a + X[j-1], a + X[j], tol);
   }


   private void reserve (int m, int n) {
      // Reserve memory for object
      A = reserve (A, m);
      F = reserve (F, m);
      C = reserve (C, m, n);
      U = reserve (U, m, n);
      X = reserve (X, m, n);
   }


   private double[] reserve (double[] T, int m) {
      if (m == 0) {
         // first call, just reserve memory.
         T = new double[K0 + 1];

      } else if (m < 0) {
         // Computation of table is complete. Table capacity is larger than
         // size: Resize table to exact size (-m + 1) and keep old values.
         m = -m;
         double[] tem = new double[m + 1];
         for (int i = 0; i <= m; i++)
            tem[i] = T[i];
         T = tem;

      } else {
         // Array is too short: reserve more memory and keep old values
         double[] tem = new double[2*m + 1];
         for (int i = 0; i <= m; i++)
            tem[i] = T[i];
         T = tem;
      }
      return T;
   }


   private double[][] reserve (double[][] T, int m, int n) {
      if (m == 0) {
         // first call, just reserve memory.
         T = new double[K0 + 1][n+1];

      } else if (m < 0) {
         // Computation of table is complete. Table capacity is larger than
         // size: Resize table to exact size (-m + 1) and keep old values.
         m = -m;
         double[][] tem = new double[m + 1][n+1];
         int j;
         for (int i = 0; i <= m; i++) {
            for (j = 0; j <= n; j++)
               tem[i][j] = T[i][j];
         }
         T = tem;

         } else {
         // Array is too short: reserve more memory and keep old values
         double[][] tem = new double[2*m + 1][n+1];
         int j;
         for (int i = 0; i <= m; i++) {
            for (j = 0; j <= n; j++)
               tem[i][j] = T[i][j];
         }
         T = tem;
      }
      return T;
   }


   private void NTest (double[] U, double[] T, int n) {
      // Routine 3 NTest in cite{rDER09a}
      // Compute test points T, given U
      int i, j, k;
      double s, sq, tem;
      T[0] = 0;
      for (k = 1; k <= n; k++) {
         T[k] = (U[k-1] + U[k]) / 2.;
         for (j = 0; j < 2 ; j++) {
            s = 0;
            sq = 0;
            for (i = 0; i <= n; i++) {
               tem = T[k] - U[i];
               if (tem == 0.)
                  break;
               tem = 1.0/tem;
               s += tem;
               sq += tem*tem;
            }
            if (sq != 0.)
               T[k] += s/sq;
         }
      }
   }


   private void calcChebyZ (double[] Z, int n) {
      // Eq. (3) in cite{rDER09a}. z_j = sin(j*phi)*sin((j+1)*phi)/cos(phi)
      // Compute normalized Chebyshev points in [0, 1]
      double phi = HALF_PI/(n+1);
      double c = Math.cos(phi);
      double y;
      double temp = 0;
      for (int j = 0; j < n; j++) {
         y = temp;
         temp = Math.sin((j+1)*phi);
         y *= temp;
         Z[j] = y/c;
      }
      Z[n] = 1;
   }


   private void calcChebyX (double[] Z, double[] X, int n, double h) {
      // Compute Chebyshev points in [0, h]
      for (int j = 1; j < n; j++)
         X[j] = h*Z[j];
      X[0] = 0;
      X[n] = h;
   }


   private double binSearch (double xa, double xb, double eps,
                                             boolean right) {
      /*
       * Binary search:
       *    find x such that   fa*epslow < f < fb*eps in the left tail
       *    find x such that   fa*eps > f > fb*epslow in the right tail
       *    where fa = density(xa), fb = density(xb),  f = density(x).
       * We find an x such that density(x) is a little smaller than eps
       */
      final double epslow = 0.1 * eps;
      double x = 0, y = 0;
      boolean fini = false;

      if (right) {    // right tail
         while (!fini) {
            x = 0.5 * (xa + xb);
            if ((xb - xa) < eps*Math.abs(x) || (xb - xa) < eps) {
               fini = true;
               if (x > supportB)
                  x = supportB;
            }
            y = m_dens.evaluate(x);
            if (y < epslow) {
               xb = x;
            } else if (y > eps) {
               xa = x;
            } else
               fini = true;
         }

      } else {   // left tail
         while (!fini) {
            x = 0.5 * (xa + xb);
            if ((xb - xa) < eps*Math.abs(x) || (xb - xa) < eps) {
               fini = true;
               if (x < supportA)
                  x = supportA;
            }
            y = m_dens.evaluate(x);
            if (y < epslow) {
               xa = x;
            } else if (y > eps) {
               xb = x;
            } else
               fini = true;
         }
      }
      if (DEBUG)
         System.out.printf(
         "binSearch   x =  %g    f =  %g     r =  %g%n", x, y, y/eps);

      return x;
   }


   private void findSupport (double xc) {
      /*
       * Find interval where density is non-negligible (above some epsilon):
       * find points bleft < xc < bright such that
       *      density(bleft) ~ density(bright) ~ 10^(-13)*density(xc)
       */
      boolean flagL = false;
      boolean flagR = false;
      final double DELTA = 1.0e-100;
      final double DELTAR = 1.0e-14;
      double x, y;
      double bl = supportA;
      double br = supportB;

      if (bl > Double.NEGATIVE_INFINITY) {
         // Density is 0 for finite x < bl
         y = m_dens.evaluate(bl);
         x = bl;
         if (y >= HUGE || y <= 0.0) {
            // density is infinite or 0 at bl; choose bl --> bl(1 + epsilon)
            x = bl + DELTAR * Math.abs(bl);
            if (x == 0)
               // bl is 0 --> choose bl = DELTA
               x = DELTA;
            y = m_dens.evaluate(x);
         }

         if (y >= HUGE)
            throw new UnsupportedOperationException
            ("Infinite density at left boundary");

         if (y >= 1.0e-50) {
            // f(bl) is large enough; we have found bl
            flagL = true;
            bl = x;
         }
      }

      if (br < Double.POSITIVE_INFINITY) {
         // Density is 0 for finite x > br
         y = m_dens.evaluate(br);
         x = br;
         if (y >= HUGE || y <= 0.0) {
            // density is infinite or 0 at br; choose br --> br(1 - epsilon)
            x = br - DELTAR * Math.abs(br);
            if (x == 0)
               // br is 0 --> choose br = -DELTA
               x = -DELTA;
            y = m_dens.evaluate(x);
         }

         if (y >= HUGE)
            throw new UnsupportedOperationException
            ("Infinite density at right boundary");

         if (y >= 1.0e-50) {
            // f(br) is large enough; we have found br
            flagR = true;
            br = x;
         }
      }

      bleft = bl;
      bright = br;
      if (flagL && flagR)
         return;

      // We have not found bl or br
      double h;
      y = m_dens.evaluate(xc);
      double epsy = 1.0e-13*y;
      double xa, xb;


      if (!flagR) {
         // Find br: start at xc; increase x until density is very small
         h = 1;
         xa = xc;
         xb = xc + h;
         while (m_dens.evaluate(xb) >= epsy) {
            xa = xb;
            h *= 2.0;
            xb += h;
         }
         // Now we have density(xa) > epsy > density(xb)

        if (xb > supportB) {
            // density = 0 outside [supportA, supportB]
            xb = supportB;
         }
         x = binSearch (xa, xb, epsy, true);
         bright = x;   // Have found br
    }

      if (!flagL) {
         h = 1;
         xb = xc;
         xa = xc - h;
         while (m_dens.evaluate(xa) >= epsy) {
            xb = xa;
            h *= 2.0;
            xa -= h;
         }
         // Now we have density(xa) < epsy < density(xb)

         if (xa < supportA) {
            // density = 0 outside [supportA, supportB]
            xa = supportA;
         }
         x = binSearch (xa, xb, epsy, false);
         bleft = x;   // Have found bl
     }
   }


   protected void setParams (ContinuousDistribution dist, MathFunction dens,
              double xc, double eps, int order) {
      // Sets the parameter of this object
      if (eps < 1.0e-15)
         throw new IllegalArgumentException ("eps < 10^{-15}");
      if (eps > 1.0e-3)
         throw new IllegalArgumentException ("eps > 10^{-3}");
      if (order < 3)
         throw new IllegalArgumentException ("order < 3");
      if (order > 12)
         throw new IllegalArgumentException ("order > 12");
      epsu0 = eps;
      this.xc = xc;
      this.order = order;

      StringBuffer sb = new StringBuffer ("InverseDistFromDensity: ");
      if (dist == null) {
         m_dens = dens;
      } else {
         m_dens = new MaDensite(dist);
         sb.append (dist.toString());
      }
      name = sb.toString();
   }


   private double uinvLeftTail (double u) {
      // Returns x = inverseF(u) in left tail

      double x = 0;
      if (llc <= epsc)
            x = bl + lc1 * Math.log (u*lc2);
      else
            x = bl + lc1 * (Math.pow (u*lc2, lc3) - 1.);
      if (x <= supportA)
         return supportA;
      return x;
   }


   private double uinvRightTail (double u) {
      // Returns x = inverseF(u) in right tail

      double x = 0;
      double v = 1. - u;
      if (rlc <= epsc)
            x = br + rc1 * Math.log (v*rc2);
      else
            x = br + rc1 * (Math.pow (v*rc2, rc3) - 1.);
      if (x >= supportB)
         return supportB;
      return x;
   }


   private void findCutoff (double x0, double eps, boolean right) {
      /*
       * Find cut-off points for the computational domain.
       * Find cut-off x in the tails such that cdf(x) = eps in the left
       *    tail, and eps = 1 - cdf(x) in the right tail.
       * Uses successive approximations starting at x = x0.
       * If right is true, case of the right tail; otherwise the left tail.
       * The program uses T_c-concavity of densities as described in
       * Leydold et al.
       */
      final double epsx = 1.0e-3;
      final double range = bright - bleft;
      double del;

      if (right) {
          del = m_dens.evaluate(bright) - m_dens.evaluate(bright - epsx);
          if ((supportB < Double.POSITIVE_INFINITY) &&
                (supportB - bright <= epsx ||
                 supportB - bright <= Math.abs(supportB)*epsx)) {
            // If density is non-negligible right up to domain limit supportB,
            // then cutoff is bright. There is no right tail. We want cutoff
            // at bright in case density(supportB) = infinite.
            if (del < 0) {
               // density decreases toward supportB;
               br = supportB;
            } else {
               // density increases toward supportB; may be infinite
               br = bright;
            }
            rcutF = false;
            return;
         } else {
            rcutF = true;   // There is a right tail
         }

      } else {
         del = m_dens.evaluate(bleft + epsx) - m_dens.evaluate(bleft);
         if ((supportA > Double.NEGATIVE_INFINITY) &&
             (bleft - supportA <= epsx ||
              bleft - supportA <= Math.abs(supportA)*epsx)) {
            // If density is non-negligible right down to domain limit supportA,
            // then cutoff is bleft. There is no left tail. We want cutoff
            // at bleft in case density(supportA) = infinite.
            if (del > 0) {
               // density decreases toward supportA
               bl = supportA;
            } else {
               // density increases toward supportA; may be infinite
               bl = bleft;
            }
            lcutF = false;
            return;
         } else {
            lcutF = true;   // There is a left tail
         }
      }

      double c = 0;
      double h = 1.0/64.0;      // step to compute derivative
      h = Math.max (h, (bright - bleft) / (1024));
      double x = x0, xnew;
      double y = 0, yl = 0, yr = 0, yprime = 0;
      double tem = 0;
      int iter = 0;
      final int ITERMAX = 30;
      boolean fini = false;

      while (!fini && iter < ITERMAX) {
         iter++;
         boolean ended = false;
         int it = 0;

         while (!ended && it < 10) {
            it++;
            if (x + h > supportB)
               h = supportB - x;
            if (x - h < supportA)
               h = x - supportA;
            yr = m_dens.evaluate(x + h);
            y = m_dens.evaluate(x);
            yl = m_dens.evaluate(x - h);
            if (!(yl == 0 || yr == 0 || y == 0))
               ended = true;
            else
               h /= 2;
         }

         c = yr / (yr - y) + yl / (yl - y) - 1.;  // the local concavity lc
         yprime = (yr - yl) / (2. * h);          // first derivative
         tem = Math.abs (y * y / ((c + 1.) * yprime));  // tail area of CDF

         if (Double.isNaN (tem))
            break;
         if (Math.abs (tem / eps - 1.) < 1.e-4)   // accuracy is good?
            break;
         if (Math.abs(c) <= epsc) {
            tem = eps * Math.abs(yprime) / (y * y);   // formula (10)
            if (tem <= 0)
               break;
            xnew = x + y / yprime * Math.log(tem);
         } else {
            tem = (1. + c) * eps * Math.abs(yprime) / (y * y); // formula(10)
            if (tem < 0)
               break;
            xnew = x + y / (c*yprime) * (Math.pow(tem, c / (1. + c)) - 1.);
         }

         if (DEBUG)
            System.out.printf(
            "Cutoff   x =  %g    y =  %g     c =  %g%n", xnew, y, c);

         if ((Math.abs(xnew - x) <= Math.abs(x)*epsx) ||
             (Math.abs(xnew - x) <= epsx))
            fini = true;     // found cut-off x
         else
            x = xnew;

         // Given good x, precompute some parameters in formula (10)
         if (right) {
            rlc = Math.abs(c);
            br = x;
            rc3 = c / (1 + c);
            rc2 = tem / eps;
            rc1 = y / yprime;
            if (Math.abs(c) > epsc)
               rc1 /= c;

         } else {
            llc = Math.abs(c);
            bl = x;
            lc3 = c / (1 + c);
            lc2 = tem / eps;
            lc1 = y / yprime;
            if (Math.abs(c) > epsc)
               lc1 /= c;
         }

       if (Math.abs(xnew - x) >= range)
           fini = true;
      }

      if (right) {
         if ((rc1 == 0 && rc2 == 0 && rc3 == 0)) {
            br = bright;
            rcutF = false;
         }
      } else {
          if ((lc1 == 0 && lc2 == 0 && lc3 == 0)) {
            bl = bleft;
            lcutF = false;
         }
      }
   }

}\end{code}
\end{hide}
