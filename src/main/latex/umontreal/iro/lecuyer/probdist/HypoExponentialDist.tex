\defclass {HypoExponentialDist}

This class implements the \emph{hypoexponential} distribution,
also called the generalized Erlang  distribution. Let the $X_j$,
$j=1,\ldots,k$, be $k$ independent exponential random variables with different
rates $\lambda_j$, i.e. assume that $\lambda_j \neq \lambda_i$ for
$i \neq j$. Then the sum $\sum_{j=1}^kX_j$ is called a \emph{hypoexponential}
random variable.

Let the  $k\times k$ upper triangular bidiagonal matrix
%
\begin{latexonly}%
\begin{equation}
\label{eq:tail-hypomatrix}
  \bA = \begin{pmatrix}
   -\lambda_1 & \lambda_1 & 0 & \ldots & 0 \\
   0 & -\lambda_2& \lambda_2 & \ldots &  0 \\
   \vdots &  \vdots & \ddots & ~~\ddots & \vdots   \\
   0 & \ldots  & 0 &  -\lambda_{k-1} & \lambda_{k-1} \\
   0 & \ldots  & 0 &   0 & -\lambda_k
  \end{pmatrix}
\end{equation}
\end{latexonly}%
%
\begin{htmlonly}%
\begin{equation}
\label{eq:tail-hypomatrix}
  \bA = \begin{array}{ccccc}
   -\lambda_1 & \lambda_1 & 0 & \ldots & 0 \\
   0 & -\lambda_2& \lambda_2 & \ldots &  0 \\
   \vdots &  \vdots & \ddots & ~~\ddots & \vdots   \\
   0 & \ldots  & 0 &  -\lambda_{k-1} & \lambda_{k-1} \\
   0 & \ldots  & 0 &   0 & -\lambda_k
  \end{array}
\end{equation}
\end{htmlonly}%
%
with  $\lambda_j$ the rates of the $k$ exponential random variables;
then the cumulative complementary probability of the hypoexponential
distribution is given by \cite{pNEU81a,pLAT99a}
\begin{equation}
\label{eq:tail-hypoexp}
\bar F(x) = 
\begin{latexonly} \mathbb{P} \end{latexonly}
\begin{htmlonly} \mathcal{P} \end{htmlonly}
\left[X_1 + \cdots + X_k > x \right] =
\sum_{j=1}^k \left(e^{\bA x}\right)_{1j},
\end{equation}
%
i.e., it is the sum of the elements of the first row of matrix $e^{\bA x}$.
The density of the hypoexponential distribution is
\eq  f(x) = \left(-e^{\bA x}\bA\right)_{1k} =
      \lambda_k  \left(e^{\bA x}\right)_{1k},
    \eqlabel{eq:fhypoexp}
\endeq
%
i.e., it is element $(1,k)$ of matrix $-e^{\bA x}\bA$.
The distribution function is as usual $F(x) = 1 - \bar F(x)$.

See the class \class{HypoExponentialDistQuick} for alternative formulae
for the probabilities.



\bigskip\hrule

%%%%%%%%%%%%%%%%%%%%%%%%%%%%%%%%%%%%%%%%
\begin{code}\begin{hide}
/*
 * Class:        HypoExponentialDist
 * Description:  Hypo-exponential distribution
 * Environment:  Java
 * Software:     SSJ 
 * Copyright (C) 2001  Pierre L'Ecuyer and Université de Montréal
 * Organization: DIRO, Université de Montréal
 * @author       Richard Simard
 * @since        January 2011

 * SSJ is free software: you can redistribute it and/or modify it under
 * the terms of the GNU General Public License (GPL) as published by the
 * Free Software Foundation, either version 3 of the License, or
 * any later version.

 * SSJ is distributed in the hope that it will be useful,
 * but WITHOUT ANY WARRANTY; without even the implied warranty of
 * MERCHANTABILITY or FITNESS FOR A PARTICULAR PURPOSE.  See the
 * GNU General Public License for more details.

 * A copy of the GNU General Public License is available at
   <a href="http://www.gnu.org/licenses">GPL licence site</a>.
 */
\end{hide}
package umontreal.iro.lecuyer.probdist;
\begin{hide}
import java.util.Formatter;
import java.util.Locale;
import cern.colt.matrix.*;
import  umontreal.iro.lecuyer.util.*;
import umontreal.iro.lecuyer.functions.MathFunction;
\end{hide}

public class HypoExponentialDist extends ContinuousDistribution\begin{hide} {
   protected double[] m_lambda;

   protected static void testLambda (double[] lambda) {
      int k = lambda.length;
      for (int j = 0; j < k; ++j) {
         if (lambda[j] <= 0)
            throw new IllegalArgumentException ("lambda_j <= 0");
      }
   }


   // Builds the bidiagonal matrix A out of the lambda
   private static DoubleMatrix2D buildMatrix (double[] lambda, double x) {
      int k = lambda.length;
      DoubleFactory2D F2 = DoubleFactory2D.dense;
      DoubleMatrix2D A = F2.make(k, k);
      for (int j = 0; j < k-1; j++) {
         A.setQuick(j, j, -lambda[j]*x);
         A.setQuick(j, j + 1, lambda[j]*x);
      }
      A.setQuick(k-1, k-1, -lambda[k-1]*x);
      return A;
   }


   private static class myFunc implements MathFunction {
      // For inverseF
      private double[] m_lam;
      private double m_u;

      public myFunc (double[] lam, double u) {
         m_lam = lam;
         m_u = u;
      }

      public double evaluate (double x) {
         return m_u - HypoExponentialDist.cdf(m_lam, x);
      }
   }
\end{hide}
\end{code}
%%%%%%%%%%%%%%%%%%%%%%%%%%%%%%%%%%%%%%%%%
\subsubsection* {Constructor}

\begin{code}

   public HypoExponentialDist (double[] lambda)\begin{hide} {
      setLambda (lambda);
  }\end{hide}
\end{code}
\begin{tabb} Constructs a \texttt{HypoExponentialDist} object,
with rates $\lambda_i = $ \texttt{lambda[$i-1$]}, $i = 1,\ldots,k$.
\end{tabb}
\begin{htmlonly}
   \param{lambda}{rates of the hypoexponential distribution}
\end{htmlonly}


%%%%%%%%%%%%%%%%%%%%%%%%%%%%%%%%%%%
\subsubsection* {Methods}

\begin{code}\begin{hide}

   public double density (double x) {
      return density (m_lambda, x);
   }

   public double cdf (double x) {
      return cdf (m_lambda, x);
   }

   public double barF (double x) {
      return barF (m_lambda, x);
   }

   public double inverseF (double u) {
      return inverseF (m_lambda, u);
   }

   public double getMean() {
      return getMean (m_lambda);
   }

   public double getVariance() {
      return getVariance (m_lambda);
   }

   public double getStandardDeviation() {
      return getStandardDeviation (m_lambda);
   }\end{hide}

   public static double density (double[] lambda, double x)\begin{hide} {
      testLambda (lambda);
      if (x < 0)
         return 0;
      DoubleMatrix2D Ax = buildMatrix (lambda, x);
	   DoubleMatrix2D M = DMatrix.expBidiagonal(Ax);

      int k = lambda.length;
      return lambda[k-1]*M.getQuick(0, k-1);
   }\end{hide}
\end{code}
\begin{tabb} Computes the density function $f(x)$, with $\lambda_i = $
\texttt{lambda[$i-1$]}, $i = 1,\ldots,k$.
\end{tabb}
\begin{htmlonly}
   \param{lambda}{rates of the hypoexponential distribution}
   \param{x}{value at which the density is evaluated}
   \return{density at $x$}
\end{htmlonly}
\begin{code}

   public static double cdf (double[] lambda, double x)\begin{hide} {
      return 1.0 - barF(lambda, x);
   }\end{hide}
\end{code}
 \begin{tabb}
  Computes the  distribution function $F(x)$, with $\lambda_i = $
\texttt{lambda[$i-1$]}, $i = 1,\ldots,k$.
 \end{tabb}
\begin{htmlonly}
   \param{lambda}{rates of the hypoexponential distribution}
   \param{x}{value at which the distribution is evaluated}
   \return{distribution at $x$}
\end{htmlonly}
\begin{code}

   public static double barF (double[] lambda, double x)\begin{hide} {
      testLambda (lambda);
      if (x <= 0.0)
         return 1.0;
      if (x >= Double.MAX_VALUE)
         return 0.0;
      DoubleMatrix2D M = buildMatrix (lambda, x);
      M = DMatrix.expBidiagonal(M);

      // prob is first row of final matrix
      int k = lambda.length;
      double sum = 0;
      for (int j = 0; j < k; j++)
         sum += M.getQuick(0, j);
      return sum;
   }\end{hide}
\end{code}
  \begin{tabb}
  Computes the complementary distribution $\bar F(x)$,
with $\lambda_i = $ \texttt{lambda[$i-1$]}, $i = 1,\ldots,k$.
 \end{tabb}
\begin{htmlonly}
   \param{lambda}{rates of the hypoexponential distribution}
   \param{x}{value at which the complementary distribution is evaluated}
   \return{complementary distribution at $x$}
\end{htmlonly}
\begin{code}

   public static double inverseF (double[] lambda, double u)\begin{hide} {
      testLambda (lambda);
      if (u < 0.0 || u > 1.0)
          throw new IllegalArgumentException ("u not in [0,1]");
      if (u >= 1.0)
          return Double.POSITIVE_INFINITY;
      if (u <= 0.0)
          return 0.0;

      final double EPS = 1.0e-12;
      myFunc fonc = new myFunc (lambda, u);
      double x1 = getMean (lambda);
      double v = cdf (lambda, x1);
      if (u <= v)
         return RootFinder.brentDekker (0, x1, fonc, EPS);

      // u > v
      double x2 = 4.0*x1 + 1.0;
      v = cdf (lambda, x2);
      while (v < u) {
         x1 = x2;
         x2 = 4.0*x2;
         v = cdf (lambda, x2);
      }
      return RootFinder.brentDekker (x1, x2, fonc, EPS);
   }\end{hide}
\end{code}
  \begin{tabb}
 Computes the inverse distribution function $F^{-1}(u)$,
with $\lambda_i = $ \texttt{lambda[$i-1$]}, $i = 1,\ldots,k$.
It uses a root-finding method and is very slow.
\end{tabb}
\begin{htmlonly}
   \param{lambda}{rates of the hypoexponential distribution}
   \param{u}{value at which the inverse distribution is evaluated}
   \return{inverse distribution at $u$}
\end{htmlonly}
\begin{code}

   public static double getMean (double[] lambda)\begin{hide} {
      testLambda (lambda);
      int k = lambda.length;
      double sum = 0;
      for (int j = 0; j < k; j++)
         sum += 1.0 / lambda[j];
      return sum;
   }\end{hide}
\end{code}
\begin{tabb} Returns the mean, $E[X] = \sum_{i=1}^k 1/\lambda_i$,
   of the hypoexponential distribution with rates $\lambda_i = $
\texttt{lambda[$i-1$]}, $i = 1,\ldots,k$.
\end{tabb}
\begin{htmlonly}
   \param{lambda}{rates of the hypoexponential distribution}
   \return{mean of the hypoexponential distribution}
\end{htmlonly}
\begin{code}

   public static double getVariance (double[] lambda)\begin{hide} {
      testLambda (lambda);
      int k = lambda.length;
      double sum = 0;
      for (int j = 0; j < k; j++)
         sum += 1.0 / (lambda[j]*lambda[j]);
      return sum;
   }\end{hide}
\end{code}
\begin{tabb}  Returns the variance,
$\mbox{Var}[X] = \sum_{i=1}^k 1/\lambda_i^2$,
of the hypoexponential distribution with rates $\lambda_i = $
\texttt{lambda[$i-1$]}, $i = 1,\ldots,k$.
\end{tabb}
\begin{htmlonly}
   \param{lambda}{rates of the hypoexponential distribution}
   \return{variance of the hypoexponential distribution}
\end{htmlonly}
\begin{code}

   public static double getStandardDeviation (double[] lambda)\begin{hide} {
      double s = getVariance (lambda);
      return Math.sqrt(s);
   }\end{hide}
\end{code}
\begin{tabb}  Returns the standard deviation
of the hypoexponential distribution  with rates $\lambda_i = $
\texttt{lambda[$i-1$]}, $i = 1,\ldots,k$.
\end{tabb}
\begin{htmlonly}
   \param{lambda}{rates of the hypoexponential distribution}
   \return{standard deviation of the hypoexponential distribution}
\end{htmlonly}
\begin{code}

   public double[] getLambda()\begin{hide} {
      return m_lambda;
   }
\end{hide}
\end{code}
  \begin{tabb}
  Returns the value of $\lambda_i$ for this object.
 \end{tabb}
\begin{code}

   public void setLambda (double[] lambda)\begin{hide} {
      testLambda (lambda);
      int k = lambda.length;
      m_lambda = new double[k];
      System.arraycopy (lambda, 0, m_lambda, 0, k);
      supportA = 0.0;
   }\end{hide}
\end{code}
  \begin{tabb}
  Sets the value of $\lambda_i = $\texttt{lambda[$i-1$]},
 $i = 1,\ldots,k$ for this object.
 \end{tabb}
\begin{code}

   public double[] getParams()\begin{hide} {
      return m_lambda;
   }\end{hide}
\end{code}
\begin{tabb}
   Same as \method{getLambda}{}.
\end{tabb}
\begin{hide}\begin{code}

   public String toString ()\begin{hide} {
      StringBuilder sb = new StringBuilder();
      Formatter formatter = new Formatter(sb, Locale.US);
      formatter.format(getClass().getSimpleName() + " : lambda = {" +
           PrintfFormat.NEWLINE);
      int k = m_lambda.length;
      for(int i = 0; i < k; i++) {
         formatter.format("   %f%n", m_lambda[i]);
      }
      formatter.format("}%n");
      return sb.toString();
   }\end{hide}
\end{code}
\begin{tabb}
   Returns a \texttt{String} containing information about the current
  distribution.
\end{tabb}
\end{hide}

\begin{code}\begin{hide}
}\end{hide}
\end{code}

