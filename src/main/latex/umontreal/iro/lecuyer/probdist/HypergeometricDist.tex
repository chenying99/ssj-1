\defclass {HypergeometricDist}

Extends the class \class{DiscreteDistributionInt} for
the {\em hypergeometric\/} distribution \cite[page 101]{rGEN98a} with
$k$ elements chosen among $l$, $m$ being
of one type, and $l-m$ of the other.
The parameters $m$, $k$ and $l$ are positive integers
where $1\le m\le l$ and $1\le k\le l$.
Its mass function is given by
\begin{htmlonly}
\eq
    p(x) =
            \mbox{nCr}(m, x) \mbox{nCr}(l - m, k-x)/\mbox{nCr}(l, k),
 \qquad \mbox{for } \max(0,k-l+m)\le x\le \min(k, m),
\endeq
where nCr is defined in \class{BinomialDist}.
\end{htmlonly}
\begin{latexonly}
\eq
    p(x) =
            \frac{ \binom{m}{x} \binom{l - m}{k-x}}{\binom{l}{k}}
 \qquad \mbox{for } \max(0,k-l+m)\le x\le \min(k, m).  \eqlabel{eq:fheperg}
\endeq
\end{latexonly}


\bigskip\hrule

%%%%%%%%%%%%%%%%%%%%%%%%%%%%%%%%%%%%%%%%
\begin{code}
\begin{hide}
/*
 * Class:        HypergeometricDist
 * Description:  hypergeometric distribution
 * Environment:  Java
 * Software:     SSJ 
 * Copyright (C) 2001  Pierre L'Ecuyer and Université de Montréal
 * Organization: DIRO, Université de Montréal
 * @author       
 * @since

 * SSJ is free software: you can redistribute it and/or modify it under
 * the terms of the GNU General Public License (GPL) as published by the
 * Free Software Foundation, either version 3 of the License, or
 * any later version.

 * SSJ is distributed in the hope that it will be useful,
 * but WITHOUT ANY WARRANTY; without even the implied warranty of
 * MERCHANTABILITY or FITNESS FOR A PARTICULAR PURPOSE.  See the
 * GNU General Public License for more details.

 * A copy of the GNU General Public License is available at
   <a href="http://www.gnu.org/licenses">GPL licence site</a>.
 */
\end{hide}
package umontreal.iro.lecuyer.probdist;\begin{hide}
import umontreal.iro.lecuyer.util.Num;
\end{hide}

public class HypergeometricDist extends DiscreteDistributionInt\begin{hide} {

   private int m;
   private int l;
   private int k;
   private double p0;
\end{hide}\end{code}\begin{code}\end{code}

\unmoved\begin{detailed}
%%%%%%%%%%%%%%%%%%%%%%%%%%%%%%%%%%%%%%%%%
\subsubsection*{Constant}
\begin{code}

   public static double MAXN = 100000;
\end{code}
 \begin{tabb}
  If the number of integers in the interval $[$max$(0,k-l+m),\,$min$(k,m)]$
  is larger than this constant, the tables will {\em not\/}
  be precomputed by the constructor.
\end{tabb}
\end{detailed}

%%%%%%%%%%%%%%%%%%%%%%%%%%%%%%%%%%%%%%%%%
\subsubsection*{Constructor}
\begin{code}

   public HypergeometricDist (int m, int l, int k)\begin{hide} {
      setParams (m, l, k);
   }\end{hide}
\end{code}
  \begin{tabb} Constructs an hypergeometric distribution with parameters
   $m$, $l$ and $k$.
 \end{tabb}

%%%%%%%%%%%%%%%%%%%%%%%%%%%%%%%%%%%
\subsubsection* {Methods}
\begin{code}\begin{hide}

   public double prob (int x) {
      if (x < supportA || x > supportB)
         return 0.0;
      if (pdf == null || x < xmin || x > xmax)
         return prob (m, l, k, x);
      return pdf[x - xmin];
   }

   public double cdf (int x) {
      if (x < supportA)
         return 0.0;
      if (x >= supportB)
         return 1.0;
      if (cdf != null) {
         if (x >= xmax)
            return 1.0;
         if (x < xmin)
            return cdf (m, l, k, x);
         if (x <= xmed)
            return cdf[x - xmin];
         else
            // We keep the complementary distribution in the upper part of cdf
            return 1.0 - cdf[x + 1 - xmin];
      }
      else
         return cdf (m, l, k, x);
   }

   public double barF (int x) {
      if (x <= supportA)
         return 1.0;
      if (x > supportB)
         return 0.0;
      if (cdf != null) {
         if (x > xmax)
            return barF (m, l, k, x);
         if (x <= xmin)
            return 1.0;
         if (x > xmed)
           // We keep the complementary distribution in the upper part of cdf
           return cdf[x - xmin];
         else
            return 1.0 - cdf[x - 1 - xmin];
      }
      else
         return barF (m, l, k, x);
   }

   public int inverseFInt (double u) {
      if (u < 0 || u > 1)
         throw new IllegalArgumentException ("u is not in [0,1]");
      if (u <= 0.0)
         return Math.max (0, k - l + m);
      if (u >= 1.0)
         return Math.min (k, m);
      double p = p0;
     // Empirical correction, the original algorithm sets x=0.
     int x = Math.max (0, k - l + m);
     if (u <= p) return x;
     do {
        u = u - p;
        p = p*(m - x)*(k - x)/((x+1)*(l - m - k + 1.0 + x));
        x++;
     } while (u > p);
     return x;
   }

   public double getMean() {
      return HypergeometricDist.getMean (m, l, k);
   }

   public double getVariance() {
      return HypergeometricDist.getVariance (m, l, k);
   }

   public double getStandardDeviation() {
      return HypergeometricDist.getStandardDeviation (m, l, k);
   }\end{hide}

   public static double prob (int m, int l, int k, int x)\begin{hide} {
      final int SLIM = 70;
      final double MAXEXP = (Num.DBL_MAX_EXP - 1)*Num.LN2;// To avoid overflow
      if (l <= 0)
         throw new IllegalArgumentException ("l must be greater than 0");
      if (m <= 0 || m > l)
         throw new IllegalArgumentException ("m is invalid: 1 <= m < l");
      if (k <= 0 || k > l)
         throw new IllegalArgumentException ("k is invalid: 1 <= k < l");
      if (x < Math.max (0, k - l + m) || x > Math.min (k, m))
         return 0;

      if (l <= SLIM)
         return Num.combination (m, x)
            * Num.combination (l - m, k - x)/Num.combination (l, k);
      else {
         double res =
             Num.lnFactorial (m) + Num.lnFactorial (l-m) - Num.lnFactorial (l)
           - Num.lnFactorial (x) - Num.lnFactorial (k - x) + Num.lnFactorial (k)
           - Num.lnFactorial (m - x) - Num.lnFactorial (l - m - k + x)
           + Num.lnFactorial (l - k);
         if (res >= MAXEXP)
            throw new IllegalArgumentException ("term overflow");
         return Math.exp (res);
      }
   }\end{hide}
\end{code}
 \begin{tabb}
   Computes the hypergeometric probability
   $p(x)$\latex{ given by (\ref{eq:fheperg})}.
 \end{tabb}
\begin{code}

   public static double cdf (int m, int l, int k, int x)\begin{hide} {
      if (l <= 0)
         throw new IllegalArgumentException ("l must be greater than 0");
      if (m <= 0 || m > l)
         throw new IllegalArgumentException ("m is invalid: 1 <= m < l");
      if (k <= 0 || k > l)
         throw new IllegalArgumentException ("k is invalid: 1 <= k < l");
      int imin = Math.max (0, k - l + m);
      int imax = Math.min (k, m);
      if (x < imin)
         return 0.0;
      if (x >= imax)
         return 1.0;
      // Very inefficient
      double res = 0.0;
      for (int i = imin; i <= x; i++)
         res += prob (m, l, k, i);
      if (res >= 1.0)
         return 1.0;
      return res;
   }\end{hide}
\end{code}
  \begin{tabb} Computes the distribution function $F(x)$.
  \end{tabb}
\begin{code}

   public static double barF (int m, int l, int k, int x)\begin{hide} {
      if (l <= 0)
         throw new IllegalArgumentException ("l must be greater than 0");
      if (m <= 0 || m > l)
         throw new IllegalArgumentException ("m is invalid: 1 < =m < l");
      if (k <= 0 || k > l)
         throw new IllegalArgumentException ("k is invalid: 1 < =k < l");
      int imin = Math.max (0, k - l + m);
      int imax = Math.min (k, m);
      if (x <= imin)
         return 1.0;
      if (x > imax)
         return 0.0;
      // Very inefficient
      double res = 0.0;
      for (int i = imax; i >= x; i--)
         res += prob (m, l, k, i);
      if (res >= 1.0)
         return 1.0;
      return res;
   }\end{hide}
\end{code}
\begin{tabb} Computes the complementary distribution function.
\emph{WARNING:} The complementary distribution function is defined as
$\bar F(x) = P[X \ge x]$.
\end{tabb}
\begin{code}

   public static int inverseF (int m, int l, int k, double u)\begin{hide} {
      // algo hin dans Kachitvichyanukul
      if (u < 0 || u >= 1)
         throw new IllegalArgumentException ("u is not in [0,1]");
      if (u <= 0.0)
         return Math.max (0, k - l + m);
      if (u >= 1.0)
         return Math.min (k, m);
      double p = 0;
      if (k < l - m)
          p = Math.exp (Num.lnFactorial (l - m) + Num.lnFactorial (l - k)
                       -Num.lnFactorial (l) - Num.lnFactorial (l - m - k));
      else
          p = Math.exp (Num.lnFactorial (m) + Num.lnFactorial (k)
                      -Num.lnFactorial (k- l + m) - Num.lnFactorial (l));

     // Empirical correction, the original algorithm sets x=0.
     int x = Math.max (0, k - l + m);
     if (u <= p) return x;

     do {
         u = u - p;
         p = p*(m - x)*(k - x)/((x+1)*(l - m - k + 1.0 + x));
         x++;
     } while (u > p && p > 0);
     return x;
   }\end{hide}
\end{code}
\begin{tabb}
  Computes $F^{-1}(u)$ for the hypergeometric distribution without
  using precomputed tables.  The inversion is computed
  using the chop-down algorithm \cite{sKAC85a}.
\end{tabb}
\begin{code}

   public static double getMean (int m, int l, int k)\begin{hide} {
      if (l <= 0)
         throw new IllegalArgumentException ("l must be greater than 0");
      if (m <= 0 || m > l)
         throw new IllegalArgumentException ("m is invalid: 1<=m<l");
      if (k <= 0 || k > l)
         throw new IllegalArgumentException ("k is invalid: 1<=k<l");

      return ((double) k *  (double) m / (double) l);
   }\end{hide}
\end{code}
\begin{tabb}  Computes and returns the mean $E[X] = km/l$
   of the Hypergeometric distribution with parameters $m$, $l$ and $k$.
\end{tabb}
\begin{htmlonly}
   \return{the mean of the hypergeometric distribution $E[X] = km / l$}
\end{htmlonly}
\begin{code}

   public static double getVariance (int m, int l, int k)\begin{hide} {
      if (l <= 0)
         throw new IllegalArgumentException ("l must be greater than 0");
      if (m <= 0 || m > l)
         throw new IllegalArgumentException ("m is invalid: 1<=m<l");
      if (k <= 0 || k > l)
         throw new IllegalArgumentException ("k is invalid: 1<=k<l");

      return (((double) k * (double) m / (double) l) *
              ( 1 - ((double) m / (double) l)) * ((double) l - (double) k) /
              ((double) l - 1));
   }\end{hide}
\end{code}
\begin{tabb}  Computes and returns the variance
\begin{latexonly}
   $\mbox{Var}[X] = \frac{(km/l)(1 - m/l)(l - k)}{l - 1}$
\end{latexonly}
   of the hypergeometric distribution with parameters $m$, $l$ and $k$.
\end{tabb}
\begin{htmlonly}
   \return{the variance of the Hypergeometric distribution
    $\mbox{Var}[X] = (km / l)(1 - m / l)(l - k) / (l - 1)$}
\end{htmlonly}
\begin{code}

   public static double getStandardDeviation (int m, int l, int k)\begin{hide} {
      return Math.sqrt (HypergeometricDist.getVariance (m, l, k));
   }\end{hide}
\end{code}
\begin{tabb}  Computes and returns the standard deviation of the hypergeometric distribution
   with parameters $m$, $l$ and $k$.
\end{tabb}
\begin{htmlonly}
   \return{the standard deviation of the hypergeometric distribution}
\end{htmlonly}
\begin{code}

   public int getM()\begin{hide} {
      return m;
   }\end{hide}
\end{code}
\begin{tabb}
   Returns the $m$ associated with this object.
\end{tabb}
\begin{code}

   public int getL()\begin{hide} {
      return l;
   }\end{hide}
\end{code}
\begin{tabb}
   Returns the $l$ associated with this object.
\end{tabb}
\begin{code}

   public int getK()\begin{hide} {
      return k;
   }\end{hide}
\end{code}
\begin{tabb}
   Returns the $k$ associated with this object.
\end{tabb}
\begin{code}\begin{hide}
   private void setHypergeometric() {
      int imin = Math.max (0, k - l + m);
      int imax = Math.min (k, m);
      supportA = imin;
      supportB = imax;
      int ns = imax - imin + 1;
      if (ns > MAXN) {
         pdf = null;
         cdf = null;
         return;
      }

      int offset = imin;
      imin = 0;
      imax -= offset;
      double[] P = new double[ns];
      double[] F = new double[ns];

      // Computes the mode (taken from UNURAN)
      int mode = (int)((k + 1.0)*(m + 1.0)/(l + 2.0));
      int imid = mode - offset;

      P[imid] = prob (m, l, k, mode);

      int i = imid;
      while (i > imin && Math.abs (P[i]) > EPSILON) {
         P[i-1] = P[i]*(i + offset)/(m-i-offset+1)
                  * (l - m - k + i + offset)/(k - i - offset + 1);
         i--;
      }
      imin = i;

      i = imid;
      while (i < imax && Math.abs (P[i]) > EPSILON) {
         P[i+1] = P[i]*(m - i - offset)/(i + offset + 1)
                  * (k - i - offset)/(l - m - k + i + offset + 1);
         i++;
      }
      imax = i;

      F[imin] = P[imin];
      i = imin;
      while (i < imax && F[i] < 0.5) {
         i++;
         F[i] = F[i-1] + P[i];
      }
      xmed = i;

      F[imax] = P[imax];
      i = imax - 1;
      while (i > xmed) {
         F[i] = P[i] + F[i + 1];
         i--;
      }

       xmin = imin + offset;
       xmax = imax + offset;
       xmed += offset;
       pdf  = new double[imax + 1 - imin];
       cdf  = new double[imax + 1 - imin];
       System.arraycopy (P, imin, pdf, 0, imax+1-imin);
       System.arraycopy (F, imin, cdf, 0, imax+1-imin);
   }
\end{hide}

   public double[] getParams ()\begin{hide} {
      double[] retour = {m, l, k};
      return retour;
   }\end{hide}
\end{code}
\begin{tabb}
   Return a table containing the parameters of the current distribution.
   This table is put in regular order: [$m$, $l$, $k$].
\end{tabb}
\begin{code}

   public void setParams (int m, int l, int k)\begin{hide} {
      if (l <= 0)
         throw new IllegalArgumentException ("l must be greater than 0");
      if (m <= 0 || m > l)
         throw new IllegalArgumentException ("m is invalid: 1 <= m < l");
      if (k <= 0 || k > l)
         throw new IllegalArgumentException ("k is invalid: 1 <= k < l");
      this.m = m;
      this.l = l;
      this.k = k;
      setHypergeometric();
      if (k < l - m)
          p0 = Math.exp (Num.lnFactorial (l - m) + Num.lnFactorial (l - k)
                               -Num.lnFactorial (l) - Num.lnFactorial (l - m - k));
      else
          p0 = Math.exp (Num.lnFactorial (m) + Num.lnFactorial (k)
                               -Num.lnFactorial (k- l + m) - Num.lnFactorial (l));
   }\end{hide}
\end{code}
 \begin{tabb}
    Resets the parameters of this object to $m$, $l$ and $k$.
 \end{tabb}
\begin{hide}\begin{code}

   public String toString ()\begin{hide} {
      return getClass().getSimpleName() + " : m = " + m + ", l = " + l + ", k = " + k;
   }\end{hide}
\end{code}
\begin{tabb}
   Returns a \texttt{String} containing information about the current distribution.
\end{tabb}\end{hide}
\begin{code}\begin{hide}
}\end{hide}\end{code}

