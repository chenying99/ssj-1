\defclass{Pearson6Dist}

Extends the class \class{ContinuousDistribution} for
the {\em Pearson type VI\/} distribution with shape parameters
$\alpha_1 > 0$ and $\alpha_2 > 0$, and scale parameter $\beta > 0$.
The density function is given by
\begin{htmlonly}
\eq
  f(x) = (x / \beta)^{\alpha_{1} - 1} / (\beta \mathcal{B}(\alpha_{1},
\alpha_{2})[1 + x/\beta]^{\alpha_{1} + \alpha_{2}})
  \qquad \mbox{for } x > 0,
\endeq
 and $f(x) = 0$ otherwise,
\end{htmlonly}
\begin{latexonly}
\eq
  f(x) =\left\{\begin{array}{ll} \displaystyle
      \frac{\left({x}/{\beta}\right)^{\alpha_{1} - 1}}{\beta
\mathcal{B}(\alpha_{1}, \alpha_{2})(1 + {x}/{\beta})^{\alpha_{1} +
\alpha_{2}}}
  & \quad \mbox{for } x > 0, \\[14pt]
   0 & \quad \mbox{otherwise,}
   \end{array} \right.
  \eqlabel{eq:fpearson6}
\endeq
\end{latexonly}
where $\mathcal{B}$ is the beta function.
The distribution function is given by
\begin{htmlonly}
\eq
   F(x) = F_{B}\left(x / (x + \beta)\right)
   \qquad \mbox{for } x > 0,
\endeq
\end{htmlonly}
\begin{latexonly}
\eq
   F(x) = F_{B}\left(\frac{x}{x + \beta}\right)
   \qquad \mbox{for } x > 0,
   \eqlabel{eq:Fpearson6}
\endeq
\end{latexonly}
 and $F(x) = 0$ otherwise, where $F_{B}(x)$ is the distribution function
of a beta distribution with shape
parameters $\alpha_1$ and $\alpha_2$.

\bigskip\hrule

%%%%%%%%%%%%%%%%%%%%%%%%%%%%%%%%%%%%%%%%
\begin{code}
\begin{hide}
/*
 * Class:        Pearson6Dist
 * Description:  Pearson type VI distribution
 * Environment:  Java
 * Software:     SSJ 
 * Copyright (C) 2001  Pierre L'Ecuyer and Université de Montréal
 * Organization: DIRO, Université de Montréal
 * @author       
 * @since

 * SSJ is free software: you can redistribute it and/or modify it under
 * the terms of the GNU General Public License (GPL) as published by the
 * Free Software Foundation, either version 3 of the License, or
 * any later version.

 * SSJ is distributed in the hope that it will be useful,
 * but WITHOUT ANY WARRANTY; without even the implied warranty of
 * MERCHANTABILITY or FITNESS FOR A PARTICULAR PURPOSE.  See the
 * GNU General Public License for more details.

 * A copy of the GNU General Public License is available at
   <a href="http://www.gnu.org/licenses">GPL licence site</a>.
 */
\end{hide}
package umontreal.iro.lecuyer.probdist;
\begin{hide}
import umontreal.iro.lecuyer.util.Num;
import optimization.*;
\end{hide}

public class Pearson6Dist extends ContinuousDistribution\begin{hide} {
   protected double alpha1;
   protected double alpha2;
   protected double beta;
   protected double logBeta; // Ln (Beta (alpha1, alpha2))

   private static class Optim implements Uncmin_methods {
      private int n;
      private double[] x;

      public Optim (double[] x, int n) {
         this.n = n;
         this.x = new double[n];
         System.arraycopy (x, 0, this.x, 0, n);
      }

      public double f_to_minimize (double[] param) {

         if ((param[1] <= 0.0) || (param[2] <= 0.0) || (param[3] <= 0.0))
            return 1e200;

         double sumLogY = 0.0;
         double sumLog1_Y = 0.0;
         for (int i = 0; i < n; i++)
         {
            if (x[i] > 0.0)
               sumLogY += Math.log (x[i] / param[3]);
            else
               sumLogY -= 709.0;
            sumLog1_Y += Math.log1p (x[i] / param[3]);
         }

         return (n * (Math.log (param[3]) + Num.lnBeta (param[1], param[2])) -
         (param[1] - 1.0) * sumLogY + (param[1] + param[2]) * sumLog1_Y);
      }

      public void gradient (double[] x, double[] g)
      {
      }

      public void hessian (double[] x, double[][] h)
      {
      }
   }
\end{hide}\end{code}
%%%%%%%%%%%%%%%%%%%%%%%%%%%%%%%%%%%%%%%%%
\subsubsection* {Constructor}

\begin{code}

   public Pearson6Dist (double alpha1, double alpha2, double beta)\begin{hide} {
      setParam (alpha1, alpha2, beta);
   }\end{hide}
\end{code}
\begin{tabb}
   Constructs a \texttt{Pearson6Dist} object with parameters $\alpha_1$ =
   \texttt{alpha1}, $\alpha_2$ = \texttt{alpha2} and $\beta$ = \texttt{beta}.
\end{tabb}

%%%%%%%%%%%%%%%%%%%%%%%%%%%%%%%%%%%
\subsubsection* {Methods}
\begin{code}\begin{hide}

   public double density (double x) {
      if (x <= 0.0)
         return 0.0;
      return Math.exp ((alpha1 - 1.0) * Math.log (x / beta) - (logBeta +
            (alpha1 + alpha2) * Math.log1p (x / beta))) / beta;
   }

   public double cdf (double x) {
      return cdf (alpha1, alpha2, beta, x);
   }

   public double barF (double x) {
      return barF (alpha1, alpha2, beta, x);
   }

   public double inverseF (double u) {
      return inverseF (alpha1, alpha2, beta, u);
   }

   public double getMean () {
      return getMean (alpha1, alpha2, beta);
   }

   public double getVariance () {
      return getVariance (alpha1, alpha2, beta);
   }

   public double getStandardDeviation () {
      return getStandardDeviation (alpha1, alpha2, beta);
   }\end{hide}

   public static double density (double alpha1, double alpha2,
                                 double beta, double x)\begin{hide} {
      if (alpha1 <= 0.0)
         throw new IllegalArgumentException("alpha1 <= 0");
      if (alpha2 <= 0.0)
         throw new IllegalArgumentException("alpha2 <= 0");
      if (beta <= 0.0)
         throw new IllegalArgumentException("beta <= 0");
      if (x <= 0.0)
         return 0.0;

      return Math.exp ((alpha1 - 1.0) * Math.log (x / beta) -
         (Num.lnBeta (alpha1, alpha2) + (alpha1 + alpha2) * Math.log1p (x / beta))) / beta;
   }\end{hide}
\end{code}
\begin{tabb}
   Computes the density function of a Pearson VI distribution with shape
parameters $\alpha_1$
   and $\alpha_2$, and scale parameter $\beta$.
\end{tabb}
\begin{code}

   public static double cdf (double alpha1, double alpha2,
                             double beta, double x)\begin{hide} {
      if (alpha1 <= 0.0)
         throw new IllegalArgumentException("alpha1 <= 0");
      if (alpha2 <= 0.0)
         throw new IllegalArgumentException("alpha2 <= 0");
      if (beta <= 0.0)
         throw new IllegalArgumentException("beta <= 0");
      if (x <= 0.0)
         return 0.0;

      return BetaDist.cdf (alpha1, alpha2, 15, x / (x + beta));
   }\end{hide}
\end{code}
\begin{tabb}
   Computes the distribution function of a Pearson VI distribution with
shape parameters $\alpha_1$
   and $\alpha_2$, and scale parameter $\beta$.
\end{tabb}
\begin{code}

   public static double barF (double alpha1, double alpha2,
                              double beta, double x)\begin{hide} {
      if (alpha1 <= 0.0)
         throw new IllegalArgumentException("alpha1 <= 0");
      if (alpha2 <= 0.0)
         throw new IllegalArgumentException("alpha2 <= 0");
      if (beta <= 0.0)
         throw new IllegalArgumentException("beta <= 0");
      if (x <= 0.0)
         return 1.0;

      return 1.0 - BetaDist.cdf (alpha1, alpha2, 15, x / (x + beta));
   }\end{hide}
\end{code}
\begin{tabb}
   Computes the complementary distribution function of a Pearson VI distribution
   with shape parameters $\alpha_1$ and $\alpha_2$, and scale parameter $\beta$.
\end{tabb}
\begin{code}

   public static double inverseF (double alpha1, double alpha2,
                                  double beta, double u)\begin{hide} {
      if (alpha1 <= 0.0)
         throw new IllegalArgumentException("alpha1 <= 0");
      if (alpha2 <= 0.0)
         throw new IllegalArgumentException("alpha2 <= 0");
      if (beta <= 0.0)
         throw new IllegalArgumentException("beta <= 0");

      double y = BetaDist.inverseF (alpha1, alpha2, 15, u);

      return ((y * beta) / (1.0 - y));
   }\end{hide}
\end{code}
\begin{tabb}
   Computes the inverse distribution function of a Pearson VI distribution
   with shape parameters $\alpha_1$ and $\alpha_2$, and scale parameter $\beta$.
\end{tabb}
\begin{code}

   public static double[] getMLE (double[] x, int n)\begin{hide} {
      if (n <= 0)
         throw new IllegalArgumentException ("n <= 0");

      double[] parameters = new double[3];
      double[] xpls = new double[4];
      double[] param = new double[4];
      double[] fpls = new double[4];
      double[] gpls = new double[4];
      int[] itrcmd = new int[2];
      double[][] h = new double[4][4];
      double[] udiag = new double[4];

      Optim system = new Optim (x, n);

      double mean = 0.0;
      double mean2 = 0.0;
      double mean3 = 0.0;
      for (int i = 0; i < n; i++)
      {
         mean += x[i];
         mean2 += x[i] * x[i];
         mean3 += x[i] * x[i] * x[i];
      }
      mean /= (double) n;
      mean2 /= (double) n;
      mean3 /= (double) n;

      double r1 = mean2 / (mean * mean);
      double r2 = mean2 * mean / mean3;

      param[1] = - (2.0 * (-1.0 + r1 * r2)) / (-2.0 + r1 + r1 * r2);
      if(param[1] <= 0) param[1] = 1;
      param[2] = (- 3.0 - r2 + 4.0 * r1 * r2) / (- 1.0 - r2 + 2.0 * r1 * r2);
      if(param[2] <= 0) param[2] = 1;
      param[3] = (param[2] - 1.0) * mean / param[1];
      if(param[3] <= 0) param[3] = 1;

      Uncmin_f77.optif0_f77 (3, param, system, xpls, fpls, gpls, itrcmd, h, udiag);

      for (int i = 0; i < 3; i++)
         parameters[i] = xpls[i+1];

      return parameters;
   }\end{hide}
\end{code}
\begin{tabb}
    Estimates the parameters $(\alpha_1,\alpha_2,\beta)$ of the Pearson VI distribution
    using the maximum likelihood method, from the $n$ observations
   $x[i]$, $i = 0, 1,\ldots, n-1$. The estimates are returned in a three-element
    array, in regular order: [$\alpha_1, \alpha_2$, $\beta$].
   \begin{detailed}
   The estimate of the parameters is given by maximizing numerically the
   log-likelihood function, using the Uncmin package \cite{iSCHa,iVERa}.
   \end{detailed}
\end{tabb}
\begin{htmlonly}
   \param{x}{the list of observations to use to evaluate parameters}
   \param{n}{the number of observations to use to evaluate parameters}
   \return{returns the parameters [$\hat{\alpha_1}, \hat{\alpha_2}, \hat{\beta}$]}
\end{htmlonly}
\begin{code}

   public static Pearson6Dist getInstanceFromMLE (double[] x, int n)\begin{hide} {
      double parameters[] = getMLE (x, n);
      return new Pearson6Dist (parameters[0], parameters[1], parameters[2]);
   }\end{hide}
\end{code}
\begin{tabb}
   Creates a new instance of a Pearson VI distribution with parameters $\alpha_1$,
   $\alpha_2$ and $\beta$, estimated using the maximum likelihood method based on
   the $n$ observations $x[i]$, $i = 0, 1, \ldots, n-1$.
\end{tabb}
\begin{htmlonly}
   \param{x}{the list of observations to use to evaluate parameters}
   \param{n}{the number of observations to use to evaluate parameters}
\end{htmlonly}
\begin{code}

   public static double getMean (double alpha1, double alpha2,
                                 double beta)\begin{hide} {
      if (alpha1 <= 0.0)
         throw new IllegalArgumentException("alpha1 <= 0");
      if (alpha2 <= 1.0)
         throw new IllegalArgumentException("alpha2 <= 1");
      if (beta <= 0.0)
         throw new IllegalArgumentException("beta <= 0");

      return ((beta * alpha1) / (alpha2 - 1.0));
   }\end{hide}
\end{code}
\begin{tabb}
   Computes and returns the mean $E[X] = (\beta \alpha_1) / (\alpha_2 - 1)$ of a
   Pearson VI distribution with shape parameters $\alpha_1$ and $\alpha_2$, and
   scale parameter $\beta$.
\end{tabb}
\begin{code}

   public static double getVariance (double alpha1, double alpha2,
                                     double beta)\begin{hide} {
      if (alpha1 <= 0.0)
         throw new IllegalArgumentException("alpha1 <= 0");
      if (alpha2 <= 0.0)
         throw new IllegalArgumentException("alpha2 <= 2");
      if (beta <= 0.0)
         throw new IllegalArgumentException("beta <= 0");

      return (((beta * beta) * alpha1 * (alpha1 + alpha2 - 1.0)) /
((alpha2 - 1.0) * (alpha2 - 1.0) * (alpha2 - 2.0)));
   }\end{hide}
\end{code}
\begin{tabb}
   Computes and returns the variance
   $\mbox{Var}[X] = [\beta^2 \alpha_1 (\alpha_1 + \alpha_2 - 1)] /
    [(\alpha_2 - 1)^2(\alpha_2 - 2)]$ of a Pearson VI distribution with shape
   parameters $\alpha_1$ and $\alpha_2$, and scale parameter $\beta$.
\end{tabb}
\begin{code}

   public static double getStandardDeviation (double alpha1, double alpha2,
                                              double beta)\begin{hide} {
      return Math.sqrt (getVariance (alpha1, alpha2, beta));
   }\end{hide}
\end{code}
\begin{tabb}
   Computes and returns the standard deviation of a Pearson VI
distribution with shape
   parameters $\alpha_1$ and $\alpha_2$, and scale parameter $\beta$.
\end{tabb}
\begin{code}

   public double getAlpha1()\begin{hide} {
      return alpha1;
   }\end{hide}
\end{code}
\begin{tabb}
   Returns the $\alpha_1$ parameter of this object.
\end{tabb}
\begin{code}

   public double getAlpha2()\begin{hide} {
      return alpha2;
   }\end{hide}
\end{code}
\begin{tabb}
   Returns the $\alpha_2$ parameter of this object.
\end{tabb}
\begin{code}

   public double getBeta()\begin{hide} {
      return beta;
   }\end{hide}
\end{code}
\begin{tabb}
   Returns the $\beta$ parameter of this object.
\end{tabb}
\begin{code}

   public void setParam (double alpha1, double alpha2, double beta)\begin{hide} {
      if (alpha1 <= 0.0)
         throw new IllegalArgumentException("alpha1 <= 0");
      if (alpha2 <= 0.0)
         throw new IllegalArgumentException("alpha2 <= 0");
      if (beta <= 0.0)
         throw new IllegalArgumentException("beta <= 0");
      supportA = 0.0;
      this.alpha1 = alpha1;
      this.alpha2 = alpha2;
      this.beta = beta;
      logBeta = Num.lnBeta (alpha1, alpha2);
   }\end{hide}
\end{code}
\begin{tabb}
   Sets the parameters $\alpha_1$, $\alpha_2$ and $\beta$ of this object.
\end{tabb}
\begin{code}

   public double[] getParams ()\begin{hide} {
      double[] retour = {alpha1, alpha2, beta};
      return retour;
   }\end{hide}
\end{code}
\begin{tabb}
   Return a table containing the parameters of the current distribution.
   This table is put in regular order: [$\alpha_1$, $\alpha_2$, $\beta$].
\end{tabb}
\begin{hide}\begin{code}

   public String toString ()\begin{hide} {
      return getClass().getSimpleName() + " : alpha1 = " + alpha1 + ", alpha2 = " + alpha2 + ", beta = " + beta;
   }\end{hide}
\end{code}
\begin{tabb}
   Returns a \texttt{String} containing information about the current distribution.
\end{tabb}\end{hide}
\begin{code}\begin{hide}
}\end{hide}
\end{code}
