\defclass {PowerDist}

Extends the class \class{ContinuousDistribution} for 
the {\em power\/} distribution
\cite[page 161]{tEVA00a} with shape parameter
$c > 0$, over the interval $[a,b]$, where $a < b$.
\begin{htmlonly}
It has density 
\eq
  f(x) = c(x - a)^{c - 1}/(b - a)^{c}
\endeq
for $a < x < b$, and 0 elsewhere.  It has distribution function
\eq
    F(x) = (x - a)^{c}/(b - a)^{c}
           \qquad  \mbox{for } a \le x \le b,
\endeq
\end{htmlonly}%
\begin{latexonly}%
This distribution has density 
\eq 
  f(x) = \frac{c(x-a)^{c - 1}} {(b - a)^{c}},
            \qquad  \mbox{for } a \le x \le b,  \eqlabel{eq:fpower}
\endeq
and  $f(x) = 0$  elsewhere. Its distribution function is
\eq
    F(x) = \frac {(x - a)^{c}} {(b - a)^{c}}, 
            \qquad  \mbox{for } a \le x \le b,     \eqlabel{eq:Fpower}
\endeq
\end{latexonly}
with $F(x) = 0$ for $x \le a$ and  $F(x) = 1$ for $x \ge b$.

\bigskip\hrule

%%%%%%%%%%%%%%%%%%%%%%%%%%%%%%%%%%%%%%%%
\begin{code}
\begin{hide}
/*
 * Class:        PowerDist
 * Description:  power distribution
 * Environment:  Java
 * Software:     SSJ 
 * Copyright (C) 2001  Pierre L'Ecuyer and Université de Montréal
 * Organization: DIRO, Université de Montréal
 * @author       
 * @since

 * SSJ is free software: you can redistribute it and/or modify it under
 * the terms of the GNU General Public License (GPL) as published by the
 * Free Software Foundation, either version 3 of the License, or
 * any later version.

 * SSJ is distributed in the hope that it will be useful,
 * but WITHOUT ANY WARRANTY; without even the implied warranty of
 * MERCHANTABILITY or FITNESS FOR A PARTICULAR PURPOSE.  See the
 * GNU General Public License for more details.

 * A copy of the GNU General Public License is available at
   <a href="http://www.gnu.org/licenses">GPL licence site</a>.
 */
\end{hide}
package umontreal.iro.lecuyer.probdist;
\begin{hide}
import umontreal.iro.lecuyer.util.Num;
\end{hide}
public class PowerDist extends ContinuousDistribution\begin{hide} {
   private double a;
   private double b;
   private double c;

\end{hide}
\end{code}
%%%%%%%%%%%%%%%%%%%%%%%%%%%%%%%%%%%%%%%%%
\subsubsection* {Constructors}

\begin{code}

   public PowerDist (double a, double b, double c)\begin{hide} {
      setParams (a, b, c);
   }\end{hide}
\end{code}
\begin{tabb} Constructs a \texttt{PowerDist} object with parameters 
     $a =$ \texttt{a}, $b =$ \texttt{b} and $c =$ \texttt{c}.
 \end{tabb}
\begin{code}

   public PowerDist (double b, double c)\begin{hide} {
      setParams (0.0, b, c);
   }\end{hide}
\end{code}
\begin{tabb} Constructs a \texttt{PowerDist} object with parameters 
     $a = 0$, $b =$ \texttt{b} and $c =$ \texttt{c}.
 \end{tabb}
\begin{code}

   public PowerDist (double c)\begin{hide} {
      setParams (0.0, 1.0, c);
   }\end{hide}
\end{code}
\begin{tabb} Constructs a \texttt{PowerDist} object with parameters 
     $a = 0$, $b =1$ and $c =$ \texttt{c}.
 \end{tabb}


%%%%%%%%%%%%%%%%%%%%%%%%%%%%%%%%%%%
\subsubsection* {Methods}

\begin{code}\begin{hide}

   public double density (double x) {
      return density (a, b, c, x);
   }

   public double cdf (double x) {
      return cdf (a, b, c, x);
   }

   public double barF (double x) {
      return barF (a, b, c, x);
   }

   public double inverseF (double u) {
      return inverseF (a, b, c, u);
   }

   public double getMean() {
      return PowerDist.getMean (a, b, c);
   }

   public double getVariance() {
      return PowerDist.getVariance (a, b, c);
   }

   public double getStandardDeviation() {
      return PowerDist.getStandardDeviation (a, b, c);
   }\end{hide}

   public static double density (double a, double b, double c, double x)\begin{hide} {
      if (c <= 0.0)
         throw new IllegalArgumentException ("c <= 0");
      if (x <= a)
         return 0.0;
      if (x >= b)
         return 0.0;
      double z = (x-a)/(b-a);
      return c*Math.pow(z, c-1.0) / (b-a);
   }\end{hide}
\end{code}
\begin{tabb} Computes the density function (\ref{eq:fpower}).
\end{tabb}
\begin{htmlonly}
   \param{a}{left limit of interval}
   \param{b}{right limit of interval}
   \param{c}{shape parameter}
   \param{x}{the value at which the density is evaluated}
   \return{returns the density function}
\end{htmlonly}
\begin{code}

   public static double cdf (double a, double b, double c, double x)\begin{hide} {
      if (c <= 0.0)
         throw new IllegalArgumentException ("c <= 0");
      if (x <= a)
         return 0.0;
      if (x >= b)
         return 1.0;
      return Math.pow((x-a)/(b-a), c);
   }\end{hide}
\end{code}
 \begin{tabb}
  Computes the distribution function (\ref{eq:Fpower}).
 \end{tabb}
\begin{htmlonly}
   \param{a}{left limit of interval}
   \param{b}{right limit of interval}
   \param{c}{shape parameter}
   \param{x}{the value at which the distribution is evaluated}
   \return{returns the distribution function}
\end{htmlonly}
\begin{code}

   public static double barF (double a, double b, double c, double x)\begin{hide} {
      if (c <= 0.0)
         throw new IllegalArgumentException ("c <= 0");
      if (x <= a)
         return 1.0;
      if (x >= b)
         return 0.0;
      return 1.0 - Math.pow((x-a)/(b-a),c);
   }\end{hide}
\end{code}
  \begin{tabb}
 Computes  the complementary distribution function.
 \end{tabb}
\begin{htmlonly}
   \param{a}{left limit of interval}
   \param{b}{right limit of interval}
   \param{c}{shape parameter}
   \param{x}{the value at which the complementary distribution is evaluated}
   \return{returns the complementary distribution function}
\end{htmlonly}
\begin{code}

   public static double inverseF (double a, double b, double c, double u)\begin{hide} {
      if (c <= 0.0)
         throw new IllegalArgumentException ("c <= 0");
      if (u < 0.0 || u > 1.0)
          throw new IllegalArgumentException ("u not in [0, 1]");
      if (u == 0.0)
         return a;
      if (u == 1.0)
         return b;

      return a + (b-a) * Math.pow(u,1.0/c);
   }\end{hide}
\end{code}
  \begin{tabb}
  Computes  the inverse of the distribution function.
 \end{tabb}
\begin{htmlonly}
   \param{a}{left limit of interval}
   \param{b}{right limit of interval}
   \param{c}{shape parameter}
   \param{u}{the value at which the inverse distribution is evaluated}
   \return{returns the inverse of the distribution function}
\end{htmlonly}
\begin{code}

   public static double[] getMLE (double[] x, int n, double a, double b)\begin{hide} {
      if (n <= 0)
         throw new IllegalArgumentException ("n <= 0");

      double d = b - a;
      double somme = 0;
      for (int i = 0 ; i < n ; ++i) somme += Math.log((x[i] - a)/d);

      double [] parametres = new double [1];
      parametres[0] = -1.0 / (somme/n);
      return parametres;
   }\end{hide}
\end{code}
\begin{tabb}
 Estimates the parameter $c$ of the power distribution from the $n$
   observations $x[i]$, $i = 0, 1, \ldots, n-1$, using the maximum
   likelihood method and assuming that $a$ and $b$ are known. 
   The estimate is returned in a one-element array: [$c$].
   \begin{detailed}
   The maximum likelihood estimator is the value
   $\hat{c}$ that satisfies the equation
   \begin{eqnarray*}
     \frac1{\hat c} =  -\frac1n \sum_{i=1}^{n} \ln\left(\frac{x_i - a}{b - a}
    \right)
   \end{eqnarray*}
   \end{detailed}
\end{tabb}
\begin{htmlonly}
   \param{x}{the list of observations to use to evaluate parameters}
   \param{n}{the number of observations to use to evaluate parameters}
   \param{a}{left limit of interval}
   \param{b}{right limit of interval}
   \return{returns the shape parameter [$\hat{c}$]}
\end{htmlonly}
\begin{code}

   public static PowerDist getInstanceFromMLE (double[] x, int n,
                                               double a, double b)\begin{hide} {
      double parameters[] = getMLE (x, n, a, b);
      return new PowerDist (a, b, parameters[0]);
   }\end{hide}
\end{code}
\begin{tabb}
   Creates a new instance of a power distribution with parameters $a$ and $b$,
   with $c$ estimated using the maximum likelihood method based on the 
   $n$ observations $x[i]$, $i = 0, \ldots, n-1$.
\end{tabb}
\begin{htmlonly}
   \param{x}{the list of observations to use to evaluate parameters}
   \param{n}{the number of observations to use to evaluate parameters}
   \param{a}{left limit of interval}
   \param{b}{right limit of interval}
\end{htmlonly}
\begin{code}

   public static double getMean (double a, double b, double c)\begin{hide} {
      return a + (b-a) * c / (c+1.0);
   }\end{hide}
\end{code}
\begin{tabb} Returns the mean $a + (b-a)c/(c+1)$ of the power distribution
with parameters  $a$, $b$ and $c$.
\end{tabb}
\begin{htmlonly}
   \param{a}{left limit of interval}
   \param{b}{right limit of interval}
   \param{c}{shape parameter}
   \return{returns the mean}
\end{htmlonly}
\begin{code}

   public static double getVariance (double a, double b, double c)\begin{hide} {
      return (b-a)*(b-a)*c / ((c+1.0)*(c+1.0)*(c+2.0));
   }\end{hide}
\end{code}
\begin{tabb}  Computes and returns the variance $(b-a)^2 c / [(c+1)^2(c+2)]$
   of the power distribution with parameters $a$, $b$ and $c$.
\end{tabb}
\begin{htmlonly}
   \param{a}{left limit of interval}
   \param{b}{right limit of interval}
   \param{c}{shape parameter}
   \return{returns the variance}
\end{htmlonly}
\begin{code}

   public static double getStandardDeviation (double a, double b, double c)\begin{hide} {
      return Math.sqrt (PowerDist.getVariance (a, b, c));
   }\end{hide}
\end{code}
\begin{tabb}  Computes and returns the standard deviation
   of the power distribution with parameters $a$, $b$ and $c$.
\end{tabb}
\begin{htmlonly}
   \return{the standard deviation of the power distribution}
\end{htmlonly}
\begin{code}

   public double getA()\begin{hide} {
      return a;
   }\end{hide}
\end{code} 
  \begin{tabb} Returns the parameter $a$.
  \end{tabb}
\begin{htmlonly}
   \return{the left limit of interval $a$}
\end{htmlonly}
\begin{code}

   public double getB()\begin{hide} {
      return b;
   }\end{hide}
\end{code} 
  \begin{tabb} Returns the parameter $b$.
  \end{tabb}
\begin{htmlonly}
   \return{the right limit of interval $b$}
\end{htmlonly}
\begin{code}

   public double getC()\begin{hide} {
      return c;
   }\end{hide}
\end{code} 
  \begin{tabb} Returns the parameter $c$.
  \end{tabb}
\begin{htmlonly}
   \return{the shape parameter $c$}
\end{htmlonly}
\begin{code}

   public void setParams (double a, double b, double c)\begin{hide} {
      this.a  = a;
      this.b  = b;
      this.c  = c;
   }\end{hide}
\end{code} 
  \begin{tabb} Sets the parameters $a$, $b$ and $c$ for this object.
  \end{tabb}
\begin{htmlonly}
   \param{a}{left limit of interval}
   \param{b}{right limit of interval}
   \param{c}{shape parameter}
\end{htmlonly}
\begin{code}

   public double[] getParams ()\begin{hide} {
      double[] retour = {a, b, c};
      return retour;
   }\end{hide}
\end{code}
\begin{tabb}
   Return a table containing the parameters of the current distribution.
   This table is put in regular order: [$a$, $b$, $c$].
\end{tabb}
\begin{htmlonly}
   \return{[$a$, $b, $c]}
\end{htmlonly}
\begin{hide}\begin{code}

   public String toString ()\begin{hide} {
      return getClass().getSimpleName() + " : a = " + a + " : b = " + b + " : c = " + c;
   }\end{hide}
\end{code}
\begin{tabb}
   Returns a \texttt{String} containing information about the current distribution.
\end{tabb}
\begin{htmlonly}
   \return{a \texttt{String} containing information about the current distribution}
\end{htmlonly}\end{hide}
\begin{code}\begin{hide}
}\end{hide}
\end{code}
