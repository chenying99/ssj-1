\defclass {GeometricDist}

Extends the class \class{DiscreteDistributionInt} for
the {\em geometric\/} distribution
\cite[page 322]{sLAW00a} with parameter
$p$, where $0 < p < 1$.
Its mass function is
\eq
 p (x) = p\, (1-p)^x, \qquad \mbox{for } x = 0, 1, 2, \ldots
                                                \eqlabel{eq:fgeom}
\endeq
The distribution function is given by
\eq
   F (x) = 1 - (1-p)^{x+1}, \qquad \mbox{for } x = 0, 1, 2, \ldots
                                                \eqlabel{eq:Fgeom}
\endeq
and its inverse is
\eq
   F^{-1}(u) = \left\lfloor \latex{\frac{\ln (1 - u)}{\ln (1 - p)}}
                \html{\ln(1-u)/\ln(1-p)}\right\rfloor,
               \qquad \mbox{for }  0 \le u < 1.
                                                \eqlabel{eq:FInvgeom}
\endeq


\bigskip\hrule

%%%%%%%%%%%%%%%%%%%%%%%%%%%%%%%%%%%%%%%%
\begin{code}
\begin{hide}
/*
 * Class:        GeometricDist
 * Description:  geometric distribution
 * Environment:  Java
 * Software:     SSJ 
 * Copyright (C) 2001  Pierre L'Ecuyer and Université de Montréal
 * Organization: DIRO, Université de Montréal
 * @author       
 * @since

 * SSJ is free software: you can redistribute it and/or modify it under
 * the terms of the GNU General Public License (GPL) as published by the
 * Free Software Foundation, either version 3 of the License, or
 * any later version.

 * SSJ is distributed in the hope that it will be useful,
 * but WITHOUT ANY WARRANTY; without even the implied warranty of
 * MERCHANTABILITY or FITNESS FOR A PARTICULAR PURPOSE.  See the
 * GNU General Public License for more details.

 * A copy of the GNU General Public License is available at
   <a href="http://www.gnu.org/licenses">GPL licence site</a>.
 */
\end{hide}
package umontreal.iro.lecuyer.probdist;
\begin{hide}
import  umontreal.iro.lecuyer.util.Num;\end{hide}

public class GeometricDist extends DiscreteDistributionInt\begin{hide} {

   private double p;
   private double vp;
\end{hide}\end{code}

%%%%%%%%%%%%%%%%%%%%%%%%%%%%%%%%%%%%%%%%%
\subsubsection* {Constructor}
\begin{code}

   public GeometricDist (double p)\begin{hide} {
      setP(p);
   }\end{hide}
\end{code}
\begin{tabb} Constructs a geometric distribution with parameter $p$.
\end{tabb}

%%%%%%%%%%%%%%%%%%%%%%%%%%%%%%%%%%%
\subsubsection* {Methods}
\begin{hide}\begin{code}

   public double prob (int x) {
      return prob (p, x);
   }

   public double cdf (int x) {
      return cdf (p, x);
   }

   public double barF (int x) {
      return barF (p, x);
   }

   public int inverseFInt (double u) {
        if (u > 1.0 || u < 0.0)
            throw new IllegalArgumentException ("u not in [0,1]");

        if (p >= 1.0)
            return 0;
        if (u <= p)
            return 0;
        if (u >= 1.0 || p <= 0.0)
            return Integer.MAX_VALUE;

        return (int)Math.floor (Math.log1p(-u)/vp);
   }

   public double getMean() {
      return GeometricDist.getMean (p);
   }

   public double getVariance() {
      return GeometricDist.getVariance (p);
   }

   public double getStandardDeviation() {
      return GeometricDist.getStandardDeviation (p);
   }

\end{code}\end{hide}
\begin{code}

   public static double prob (double p, int x)\begin{hide} {
      if (p < 0 || p > 1)
         throw new IllegalArgumentException ("p not in range (0,1)");
      if (p <= 0)
         return 0;
      if (p >= 1)
         return 0;
      if (x < 0)
         return 0;
      return p*Math.pow (1 - p, x);
   }\end{hide}
\end{code}
 \begin{tabb}
   Computes the geometric probability $p(x)$%
\latex{ given in (\ref{eq:fgeom}) }.
 \end{tabb}
\begin{code}

   public static double cdf (double p, int x)\begin{hide} {
      if (p < 0.0 || p > 1.0)
        throw new IllegalArgumentException ("p not in [0,1]");
      if (x < 0)
         return 0.0;
      if (p >= 1.0)                  // In fact, p == 1
         return 1.0;
      if (p <= 0.0)                  // In fact, p == 0
         return 0.0;
      return 1.0 - Math.pow (1.0 - p, (double)x + 1.0);
   }\end{hide}
\end{code}
  \begin{tabb} Computes the distribution function $F(x)$.
  \end{tabb}
\begin{code}

   public static double barF (double p, int x)\begin{hide} {
      if (p < 0.0 || p > 1.0)
         throw new IllegalArgumentException ("p not in [0,1]");
      if (x < 0)
         return 1.0;
      if (p >= 1.0)                  // In fact, p == 1
         return 0.0;
      if (p <= 0.0)                  // In fact, p == 0
         return 1.0;

      return Math.pow (1.0 - p, x);
   }\end{hide}
\end{code}
\begin{tabb} Computes the complementary distribution function.
\emph{WARNING:} The complementary distribution function is defined as
$\bar F(x) = P[X \ge x]$.
\end{tabb}
\begin{code}

   public static int inverseF (double p, double u)\begin{hide} {
        if (p > 1.0 || p < 0.0)
            throw new IllegalArgumentException ( "p not in [0,1]");
        if (u > 1.0 || u < 0.0)
            throw new IllegalArgumentException ("u not in [0,1]");
        if (p >= 1.0)
            return 0;
        if (u <= p)
           return 0;
        if (u >= 1.0 || p <= 0.0)
            return Integer.MAX_VALUE;

         double v = Math.log1p (-p);
         return (int)Math.floor (Math.log1p (-u)/v);
   }\end{hide}
\end{code}
\begin{tabb} Computes the inverse of the geometric
 distribution\latex{, given by (\ref{eq:FInvgeom})}.
\end{tabb}
\begin{code}

   public static double[] getMLE (int[] x, int n)\begin{hide} {
      if (n <= 0)
         throw new IllegalArgumentException ("n <= 0");

      double parameters[];
      parameters = new double[1];
      double sum = 0.0;
      for (int i = 0; i < n; i++) {
         sum += x[i];
      }

      parameters[0] = 1.0 / (((double) sum / (double) n) + 1.0);

      return parameters;
   }\end{hide}
\end{code}
\begin{tabb}
   Estimates the parameter $p$ of the geometric distribution
   using the maximum likelihood method, from the $n$ observations
   $x[i]$, $i = 0, 1, \ldots, n-1$. The estimate is returned in element 0
   of the returned array.
   \begin{detailed}
   The maximum likelihood estimator $\hat{p}$ satisfies the equation
   (see \cite[page 323]{sLAW00a})
   \begin{eqnarray*}
      \hat{p} = \frac{1}{\bar{x}_n + 1\rule{0pt}{1em}}
   \end{eqnarray*}
   where  $\bar{x}_n$ is the average of $x[0], \ldots, x[n-1]$.
   \end{detailed}
\end{tabb}
\begin{htmlonly}
   \param{x}{the list of observations used to evaluate parameters}
   \param{n}{the number of observations used to evaluate parameters}
   \return{returns the parameter [$\hat{p}$]}
\end{htmlonly}
\begin{code}

   public static GeometricDist getInstanceFromMLE (int[] x, int n)\begin{hide} {

      double parameters[] = getMLE (x, n);

      return new GeometricDist (parameters[0]);
   }\end{hide}
\end{code}
\begin{tabb}
   Creates a new instance of a geometric distribution with parameter $p$
   estimated using the maximum likelihood method based on the $n$
   observations $x[i]$, $i = 0, 1, \ldots, n-1$.
\end{tabb}
\begin{htmlonly}
   \param{x}{the list of observations to use to evaluate parameters}
   \param{n}{the number of observations to use to evaluate parameters}
\end{htmlonly}
\begin{code}

   public static double getMean (double p)\begin{hide} {
      if (p < 0.0 || p > 1.0)
         throw new IllegalArgumentException ("p not in range (0,1)");

      return (1 - p) / p;
   }\end{hide}
\end{code}
\begin{tabb}  Computes and returns the mean  $E[X] = (1 - p)/p$ of the
   geometric distribution with parameter $p$.
\end{tabb}
\begin{htmlonly}
   \return{the mean of the geometric distribution $E[X] = (1 - p) / p$}
\end{htmlonly}
\begin{code}

   public static double getVariance (double p)\begin{hide} {
      if (p < 0.0 || p > 1.0)
         throw new IllegalArgumentException ("p not in range (0,1)");

      return ((1 - p) / (p * p));
   }\end{hide}
\end{code}
\begin{tabb}  Computes and returns the variance $\mbox{Var}[X] = (1 - p)/p^2$
   of the geometric distribution with parameter $p$.
\end{tabb}
\begin{htmlonly}
   \return{the variance of the Geometric distribution
    $\mbox{Var}[X] = (1 - p) / p^2$}
\end{htmlonly}
\begin{code}

   public static double getStandardDeviation (double p)\begin{hide} {
      return Math.sqrt (GeometricDist.getVariance (p));
   }\end{hide}
\end{code}
\begin{tabb}  Computes and returns the standard deviation of the geometric
   distribution with parameter $p$.
\end{tabb}
\begin{htmlonly}
   \return{the standard deviation of the geometric distribution}
\end{htmlonly}
\begin{code}

   public double getP()\begin{hide} {
      return p;
   }\end{hide}
\end{code}
\begin{tabb}
   Returns the $p$ associated with this object.
\end{tabb}
\begin{code}

   public void setP (double p)\begin{hide} {
      if (p < 0 || p > 1)
         throw new IllegalArgumentException ("p not in range (0,1)");
      vp = Math.log1p (-p);
      this.p = p;
      supportA = 0;
   }\end{hide}
\end{code}
\begin{tabb}
   Resets the value of $p$ associated with this object.
\end{tabb}
\begin{code}

   public double[] getParams ()\begin{hide} {
      double[] retour = {p};
      return retour;
   }\end{hide}
\end{code}
\begin{tabb}
   Return a table containing the parameters of the current distribution.
\end{tabb}
\begin{hide}\begin{code}

   public String toString ()\begin{hide} {
      return getClass().getSimpleName() + " : p = " + p;
   }\end{hide}
\end{code}
\begin{tabb}
   Returns a \texttt{String} containing information about the current distribution.
\end{tabb}\end{hide}
\begin{code}\begin{hide}
}\end{hide}
\end{code}
