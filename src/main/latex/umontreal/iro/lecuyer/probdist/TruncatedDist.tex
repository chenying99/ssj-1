\defclass{TruncatedDist}

This container class takes an arbitrary continuous distribution and truncates
it to an interval $[a,b]$, where $a$ and $b$ can be finite or infinite.
If the original density and distribution function are $f_0$ and $F_0$,
the new ones are $f$ and $F$, defined by
\begin{latexonly}%
\[
  f(x) = \frac {f_0(x)}{F_0(b) - F_0(a)} \qquad\mbox { for } a\le x\le b
\]
\end{latexonly}
\begin{htmlonly}
\[
  f(x) =  f_0(x) / (F_0(b) - F_0(a)) \qquad\mbox { for } a\le x\le b
\]
\end{htmlonly}%
and $f(x)=0$ elsewhere, and
\begin{latexonly}%
\[
  F(x) = \frac {F_0(x)-F_0(a)}{F_0(b)-F_0(a)}  \qquad\mbox { for } a\le x\le b.
\]
\end{latexonly}
\begin{htmlonly}
\[
  F(x) = (F_0(x)-F_0(a))/(F_0(b)-F_0(a))  \qquad\mbox { for } a\le x\le b.
\]
\end{htmlonly}

The inverse distribution function of the truncated distribution is
\[
  {F^{-1}}(u) = F_0^{-1}(F_0(a) + (F_0(b) - F_0(a))u)
\]
where $F_0^{-1}$ is the inverse distribution function of the
original distribution.

\bigskip\hrule

%%%%%%%%%%%%%%%%%%%%%%%%%%%%%%%%%%%%%%%%%%%%%%%%%%%%%
\begin{code}
\begin{hide}
/*
 * Class:        TruncatedDist
 * Description:  an arbitrary continuous distribution truncated
 * Environment:  Java
 * Software:     SSJ 
 * Copyright (C) 2001  Pierre L'Ecuyer and Université de Montréal
 * Organization: DIRO, Université de Montréal
 * @author       Richard Simard
 * @since

 * SSJ is free software: you can redistribute it and/or modify it under
 * the terms of the GNU General Public License (GPL) as published by the
 * Free Software Foundation, either version 3 of the License, or
 * any later version.

 * SSJ is distributed in the hope that it will be useful,
 * but WITHOUT ANY WARRANTY; without even the implied warranty of
 * MERCHANTABILITY or FITNESS FOR A PARTICULAR PURPOSE.  See the
 * GNU General Public License for more details.

 * A copy of the GNU General Public License is available at
   <a href="http://www.gnu.org/licenses">GPL licence site</a>.
 */
\end{hide}
package umontreal.iro.lecuyer.probdist;\begin{hide}
import umontreal.iro.lecuyer.functions.MathFunctionUtil;
import umontreal.iro.lecuyer.functions.MathFunction;
\end{hide}

public class TruncatedDist extends ContinuousDistribution\begin{hide} {
   public static int NUMINTERVALS = 500;

   private ContinuousDistribution dist0;  // The original (non-truncated) dist.
   private double fa;                    // F(a)
   private double fb;                    // F(b)
   private double barfb;                 // bar F(b)
   private double fbfa;                  // F(b) - F(a)
   private double a;
   private double b;
   private double approxMean;
   private double approxVariance;
   private double approxStandardDeviation;\end{hide}
\end{code}
%%%%%%%%%%%%%%%%%%%%%%%%%%%%%%%%%%%%%%%%%
\subsubsection* {Constructor}

\begin{code}

   public TruncatedDist (ContinuousDistribution dist, double a, double b)\begin{hide} {
      setParams (dist, a, b);
   }\end{hide}
\end{code}
\begin{tabb}
  Constructs a new distribution by truncating distribution \texttt{dist}
  to the interval $[a,b]$. Restrictions: $a$ and $b$ must be finite.
%  If \texttt{a = Double.NEGATIVE\_INFINITY}, $F_0(a)$ is assumed to be 0.
%  If \texttt{b = Double.POSITIVE\_INFINITY}, $F_0(b)$ is assumed to be 1.
\end{tabb}

%%%%%%%%%%%%%%%%%%%%%%%%%%%%%%%%%%%
\subsubsection* {Methods}
\begin{hide}\begin{code}

   public double density (double x) {
      if ((x < a) || (x > b))
         return 0;
      return dist0.density (x) / fbfa;
   }

   public double cdf (double x) {
      if (x <= a)
         return 0;
      else if (x >= b)
         return 1;
      else
         return (dist0.cdf (x) - fa) / fbfa;
   }

   public double barF (double x) {
      if (x <= a)
         return 1;
      else if (x >= b)
         return 0;
      else
         return (dist0.barF (x) - barfb) / fbfa;
   }

   public double inverseF (double u) {
      if (u == 0)
         return a;
      if (u == 1)
         return b;
      return dist0.inverseF (fa + fbfa * u);
   }\end{code}\end{hide}
\begin{code}

   public double getMean()\begin{hide} {
      if (Double.isNaN (approxMean))
         throw new UnsupportedOperationException("Undefined mean");
      return approxMean;
   }\end{hide}
\end{code}
\begin{tabb}
  Returns an approximation of the mean computed with the
  Simpson $1/3$ numerical integration rule.
\end{tabb}
\begin{htmlonly}
   \exception{UnsupportedOperationException}{the mean of the truncated distribution is unknown}
\end{htmlonly}
\begin{code}

   public double getVariance()\begin{hide} {
      if (Double.isNaN (approxVariance))
         throw new UnsupportedOperationException("Unknown variance");
      return approxVariance;
   }\end{hide}
\end{code}
\begin{tabb}
  Returns an approximation of the variance computed with the
  Simpson $1/3$ numerical integration rule.
\end{tabb}
\begin{htmlonly}
   \exception{UnsupportedOperationException}{the mean of the truncated distribution is unknown}
\end{htmlonly}
\begin{code}

   public double getStandardDeviation()\begin{hide} {
      if (Double.isNaN (approxStandardDeviation))
         throw new UnsupportedOperationException("Unknown standard deviation");
      return approxStandardDeviation;
   }\end{hide}
\end{code}
\begin{tabb}
  Returns the square root of the approximate variance.
\end{tabb}
\begin{htmlonly}
   \exception{UnsupportedOperationException}{the mean of the truncated distribution is unknown}
\end{htmlonly}
\begin{code}

   public double getA()\begin{hide} {
      return a;
   }\end{hide}
\end{code}
\begin{tabb}  Returns the value of $a$.
\end{tabb}
\begin{code}

   public double getB()\begin{hide} {
      return b;
   }\end{hide}
\end{code}
\begin{tabb}   Returns the value of $b$.
\end{tabb}
\begin{code}

   public double getFa()\begin{hide} {
      return fa;
   }\end{hide}
\end{code}
\begin{tabb}   Returns the value of $F_0(a)$.
\end{tabb}
\begin{code}

   public double getFb()\begin{hide} {
      return fb;
   }\end{hide}
\end{code}
\begin{tabb}   Returns the value of $F_0(b)$.
\end{tabb}
\begin{code}

   public double getArea()\begin{hide} {
      return fbfa;
   }
\end{hide}
\end{code}
\begin{tabb}  Returns the value of $F_0(b) - F_0(a)$,
  the area under the truncated density function.
\end{tabb}
\begin{code}

   public void setParams (ContinuousDistribution dist, double a, double b)\begin{hide} {
      if (a >= b)
         throw new IllegalArgumentException ("a must be smaller than b.");
      this.dist0 = dist;
      if (a <= dist.getXinf())
         a = dist.getXinf();
      if (b >= dist.getXsup())
         b = dist.getXsup();
      supportA = this.a = a;
      supportB = this.b = b;
      fa = dist.cdf (a);
      fb = dist.cdf (b);
      fbfa = fb - fa;
      barfb = dist.barF (b);

      if (((a <= dist.getXinf()) && (b >= dist.getXsup())) ||
       ((a == Double.NEGATIVE_INFINITY) && (b == Double.POSITIVE_INFINITY))) {
         approxMean = dist.getMean();
         approxVariance = dist.getVariance();
         approxStandardDeviation = dist.getStandardDeviation();

      } else {
         // The mean is the integral of xf*(x) over [a, b].
         MomentFunction func1 = new MomentFunction (dist, 1);
         approxMean = MathFunctionUtil.simpsonIntegral (func1, a, b, NUMINTERVALS) / fbfa;

         // Estimate the integral of (x-E[X])^2 f*(x) over [a, b]
         MomentFunction func2 = new MomentFunction (dist, 2, approxMean);
         approxVariance = MathFunctionUtil.simpsonIntegral (func2, a, b, NUMINTERVALS) / fbfa;

         approxStandardDeviation = Math.sqrt (approxVariance);
      }
   }\end{hide}
\end{code}
\begin{tabb}  Sets the parameters \texttt{dist}, $a$ and $b$ for this object. See the
  constructor for details.
\end{tabb}
\begin{code}

   public double[] getParams ()\begin{hide} {
      double[] retour = {a, b, fa, fb, fbfa};
      return retour;
   }\end{hide}
\end{code}
\begin{tabb}
   Return a table containing the parameters of the current distribution.
   This table is put in order: [$a$, $b$, $F_0(a)$, $F_0(b)$, $F_0(b) - F_0(a)$].
\end{tabb}
\begin{code}

   public String toString ()\begin{hide} {
      return getClass().getSimpleName() + " : a = " + a + ", b = " + b + ", F(a) = " + fa + ", F(b) = " + fb + ", F(b)-F(a) = " + fbfa;
   }\end{hide}
\end{code}
\begin{tabb}
   Returns a \texttt{String} containing information about the current distribution.
\end{tabb}
\begin{code}\begin{hide}

   private static class MomentFunction implements MathFunction {
      private ContinuousDistribution dist;
      private int moment;
      private double offset;

      public MomentFunction (ContinuousDistribution dist, int moment) {
         this.dist = dist;
         this.moment = moment;
         offset = 0;
      }

      public MomentFunction (ContinuousDistribution dist, int moment, double offset) {
         this (dist, moment);
         this.offset = offset;
      }

      public double evaluate (double x) {
         double res = dist.density (x);
         final double offsetX = x - offset;
         for (int i = 0; i < moment; i++)
            res *= offsetX;
         return res;
      }
   }
}\end{hide}
\end{code}
