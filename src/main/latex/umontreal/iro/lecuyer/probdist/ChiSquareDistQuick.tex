\defclass {ChiSquareDistQuick}

Provides a variant of \class{ChiSquareDist} with
faster but less accurate methods.
The non-static version of 
% the methods \texttt{cdf}, \texttt{barF}, 
\texttt{inverseF} calls the static version.
This method is not very accurate for small $n$ but becomes
better as $n$ increases.
The other methods are the same as in \class{ChiSquareDist}.

\bigskip\hrule

%%%%%%%%%%%%%%%%%%%%%%%%%%%%%%%%%%%%%%%%
\begin{code}
\begin{hide}
/*
 * Class:        ChiSquareDistQuick
 * Description:  chi-square distribution
 * Environment:  Java
 * Software:     SSJ 
 * Copyright (C) 2001  Pierre L'Ecuyer and Université de Montréal
 * Organization: DIRO, Université de Montréal
 * @author       
 * @since

 * SSJ is free software: you can redistribute it and/or modify it under
 * the terms of the GNU General Public License (GPL) as published by the
 * Free Software Foundation, either version 3 of the License, or
 * any later version.

 * SSJ is distributed in the hope that it will be useful,
 * but WITHOUT ANY WARRANTY; without even the implied warranty of
 * MERCHANTABILITY or FITNESS FOR A PARTICULAR PURPOSE.  See the
 * GNU General Public License for more details.

 * A copy of the GNU General Public License is available at
   <a href="http://www.gnu.org/licenses">GPL licence site</a>.
 */
\end{hide}
package umontreal.iro.lecuyer.probdist;


public class ChiSquareDistQuick extends ChiSquareDist\begin{hide} {
\end{hide}
\end{code}
%%%%%%%%%%%%%%%%%%%%%%%%%%%%%%%%%%%%%%%%%
\subsubsection* {Constructor}

\begin{code}

   public ChiSquareDistQuick (int n)\begin{hide} {
      super (n);
   }\end{hide}
\end{code}
\begin{tabb}
   Constructs a chi-square distribution with \texttt{n} degrees of freedom.
\end{tabb}

%%%%%%%%%%%%%%%%%%%%%%%%%%%%%%%%%%%
\subsubsection* {Methods}
\begin{code}\begin{hide}

   public double inverseF (double u) {
      return inverseF (n, u);
   }\end{hide}

   public static double inverseF (int n, double u)\begin{hide} {
      /*
       * Returns an approximation of the inverse of Chi square cdf
       * with n degrees of freedom.
       * As in Figure L.24 of P.Bratley, B.L.Fox, and L.E.Schrage.
       *         A Guide to Simulation Springer-Verlag,
       *         New York, second edition, 1987.
       */

      if (u < 0.0 || u > 1.0)
         throw new IllegalArgumentException ("u is not in [0,1]");
      if (u <= 0.0)
         return 0.0;
      if (u >= 1.0)
         return Double.POSITIVE_INFINITY;

      final double SQP5 = 0.70710678118654752440;
      final double DWARF = 0.1e-15;
      final double ULOW = 0.02;
      double Z, arg, v, ch, sqdf;

      if (n == 1) {
          Z = NormalDist.inverseF01 ((1.0 + u)/2.0);
          return Z*Z;

      } else if (n == 2) {
         arg = 1.0 - u;
         if (arg < DWARF)
            arg = DWARF;
         return -Math.log (arg)*2.0;

     } else if ((u > ULOW) && (u < 1.0 - ULOW)) {
        Z = NormalDist.inverseF01 (u);
        sqdf = Math.sqrt ((double)n);
        v = Z * Z;

        ch = -(((3753.0 * v + 4353.0) * v - 289517.0) * v -
           289717.0) * Z * SQP5 / 9185400;

        ch = ch / sqdf + (((12.0 * v - 243.0) * v - 923.0)
           * v + 1472.0) / 25515.0;

        ch = ch / sqdf + ((9.0 * v + 256.0) * v - 433.0)
           * Z * SQP5 / 4860;

        ch = ch / sqdf - ((6.0 * v + 14.0) * v - 32.0) / 405.0;
        ch = ch / sqdf + (v - 7.0) * Z * SQP5 / 9;
        ch = ch / sqdf + 2.0 * (v - 1.0) / 3.0;
        ch = ch / sqdf + Z / SQP5;
        return n * (ch / sqdf + 1.0);

     } else if (n >= 10) {
        Z = NormalDist.inverseF01 (u);
        v = Z * Z;
        double temp;
        temp = 1.0 / 3.0 + (-v + 3.0) / (162.0 * n) -
           (3.0 * v * v + 40.0 * v + 45.0) / (5832.0 * n * n) +
           (301.0 * v * v * v - 1519.0 * v * v - 32769.0 * v -
           79349.0) / (7873200.0 * n * n * n);
        temp *= Z * Math.sqrt (2.0 / n);

        ch = 1.0 - 2.0 / (9.0 * n) + (4.0 * v * v + 16.0 * v -
           28.0) / (1215.0 * n * n) + (8.0 * v * v * v + 720.0 * v * v +
           3216.0 * v + 2904.0) / (229635.0 * n * n * n) + temp;

        return n * ch * ch * ch;

    } else {
    // Note: this implementation is quite slow.
    // Since it is restricted to the tails, we could perhaps replace
    // this with some other approximation (series expansion ?)
        return 2.0*GammaDist.inverseF (n/2.0, 6, u);
    }
}\end{hide}
\end{code}
\begin{tabb}  Computes a quick-and-dirty approximation of $F^{-1}(u)$, 
  where $F$ is the {\em chi-square\/} distribution with $n$ degrees of freedom.
  Uses the approximation given in  Figure L.24 of  \latex{\cite{sBRA87a}}
  \html {Bratley, Fox and Schrage (1987)} over most of the range.
  For $u < 0.02$ or $u > 0.98$, it uses the approximation given in 
  \latex{\cite{tGOL73a}} \html {Goldstein} for $n \ge 10$, and returns 
  \texttt{2.0 *}
  \clsexternalmethod{}{GammaDist}{inverseF}{double,int,double}~\texttt{(n/2, 6, u)}
   for $n < 10$ in order to avoid
  the loss of precision of the above approximations. 
   When $n \ge 10$ or $0.02 < u < 0.98$,
  it is between 20 to 30 times faster than the same method in
  \class{ChiSquareDist} for $n$ between $10$ and $1000$ and even faster
  for larger $n$. 

  Note that the number $d$ of decimal digits of precision
  generally increases with $n$. For $n=3$, we only have $d = 3$ over
  most of the range. For $n=10$, $d=5$ except far in the tails where $d = 3$.
  For $n=100$, one has more than $d=7$ over most of the range and for
  $n=1000$, at least $d=8$.
  The cases $n = 1$ and $n = 2$ are exceptions, with precision of about $d=10$.
\end{tabb}
\begin{code}\begin{hide}
}\end{hide}
\end{code}
