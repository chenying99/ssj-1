
\begin{latexonly} %%%%%%%

The packages currently offered are the following:
%
\begin{verse}
\externalclass{umontreal.iro.lecuyer}{util}
  contains utility classes used in the implementation of SSJ,
  and which are often useful elsewhere.
  For example, there are timers (for CPU usage),
  utilities to read or format numbers and arrays from/to text,
  operations on binary vectors and matrices,
  some mathematical functions and constants,
  root-finding tools,
  facilities for SQL database interface, and so on.

\externalclass{umontreal.iro.lecuyer}{probdist}
  contains a set of Java classes providing methods to compute mass,
  density, distribution, complementary distribution,
  and inverse distribution functions for many discrete and continuous
  probability distributions, as well as estimating the parameters of
  these distributions.

\externalclass{umontreal.iro.lecuyer}{probdistmulti}
  contains a set of Java classes providing methods to compute mass,
  density, distribution, complementary distribution,
  for some multi-dimensionnal discrete and continuous
  probability distributions.

\externalclass{umontreal.iro.lecuyer}{rng}
  provides facilities for generating uniform random numbers over the
  interval $(0,1)$, or over a given range of integer values, and other
  types of simple random objects such as random permutations.

\externalclass{umontreal.iro.lecuyer}{hups}
 provides classes implementing highly uniform point sets and
 sequences (HUPS), also called low-discrepancy sets and sequences,
 and  tools for their randomization.

\externalclass{umontreal.iro.lecuyer}{randvar}
  provides a collection of classes for non-uniform random variate
  generation, primarily from standard distributions.

\externalclass{umontreal.iro.lecuyer}{randvarmulti}
  provides a collection of classes for
  random number generators for some multi-dimensional distributions.

\externalclass{umontreal.iro.lecuyer}{gof}
  contains tools for performing univariate goodness-of-fit
  (GOF) statistical tests.

\externalclass{umontreal.iro.lecuyer}{stat}
  provides elementary tools for collecting statistics
  and computing confidence intervals.

\externalclass{umontreal.iro.lecuyer}{stat.list}
  this subpackage of \texttt{stat}
  provides support to manage lists of statistical collectors.
% Statistical operations may be applied on all the collectors easily.

\externalclass{umontreal.iro.lecuyer}{simevents}
  provides and manages the event-driven simulation facilities as well
  as the simulation clock.  Can manage several simulations in parallel,
  in the same program.

\externalclass{umontreal.iro.lecuyer}{simevents.eventlist}
 this subpackage of \texttt{simevents} offers several kinds of event
 list implementations.

\externalclass{umontreal.iro.lecuyer}{simprocs}
  provides and manages the process-driven simulation facilities.
%  This package requires \emph{green threads}, which are available
%  only in JDK version 1.3.1 or earlier (unfortunately).
%
%  simprocs ne nécessite pas les green threads absolument.
%  Premièrement, si l'interpréteur D-SOL est employé, tout fonctionne #1, même
%  sans green threads. Avec l'implantation par défaut de SimProcess, qui utilise
%  les threads, c'est mieux avec les green threads, mais cela peut fonctionner
%  (quoique pas toujours stable) avec les native threads. Il faudrait donc dire,
%  dans le texte, qu'il est recommandé d'utiliser une JVM avec green threads,
%  comme Sun JRE <=1.3.1.

\externalclass{umontreal.iro.lecuyer}{functions}
 contains classes that allow one to pass an arbitrary function of one variable
 as argument to a method
 and to apply elementary mathematical operations on generic functions.

\externalclass{umontreal.iro.lecuyer}{functionfit}
provides basic facilities for curve fitting and interpolation
with polynomials. % as, for example, least square fit and spline interpolation.

\externalclass{umontreal.iro.lecuyer}{charts}
provides tools for easy construction, visualization, and customization
of $xy$ plots, histograms, and empirical styled charts
from a Java program.

\externalclass{umontreal.iro.lecuyer}{stochprocess}
implements different kinds of stochastic processes.
%%
\end{verse}
\end{latexonly}  %%%%%%%%%



%%%%%%%%%%%%%%%%%%%%%%%%%%%%%%%%%%%%%%%%%%%%
\subsection*{Dependence on other libraries}

SSJ uses some classes from other free Java libraries.

The \htmladdnormallink{Colt library}{http://acs.lbl.gov/software/colt/},
 developed at the Centre Europ\'een de Recherche
Nucl\'eaire (CERN) in Geneva \cite{iHOS04a},
is a large library that provides a wide range of facilities for
high performance scientific and technical computing in Java.
SSJ uses the class \externalclass{cern.colt.list}{DoubleArrayList}
from Colt in a few of its classes, namely in packages
\externalclass{umontreal.iro.lecuyer}{stat} and
\externalclass{umontreal.iro.lecuyer}{hups}.
The reason is that this class provides a very efficient and convenient
implementation of an (automatically) extensible array of {\tt double},
together with several methods for computing statistics for the observations
% Javadoc does not find cern.jet.stat.Descriptive and issues
% a warning; maybe a bug in Javadoc.
stored in the array (see, e.g., {\tt Descriptive}).
% The Colt library must be installed (see the installation instructions
% of SSJ) if one wishes to use a class relying on it.
% Otherwise, Colt is not needed.
The Colt library is distributed with the SSJ package as \textbf{colt.jar}.
It  must be added in the CLASSPATH environment variable.


The \textbf{linear\_algebra} library is based on public domain LINPACK routines.
They were translated from Fortran to Java by Steve Verrill at the
    USDA Forest Products Laboratory
    Madison, Wisconsin, USA.
This software is also in the public domain and is included in the
SSJ distribution as the \textbf{Blas.jar} archive. It is used only in the
\texttt{probdist} package to compute maximum likelihood estimators.


The optimization package of Steve Verrill includes Java translations of the
\htmladdnormallink{MINPACK}{http://www.iro.umontreal.ca/~simardr/Uncmin_f77/Minpack_f77.html}
 routines \cite{iMOR80a} for  nonlinear least squares problems as well as
\htmladdnormallink{UNCMIN}{http://www.iro.umontreal.ca/~simardr/Uncmin_f77/Uncmin_f77.html}
 routines \cite{iSCHa} for
unconstrained optimization. They were translated from Fortran to Java by
Steve Verrill and are in the public domain. They are included in the SSJ
distribution as the \textbf{optimization.jar} archive. It is used only in the \texttt{probdist}
 package to compute maximum likelihood estimators.


\htmladdnormallink{JFreeChart}{http://www.jfree.org/jfreechart/index.html} is a free
Java library that can generate a wide variety of charts and plots for use in
applications, applets and servlets. \textbf{JFreeChart}  currently supports, amongst
others, bar charts, pie charts, line charts, XY-plots, histograms, scatter plots and
time series plots. It is distributed with SSJ as \textbf{jfreechart-*.jar}.
\htmladdnormallink{JCommon}{http://www.jfree.org/jcommon/index.php} is a free
general purpose Java library containing many useful classes used by JFreeChart and
 other Java packages. It is distributed with SSJ as \textbf{jcommon-*.jar}.
JFreeChart (and JCommon) are used in the SSJ package \textbf{charts} to create
different kinds of charts.

SSJ also provides an interface to the
\htmladdnormallink{UNURAN}{http://statistik.wu-wien.ac.at/unuran/}
library for nonuniform random number generation \cite{iLEY02a}, in the
\externalclass{umontreal.iro.lecuyer}{randvar} package.
UNURAN does not have to be installed to be used with SSJ, because it is
linked statically with the appropriate SSJ native library.
However, the UNURAN documentation will be required
to take full advantage of the library.
