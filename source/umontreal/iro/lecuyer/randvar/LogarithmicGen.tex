\defclass {LogarithmicGen}

This class implements random variate generators for the (discrete)
{\em logarithmic\/} distribution. Its  mass function is
\eq
\begin{latexonly}
  p(x) = \frac{-\theta^x}{x \log(1 - \theta)} \qquad\mbox{ for } x = 1,2,\ldots,
                                       \eqlabel{eq:flogar}
\end{latexonly}
\begin{htmlonly}
  p(x) = {-\theta^x} / {x \log(1 - \theta)} \qquad\mbox{ for } x = 1,2,\ldots,
                                       \eqlabel{eq:flogar}
\end{htmlonly}
\endeq
where $0 < \theta <1$.
It uses inversion with the LS chop-down algorithm if $\theta < \theta_0$
and the LK transformation algorithm if $\theta \ge \theta_0$,
as described in \cite{rKEM81a}.
The threshold $\theta_0$ can be specified when invoking the constructor.
Its default value is $\theta_0 = 0.96$, as suggested in \cite{rKEM81a}.
%
\hpierre{Does this work for any $\theta_0$?  Should we add constraints?}

% With chop-down, we get wrong inverses and it is used
% in UNURAN and Colt.
% With linear search, we get an infinite loop.

A local copy of the parameter $\theta$ is maintained in this class.

\bigskip\hrule

\begin{code}
\begin{hide}
/*
 * Class:        LogarithmicGen
 * Description:  random variate generators for the (discrete) logarithmic distribution
 * Environment:  Java
 * Software:     SSJ 
 * Copyright (C) 2001  Pierre L'Ecuyer and Université de Montréal
 * Organization: DIRO, Université de Montréal
 * @author       
 * @since

 * SSJ is free software: you can redistribute it and/or modify it under
 * the terms of the GNU General Public License (GPL) as published by the
 * Free Software Foundation, either version 3 of the License, or
 * any later version.

 * SSJ is distributed in the hope that it will be useful,
 * but WITHOUT ANY WARRANTY; without even the implied warranty of
 * MERCHANTABILITY or FITNESS FOR A PARTICULAR PURPOSE.  See the
 * GNU General Public License for more details.

 * A copy of the GNU General Public License is available at
   <a href="http://www.gnu.org/licenses">GPL licence site</a>.
 */
\end{hide}
package umontreal.iro.lecuyer.randvar;\begin{hide}
import umontreal.iro.lecuyer.rng.*;
import umontreal.iro.lecuyer.probdist.*;
import umontreal.iro.lecuyer.util.Num;
\end{hide}

public class LogarithmicGen extends RandomVariateGenInt \begin{hide} {
   private static final double default_theta_limit = 0.96;

   private double theta_limit = default_theta_limit;
   private double theta;
   private double t;      // = log (1.0-theta).
   private double h;      // = -theta/log (1.0-theta).
\end{hide}
\end{code}

%%%%%%%%%%%%%%%%%%%%%%%%%%%%%%%%%%%%%%%%%%%%%
\subsubsection* {Constructors}
\begin{code}

   public LogarithmicGen (RandomStream s, double theta)\begin{hide} {
      this (s, theta, default_theta_limit);
   }\end{hide}
\end{code}
\begin{tabb} Creates a logarithmic random variate generator with parameters
  $\theta = $ \texttt{theta} and default value $\theta_0 = 0.96$,
  using stream \texttt{s}.
\end{tabb}
\begin{code}

   public LogarithmicGen (RandomStream s, double theta, double theta0)\begin{hide} {
      super (s, null);
      this.theta = theta;
      theta_limit = theta0;
      init();
   }\end{hide}
\end{code}
\begin{tabb} Creates a logarithmic random variate generator with parameters
  $\theta = $ \texttt{theta} and $\theta_0 = \texttt{theta0}$,
  using stream \texttt{s}.
\end{tabb}
\begin{code}

   public LogarithmicGen (RandomStream s, LogarithmicDist dist)\begin{hide} {
      this (s, dist, default_theta_limit);
   }\end{hide}
\end{code}
\begin{tabb} Creates a new generator with distribution \texttt{dist} and
  stream \texttt{s}, with default value $\theta_0 = 0.96$.
\end{tabb}
\begin{code}

   public LogarithmicGen (RandomStream s, LogarithmicDist dist,
                          double theta0)\begin{hide} {
      super (s, dist);
      theta_limit = theta0;
      if (dist != null)
         theta = dist.getTheta();
      init();
   }\end{hide}
\end{code}
\begin{tabb} Creates a new generator with distribution \texttt{dist}
   and stream \texttt{s}, with $\theta_0 = \texttt{theta0}$.
\end{tabb}


%%%%%%%%%%%%%%%%%%%%%%%%%%%%%%%%%%%%%%%%%%%%%%%%
\subsubsection* {Methods}
\begin{code}\begin{hide}

   private void init () {
      if (theta <= 0.0 || theta >= 1.0)
         throw new IllegalArgumentException ("theta not in (0, 1)");
      if (theta >= theta_limit)
         h = Math.log1p(-theta);
      else
         t = -theta / Math.log1p(-theta);
   }

   public int nextInt() {
      if (theta < theta_limit)
         return ls (stream, theta, t);
      else   // Transformation
         return lk (stream, theta, h);
   }\end{hide}

   public static int nextInt (RandomStream s, double theta)\begin{hide} {
      if (theta < default_theta_limit)
         return ls (s, theta, -theta/Math.log1p(-theta));
      else   // Transformation
         return lk (s, theta, Math.log1p(-theta));
   }\end{hide}
\end{code}
\begin{tabb} Uses stream \texttt{s} to generate
   a new variate from the {\em logarithmic\/} distribution with parameter
 $\theta =$ \texttt{theta}.
\end{tabb}
\begin{code}\begin{hide}


//>>>>>>>>>>>>>>>>>>>>  P R I V A T E    M E T H O D S   <<<<<<<<<<<<<<<<<<<<


   private static int ls (RandomStream stream, double theta, double t) {
      double u = stream.nextDouble();
      int x = 1;

      double p =  t;

      while (u > p) {
            u -= p;
            x++;
            p *= theta*((double) x - 1.0)/((double)x);
      }
      return x;
   }

   private static int lk (RandomStream stream, double theta, double h) {
      double u, v, p, q;
      int x;

      u = stream.nextDouble();
      if (u > theta)
            return 1;
      v = stream.nextDouble();
      q = 1.0 - Math.exp(v * h);
      if ( u <= q * q) {
           x = (int)(1. + (Math.log(u) / Math.log(q)));
           return x;
      }
      return ((u > q) ? 1 : 2);
   }\end{hide}

   public double getTheta()\begin{hide} {
      return theta;
   }\end{hide}
\end{code}
\begin{tabb}
   Returns the $\theta$ associated with this object.
\end{tabb}
\begin{code}

   public double getTheta0()\begin{hide} {
      return theta_limit;
   }\end{hide}
\end{code}
\begin{tabb}
   Returns the $\theta_0$ associated with this object.
\end{tabb}
\begin{code}\begin{hide}
}\end{hide}
\end{code}
