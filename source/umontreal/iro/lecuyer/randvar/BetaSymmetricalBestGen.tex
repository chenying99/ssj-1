\defclass {BetaSymmetricalBestGen}

This class implements {\em symmetrical beta\/} random variate generators using
Devroye's one-liner method. It is based on Best's relation  \cite{rBES78a} 
between a Student-$t$ variate and a symmetrical beta variate:
\[
  B_{\alpha, \alpha} \stackrel{\mathcal L}{=} \frac 1 2 \left(
  1 + \frac{T_{2\alpha}}{\sqrt{2\alpha + T_{2\alpha}^2}}\right).
\]
If $S$ is a random sign and $U_1$, $U_2$ are two independent uniform $[0,1]$
random variates, then the following gives a symmetrical
 beta variate \cite{rDEV96a}:
\begin{equation}
  B_{\alpha, \alpha} \stackrel{\mathcal L}{=} \frac 1 2 + \frac{S}{2
  \sqrt{1 + \frac{1}{\left(U_1^{-1/\alpha} - 1\right)\cos^2(2\pi U_2)}}}
\label{eq.beta.best}
\end{equation}
valid for any shape parameter $\alpha > 0$.



\bigskip\hrule

\begin{code}
\begin{hide}
/*
 * Class:        BetaSymmetricalBestGen
 * Description:  symmetrical beta random variate generators using
                 Devroye's one-liner method.
 * Environment:  Java
 * Software:     SSJ 
 * Copyright (C) 2001  Pierre L'Ecuyer and Université de Montréal
 * Organization: DIRO, Université de Montréal
 * @author       Richard Simard
 * @since

 * SSJ is free software: you can redistribute it and/or modify it under
 * the terms of the GNU General Public License (GPL) as published by the
 * Free Software Foundation, either version 3 of the License, or
 * any later version.

 * SSJ is distributed in the hope that it will be useful,
 * but WITHOUT ANY WARRANTY; without even the implied warranty of
 * MERCHANTABILITY or FITNESS FOR A PARTICULAR PURPOSE.  See the
 * GNU General Public License for more details.

 * A copy of the GNU General Public License is available at
   <a href="http://www.gnu.org/licenses">GPL licence site</a>.
 */
\end{hide}
package umontreal.iro.lecuyer.randvar;\begin{hide}
import umontreal.iro.lecuyer.rng.*;
import umontreal.iro.lecuyer.probdist.BetaSymmetricalDist;
\end{hide}

public class BetaSymmetricalBestGen extends BetaSymmetricalGen \begin{hide} {
   private RandomStream stream2;
   private RandomStream stream3;
   private double afactor;          // = 1/alpha
   private static final double TWOPI = 2.0*Math.PI;      // = 2 Pi
\end{hide}
\end{code}

%%%%%%%%%%%%%%%%%%%%%%%%%%%%%%%%%%%%%%%%%%%%%%%%
\subsubsection* {Constructors}

\begin{code}

   public BetaSymmetricalBestGen (RandomStream s1, RandomStream s2,
                                  RandomStream s3, double alpha) \begin{hide} {
      super (s1, null);
      stream2 = s2;
      stream3 = s3;
      afactor = 1.0/alpha;
      setParams (alpha, alpha, 0.0, 1.0);
   }\end{hide}
\end{code}
  \begin{tabb} Creates a symmetrical beta random variate generator with
 parameter $\alpha =$ \texttt{alpha}, using stream \texttt{s1} to generate
  $U_1$, stream \texttt{s2} to generate $U_2$ and stream \texttt{s3} to
  generate $S$, as given in equation (\ref{eq.beta.best}).
  \end{tabb}
\begin{code}

   public BetaSymmetricalBestGen (RandomStream s1, double alpha) \begin{hide} {
     this (s1, s1, s1, alpha);
   }\end{hide}
\end{code}
  \begin{tabb}  Creates a symmetrical beta random variate generator with
 parameter $\alpha =$ \texttt{alpha},  using only one stream \texttt{s1}
  to generate $U_1$, $U_2$, and $S$ as given in equation (\ref{eq.beta.best}).
  \end{tabb}
\begin{code}

   public BetaSymmetricalBestGen (RandomStream s1, RandomStream s2,
                                  RandomStream s3, BetaSymmetricalDist dist) \begin{hide} {
      super (s1, dist);
      stream2 = s2;
      stream3 = s3;
      afactor = 1.0/dist.getAlpha();
      if (dist != null)
         setParams (dist.getAlpha(), dist.getAlpha(), dist.getA(), dist.getB());
   }\end{hide}
\end{code}
  \begin{tabb}  Creates a new generator for the distribution \texttt{dist},
 using stream \texttt{s1} to generate $U_1$, stream \texttt{s2} to generate $U_2$
 and stream \texttt{s3} to generate $S$ as given in equation (\ref{eq.beta.best}).
  \end{tabb}
\begin{code}

   public BetaSymmetricalBestGen (RandomStream s1, BetaSymmetricalDist dist) \begin{hide} {
     this (s1, s1, s1, dist);
   }\end{hide}
\end{code}
  \begin{tabb}  Creates a new generator for the distribution \texttt{dist},
     using only one stream \texttt{s1}.
  \end{tabb}

%%%%%%%%%%%%%%%%%%%%%%%%%%%%%%%%%%%%%%%%%%%%%%%%5
\subsubsection* {Methods}

\begin{code}

   public static double nextDouble (RandomStream s1, RandomStream s2,
                                    RandomStream s3, double alpha) \begin{hide}  {
      double cos, temp, v, S;
      cos = Math.cos (TWOPI * s2.nextDouble());
      temp = 1.0/Math.pow(s1.nextDouble(), 1.0/alpha) - 1.0;
      v = Math.sqrt(1.0 + 1.0 / (temp*cos*cos));
      S = s3.nextDouble();
      if (S < 0.5)
         return 0.5 - 0.5/v;
      else
         return 0.5 + 0.5/v;
   }\end{hide}
\end{code}
  \begin{tabb} Generates a random number using Devroye's one-liner method.
   Restriction:  $\alpha > 0$.
  \end{tabb}
\begin{code}

   public static double nextDouble (RandomStream s, double alpha) \begin{hide}  {
      return nextDouble (s, s, s, alpha);
   }\end{hide}
\end{code}
  \begin{tabb} Generates a random number using Devroye's one-liner method with
  only one stream \texttt{s}.
  Restriction:  $\alpha > 0$.
  \end{tabb}
\begin{code}\begin{hide}

   public double nextDouble() {
      // Generates a random number using Devroye's one liner method
      double cos, temp, v, S;
      cos = Math.cos (TWOPI * stream2.nextDouble());
      temp = 1.0/Math.pow(stream.nextDouble(), afactor) - 1.0;
      v = Math.sqrt(1.0 + 1.0 / (temp*cos*cos));
      S = stream3.nextDouble();
      if (S < 0.5)
         return 0.5 - 0.5/v;
      else
         return 0.5 + 0.5/v;
  }\end{hide}

   public RandomStream getStream2()\begin{hide} {
      return stream2;
   }\end{hide}
\end{code}
\begin{tabb}   Returns stream \texttt{s2} associated with this object.
\end{tabb}
\begin{code}

   public RandomStream getStream3()\begin{hide} {
      return stream3;
   }\end{hide}
\end{code}
\begin{tabb}   Returns stream \texttt{s3} associated with this object.
\end{tabb}

\begin{code}\begin{hide}
}
\end{hide}
\end{code}
