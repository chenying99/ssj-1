\defclass {InverseGaussianMSHGen}

This class implements \emph{inverse gaussian} random variate generators using
the many-to-one transformation method of Michael, Schucany and Haas (MHS)
 \cite{rMIC76a,rDEV06a}.

\bigskip\hrule

%%%%%%%%%%%%%%%%%%%%%%%%%%%%%%%%%%%%%%%%%%%
\begin{code}
\begin{hide}
/*
 * Class:        InverseGaussianMSHGen
 * Description:  inverse gaussian random variate generators
 * Environment:  Java
 * Software:     SSJ 
 * Copyright (C) 2001  Pierre L'Ecuyer and Université de Montréal
 * Organization: DIRO, Université de Montréal
 * @author       
 * @since

 * SSJ is free software: you can redistribute it and/or modify it under
 * the terms of the GNU General Public License (GPL) as published by the
 * Free Software Foundation, either version 3 of the License, or
 * any later version.

 * SSJ is distributed in the hope that it will be useful,
 * but WITHOUT ANY WARRANTY; without even the implied warranty of
 * MERCHANTABILITY or FITNESS FOR A PARTICULAR PURPOSE.  See the
 * GNU General Public License for more details.

 * A copy of the GNU General Public License is available at
   <a href="http://www.gnu.org/licenses">GPL licence site</a>.
 */
\end{hide}
package umontreal.iro.lecuyer.randvar;\begin{hide}
import umontreal.iro.lecuyer.rng.*;
import umontreal.iro.lecuyer.probdist.*;
\end{hide}

public class InverseGaussianMSHGen extends InverseGaussianGen \begin{hide} {
   private NormalGen genN;
\end{hide}\end{code}

%%%%%%%%%%%%%%%%%%%%%%%%%%%%%%%%%%%%%%%%%%%%
\subsubsection* {Constructors}
\begin{code}

   public InverseGaussianMSHGen (RandomStream s, NormalGen sn,
                                 double mu, double lambda) \begin{hide} {
      super (s, null);
      setParams (mu, lambda, sn);
   }\end{hide}
\end{code} 
\begin{tabb}  Creates an \emph{inverse gaussian} random variate generator with
 parameters $\mu = $ \texttt{mu} and $\lambda= $ \texttt{lambda},
 using streams \texttt{s} and \texttt{sn}.
\end{tabb}
\begin{code}
 
   public InverseGaussianMSHGen (RandomStream s, NormalGen sn,
                                 InverseGaussianDist dist) \begin{hide} {
      super (s, dist);
      if (dist != null)
         setParams (dist.getMu(), dist.getLambda(), sn);
   }\end{hide}
\end{code}
\begin{tabb}  Creates a new generator for the distribution \texttt{dist}
 using streams \texttt{s} and \texttt{sn}.
\end{tabb}

%%%%%%%%%%%%%%%%%%%%%%%%%%%%%%%%%%%%%%%%%%%%%%%%5
\subsubsection* {Methods}
\begin{code}

   public static double nextDouble (RandomStream s, NormalGen sn,
                                    double mu, double lambda) \begin{hide} {
      return mhs(s, sn, mu, lambda);
   }\end{hide}
\end{code}
\begin{tabb} Generates a new variate from the \emph{inverse gaussian}
distribution with parameters $\mu = $ \texttt{mu} and $\lambda= $
\texttt{lambda}, using streams \texttt{s} and \texttt{sn}.
\end{tabb}
\begin{code}\begin{hide} 

   public double nextDouble() {
      return mhs(stream, genN, mu, lambda);
   }


//>>>>>>>>>>>>>>>>>>>>  P R I V A T E     M E T H O D S   <<<<<<<<<<<<<<<<<<<<

   private static double mhs (RandomStream stream, NormalGen genN,
                              double mu, double lambda) {
      if (lambda <= 0.0)
         throw new IllegalArgumentException ("lambda <= 0");
      if (mu <= 0.0)
         throw new IllegalArgumentException ("mu <= 0");

      double z = genN.nextDouble ();
      double ymu = mu*z*z;
      double x1 = mu + 0.5*mu*ymu/lambda - 0.5*mu/lambda *
                  Math.sqrt(4.0*ymu*lambda + ymu*ymu);
      double u = stream.nextDouble();
      if (u <= mu/(mu + x1))
         return x1;
      return mu*mu/x1;
   }


   protected void setParams (double mu, double lambda, NormalGen sn) {
      setParams (mu, lambda);
      this.genN = sn;
   }

}\end{hide}
\end{code}
