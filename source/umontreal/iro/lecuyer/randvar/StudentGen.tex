\defclass {StudentGen}

This class implements methods for generating random variates from the
{\em Student\/} distribution with $n>0$ degrees of freedom.
Its density function is
\begin{htmlonly}
 \eq
        f (x) = [\Gamma((n + 1)/2 )/
                        (\Gamma (n/2) \sqrt{\pi n})]
            [1 + x^2/n]^{-(n+1)/2}
            \qquad \qquad \mbox {for } -\infty < x < \infty,
   \endeq
\end{htmlonly}
\begin{latexonly}
 \eq
        f (x) = \frac{\Gamma\left ((n + 1)/2 \right)}
                        {\Gamma (n/2) \sqrt{\pi n}}
            \left[1 + \frac{x^2}{n}\right]^{-(n+1)/2}
            \qquad \qquad \mbox {for } -\infty < x < \infty,    \eqlabel{eq:fstudent}
   \endeq
\end{latexonly}
where $\Gamma (x)$ is the gamma function defined in
\latex{(\ref{eq:Gamma})}\html{\class{GammaGen}}.

The \texttt{nextDouble} method simply calls \texttt{inverseF} on the
distribution.

The following table gives the CPU time needed to generate $10^7$ Student
 random variates using the different implementations available in SSJ.
 The second test ({Q}) was made with the inverse in
  \class{StudentDistQuick},
 while the first test was made with the inverse in \class{StudentDist}.
These tests were made on a machine with processor AMD Athlon 4000, running
Red Hat Linux, with clock speed at 2400 MHz.

\begin{center}
\begin{tabular}{|l|c|}
\hline
 Generator  &  time in seconds  \\
\hline
\texttt{StudentGen}       &   22.4  \\
\texttt{StudentGen(Q)}   &  \phantom{2}6.5   \\
\texttt{StudentPolarGen}  &   \phantom{2}1.4 \\
\hline
\end{tabular}
\end{center}

\bigskip\hrule

%%%%%%%%%%%%%%%%%%%%%%%%%%%%%%%%%%%%%%%%%%%
\begin{code}
\begin{hide}
/*
 * Class:        StudentGen
 * Description:  Student-t random variate generators 
 * Environment:  Java
 * Software:     SSJ 
 * Copyright (C) 2001  Pierre L'Ecuyer and Université de Montréal
 * Organization: DIRO, Université de Montréal
 * @author       
 * @since

 * SSJ is free software: you can redistribute it and/or modify it under
 * the terms of the GNU General Public License (GPL) as published by the
 * Free Software Foundation, either version 3 of the License, or
 * any later version.

 * SSJ is distributed in the hope that it will be useful,
 * but WITHOUT ANY WARRANTY; without even the implied warranty of
 * MERCHANTABILITY or FITNESS FOR A PARTICULAR PURPOSE.  See the
 * GNU General Public License for more details.

 * A copy of the GNU General Public License is available at
   <a href="http://www.gnu.org/licenses">GPL licence site</a>.
 */
\end{hide}
package umontreal.iro.lecuyer.randvar;\begin{hide}
import umontreal.iro.lecuyer.rng.*;
import umontreal.iro.lecuyer.probdist.*;
\end{hide}

public class StudentGen extends RandomVariateGen \begin{hide} {
   protected int n = -1;
\end{hide}\end{code}

%%%%%%%%%%%%%%%%%%%%%%%%%%%%%%%%%%%%%%%%%%%%
\subsubsection* {Constructors}
\begin{code}

   public StudentGen (RandomStream s, int n) \begin{hide} {
      super (s, new StudentDist(n));
      setN (n);
   }\end{hide}
\end{code}
\begin{tabb}  Creates a Student random variate generator with
 $n$ degrees of freedom, using stream \texttt{s}.
\end{tabb}
\begin{code}

   public StudentGen (RandomStream s, StudentDist dist) \begin{hide} {
      super (s, dist);
      if (dist != null)
         setN (dist.getN ());
   }\end{hide}
\end{code}
 \begin{tabb} Creates a new generator for the Student distribution \texttt{dist}
     and stream \texttt{s}.
 \end{tabb}

%%%%%%%%%%%%%%%%%%%%%%%%%%%%%%%%%%%%%%%%%%%%%%%%5
\subsubsection* {Methods}
\begin{code}

   public static double nextDouble (RandomStream s, int n) \begin{hide} {
      return StudentDist.inverseF (n, s.nextDouble());
    }\end{hide}
\end{code}
 \begin{tabb}  Generates a new variate from the Student distribution
   with $n = $~\texttt{n} degrees of freedom, using stream \texttt{s}.
 \end{tabb}
\begin{code}

   public int getN()\begin{hide} {
      return n;
   }
\end{hide}
\end{code}
\begin{tabb}
  Returns the value of $n$ for this object.
\end{tabb}
\begin{hide}\begin{code}

   protected void setN (int nu) {
      if (nu <= 0)
         throw new IllegalArgumentException ("n <= 0");
      this.n = nu;
   }
}
\end{code}
\end{hide}
