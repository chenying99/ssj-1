\defclass {StudentPolarGen}

This class implements {\em Student\/} random variate generators using
 the {\em polar\/} method of \cite{rBAI94a}.
The code is adapted from UNURAN\latex{ (see \cite{iLEY02a})}.

The non-static \texttt{nextDouble} method generates two variates at a time
and the second one is saved for the next call.  
A pair of variates is generated every second call. 
In the static case, two variates are generated per
call but only the first one is returned and the second is discarded.

\bigskip\hrule

%%%%%%%%%%%%%%%%%%%%%%%%%%%%%%%%%%%%%%%%%%%
\begin{code}
\begin{hide}
/*
 * Class:        StudentPolarGen
 * Description:  Student-t random variate generators using the polar method
 * Environment:  Java
 * Software:     SSJ 
 * Copyright (C) 2001  Pierre L'Ecuyer and Université de Montréal
 * Organization: DIRO, Université de Montréal
 * @author       
 * @since

 * SSJ is free software: you can redistribute it and/or modify it under
 * the terms of the GNU General Public License (GPL) as published by the
 * Free Software Foundation, either version 3 of the License, or
 * any later version.

 * SSJ is distributed in the hope that it will be useful,
 * but WITHOUT ANY WARRANTY; without even the implied warranty of
 * MERCHANTABILITY or FITNESS FOR A PARTICULAR PURPOSE.  See the
 * GNU General Public License for more details.

 * A copy of the GNU General Public License is available at
   <a href="http://www.gnu.org/licenses">GPL licence site</a>.
 */
\end{hide}
package umontreal.iro.lecuyer.randvar;\begin{hide}
import umontreal.iro.lecuyer.rng.*;
import umontreal.iro.lecuyer.probdist.*;
\end{hide}

public class StudentPolarGen extends StudentGen \begin{hide} {

    private boolean available = false;
    private double[] variates = new double[2];
    private static double[] staticVariates = new double[2];
    // Used by the polar method.
\end{hide}\end{code}

%%%%%%%%%%%%%%%%%%%%%%%%%%%%%%%%%%%%%%%%%%%%
\subsubsection* {Constructors}
\begin{code}

   public StudentPolarGen (RandomStream s, int n) \begin{hide} {
      super (s, null);
      setN (n);
   }\end{hide}
\end{code} 
\begin{tabb}  Creates a Student random variate generator with $n$
  degrees of freedom, using stream \texttt{s}. 
\end{tabb}
\begin{code}
 
   public StudentPolarGen (RandomStream s, StudentDist dist) \begin{hide} {
      super (s, dist);
      if (dist != null)
         setN (dist.getN ());
   }\end{hide}
\end{code}
\begin{tabb}  Creates a new generator for the Student distribution \texttt{dist}
     and stream \texttt{s}. 
\end{tabb}

%%%%%%%%%%%%%%%%%%%%%%%%%%%%%%%%%%%%%%%%%%%%%%%%5
%%\subsubsection* {Methods}
\begin{code}\begin{hide} 

   public double nextDouble() {
       if (available) {
          available = false;
          return variates[1];
       }
       else {
          polar (stream, n, variates);
          available = true;
          return variates[0];
       }
   }

   public static double nextDouble (RandomStream s, int n) {
      polar (s, n, staticVariates);
      return staticVariates[0];
   }
\end{code}
\begin{tabb}  Generates a new variate from the Student distribution
   with $n = $~\texttt{n} degrees of freedom, using stream \texttt{s}.
\end{tabb}
\begin{code}


//>>>>>>>>>>>>>>>>>>>>  P R I V A T E S    M E T H O D S   <<<<<<<<<<<<<<<<<<<<


/*****************************************************************************
 *                                                                           *
 * Student's t Distribution: Polar Method                                    *
 *                                                                           *
 *****************************************************************************
 *                                                                           *
 * FUNCTION:   - samples a random number from Student's t distribution with  *
 *               parameters n > 0.                                          *
 *                                                                           *
 * REFERENCE:  - R.W. Bailey (1994): Polar generation of random variates     *
 *               with the t-distribution,                                    *
 *               Mathematics of Computation 62, 779-781.                     *
 *                                                                           *
 * Implemented by F. Niederl, 1994                                           *
 *****************************************************************************
 *                                                                           *
 * The polar method of Box/Muller for generating Normal variates is adapted  *
 * to the Student-t distribution. The two generated variates are not         *
 * independent and the expected no. of uniforms per variate is 2.5464.       *
 *                                                                           *
 *****************************************************************************
 *    WinRand (c) 1995 Ernst Stadlober, Institut fuer Statistitk, TU Graz    *
 *****************************************************************************
 * UNURAN (c) 2000  W. Hoermann & J. Leydold, Institut f. Statistik, WU Wien *
 ****************************************************************************/

   private static void polar (RandomStream stream, int n, double[] variates) {
      double u,v,w;
      do {
         u = 2. * stream.nextDouble() - 1.;
         v = 2. * stream.nextDouble() - 1.;
         w = u*u + v*v;
       } while (w > 1.);

      double temp = Math.sqrt (n*(Math.exp (-2./n*Math.log (w)) - 1.)/w);
      variates[0] = u*temp;
      variates[1] = v*temp;
   }


}\end{hide}
\end{code}
