\defmodule {F2NL607}

Implements the \class{RandomStream} interface by using as a backbone
generator the combination of the WELL607 proposed in \cite{rPAN04t,rPAN06b}
(and implemented in \class{WELL607}) with a nonlinear generator.
This nonlinear generator is made up of a small number of components (say $n$)
combined via addition modulo 1. Each component contains an array already
filled with a ``random'' permutation of $\{0,...,s-1\}$ where \texttt{s} is the
size of the array.
These numbers and the lengths of the components can be changed by the user.
Each call to the generator uses the next number in each array
(or the first one if we are at the end of the array). By default,
there are 3 components of lengths 1019, 1021, and 1031, respectively.
The non-linear generator is combined with the WELL using a bitwise XOR operation.
This ensures that the new generator has at least as much equidistribution
as the WELL607, as shown in \cite{rLEC03c}.

The combined generator has a period length of \html{approximatively}
\latex{$\rho\approx$ } $2^{637}$. The values of $V$, $W$ and $Z$
 are $2^{250}$,
$2^{150}$, and $2^{400}$, respectively (see \class{RandomStream} for their
definition). The seed of the RNG has two part: the linear part is a
19-dimensional vector of 32-bit integers, while the nonlinear part is
made up of a $n$-dimensional vector of indices, representing the position
of the generator in each array of the nonlinear components.
%The default initial seed of the linear part is $(0xD6AFB71C,$ $ 0x82ADB18E,$ $
%0x326E714E,$ $ 0xB1EE42B6,$ $ 0xF1A834ED,$ $ 0x04AE5721,$ $ 0xC5EA2843,$ $
%0xFA04116B,$ $ 0x6ACE14EF,$ $ 0xCD5781A0,$ $ 0x6B1F731C,$ $ 0x7E3B8E3D,$ $
%0x8B34DE2A,$ $ 0x74EC15F5,$ $ 0x84EBC216,$ $ 0x83EA2C61,$ $ 0xE4A83B1E,$ $
%0xA5D82CB9,$ $ 0x9E1A6C89)$,
%while the default initial seed for the non-linear part is $(0, 0, 0)$.


%%%%%%%%%%%%%%%%%%%%%%%%%%%%%%%%%%%%%%%%%%%%%%%%%%%%
\bigskip\hrule

\begin{code}
\begin{hide}
/*
 * Class:        F2NL607
 * Description:  generator: combination of the WELL607 proposed by
                 Panneton with a nonlinear generator.
 * Environment:  Java
 * Software:     SSJ 
 * Copyright (C) 2001  Pierre L'Ecuyer and Université de Montréal
 * Organization: DIRO, Université de Montréal
 * @author       
 * @since

 * SSJ is free software: you can redistribute it and/or modify it under
 * the terms of the GNU General Public License (GPL) as published by the
 * Free Software Foundation, either version 3 of the License, or
 * any later version.

 * SSJ is distributed in the hope that it will be useful,
 * but WITHOUT ANY WARRANTY; without even the implied warranty of
 * MERCHANTABILITY or FITNESS FOR A PARTICULAR PURPOSE.  See the
 * GNU General Public License for more details.

 * A copy of the GNU General Public License is available at
   <a href="http://www.gnu.org/licenses">GPL licence site</a>.
 */
\end{hide}
package umontreal.iro.lecuyer.rng; \begin{hide}

import umontreal.iro.lecuyer.util.ArithmeticMod;
import umontreal.iro.lecuyer.util.PrintfFormat;

import java.util.Random;
\end{hide}

public class F2NL607 \begin{hide} extends WELL607base {

   private static final long serialVersionUID = 70510L;
   //La date de modification a l'envers, lire 10/05/2007

   //stream variable for the WELL607 part
   private static int[] curr_stream = {0xD6AFB71C, 0x82ADB18E, 0x326E714E,
                                       0xB1EE42B6, 0xF1A834ED, 0x04AE5721,
                                       0xC5EA2843, 0xFA04116B, 0x6ACE14EF,
                                       0xCD5781A0, 0x6B1F731C, 0x7E3B8E3D,
                                       0x8B34DE2A, 0x74EC15F5, 0x84EBC216,
                                       0x83EA2C61, 0xE4A83B1E, 0xA5D82CB9,
                                       0x9E1A6C89};

   //data for the non-linear part
   private static int[][] nlData;
   private static int[] nlJumpW, nlJumpZ;
   private int[] nlState;

   //stream and substream variables for the non-linear part
   private int[] nlStream;
   private int[] nlSubstream;
   private static int[] curr_nlStream;

   //set to true when a instance of F2NL607 is constructed
   private static boolean constructed = false;

   //initialisation of the generator
   static
   {
      curr_nlStream = new int[]{ 0, 0, 0};

      nlData = new int[][]{ new int[1019],
                            new int[1021],
                            new int[1031]};

      Random rand = new Random(0);

      for(int i = 0; i < nlData.length; i++) {
         for(int j = 0; j < nlData[i].length; j++)
            nlData[i][j] = (int)((((long)j) << 32) / nlData[i].length);
         scramble(nlData[i], rand);

         /*
         for(int j = 0; j < nlData[i].length; j++)
            if(nlData[i][j] >= 0)
               System.out.print(nlData[i][j] + "U, ");
            else
               System.out.print(((long)nlData[i][j] + 0x100000000L) + "U, ");
         System.out.println(PrintfFormat.NEWLINE +
            PrintfFormat.NEWLINE + PrintfFormat.NEWLINE);
         */
      }

      //computing the jumps length
      //A FAIRE EN PRE-CALCUL
      nlJumpW = new int[3];
      nlJumpZ = new int[3];
      for(int i = 0; i < nlJumpW.length; i++) {
         int temp = 1;
         for(int j = 0; j < w; j++)
            temp = (2 * temp) % nlData[i].length;
         nlJumpW[i] = temp;
         for(int j = 0; j < v; j++)
            temp = (2 * temp) % nlData[i].length;
         nlJumpZ[i] = temp;
      }
   }

   private static void scramble(int[] data, Random rand) {
      int buffer;
      int j;

      for(int i = 0; i < data.length - 1; i++) {
         //choose a random number between i and data.length - 1
         j = (int)(rand.nextDouble() * (data.length - i)) + i;
         //exchange data[i] and data[j]
         buffer = data[i];
         data[i] = data[j];
         data[j] = buffer;
      }
   }

   private static void scramble(int[] data, RandomStream rand) {
      int buffer;
      int j;

      for(int i = 0; i < data.length - 1; i++) {
         //choose a random number between i and data.length - 1
         j = rand.nextInt(i, data.length - 1);
         //exchange data[i] and data[j]
         buffer = data[i];
         data[i] = data[j];
         data[j] = buffer;
      }
   }
 \end{hide}
\end{code}

%%%%%%%%%%%%%%%%%%%%%%%%%%%%%%%%%%%%%%%%%%%%%%%%%%%%
\subsubsection* {Constructors}

\begin{code}
   public F2NL607() \begin{hide} {
      //linear part
      initialisation();

      constructed = true;

      state = new int[BUFFER_SIZE];
      stream = new int[R];
      substream = new int[R];

      for(int i = 0; i < R; i++)
         stream[i] = curr_stream[i];

//    advanceSeed(curr_stream, Apz);
      advanceSeed(curr_stream, WELL607.pz);

      //non-linear part
      nlState = new int[nlData.length];
      nlStream = new int[nlData.length];
      nlSubstream = new int[nlData.length];

      for(int i = 0; i < nlData.length; i++) {
         nlStream[i] = curr_nlStream[i];
         curr_nlStream[i] += nlJumpZ[i];
      }

      resetStartStream();
   } \end{hide}
\end{code}
\begin{tabb} Constructs a new stream, initializing it at its beginning.
  Also makes sure that the seed of the next constructed stream is
  $Z$ steps away. Sets its antithetic switch to \texttt{false} and sets
  the stream to normal precision mode (offers 32 bits of precision).
\end{tabb}
\begin{code}

   public F2NL607 (String name) \begin{hide} {
      this();
      this.name = name;
   }\end{hide}
\end{code}
\begin{tabb} Constructs a new stream with the identifier \texttt{name}
  (used in the \texttt{toString} method).
\end{tabb}
\begin{htmlonly}
  \param{name}{name of the stream}
\end{htmlonly}
\begin{code}
  \begin{hide}
   public void resetStartStream() {
      for(int i = 0; i < R; i++)
         substream[i] = stream[i];
      for(int i = 0; i < nlSubstream.length; i++)
         nlSubstream[i] = nlStream[i];

      resetStartSubstream();
   }

   public void resetStartSubstream() {
      state_i = 0;
      for(int i = 0; i < R; i++)
         state[i] = substream[i];
      for(int i = 0; i < nlState.length; i++)
         nlState[i] = nlSubstream[i];
   }

   public void resetNextSubstream() {
//      advanceSeed(substream, WELL607.Apw);
      advanceSeed(substream, WELL607.pw);
      for(int i = 0; i < nlState.length; i++)
         nlSubstream[i] = (nlSubstream[i] + nlJumpW[i]) %
                          nlData[i].length;

      resetStartSubstream();
   }\end{hide}
\end{code}

%%%%%%%%%%%%%%%%%%%%%%%%%%%%%%%%%%%%%%%%%
\subsubsection* {Methods}
\begin{code}
   public static void setPackageLinearSeed (int seed[]) \begin{hide} {
      verifySeed(seed);

      for(int i = 0; i < R; i++)
         curr_stream[i] = seed[i];
   } \end{hide}
\end{code}
\begin{tabb} Sets the initial seed of the linear part of the class
  \texttt{F2NL607} to the 19 integers of the vector \texttt{seed[0..18]}.
  This will be the initial seed (or seed) of the next created stream.
  At least one of the integers must be non-zero and if this integer is
  the last one, it must not be equal to \texttt{0x7FFFFFFF}.
\end{tabb}
\begin{htmlonly}
  \param{seed}{array of 19 elements representing the seed}
\end{htmlonly}
\begin{code}

   public void setLinearSeed (int seed[]) \begin{hide} {
      verifySeed(seed);

      for(int i = 0; i < R; i++)
         stream[i] = seed[i];

      resetStartStream();
   } \end{hide}
\end{code}
\begin{tabb} This method is discouraged for normal use.
  Initializes the stream at the beginning of a stream with the initial linear
  seed \texttt{seed[0..18]}. The seed must satisfy the same conditions as
  in \texttt{setPackageSeed}.
  The non-linear seed is not modified; thus the non-linear part of the random
  number generator is reset to the beginning of the old stream.
  This method only affects the specified stream; the others are not
  modified. Hence after calling this method, the beginning of the streams
  will no longer be spaced $Z$ values apart.
  For this reason, this method should only be used in very exceptional cases;
  proper use of the \texttt{reset...} methods and of the stream constructor is
  preferable.
\end{tabb}
\begin{htmlonly}
  \param{seed}{array of 19 elements representing the seed}
\end{htmlonly}
\begin{code}

   public int[] getLinearState() \begin{hide} {
      return getState();
   } \end{hide}
\end{code}
\begin{tabb} Returns the current state of the linear part of the stream,
  represented as an array of 19 integers.
\end{tabb}
\begin{htmlonly}
  \return{the current state of the stream}
\end{htmlonly}
\begin{code}

   public static void setPackageNonLinearSeed (int seed[]) \begin{hide} {
      if (seed.length < nlData.length)
         throw new IllegalArgumentException("Seed must contain " +
                                            nlData.length + " values");
      for (int i = 0; i < nlData.length; i++)
         if (seed[i] < 0 || seed[i] >= nlData[i].length)
            throw new IllegalArgumentException("Seed number " + i +
                                               " must be between 0 and " +
                                               (nlData[i].length - 1));

      for(int i = 0; i < nlData.length; i++)
         curr_nlStream[i] = seed[i];
   } \end{hide}
\end{code}
\begin{tabb} Sets the non-linear part of the initial seed of the class
  \texttt{F2NL607} to the $n$ integers of the vector \texttt{seed[0..n-1]},
  where $n$ is the number of components of the non-linear part. The
  default is $n = 3$.
  Each of the integers must be between 0 and the length of the corresponding
  component minus one. By default, the lengths are $(1019, 1021, 1031)$.
\end{tabb}
\begin{htmlonly}
  \param{seed}{array of $n$ elements representing the non-linear seed}
\end{htmlonly}
\begin{code}

   public void setNonLinearSeed (int seed[]) \begin{hide} {
      if(seed.length != nlData.length)
         throw new IllegalArgumentException("Seed must contain " +
                                            nlData.length + " values");
      for(int i = 0; i < nlData.length; i++)
         if(seed[i] < 0 || seed[i] >= nlData[i].length)
            throw new IllegalArgumentException("Seed number " + i +
                                               " must be between 0 and " +
                                               (nlData[i].length - 1));

      for(int i = 0; i < nlData.length; i++)
         nlStream[i] = seed[i];

      resetStartStream();
   } \end{hide}
\end{code}
\begin{tabb} This method is discouraged for normal use.
  Initializes the stream at the beginning of a stream with the initial
  non-linear seed \texttt{seed[0..n-1]}, where $n$ is the number of components
  of the non-linear part of the generator.
  The linear seed is not modified so the linear part of the random number
  generator is reset to the beginning of the old stream.
  This method only affects the specified stream; the others are not
  modified. Hence after calling this method, the beginning of the streams
  will no longer be spaced $Z$ values apart.
  For this reason, this method should only be used in very exceptional cases;
  proper use of the \texttt{reset...} methods and of the stream constructor is
  preferable.
\end{tabb}
\begin{htmlonly}
  \param{seed}{}
\end{htmlonly}
\begin{code}

   public int[] getNonLinearState() \begin{hide} {
      //we must prevent the user to change the state, so we give him a copy
      int[] state = new int[nlState.length];
      for(int i = 0; i < nlState.length; i++)
         state[i] = nlState[i];
      return state;
   } \end{hide}
\end{code}
\begin{tabb} Returns the current state of the non-linear part of the stream,
  represented as an array of $n$ integers, where $n$ is the number
  of components in the non-linear generator.
\end{tabb}
\begin{htmlonly}
  \return{the current state of the stream}
\end{htmlonly}
\begin{code}

   public static int[][] getNonLinearData() \begin{hide} {
      //we must prevent the user to change the data, so we give him a copy
      int[][] data = new int[nlData.length][];
      for(int i = 0; i < nlData.length; i++) {
         data[i] = new int[nlData[i].length];
         for(int j = 0; j < nlData[i].length; j++)
            data[i][j] = nlData[i][j];
      }
      return data;
   } \end{hide}
\end{code}
\begin{tabb} Return the data of all the components of the non-linear
  part of the random number generator. This data is explained in the
  introduction.
\end{tabb}
\begin{htmlonly}
  \return{the data of the components}
\end{htmlonly}
\begin{code}

   public static void setNonLinearData (int[][] data) \begin{hide} {
      if(constructed)
         throw new IllegalStateException("setNonLinearData can only be " +
                                         "called before the creation of " +
                                         "any F2NL607");

      nlData = new int[data.length][];
      curr_nlStream = new int[data.length];
      for(int i = 0; i < data.length; i++) {
         nlData[i] = new int[data[i].length];
         for(int j = 0; j < data[i].length; j++)
            nlData[i][j] = data[i][j];
         curr_nlStream[i] = 0;
      }

      //recomputing the jumps length (an inefficient method is used)
      nlJumpW = new int[data.length];
      nlJumpZ = new int[data.length];
      for(int i = 0; i < nlJumpW.length; i++) {
         int temp = 1;
         for(int j = 0; j < w; j++)
            temp = (2 * temp) % nlData[i].length;
         nlJumpW[i] = temp;
         for(int j = 0; j < v; j++)
            temp = (2 * temp) % nlData[i].length;
         nlJumpZ[i] = temp;
      }
   } \end{hide}
\end{code}
\begin{tabb} Selects new data for the components of the non-linear generator.
  The number of arrays in \texttt{data} will decide the number of components.
  Each of the arrays will be assigned to one of the components. The period
  of the resulting non-linear generator will be equal to the lowest common
  multiple of the lengths of the arrays. It is thus recommended to choose
  only prime length for the best results.

  NOTE : This method cannot be called if at least one instance of
  \texttt{F2NL607} has been constructed. In that case, it will throw
  an \class{IllegalStateException}.
\end{tabb}
\begin{htmlonly}
  \param{data}{the new data of the components}
  \exception{IllegalStateException}{if an instance of the class was
    constructed before}
\end{htmlonly}
\begin{code}

   public static void setScrambleData (RandomStream rand, int steps,
                                       int[] size) \begin{hide} {
      if (constructed)
         throw new IllegalStateException("setScrambleData can only be " +
                                         "called before the creation of " +
                                         "any F2NL607");

      curr_nlStream = new int[size.length];
      for(int i = 0; i < size.length; i++)
         curr_nlStream[i] = 0;

      nlData = new int[size.length][];
      for(int i = 0; i < size.length; i++)
         nlData[i] = new int[size[i]];

      for(int i = 0; i < nlData.length; i++) {
         for(int j = 0; j < nlData[i].length; j++)
            nlData[i][j] = (int)((((long)j) << 32) / nlData[i].length);
         for(int j = 0; j < steps; j++)
            scramble(nlData[i], rand);
      }

      //computing the jumps length
      //inefficient algorithm alert!
      nlJumpW = new int[size.length];
      nlJumpZ = new int[size.length];
      for(int i = 0; i < nlJumpW.length; i++) {
         int temp = 1;
         for(int j = 0; j < w; j++)
            temp = (2 * temp) % nlData[i].length;
         nlJumpW[i] = temp;
         for(int j = 0; j < v; j++)
            temp = (2 * temp) % nlData[i].length;
         nlJumpZ[i] = temp;
      }
   } \end{hide}
\end{code}
\begin{tabb} Selects new data for the components of the non-linear generator.
  The number of arrays in \texttt{data} will decide the number of components.
  Each of the arrays will be assigned to one of the components. The period
  of the resulting non-linear generator will be equal to the lowest common
  multiple of the lengths of the arrays. It is thus recommended to choose
  only prime length for the best results.

  NOTE : This method cannot be called if at least one instance of
  \texttt{F2NL607} has been constructed. In that case, it will throw
  an \class{IllegalStateException}.
\end{tabb}
\begin{htmlonly}
  \param{rand}{the random numbers source to do the scrambling}
  \param{steps}{number of time to do the scrambling}
  \param{size}{size of each components}
  \exception{IllegalStateException}{if an instance of the class was
    constructed before}
\end{htmlonly}
\begin{code}

   public F2NL607 clone() \begin{hide} {
      F2NL607 retour = null;
      retour = (F2NL607)super.clone();
      retour.state = new int[BUFFER_SIZE];
      retour.substream = new int[R];
      retour.stream = new int[R];
      retour.nlState = new int[nlData.length];
      retour.nlStream = new int[nlData.length];
      retour.nlSubstream = new int[nlData.length];

      for (int i = 0; i<R; i++) {
         retour.substream[i] = substream[i];
         retour.stream[i] = stream[i];
      }
      for (int i = 0; i<BUFFER_SIZE; i++) {
         retour.state[i] = state[i];
      }
      for (int i = 0; i<nlData.length; i++) {
         retour.nlState[i] = nlState[i];
         retour.nlStream[i] = nlStream[i];
         retour.nlSubstream[i] = nlSubstream[i];
      }
      return retour;
   }\end{hide}
\end{code}
 \begin{tabb} Clones the current generator and return its copy.
 \end{tabb}
 \begin{htmlonly}
   \return{A deep copy of the current generator}
\end{htmlonly}
\begin{code}
\begin{hide}
   public String toString() {
      StringBuffer sb = new StringBuffer();

      if(name == null)
         sb.append("The state of this F2NL607 is : " + PrintfFormat.NEWLINE);
      else
         sb.append("The state of " + name + " is : " + PrintfFormat.NEWLINE);
      sb.append(" Linear part : { ");
      sb.append(super.stringState());
      sb.append(PrintfFormat.NEWLINE + " Non-linear part : { ");
      for(int i = 0; i < nlState.length; i++)
         sb.append(nlState[i] + " ");
      sb.append("}");

      return sb.toString();
   }



   protected double nextValue() {
      for (int i = 0; i < nlData.length; i++)
         if (nlState[i] >= nlData[i].length - 1)
            nlState[i] = 0;
         else
            nlState[i]++;

      int nonLin = 0;
      for (int i = 0; i < nlData.length; i++)
         nonLin += nlData[i][nlState[i]];

      long result = (nextInt() ^ nonLin);
      if(result <= 0)
         result += 0x100000000L;
      return result * NORM;
   }
}
\end{hide}
\end{code}
