\defclass{F2wNetPolyLCG}


This class implements a digital net in base 2 starting from a
polynomial LCG in $\latex{\mathbb{F}}\html{\mathbf{F}}_{2^w}[z]/P(z)$.
 It is exactly the same
point set as the one defined in the class 
 \externalclass{umontreal.iro.lecuyer.hups}{F2wCycleBasedPolyLCG}.
 See the description
of this class for more details on the way the point set is constructed.

Constructing a point set using this class, instead of using
 \externalclass{umontreal.iro.lecuyer.hups}{F2wCycleBasedPolyLCG},
makes SSJ construct a digital net in base 2.  This is useful if one
wants to randomize using one of the randomizations included in the class
 \externalclass{umontreal.iro.lecuyer.hups}{DigitalNet}.

\textbf{Note: This class in not operational yet!}

\bigskip\hrule\bigskip

%%%%%%%%%%%%%%%%%%%%%%%%%%%%%%%%%%%%%%%%%%%%%%%%%%%%%%%%%%%%%%%%%%
\begin{code}
\begin{hide}
/*
 * Class:        F2wNetPolyLCG
 * Description:  digital nets in base 2 starting from a polynomial LCG 
 * Environment:  Java
 * Software:     SSJ 
 * Copyright (C) 2001  Pierre L'Ecuyer and Université de Montréal
 * Organization: DIRO, Université de Montréal
 * @author       
 * @since

 * SSJ is free software: you can redistribute it and/or modify it under
 * the terms of the GNU General Public License (GPL) as published by the
 * Free Software Foundation, either version 3 of the License, or
 * any later version.

 * SSJ is distributed in the hope that it will be useful,
 * but WITHOUT ANY WARRANTY; without even the implied warranty of
 * MERCHANTABILITY or FITNESS FOR A PARTICULAR PURPOSE.  See the
 * GNU General Public License for more details.

 * A copy of the GNU General Public License is available at
   <a href="http://www.gnu.org/licenses">GPL licence site</a>.
 */
\end{hide}
package umontreal.iro.lecuyer.hups; \begin{hide} 

import umontreal.iro.lecuyer.util.PrintfFormat;
\end{hide}

public class F2wNetPolyLCG extends DigitalNetBase2 \begin{hide} 
{
   private F2wStructure param;

    /**
     * Constructs and stores the set of cycles for an LCG with
     *    modulus <SPAN CLASS="MATH"><I>n</I></SPAN> and multiplier <SPAN CLASS="MATH"><I>a</I></SPAN>.
     *   If pgcd<SPAN CLASS="MATH">(<I>a</I>, <I>n</I>) = 1</SPAN>, this constructs a full-period LCG which has two
     *   cycles, one containing 0 and one, the LCG period.
     *
     * @param n required number of points and modulo of the LCG
     *
     *    @param a generator <SPAN CLASS="MATH"><I>a</I></SPAN> of the LCG
     *
     *
     */
\end{hide}
\end{code}

%%%%%%%%%%%%%%%%%%%%%%%%%%%%
\subsubsection*{Constructors}
\begin{code}

   public F2wNetPolyLCG (int type, int w, int r, int modQ, int step,
                         int nbcoeff, int coeff[], int nocoeff[], int dim) \begin{hide} 
   {
      param = new F2wStructure (w, r, modQ, step, nbcoeff, coeff, nocoeff);
      initNet (r, w, dim);
   }
\end{hide}
\end{code}
 \begin{tabb}
Constructs a point set with $2^{rw}$ points.  See the description of the class
\externalclass{umontreal.iro.lecuyer.hups}{F2wStructure} for the meaning of the 
 parameters.
 \end{tabb}
\begin{code}

   public F2wNetPolyLCG (String filename, int no, int dim) \begin{hide} 
   {
      param = new F2wStructure (filename, no);
      initNet (param.r, param.w, dim);
   }\end{hide}
\end{code}
 \begin{tabb}
   Constructs a point set after reading its parameters from
   file \texttt{filename}; the parameters are located at line numbered \texttt{no}
   of \texttt{filename}. The available files are listed in the description of class
\externalclass{umontreal.iro.lecuyer.hups}{F2wStructure}.
 \end{tabb}
\begin{code}
\begin{hide} 

   public String toString ()
   {
       String s = "F2wNetPolyLCG:" + PrintfFormat.NEWLINE;
       return s + param.toString ();
   }


   private void initNet (int r, int w, int dim)
   {
      normFactor = param.normFactor;
   }
}
\end{hide}
\end{code}
