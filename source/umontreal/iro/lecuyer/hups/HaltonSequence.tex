\defclass{HaltonSequence}

This class implements the sequence of Halton \cite{rHAL60a},
which is essentially a modification of Hammersley nets for producing 
an infinite sequence of points having low discrepancy.
The $i$th point in $s$ dimensions is 
\eq
 \bu_i = (\psi_{b_1}(i),\psi_{b_2}(i),\dots, \psi_{b_s}(i)),
                                            \eqlabel{eq:Halton-point2}
\endeq
for $i=0,1,2,\dots$, where $\psi_b$ is the radical inverse function
in base $b$, defined in class \class{RadicalInverse}, and where
$2 = b_1 < \cdots < b_s$ are the $s$ smallest prime numbers in 
increasing order.

A fast method is implemented to generate randomized Halton sequences\latex{
\cite{rSTR95a,vWAN99a}}, starting from an arbitrary point $x_0$.

The points can be ``scrambled'' by applying a permutation to the 
digits of $i$ before computing each coordinate\latex{
via (\ref{eq:Halton-point})}, in the same way as for the class
\class{HammersleyPointSet}, for all coordinates $j\ge 0$.

\bigskip\hrule\bigskip
%%%%%%%%%%%%%%%%%%%%%%%%%%%%%%%%%%%%%%%%%%%%%%%%%%%%%%%%%%%%%%%%%%

\begin{code}

\begin{hide}
/*
 * Class:        HaltonSequence
 * Description:  
 * Environment:  Java
 * Software:     SSJ 
 * Copyright (C) 2001  Pierre L'Ecuyer and Université de Montréal
 * Organization: DIRO, Université de Montréal
 * @author       
 * @since

 * SSJ is free software: you can redistribute it and/or modify it under
 * the terms of the GNU General Public License (GPL) as published by the
 * Free Software Foundation, either version 3 of the License, or
 * any later version.

 * SSJ is distributed in the hope that it will be useful,
 * but WITHOUT ANY WARRANTY; without even the implied warranty of
 * MERCHANTABILITY or FITNESS FOR A PARTICULAR PURPOSE.  See the
 * GNU General Public License for more details.

 * A copy of the GNU General Public License is available at
   <a href="http://www.gnu.org/licenses">GPL licence site</a>.
 */
\end{hide}
package umontreal.iro.lecuyer.hups;


public class HaltonSequence extends PointSet\begin{hide} { 
   private int[] base;           // Vector of prime bases.
   private int[][] permutation;  // Digits permutation, for each dimension.
   private boolean permuted;     // Permute digits?
   private RadicalInverse[] radinv; // Vector of RadicalInverse's.
   private int[] start;          // starting indices
   private final static int positiveBitMask = ~Integer.reverse(1);
\end{hide}
\end{code}

%%%%%%%%%%%%%%%%%%%%%%%%%%%%%%%%%
\subsubsection*{Constructor}

\begin{code}
   public HaltonSequence (int dim) \begin{hide} {
      if (dim < 1)
         throw new IllegalArgumentException
            ("Halton sequence must have positive dimension dim");
      this.dim  = dim;
      numPoints = Integer.MAX_VALUE;
      base = RadicalInverse.getPrimes (dim);
      start = new int[dim];
      java.util.Arrays.fill(start, 0);
   }\end{hide}
\end{code}
 \begin{tabb}
   Constructs a new Halton sequence %% of $n$ points
    in \texttt{dim} dimensions.
%%   The number of points is infinite.
 \end{tabb}
\begin{htmlonly}
   \param{dim}{dimension}
\end{htmlonly}


%%%%%%%%%%%%%%%%%%%%%%%%%%%%%%%%%
\subsubsection*{Methods}

\begin{code}

   public void setStart (double[] x0) \begin{hide} {
      for (int i = 0; i < dim; i++)
         start[i] = RadicalInverse.radicalInverseInteger(base[i], x0[i]);
   }\end{hide}
\end{code}
 \begin{tabb}
   Initializes the Halton sequence starting at point \texttt{x0}.
   For each coordinate $j$, the sequence starts at index $i_j$ such that 
   \texttt{x0[$j$]} is the radical inverse of $i_j$.
   The dimension of \texttt{x0} must be at least as large as the dimension
   of this object.
 \end{tabb}
\begin{htmlonly}
   \param{x0}{starting point of the Halton sequence}
\end{htmlonly}
\begin{code}

   public void init (double[] x0) \begin{hide} {
      radinv = new RadicalInverse[dim];
      for (int i = 0; i < dim; i++)
         radinv[i] = new RadicalInverse (base[i], x0[i]);
   }\end{hide}
\end{code}
 \begin{tabb}
   Initializes the Halton sequence starting at point \texttt{x0}.
   The dimension of \texttt{x0} must be at least as large as the dimension
   of this object.
%%   The number of points is infinite.
 \end{tabb}
\begin{htmlonly}
   \param{x0}{starting point of the Halton sequence}
\end{htmlonly}
\begin{code}

   public void addFaureLemieuxPermutations()\begin{hide} {
      permutation = new int[dim][];
      for (int i = 0; i < dim; i++) {
         permutation[i] = new int[base[i]];
         RadicalInverse.getFaureLemieuxPermutation (i, permutation[i]);
      }
      permuted = true;
   }
\end{hide}
\end{code}
 \begin{tabb}
Permutes the digits using permutations from \cite{vFAU09a} for all coordinates.
After the method is called, the coordinates $u_{i,j}$ are generated via
\[
  u_{i,j} = \sum_{r=0}^{k-1} \pi_j[a_r] b_j^{-r-1},
\]
 for $j=0,\dots,s-1$,
 where $\pi_j$ is the Faure-Lemieux (2008) permutation of $\{0,\dots,b_j-1\}$.
 \end{tabb}
\begin{code}

   public void addFaurePermutations()\begin{hide} {
      permutation = new int[dim][];
      for (int i = 0; i < dim; i++) {
         permutation[i] = new int[base[i]];
         RadicalInverse.getFaurePermutation (base[i], permutation[i]);
      }
      permuted = true;
   }
\end{hide}
\end{code}
 \begin{tabb}
  Permutes the digits using Faure permutations for all coordinates.
  After the method is called, the coordinates $u_{i,j}$ are generated via
\[
  u_{i,j} = \sum_{r=0}^{k-1} \pi_j[a_r] b_j^{-r-1},
\]
 for $j=0,\dots,s-1$,
 where $\pi_j$ is the Faure permutation of $\{0,\dots,b_j-1\}$.
 \end{tabb}
\begin{code}

   public void ErasePermutations()\begin{hide} {
      permuted = false;
      permutation = null;
   }
\end{hide}
\end{code}
 \begin{tabb}
  Erases the permutations: from now on, the digits will not be
  permuted.
 \end{tabb}
\begin{code}
\begin{hide}
    
   public int getNumPoints () {
      return Integer.MAX_VALUE;
   }

   public double getCoordinate (int i, int j) {
      if (radinv != null) {
         if (!permuted) {
            return radinv[j].nextRadicalInverse ();
         } else {
            throw new UnsupportedOperationException (
            "Fast radical inverse is not implemented in case of permutation");
         }
      } else {
         int k = start[j] + i;
         // if overflow, restart at first nonzero point
         // (Struckmeier restarts at zero)
         if (k < 0)
            k = (k & positiveBitMask) + 1;
         if (permuted)
            return RadicalInverse.permutedRadicalInverse 
            (base[j], permutation[j], k);
         else 
            return RadicalInverse.radicalInverse (base[j], k);
      }
   }
}
\end{hide}
\end{code}
