\defclass {KernelDensity}

This class provides methods to compute a kernel density estimator from a set
of $n$ individual observations $x_0, \ldots, x_{n-1}$, and returns its value
at $m$ selected points. For details on how the kernel density is defined,
and how to select the kernel and the bandwidth $h$,
see the documentation of class 
\externalclass{umontreal.iro.lecuyer.randvar}{KernelDensityGen}
in package \texttt{randvar}.

%%%%%%%%%%%%%%%%%%%%%%%%%%%%%%%%%%%%%%%%%
\bigskip\hrule

\begin{code}
\begin{hide}
/*
 * Class:        KernelDensity
 * Description:  Kernel density estimators
 * Environment:  Java
 * Software:     SSJ 
 * Copyright (C) 2001  Pierre L'Ecuyer and Université de Montréal
 * Organization: DIRO, Université de Montréal
 * @author       
 * @since

 * SSJ is free software: you can redistribute it and/or modify it under
 * the terms of the GNU General Public License (GPL) as published by the
 * Free Software Foundation, either version 3 of the License, or
 * any later version.

 * SSJ is distributed in the hope that it will be useful,
 * but WITHOUT ANY WARRANTY; without even the implied warranty of
 * MERCHANTABILITY or FITNESS FOR A PARTICULAR PURPOSE.  See the
 * GNU General Public License for more details.

 * A copy of the GNU General Public License is available at
   <a href="http://www.gnu.org/licenses">GPL licence site</a>.
 */
\end{hide}
package umontreal.iro.lecuyer.gof;
   import umontreal.iro.lecuyer.probdist.*;
\begin{hide}
import umontreal.iro.lecuyer.randvar.KernelDensityGen;
\end{hide}

public class KernelDensity \begin{hide} {

   private static double estimate (EmpiricalDist dist,
                                   ContinuousDistribution kern,
                                   double h, double y) {
      // Computes and returns the kernel density estimate at $y$, where the 
      // kernel is the density kern.density(x), and the bandwidth is $h$.
      double z;
      double a = kern.getXinf();       // lower limit of density
      double b = kern.getXsup();       // upper limit of density
      double sum = 0;
      int n = dist.getN();
      for (int i = 0; i < n; i++) {
         z = (y - dist.getObs(i))/h;
         if ((z >= a) && (z <= b))
            sum += kern.density(z);
      }

      sum /= (h*n);
      return sum;
   }\end{hide}
\end{code}
%%%%%%%%%%%%%%%%%%%%%%%%%%%%%%%%%%%%%%%%%


%%%%%%%%%%%%%%%%%%%%%%%%%%%%%%%%%%%
\subsubsection* {Methods}

\begin{code}

   public static double[] computeDensity (EmpiricalDist dist,
                                          ContinuousDistribution kern,
                                          double h, double[] Y)\begin{hide} {
      int m = Y.length;
      double[] u = new double[m];
      for (int j = 0; j < m; j++)
         u[j] = estimate (dist, kern, h, Y[j]);
      return u;
   }\end{hide}
\end{code}
 \begin{tabb}
Given the empirical distribution \texttt{dist}, this method computes the 
kernel density estimate at each of the $m$ points \texttt{Y[$j$]},
 $j= 0, 1, \ldots, (m-1)$, where $m$ is the length of \texttt{Y}, the kernel
 is \texttt{kern.density(x)},
 and the bandwidth is $h$. Returns the estimates as an array of $m$ values.
\end{tabb}
\begin{code}

   public static double[] computeDensity (EmpiricalDist dist,
                                          ContinuousDistribution kern,
                                          double[] Y)\begin{hide} {
      double h = KernelDensityGen.getBaseBandwidth(dist);
      return computeDensity (dist, kern, h, Y);
   }\end{hide}
\end{code}
 \begin{tabb}
Similar to method 
\method{computeDensity}{(EmpiricalDist,ContinuousDistribution,double,double[])}
\begin{latexonly}{ above}\end{latexonly},
but the bandwidth $h$ is obtained from the method
\clsexternalmethod{umontreal.iro.lecuyer.randvar}{KernelDensityGen}{getBaseBandwidth}{EmpiricalDist}\texttt{(dist)} in package \texttt{randvar}.
\end{tabb}

\begin{code}\begin{hide}
}\end{hide}
\end{code}
