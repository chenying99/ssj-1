\defclass {StatProbe}

The objects of this class are \emph{statistical probes} or 
\emph{collectors}, which are elementary devices for collecting
statistics. Each probe collects statistics on a given variable.
The subclasses \class{Tally}, \class{TallyStore}, and
\externalclass{umontreal.iro.lecuyer.simevents}{Accumulate} (from package
\externalclass{umontreal.iro.lecuyer}{simevents})
implement two kinds of probes, for the case of successive observations
$X_1,X_2,X_3,\dots$, and for the case of a variable whose value
evolves in time, respectively.

Each instance of
\class{StatProbe} contains a list of \class{ObservationListener} that can listen
to individual observations.  
When a probe is updated, i.e., receives a new statistical observation,
it broadcasts this new data to all registered observers.
The broadcasting of observations to registered observers can be turned
ON or OFF at any time.  
It is initially OFF by default and should stay OFF when there are 
no registered observers, to avoid unnecessary overhead.
%  and performance degradation.
% but is turned ON automatically
% when a \emph{first} \externalclass{java.util}{Observer} object registers.
% Afterwards, it can be turned ON or OFF at any time.

The data collection by the statistical probe itself can also be turned
ON or OFF.  By default, it is initially ON.
We can turn it OFF, for example, if we want to use the statistical probe
only to pass data to the observers, and do not need it to store
any information.

In the simplest programs, collection is ON, broadcast is OFF, and the overall
stats are accessed via the methods 
\texttt{min}, \texttt{max}, \texttt{sum}, \texttt{average}, ... of the
collector.

\bigskip\hrule

%%%%%%%%%%%%%%%%%%%%%%%%%%%%%%%%%%%%%%%%%%%%%%%%%%%%%%%%%%%%%%%%%%
\begin{code}
\begin{hide}
/*
 * Class:        StatProbe
 * Description:  statistical probe
 * Environment:  Java
 * Software:     SSJ 
 * Copyright (C) 2001  Pierre L'Ecuyer and Université de Montréal
 * Organization: DIRO, Université de Montréal
 * @author       
 * @since

 * SSJ is free software: you can redistribute it and/or modify it under
 * the terms of the GNU General Public License (GPL) as published by the
 * Free Software Foundation, either version 3 of the License, or
 * any later version.

 * SSJ is distributed in the hope that it will be useful,
 * but WITHOUT ANY WARRANTY; without even the implied warranty of
 * MERCHANTABILITY or FITNESS FOR A PARTICULAR PURPOSE.  See the
 * GNU General Public License for more details.

 * A copy of the GNU General Public License is available at
   <a href="http://www.gnu.org/licenses">GPL licence site</a>.
 */
\end{hide}
package umontreal.iro.lecuyer.stat;\begin{hide}
import java.util.List;
import java.util.ArrayList;
import umontreal.iro.lecuyer.util.PrintfFormat;
\end{hide}

public abstract class StatProbe\begin{hide} { 

   private List<ObservationListener> listeners = new ArrayList<ObservationListener>();
   protected String name;
   protected double maxValue;
   protected double minValue;
   protected double sumValue;
   protected boolean collect = true;
   protected boolean broadcast = false;
   protected boolean showNobs = true;

\end{hide}
\end{code}

%%%%%%%%%%%%%%%%%%%%%%%%%%%%%%%%%
\subsubsection* {Methods}
\begin{code}

   abstract public void init();
\end{code}
  \begin{tabb}  Initializes the statistical collector.
 \end{tabb}
\begin{code}

   public void setName (String name) \begin{hide} {
      this.name = name; 
   } \end{hide}
\end{code}
  \begin{tabb} Sets the name of this statistical collector to \texttt{name}.
 \end{tabb}
\begin{code}

   public String getName()\begin{hide} {
      return name;
   }\end{hide}
\end{code}
\begin{tabb}   Returns the name associated with this probe,
   or \texttt{null} if no name was specified upon construction.
\end{tabb}
\begin{htmlonly}
   \return{the name associated to this collector, 
   or \texttt{null} if not specified}
\end{htmlonly}
\begin{code}

   public double min() \begin{hide} { 
      return minValue; 
   } \end{hide}
\end{code}
  \begin{tabb}  Returns the smallest value taken by the variable 
   since the last initialization
   of this probe.
   This returns \texttt{Double.POSITIVE\_INFINITY}
   if the probe was not updated since the last initialization.
 \end{tabb}
\begin{htmlonly}
   \return{the smallest value taken by the collector since last initialization}
\end{htmlonly}
\begin{code}

   public double max() \begin{hide} { 
      return maxValue; 
   } \end{hide}
\end{code}
  \begin{tabb}  Returns the largest value taken by the variable 
   since the last initialization
   of this probe.
   This returns \texttt{Double.NEGATIVE\_INFINITY}
   if the probe was not updated since the last initialization.
 \end{tabb}
\begin{htmlonly}
   \return{the largest value taken by the collector since last initialization}
\end{htmlonly}
\begin{code}

   public double sum() \begin{hide} { 
      return sumValue; 
   } \end{hide}
\end{code}
  \begin{tabb}  Returns the sum cumulated so far for this probe.
   The meaning of this sum depends on the subclass (e.g., \class{Tally} or
   \externalclass{umontreal.iro.lecuyer.simevents}{Accumulate}).
   This returns 0
   if the probe was not updated since the last initialization.
 \end{tabb}
\begin{htmlonly}
   \return{the sum for this probe}
\end{htmlonly}
\begin{code}

   abstract public double average();
\end{code}
  \begin{tabb}  Returns the average for this collector.
   This returns \texttt{Double.NaN}
   if the probe was not updated since the last initialization.
 \end{tabb}
\begin{htmlonly}
   \return{the average value of the collected observations}
\end{htmlonly}
\begin{code}

   abstract public String report();
\end{code}
  \begin{tabb}  Returns a string containing a report for this statistical
   collector. The contents of this report depends on the statistical probe as
   well as on the parameters set by the user through probe-specific methods.
 \end{tabb}
\begin{htmlonly}
   \return{a report for this probe, represented as a string}
\end{htmlonly}
\begin{code}

   abstract public String shortReport();
\end{code}
\begin{tabb}   Formats and returns a short, one-line report
 about this statistical probe.  This line is composed of
 whitespace-separated fields which must
 correspond to the column names given by \method{shortReportHeader}{}{()}.
  This report should not contain any end-of-line character, and does not
   include the name of the probe.
   Its contents depends on the statistical probe as well as  on the
   parameters set by the user through probe-specific methods.
\end{tabb}
\begin{htmlonly}
   \return{the short report for the probe.}
\end{htmlonly}
\begin{code}

   abstract public String shortReportHeader();
\end{code}
\begin{tabb}   Returns a string containing
 the name of the values returned in the report
 strings.  The returned string must depend on the type of probe and on
the reporting options only.  It must not depend on the observations
received by the probe.
This can be used as header when printing
 several reports. For example,
\begin{verbatim}
         System.out.println (probe1.shortReportHeader());
         System.out.println (probe1.getName() + " " + probe1.shortReport());
         System.out.println (probe2.getName() + " " + probe2.shortReport());
         ...
\end{verbatim}
  Alternatively, one can use \method{report}{}~\texttt{(String,StatProbe[])}
  to get a report with aligned probe names.
\end{tabb}
\begin{htmlonly}
   \return{the header string for the short reports.}
\end{htmlonly}
\begin{code}

   public static String report (String globalName, StatProbe[] probes)\begin{hide} {
      int maxn = 0;
      StatProbe firstProbe = null;
      for (StatProbe probe : probes) {
         if (firstProbe == null)
            firstProbe = probe;
         String s = probe.getName();
         if (s != null && s.length() > maxn)
            maxn = s.length();
      }
      if (firstProbe == null)
         return "";
      StringBuffer sb = new StringBuffer ("Report for ");
      sb.append (globalName).append (PrintfFormat.NEWLINE);
      for (int i = 0; i < maxn; i++)
         sb.append (' ');
      sb.append ("   ");
      sb.append (firstProbe.shortReportHeader()).append (PrintfFormat.NEWLINE);
      for (StatProbe probe : probes) {
         sb.append
            (PrintfFormat.s (-maxn, probe.getName()));
         sb.append ("   ");
         sb.append (probe.shortReport()).append (PrintfFormat.NEWLINE);
      }
      return sb.toString();
   }\end{hide}
\end{code}
\begin{tabb}  Formats short reports for each statistical probe in the array
   \texttt{probes} while aligning the probes' names.
   This method first formats the given global name.
   It then determines the maximum length $\ell$ of the names of probes in the
   given array.
   The first line of the report is composed of $\ell+3$ spaces followed by the
   string returned by \method{shortReportHeader}{} called on the first probe in
   \texttt{probes}. Each remaining line corresponds to a statistical probe; it
   contains the probe's name followed by the contents returned by
   \method{shortReport}{}. Note that this method assumes that \texttt{probes}
   contains no \texttt{null} element.
\end{tabb}
\begin{htmlonly}
   \param{globalName}{the global name given to the formatted report.}
   \param{probes}{the probes to include in the report.}
   \return{the formatted report.}
\end{htmlonly}
\begin{code}

   public static String report (String globalName,
                                Iterable<? extends StatProbe> probes)\begin{hide} {
      int maxn = 0;
      StatProbe firstProbe = null;
      for (StatProbe probe : probes) {
         if (firstProbe == null)
            firstProbe = probe;
         String s = probe.getName();
         int sl = s == null ? 4 : s.length();
         if (sl > maxn)
            maxn = sl;
      }
      if (firstProbe == null)
         return "";
      StringBuffer sb = new StringBuffer ("Report for ");
      sb.append (globalName).append (PrintfFormat.NEWLINE);
      for (int i = 0; i < maxn; i++)
         sb.append (' ');
      sb.append ("   ");
      sb.append (firstProbe.shortReportHeader()).append (PrintfFormat.NEWLINE);
      for (StatProbe probe : probes) {
         sb.append
            (PrintfFormat.s (-maxn, probe.getName()));
         sb.append ("   ");
         sb.append (probe.shortReport()).append (PrintfFormat.NEWLINE);
      }
      return sb.toString();
   }\end{hide}
\end{code}
\begin{tabb}  Equivalent to \method{report}{(String,StatProbe[])}, except that
   \texttt{probes} is an \class{Iterable} object instead of an array.
   Of course, the iterator returned by \texttt{probes} should enumerate the
   statistical probes to include in the report in a consistent and sensible order.
\end{tabb}
\begin{htmlonly}
   \param{globalName}{the global name given to the formatted report.}
   \param{probes}{the probes to include in the report.}
   \return{the formatted report.}
\end{htmlonly}
\begin{code}

   public boolean isBroadcasting()\begin{hide} {
      return broadcast;
   }\end{hide}
\end{code}
\begin{tabb}   Determines if this statistical probe
 is broadcasting observations to registered observers.
 The default is \texttt{false}.
\end{tabb}
\begin{htmlonly}
   \return{the status of broadcasting.}
\end{htmlonly}
\begin{code}

   public void setBroadcasting (boolean b)\begin{hide} {
      broadcast = b;
   }\end{hide}
\end{code}
\begin{tabb}  Instructs the probe to turn its broadcasting ON or OFF.
  The default value is OFF. 

  Warning: To avoid useless overhead and performance degradation, broadcasting
  should never be turned ON when there are no registered observers.
\end{tabb}
\begin{htmlonly}
   \param{b}{\texttt{true} to turn broadcasting ON, \texttt{false} to turn it OFF}
\end{htmlonly}
\begin{code}

   public boolean isCollecting()\begin{hide} {
      return collect;
   }\end{hide}
\end{code}
\begin{tabb}   Determines if this statistical probe
 is collecting values. The default is \texttt{true}.
\end{tabb}
\begin{htmlonly}
   \return{the status of statistical collecting.}
\end{htmlonly}
\begin{code}

   public void setCollecting (boolean b)\begin{hide} {
      collect = b;
   }\end{hide}
\end{code}
\begin{tabb}  Turns ON or OFF the collection of statistical
  observations.  The default value is ON.
  When statistical collection is turned OFF, 
  observations added to the probe are passed to the
  registered observers if broadcasting is turned ON, but are not
  counted as observations by the probe itself.
\end{tabb}
\begin{htmlonly}
   \param{b}{\texttt{true} to activate statistical collection, 
   \texttt{false} to deactivate it}
\end{htmlonly}
\begin{code}

   public void addObservationListener (ObservationListener l)\begin{hide} {
      if (l == null)
         throw new NullPointerException();
      if (!listeners.contains (l))
         listeners.add (l);
   }\end{hide}
\end{code}
\begin{tabb}   Adds the observation listener \texttt{l} to the list of observers of
    this statistical probe.
\end{tabb}
\begin{htmlonly}
   \param{l}{the new observation listener.}
   \exception{NullPointerException}{if \texttt{l} is \texttt{null}.}
\end{htmlonly}
\begin{code}

   public void removeObservationListener (ObservationListener l)\begin{hide} {
      listeners.remove (l);
   }\end{hide}
\end{code}
\begin{tabb}   Removes the observation listener \texttt{l} from the list of observers of
    this statistical probe.
\end{tabb}
\begin{htmlonly}
   \param{l}{the observation listener to be deleted.}
\end{htmlonly}
\begin{code}

   public void clearObservationListeners()\begin{hide} {
      listeners.clear();
   }\end{hide}
\end{code}
\begin{tabb}   Removes all observation listeners from the list of observers of
    this statistical probe.
\end{tabb}
\begin{code}

   public void notifyListeners (double x)\begin{hide} {
      if (!broadcast)
         return;
      // We could also use the enhanced for loop here, but this is less efficient.
      final int nl = listeners.size();
      for (int i = 0; i < nl; i++)
         listeners.get (i).newObservation (this, x);
   }\end{hide}
\end{code}
\begin{tabb}   Notifies the observation \texttt{x} to all registered observers
   if broadcasting is ON.  Otherwise, does nothing.
\end{tabb}
\begin{code}\begin{hide}

   public StatProbe clone() throws CloneNotSupportedException {
      StatProbe s = (StatProbe)super.clone();
      s.listeners = new ArrayList<ObservationListener>(listeners);
      return s;
   }
} 
\end{hide}
\end{code}
