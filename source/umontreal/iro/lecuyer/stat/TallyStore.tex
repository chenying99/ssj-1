\defclass {TallyStore}

This class is a variant of \class{Tally} for which the individual
observations are stored in a list implemented as a
\externalclass{cern.colt.list}{DoubleArrayList}.
\begin{latexonly}%
The class \texttt{DoubleArrayList} is imported from the COLT library and
provides an efficient way of storing and manipulating a list of
real-valued numbers in a dynamic array.
\end{latexonly}%
The \externalclass{cern.colt.list}{DoubleArrayList} object used to
store the values can be either passed to the constructor or created by
the constructor, and can be accessed via the
\method{getDoubleArrayList}{} method.

The same counters as in \class{Tally} are maintained and are used by
the inherited methods.  One must access the list of observations
to compute quantities not supported by the methods in
\class{Tally}, and/or to use methods provided by the COLT package.


\emph{Never add or remove observations directly} on the
\externalclass{cern.colt.list}{DoubleArrayList} object,
because this would put the counters of the \texttt{TallyStore} object in an
inconsistent state.

There are two potential reasons for using a \class{TallyStore} object
instead of directly using a \externalclass{cern.colt.list}{DoubleArrayList} object:
(a) it can broadcast observations and
(b) it maintains a few additional counters that may speed up some operations
such as computing the average.



\bigskip\hrule

%%%%%%%%%%%%%%%%%%%%%%%%%%%%%%%%%%%%%%%%%%%%%%%%%%%%%%%%%%%%%%%%%%
\begin{code}
\begin{hide}
/*
 * Class:        TallyStore
 * Description:  
 * Environment:  Java
 * Software:     SSJ 
 * Copyright (C) 2001  Pierre L'Ecuyer and Université de Montréal
 * Organization: DIRO, Université de Montréal
 * @author       
 * @since

 * SSJ is free software: you can redistribute it and/or modify it under
 * the terms of the GNU General Public License (GPL) as published by the
 * Free Software Foundation, either version 3 of the License, or
 * any later version.

 * SSJ is distributed in the hope that it will be useful,
 * but WITHOUT ANY WARRANTY; without even the implied warranty of
 * MERCHANTABILITY or FITNESS FOR A PARTICULAR PURPOSE.  See the
 * GNU General Public License for more details.

 * A copy of the GNU General Public License is available at
   <a href="http://www.gnu.org/licenses">GPL licence site</a>.
 */
\end{hide}
package umontreal.iro.lecuyer.stat;\begin{hide}
import cern.colt.list.DoubleArrayList;
import java.util.logging.Level;
import java.util.logging.Logger;
import umontreal.iro.lecuyer.util.PrintfFormat;
\end{hide}


public class TallyStore extends Tally \begin{hide} {

   private DoubleArrayList array = null;  // Where the observations are stored.
   private Logger log = Logger.getLogger ("umontreal.iro.lecuyer.stat");
\end{hide}
\end{code}

%%%%%%%%%%%%%%%%%%%%%%%%%%%%%%%%%
\subsubsection* {Constructors}
\begin{code}

   public TallyStore() \begin{hide} {
      super();
      array = new DoubleArrayList();
   }\end{hide}
\end{code}
\begin{tabb}  Constructs a new \texttt{TallyStore} statistical probe.
\end{tabb}
\begin{code}

   public TallyStore (String name) \begin{hide} {
      super (name);
      array = new DoubleArrayList();
   }\end{hide}
\end{code}
\begin{tabb}  Constructs a new \texttt{TallyStore} statistical probe with name \texttt{name}.
\end{tabb}
\begin{htmlonly}
   \param{name}{the name of the tally.}
\end{htmlonly}
\begin{code}

   public TallyStore (int capacity) \begin{hide} {
      super();
      array = new DoubleArrayList (capacity);
   }\end{hide}
\end{code}
\begin{tabb}  Constructs a new \texttt{TallyStore} statistical probe
    with given initial capacity \texttt{capacity} for its associated array.
\end{tabb}
\begin{htmlonly}
   \param{capacity}{initial capacity of the array of observations}
\end{htmlonly}
\begin{code}

   public TallyStore (String name, int capacity) \begin{hide} {
      super (name);
      array = new DoubleArrayList (capacity);
   }\end{hide}
\end{code}
\begin{tabb}  Constructs a new \texttt{TallyStore} statistical probe with
name \texttt{name} and given initial capacity \texttt{capacity} for its
 associated array.
\end{tabb}
\begin{htmlonly}
   \param{name}{the name of the tally.}
   \param{capacity}{initial capacity of the array of observations}
\end{htmlonly}
\begin{code}

   public TallyStore (DoubleArrayList a) \begin{hide} {
      super();
      array = a;
      array.clear();
   }\end{hide}
\end{code}
\begin{tabb}  Constructs a new \texttt{TallyStore} statistical probe
    with given associated array.  This array must be empty.
\end{tabb}
\begin{htmlonly}
   \param{a}{array that will contain observations}
\end{htmlonly}

%%%%%%%%%%%%%%%%%%%%%%%%%%%%%%%%%
\subsubsection* {Methods}
\begin{code}\begin{hide}

   public void init() {
       super.init();
       // We must call super before any actions inside constructors.
       // Unfortunately, the base class calls init, which would
       // result in a NullPointerException.
       if (array != null)
          array.clear();
   }

   public void add (double x) {
      if (collect) array.add (x);
      super.add(x);
   }\end{hide}

   public double[] getArray() \begin{hide} {
      array.trimToSize();
      return array.elements();
   }\end{hide}
\end{code}
\begin{tabb} Returns the observations stored in this probe.
%  \textbf{WARNING:}
%     The returned array is the stored internal array itself, including invalid
%     elements between size and capacity. If the returned \texttt{double} array is
%     modified, then this probe itself will be modified. To get the number of
%     valid elements (observations) in the array, use
%     \externalmethod{umontreal.iro.lecuyer.stat}{Tally}{numberObs}{}.
% \richard{I now use trimToSize to remove the extra elements.}
\end{tabb}
\begin{htmlonly}
   \return{the array of observations associated with this object}
\end{htmlonly}
\begin{code}

   public DoubleArrayList getDoubleArrayList() \begin{hide} {
      array.trimToSize();
      return array;
   }\end{hide}
\end{code}
\begin{tabb}  Returns the \externalclass{cern.colt.list}{DoubleArrayList}
   object that contains the observations for this probe. \textbf{WARNING:}
   In previous releases, this function was named \texttt{getArray}.
\end{tabb}
\begin{htmlonly}
   \return{the array of observations associated with this object}
\end{htmlonly}
\begin{code}

   public void quickSort() \begin{hide} {
       array.quickSort();
   }\end{hide}
\end{code}
\begin{tabb} Sorts the elements of this probe using the \texttt{quicksort}
  from Colt.
\end{tabb}
\begin{code}

   public double covariance (TallyStore t2)\begin{hide} {
      if (numberObs() != t2.numberObs()) {
         //System.err.println
         //   ("******* TallyStore.covariance():   Tally's with different number of observations");
         log.logp (Level.WARNING, "TallyStore", "covariance",
            "This tally, with name " + getName() + ", contains " + numberObs() + " observations while " +
              "the given tally, with name " + t2.getName() + ", contains " + t2.numberObs()
              + "observations");
         return Double.NaN;
      }

      if (numberObs() < 2 || t2.numberObs() < 2) {
         //System.err.println
         //   ("******* TallyStore.covariance()   with " + numberObs() + " Observation");
         log.logp (Level.WARNING, "TallyStore", "covariance",
            "This tally, with name " + getName() + ", contains " + numberObs() + " observation");
         return Double.NaN;
      }

      return cern.jet.stat.Descriptive.covariance (
          getDoubleArrayList(), t2.getDoubleArrayList());
   }\end{hide}
\end{code}
\begin{tabb}   Returns the sample covariance of the observations contained
 in this tally, and the other tally \texttt{t2}.
 Both tallies must have the same number of observations.
   This returns \texttt{Double.NaN}
   if the tallies do not contain the same number of observations, or
 if they contain less than two observations.
\end{tabb}
\begin{htmlonly}
   \param{t2}{the other tally.}
   \return{the sample covariance.}
\end{htmlonly}
\begin{code}

   public TallyStore clone()\begin{hide} {
      TallyStore t = (TallyStore)super.clone();
      t.array = (DoubleArrayList)array.clone();
      return t;
   }\end{hide}
\end{code}
\begin{tabb} Clones this object and the array which stores the observations.
\end{tabb}
\begin{code}

   public String toString()\begin{hide} {
      StringBuffer sb = new StringBuffer ();
      for (int i=0; i<numberObs(); i++)
         sb.append (i + "    " + array.getQuick(i) +
                           PrintfFormat.NEWLINE);
      return sb.toString();
   }\end{hide}
\end{code}
\begin{tabb} Returns the observations stored in this object as a \texttt{String}.
\end{tabb}
\begin{code}\begin{hide}
}\end{hide}
\end{code}
