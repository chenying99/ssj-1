\defclass{LeastSquares}

Represents a polynomial obtained by the least squares method on a set of points.
More specifically, let $(x_0,y_0), \ldots, (x_n, y_n)$ be a set of points and
$p(x)$ the constructed polynomial of degree $m$. The constructed polynomial
minimizes the square error \[ E^2=\sum_{i=0}^n [y_i - p(x_i)]^2.\]

\bigskip\hrule

\begin{code}
\begin{hide}
/*
 * Class:        LeastSquares
 * Description:  polynomial obtained by the least squares method on set of points
 * Environment:  Java
 * Software:     SSJ 
 * Copyright (C) 2001  Pierre L'Ecuyer and Université de Montréal
 * Organization: DIRO, Université de Montréal
 * @author       
 * @since

 * SSJ is free software: you can redistribute it and/or modify it under
 * the terms of the GNU General Public License (GPL) as published by the
 * Free Software Foundation, either version 3 of the License, or
 * any later version.

 * SSJ is distributed in the hope that it will be useful,
 * but WITHOUT ANY WARRANTY; without even the implied warranty of
 * MERCHANTABILITY or FITNESS FOR A PARTICULAR PURPOSE.  See the
 * GNU General Public License for more details.

 * A copy of the GNU General Public License is available at
   <a href="http://www.gnu.org/licenses">GPL licence site</a>.
 */
\end{hide}
package umontreal.iro.lecuyer.functionfit;
   import umontreal.iro.lecuyer.functions.Polynomial;
\begin{hide}

import java.io.Serializable;
import cern.colt.matrix.DoubleMatrix2D;
import cern.colt.matrix.impl.DenseDoubleMatrix2D;
import cern.colt.matrix.linalg.Algebra;
\end{hide}

public class LeastSquares extends Polynomial implements Serializable\begin{hide} {
   private static final long serialVersionUID = -4997132164503234983L;
   private static final Algebra alg = new Algebra ();
   private double[] x;
   private double[] y;
\end{hide}
\end{code}

%%%%%%%%%%%%%%%%%%%%%%%%%%%%%%%%%%%%%%%%%
\subsubsection* {Constructors}
\begin{code}

   public LeastSquares (double[] x, double[] y, int degree)\begin{hide} {
      super (getCoefficients (x, y, degree));
      this.x = x.clone ();
      this.y = y.clone ();
   }\end{hide}
\end{code}
\begin{tabb}   Constructs a new least squares polynomial with points \texttt{(x[0],
 y[0]),..., (x[n], y[n])}. The constructed polynomial has degree
 \texttt{degree}.
\end{tabb}
\begin{htmlonly}
   \param{x}{the $x$ coordinates of the points.}
   \param{y}{the $y$ coordinates of the points.}
   \param{degree}{the degree of the polynomial.}
   \exception{NullPointerException}{if \texttt{x} or \texttt{y} are \texttt{null}.}
   \exception{IllegalArgumentException}{if the lengths of \texttt{x} and \texttt{y} are different,
               or if less than \texttt{degree + 1} points are specified.}
\end{htmlonly}

%%%%%%%%%%%%%%%%%%%%%%%%%%%%%%%%%%%
\subsubsection* {Methods}

\begin{code}

   public static double[] getCoefficients (double[] x, double[] y,
                                           int degree)\begin{hide} {
      if (x.length != y.length)
         throw new IllegalArgumentException ("Length of x and y not equal");
      if (x.length < degree + 1)
         throw new IllegalArgumentException ("Not enough points");

      final double[] xSums = new double[2 * degree + 1];
      final double[] xySums = new double[degree + 1];
      xSums[0] = x.length;
      for (int i = 0; i < x.length; i++) {
         double xv = x[i];
         xySums[0] += y[i];
         for (int j = 1; j <= 2 * degree; j++) {
            xSums[j] += xv;
            if (j <= degree)
               xySums[j] += xv * y[i];
            xv *= x[i];
         }
      }
      final DoubleMatrix2D A = new DenseDoubleMatrix2D (degree + 1, degree + 1);
      final DoubleMatrix2D B = new DenseDoubleMatrix2D (degree + 1, 1);
      for (int i = 0; i <= degree; i++) {
         for (int j = 0; j <= degree; j++) {
            final int d = i + j;
            A.setQuick (i, j, xSums[d]);
         }
         B.setQuick (i, 0, xySums[i]);
      }
      final DoubleMatrix2D aVec = alg.solve (A, B);
      return aVec.viewColumn (0).toArray ();
   }\end{hide}
\end{code}
\begin{tabb} Computes and returns the coefficients of the fitting polynomial of
  degree \texttt{degree}.
 The coordinates of the given points are  \texttt{(x[i], y[i])}.
\end{tabb}
\begin{htmlonly}
   \param{x}{the $x$ coordinates of the points.}
   \param{y}{the $y$ coordinates of the points.}
   \param{degree}{the degree of the polynomial.}
   \return{the coefficients of the fitting polynomial.}
\end{htmlonly}
\begin{code}

   public double[] getX()\begin{hide} {
      return x;
   }\end{hide}
\end{code}
\begin{tabb}   Returns the $x$ coordinates of the fitted points.
\end{tabb}
\begin{htmlonly}
   \return{the $x$ coordinates of the fitted points.}
\end{htmlonly}
\begin{code}

   public double[] getY()\begin{hide} {
      return y;
   }\end{hide}
\end{code}
\begin{tabb}   Returns the $y$ coordinates of the fitted points.
\end{tabb}
\begin{htmlonly}
   \return{the $y$ coordinates of the fitted points.}
\end{htmlonly}
\begin{code}

   public String toString()\begin{hide} {
      return PolInterp.toString (x, y);
   }

   @Override
\end{hide}
\end{code}
\begin{tabb}   Calls \externalmethod{umontreal.iro.lecuyer.functionfit}{PolInterp}{toString}{double[], double[]}
 with the associated points.
\end{tabb}
\begin{htmlonly}
   \return{a string containing the points.}
\end{htmlonly}
\begin{code}\begin{hide}

   public LeastSquares clone() {
      final LeastSquares ls = (LeastSquares) super.clone ();
      ls.x = x.clone ();
      ls.y = y.clone ();
      return ls;
   }
}\end{hide}
\end{code}
