\defclass {Sim}

This static class contains the executive of a discrete-event simulation.
It maintains the simulation clock and starts executing the events 
%and processes
in the appropriate order.
Its methods permit one to start, stop, and (re)initialize the simulation,
and read the simulation clock.

Starting from SSJ-2.0, the \texttt{Sim} class now uses
the default simulator returned by the \texttt{getDefaultSimulator()}
method in the \class{Simulator} class. Although the \texttt{Sim} class is 
perfectly adequate for simple simulations, the \class{Simulator} class
is more general and supports more functionnalities.
For example, if one needs to have more than one simulation clock and event list,
one will have to use the \class{Simulator} class instead of the simpler
\texttt{Sim} class.

\bigskip\hrule

%%%%%%%%%%%%%%%%%%%%%%%%%%%%%%%%%%%%%%%%%%%%%%%%%%%%%%%%%%%%%%%%%%
\begin{code}
\begin{hide}
/*
 * Class:        Sim
 * Description:  contains the executive of a discrete-event simulation
 * Environment:  Java
 * Software:     SSJ 
 * Copyright (C) 2001  Pierre L'Ecuyer and Université de Montréal
 * Organization: DIRO, Université de Montréal
 * @author       
 * @since

 * SSJ is free software: you can redistribute it and/or modify it under
 * the terms of the GNU General Public License (GPL) as published by the
 * Free Software Foundation, either version 3 of the License, or
 * any later version.

 * SSJ is distributed in the hope that it will be useful,
 * but WITHOUT ANY WARRANTY; without even the implied warranty of
 * MERCHANTABILITY or FITNESS FOR A PARTICULAR PURPOSE.  See the
 * GNU General Public License for more details.

 * A copy of the GNU General Public License is available at
   <a href="http://www.gnu.org/licenses">GPL licence site</a>.
 */
\end{hide}
package umontreal.iro.lecuyer.simevents;\begin{hide}
import umontreal.iro.lecuyer.simevents.eventlist.EventList;
import umontreal.iro.lecuyer.simevents.eventlist.SplayTree;
import umontreal.iro.lecuyer.simevents.Event;
import umontreal.iro.lecuyer.simevents.Simulator;\end{hide}

public final class Sim \begin{hide} {

   // Prevents construction of this object
   private Sim() {}   
\end{hide}
\end{code}
%%%%%%%%%%%%%%%%%%%%%%%%%%%%%%%%%%%
\subsubsection* {Methods}

\begin{code}

   public static double time() \begin{hide} {
      return Simulator.getDefaultSimulator().time();
   }\end{hide}
\end{code}
   \begin{tabb} Returns the current value of the simulation clock. \end{tabb}
\begin{htmlonly}
   \return{the current simulation time}
\end{htmlonly}
\begin{code}

   public static void init() \begin{hide} {
     // This has to be done another way in order to separate events and processes.
      Simulator.getDefaultSimulator().init();
   }\end{hide}
\end{code}
  \begin{tabb} Reinitializes the simulation executive by clearing up the event
   list, and resetting the simulation clock to zero.
   This method must not be used to initialize process-driven
   simulation; \clsexternalmethod{umontreal.iro.lecuyer.simprocs}{SimProcess}{init}{}
   must be used instead.
  \end{tabb}
\begin{code}
 
   public static void init (EventList evlist) \begin{hide} {
     Simulator.getDefaultSimulator().init(evlist);
   }\end{hide}
\end{code}
  \begin{tabb} Same as \method{init}{}, but also chooses \texttt{evlist} as the 
    event list to be used. For example, calling 
    \texttt{init(new DoublyLinked())} initializes the simulation
    with a doubly linked linear structure for the event list.
   This method must not be used to initialize process-driven
   simulation; 
\clsexternalmethod{}{umontreal.iro.lecuyer.simprocs}{DSOLProcessSimulator}{init}{}~\texttt{(EventList)} or \\
\clsexternalmethod{}{umontreal.iro.lecuyer.simprocs}{ThreadProcessSimulator}{init}{}~\texttt{(EventList)} 
   must be used instead.
  \end{tabb}
\begin{htmlonly}
   \param{evlist}{selected event list implementation}
\end{htmlonly}
\begin{code}

   public static EventList getEventList()\begin{hide} {
      return Simulator.getDefaultSimulator().getEventList();
   }\end{hide}
\end{code}
\begin{tabb}  Gets the currently used event list.
\end{tabb}
\begin{htmlonly}
   \return{the currently used event list}
\end{htmlonly}
\begin{hide}
\begin{code}
   protected static Event removeFirstEvent() {
       return Simulator.getDefaultSimulator().removeFirstEvent();
   }
\end{code}
\begin{tabb}  This method is used by the package
  \externalclass{umontreal.iro.lecuyer}{simprocs};
  \emph{it should not be used directly by a simulation program}.
  It removes the first event from the event list and sets the simulation
  clock to its event time.
\end{tabb}
\begin{htmlonly}
   \return{the first planned event, or \texttt{null} if there is no such event}
\end{htmlonly}
\end{hide}
\begin{code}

   public static void start() \begin{hide} {
      Simulator.getDefaultSimulator().start();
   }\end{hide}
\end{code}
  \begin{tabb}  Starts the simulation executive.
   There must be at least one \texttt{Event} in the
   event list when this method is called.
  \end{tabb}
\begin{code}

   public static void stop() \begin{hide} {
      Simulator.getDefaultSimulator().stop();
   }\end{hide}
\end{code}
  \begin{tabb} Tells the simulation executive to stop as soon as it takes control,
   and to return control to the program that called \method{start}{}.
   This program will then continue
   executing from the instructions right after its call to \texttt{Sim.start}.
   If an \class{Event} is currently executing (and this event has just called
   \texttt{Sim.stop}), the executive will take
   control when the event terminates its execution.
%   If a \texttt{SimProcess} is currently executing, the executive will take
%   control when the process terminates or suspends itself by invoking
%   one of its methods such as \texttt{delay}, \texttt{suspend}, 
%   \texttt{request}, etc.
  \end{tabb}
\begin{code}\begin{hide}

   // Used to passivate and activate the main process.
   // See the SimProcess.dispatch() and SimThread.actions()
 
//    protected static void activate() { 
 //       synchronized (eventList) {eventList.notify(); } 
 //    }
         
//     protected static void passivate() {
//         synchronized (eventList) {
  //          try { eventList.wait(); } catch (InterruptedException e) {}
  //       }
  //   }
}
\end{hide}\end{code}
