\defclass {Condition}

A \texttt{Condition} is a boolean indicator, with a list of processes  
waiting for the indicator to be \texttt{true} (when it is \texttt{false}).
A process calling \method{waitFor}{} on a condition that is currently
\texttt{false} is suspended until the condition becomes \texttt{true}.
The list of waiting processes can be accessed via \method{waitList}{}.

\bigskip\hrule

%%%%%%%%%%%%%%%%%%%%%%%%%%%%%%%%%%%%%%%%%%%%%%%%%%%%%%%%%%%%%%%%%%
\begin{code}
\begin{hide}
/*
 * Class:        Condition
 * Description:  boolean indicator with a list of processes waiting for 
                 the indicator to be true
 * Environment:  Java
 * Software:     SSJ 
 * Copyright (C) 2001  Pierre L'Ecuyer and Université de Montréal
 * Organization: DIRO, Université de Montréal
 * @author       
 * @since

 * SSJ is free software: you can redistribute it and/or modify it under
 * the terms of the GNU General Public License (GPL) as published by the
 * Free Software Foundation, either version 3 of the License, or
 * any later version.

 * SSJ is distributed in the hope that it will be useful,
 * but WITHOUT ANY WARRANTY; without even the implied warranty of
 * MERCHANTABILITY or FITNESS FOR A PARTICULAR PURPOSE.  See the
 * GNU General Public License for more details.

 * A copy of the GNU General Public License is available at
   <a href="http://www.gnu.org/licenses">GPL licence site</a>.
 */
\end{hide}
package umontreal.iro.lecuyer.simprocs;\begin{hide}
import java.util.Observable;
import umontreal.iro.lecuyer.simevents.Simulator;
import umontreal.iro.lecuyer.simevents.LinkedListStat;\end{hide}

public class Condition extends Observable \begin{hide} {

   private String name;
   private LinkedListStat<UserRecord> waitingList;
   private boolean state;
   private boolean broadcasting;
   private ProcessSimulator sim;
\end{hide}\end{code}

%%%%%%%%%%%%%%%%%%%%%%%%%%%%%%%%%%%
\subsubsection* {Constructors}
\begin{code}

   public Condition (boolean val) \begin{hide} {
      this (val, "");
   }\end{hide}
\end{code}
 \begin{tabb}  Constructs a new \texttt{Condition} with initial value \texttt{val}, linked with the default simulator.
 \end{tabb}
\begin{htmlonly}
   \param{val}{initial state of the condition}
\end{htmlonly}
\begin{code}

   public Condition (ProcessSimulator sim, boolean val) \begin{hide} {
      this (sim, val, "");
   }\end{hide}
\end{code}
 \begin{tabb}  Constructs a new \texttt{Condition} with initial value \texttt{val}, linked with simulator \texttt{sim}.
 \end{tabb}
\begin{htmlonly}
   \param{sim}{current simulator of the variable}
   \param{val}{initial state of the condition}
\end{htmlonly}
\begin{code}

   public Condition (boolean val, String name) \begin{hide} {
      super();
      try {
         ProcessSimulator.initDefault();
         this.sim = (ProcessSimulator)Simulator.getDefaultSimulator();
         waitingList = new LinkedListStat<UserRecord>(sim);
         this.name = name;
         broadcasting = false;
         state = val;
      }
      catch (ClassCastException e) {
         throw new IllegalArgumentException("Wrong default Simulator type");
      }
   }\end{hide}
\end{code}
\begin{tabb}  Constructs a new \texttt{Condition} with initial value \texttt{val}, 
 identifier \texttt{name} and linked with the default simulator.
\end{tabb}
\begin{htmlonly}
   \param{val}{initial state of the condition}
   \param{name}{identifier or name for the condition}
\end{htmlonly}
\begin{code}

   public Condition (ProcessSimulator sim, boolean val, String name) \begin{hide} {
      super();
      this.sim = sim;
      waitingList = new LinkedListStat<UserRecord>(sim);
      this.name = name;
      broadcasting = false;
      state = val;
   }\end{hide}
\end{code}
 \begin{tabb}  Constructs a new \texttt{Condition} with initial value \texttt{val}, 
 identifier \texttt{name} and linked with simulator \texttt{sim}.
 \end{tabb}
\begin{htmlonly}
   \param{sim}{current simulator of the variable}
   \param{val}{initial state of the condition}
   \param{name}{identifier or name for the condition}
\end{htmlonly}

%%%%%%%%%%%%%%%%%%%%%%%%%%%%%%%%%%%
\subsubsection* {Methods}
\begin{code}

   public void init (boolean val) \begin{hide} {
      waitingList.clear();
      state = val;
      if (broadcasting) {
         setChanged();
         notifyObservers (new Boolean (state));
      }
   }\end{hide}
\end{code}
 \begin{tabb}  Reinitializes this \texttt{Condition} by clearing up 
   its waiting list and resetting its state to \texttt{val}.
   The processes in this list (if any) remain in the same states.
 \end{tabb}
\begin{htmlonly}
   \param{val}{initial state of the condition}
\end{htmlonly}
\begin{code}

   public void set (boolean val) \begin{hide} { 
      state = val;
      if (state) {
         UserRecord record;
         while (!waitingList.isEmpty()) {
            record= waitingList.removeLast();
            record.process.resume();
         }    
      } 
      if (broadcasting) {
         setChanged();
         notifyObservers (new Boolean (val));
      }
   }\end{hide}
\end{code}
 \begin{tabb}  Sets the condition to \texttt{val}.
   If \texttt{val} is \texttt{true}, all the processes waiting 
   for this condition now resume their execution, in the same order as
   they have called \method{waitFor}{} for this condition.
   (Note that a process can never wait for more than one condition at
   a time, because it cannot call \method{waitFor}{} while it is suspended.
 \end{tabb}
\begin{htmlonly}
   \param{val}{new state of the condition}
\end{htmlonly}
\begin{code}

   public boolean state() \begin{hide} { 
      return state;
   }\end{hide}
\end{code}
 \begin{tabb}  Returns the state (\texttt{true} or \texttt{false}) of the condition.
 \end{tabb}
\begin{htmlonly}
   \return{the current state of the condition}
\end{htmlonly}
\begin{code}

   public void waitFor() \begin{hide} {
      SimProcess p = sim.currentProcess();
      if (state == true) {
         // The process can continus right away.
         if (waitingList.statSojourn() != null)
            waitingList.statSojourn().add (0.0);
         return;
      }
      else {
         // The process joins the waitingList;
         UserRecord record = new UserRecord (1, p, sim.time());
         waitingList.addLast (record);
         p.suspend();
      }
   }\end{hide}
\end{code}
 \begin{tabb}  The executing process invoking this method
  must wait for this condition to be \texttt{true}.
  If it is already \texttt{true}, the 
  process continues its execution immediately
   Otherwise, the executing process is placed into the 
   \method{waitList}{} list for this condition and is suspended
   until the condition becomes \texttt{true}.
 \end{tabb}
\begin{code}

   public LinkedListStat waitList() \begin{hide} { 
      return waitingList;
   }\end{hide}
\end{code}
 \begin{tabb}  Returns the list of \class{UserRecord}
   for the processes waiting for this condition.
 \end{tabb}
\begin{htmlonly}
   \return{the list of processes user records waiting for the condition}
\end{htmlonly}
\begin{code}

   public String getName()\begin{hide} {
      return name;
   }\end{hide}
\end{code}
\begin{tabb}   Returns the name (or identifier) associated
  to this condition.
\end{tabb}
\begin{htmlonly}
   \return{the name associated to this object}
\end{htmlonly}
\begin{code}

   public void setBroadcasting (boolean b)\begin{hide} {
      broadcasting = b;
   }
}\end{hide}
\end{code}
\begin{tabb}  Instructs the condition to start or stop observation broadcasting.
  When this is turned ON, a \class{Boolean} observation is notified to
  registered observers when the state of the condition changes.  This
  boolean gives the new state of the condition.

Warning: Since this can reduce program performance, this should be
turned on only when there are registered observers.
\end{tabb}
\begin{htmlonly}
   \param{b}{\texttt{true} to turn broadcasting ON, \texttt{false} to turn it OFF}
\end{htmlonly}
