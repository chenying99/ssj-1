\defclass{ProcessSimulator}

Defines a special type of simulator capable of managing processes.
This class extends the \class{Simulator} class
with methods providing basic facilities used by process simulators
to suspend, resume, and kill processes.
It also specifies a mechanism for creating control events used for process synchronization.
These methods are not called directly by the user, but they
need to be implemented by any framework managing processes.

A process simulator is usually constructed by the \method{newInstance}{} method which
selects the appropriate implementation based on system properties.
It can also be constructed manually  by calling the appropriate constructor explicitly.
A default, implicit, simulator can also be used for compatibility with older programs.
This can be done by calling the \method{initDefault}{} method or setting the
\texttt{ssj.processes} property.

\bigskip\hrule

%%%%%%%%%%%%%%%%%%%%%%%%%%%%%%%%%%%%%%%%%%%%%%%%%%%%%%%%%%%%%%%%%%
\begin{code}

\begin{hide}
/*
 * Class:        ProcessSimulator
 * Description:  Special type of simulator capable of managing processes
 * Environment:  Java
 * Software:     SSJ 
 * Copyright (C) 2001  Pierre L'Ecuyer and Université de Montréal
 * Organization: DIRO, Université de Montréal
 * @author       
 * @since

 * SSJ is free software: you can redistribute it and/or modify it under
 * the terms of the GNU General Public License (GPL) as published by the
 * Free Software Foundation, either version 3 of the License, or
 * any later version.

 * SSJ is distributed in the hope that it will be useful,
 * but WITHOUT ANY WARRANTY; without even the implied warranty of
 * MERCHANTABILITY or FITNESS FOR A PARTICULAR PURPOSE.  See the
 * GNU General Public License for more details.

 * A copy of the GNU General Public License is available at
   <a href="http://www.gnu.org/licenses">GPL licence site</a>.
 */
\end{hide}
package umontreal.iro.lecuyer.simprocs;\begin{hide}

import umontreal.iro.lecuyer.simevents.Event;
import umontreal.iro.lecuyer.simevents.Simulator;
\end{hide}

public abstract class ProcessSimulator extends Simulator \begin{hide} {

   protected SimProcess currentProcess;
   // The process who has control right now (current process).
   // If no process has control (an event is executing), current = null.
\end{hide}
\end{code}

% %%%%%%%%%%%%%%%%%%%%%%%%%%%%%%%%%%%
% \subsubsection* {Constructor}
% \begin{code}
%    public ProcessSimulator() \begin{hide} {
%       currentProcess = null;
%    }\end{hide}
% \end{code}
% \begin{tabb} Creates a new Simulator for process-oriented simulation
% \end{tabb}

%%%%%%%%%%%%%%%%%%%%%%%%%%%%%%%%%%%
\subsubsection* {Methods}
\begin{code}

   public SimProcess currentProcess()\begin{hide} {
      return currentProcess;
   }\end{hide}
\end{code}
\begin{tabb}  Returns the currently active process for this simulator.
\end{tabb}
\begin{code}\begin{hide}

   protected void setCurrentProcess (SimProcess currentProcess) {
      this.currentProcess = currentProcess;
   }
\end{hide}

   public abstract Event createControlEvent (SimProcess process);
\end{code}
\begin{tabb} Constructs and returns a new \class{Event} object used for synchronization.
Such control events are used by process simulator to start a process or to resume it if it is already
started.
\end{tabb}
\begin{code}

   public abstract void delay (SimProcess process, double delay);
\end{code}
\begin{tabb} Suspends the execution of \texttt{process} and
    schedules it to resume its execution in \texttt{delay} units of simulation
    time.  The state of the process also changes to \texttt{DELAYED}.
\end{tabb}
\begin{htmlonly}
   \param{process}{SimProcess variable to delay.}
   \param{delay}{delay value, in simulation time units.}
\end{htmlonly}
\begin{code}

   public abstract void suspend (SimProcess process);
\end{code}
\begin{tabb} Suspends \texttt{process}.
   If the process is \texttt{EXECUTING}, this suspends its execution.
   If the process is \texttt{DELAYED}, this cancels its control event.
   This method also places the process in the \texttt{SUSPENDED} state.
\end{tabb}
\begin{htmlonly}
   \param{process}{SimProcess variable to suspend.}
\end{htmlonly}
\begin{code}

   public abstract void kill (SimProcess process);
\end{code}
  \begin{tabb}  Terminates the life of \texttt{process} and sets its state to
   \texttt{DEAD}, after canceling its control event if there is one.
  \end{tabb}
\begin{code}

   public abstract void killAll();
\end{code}
\begin{tabb}  Kills all currently living (active, delayed, or suspended) processes managed by
  this simulator.
\end{tabb}
\begin{code}

   public static void initDefault() \begin{hide} {
      if (defaultSimulator instanceof ProcessSimulator) {
         defaultSimulator.init();
         return;
      }
      if (defaultSimulator instanceof Simulator) {
         Simulator temp = defaultSimulator;
         defaultSimulator = newInstance();
         defaultSimulator.init(temp.getEventList());
         return;
      }
      defaultSimulator = newInstance();
      defaultSimulator.init();
   }

   static {
      if (System.getProperty ("ssj.processes") != null)
         initDefault();
   } \end{hide}
\end{code}
  \begin{tabb}  Initializes the default simulator to use processes.
    If the field \texttt{Simulator.defaultSimulator} is already initialized to a class
    extending \texttt{ProcessSimulator}, this
    method does nothing.
    Otherwise, it initializes the field to the return value of
    \method{newInstance}{}.

   Warning: \textbf{this method must be called before any event or process is
   constructed},  otherwise the program could return some exceptions like
   \class{ClassCastException} or \class{NullPointerException}.
   Alternatively, one can set the \texttt{ssj.processes} system property, which
   instructs this class to call this method at the time it is loaded.

   The aim of this method is to allow the user to use default process-oriented
   simulation without giving options to the Java Virtual Machine.
   See package
   \texttt{simevents} and class \class{Simulator} for more information about the default simulator.
  \end{tabb}
\begin{code}

   public static ProcessSimulator newInstance()\begin{hide} {
      if(System.getProperty("ssj.processSimulator") != null) {
         ProcessSimulator myProcessSimulator = null;
         try{
            myProcessSimulator = (ProcessSimulator)(Class.forName(System.getProperty("ssj.processSimulator"))).newInstance();
         }
         catch (ClassNotFoundException cnfe) { cnfe.printStackTrace(); }
         catch (IllegalAccessException iae)  { iae.printStackTrace(); }
         catch (InstantiationException ie)   { ie.printStackTrace(); }
         return myProcessSimulator;
      }
      else if(System.getProperty("ssj.withThread") != null)
         return new ThreadProcessSimulator();
      else if (System.getProperty("ssj.withDSOL") != null)
         return new DSOLProcessSimulator();
      else
         return new ThreadProcessSimulator();
   }\end{hide}
\end{code}
\begin{tabb}  Constructs and returns a new process-oriented simulator.
  This method selects the concrete subclass of process simulator as follows.
  If the \texttt{ssj.processSimulator} system property is set, this gives the
  fully qualified class name of the process simulator to be instantiated.
  The given class must not be abstract, and must have a no-argument constructor.
  Otherwise, if the \texttt{ssj.withThread} system property is set, this returns a
  \class{ThreadProcessSimulator}.
  Otherwise, if the \texttt{ssj.withDSOL} system property is set, this
  returns a \class{DSOLProcessSimulator} instance.
  If no system property is set, this returns a \class{ThreadProcessSimulator}.

  For example, if a program is called using \texttt{java -Dssj.withDSOL ...}, it will use DSOL
  for process simulation.
\end{tabb}
\begin{code}\begin{hide}
}
\end{hide}\end{code}
