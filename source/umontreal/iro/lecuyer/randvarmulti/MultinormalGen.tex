\defclass{MultinormalGen}

Extends \class{RandomMultivariateGen} for a
\emph{multivariate normal} (or {\em multinormal\/}) distribution \cite{tJOH72a}.
The $d$-dimensional multivariate normal distribution
with mean vector $\boldmu\in\RR^d$ and (symmetric positive-definite)
covariance matrix $\boldSigma$, denoted $N(\boldmu, \boldSigma)$, has density
\begin{latexonly}
\[
  f(\bX)=\frac1{\sqrt{(2\pi)^d\det(\boldSigma)}}
    \exp\left(-(\bX - \boldmu)^{\!\tr}\boldSigma^{-1}(\bX -
  \boldmu)/2\right),
\]
\end{latexonly}
\begin{htmlonly}
\[
  f(\bX)=
    \exp\left(-(\bX - \boldmu)^{\!\tr}\boldSigma^{-1}(\bX -
  \boldmu)/2\right) / \sqrt{(2\pi)^d\;\det{\boldSigma}},
\]
\end{htmlonly}
for all $\bX\in\RR^d$, and  $\bX^{\tr}$ is the transpose vector of $\bX$.
If $\bZ \sim N(\bzero, \bI)$ where $\bI$ is the
identity matrix, $\bZ$ is said to have the {\em standard multinormal\/}
 distribution.

For the special case $d=2$, if the random vector $\bX = (X_1, X_2)^\tr$
has a bivariate normal distribution, then it has mean
$\boldmu = (\mu_1, \mu_2)^\tr$, and covariance matrix
\[
\boldSigma =
\left[\begin{array}{cc}
\sigma_1^2 & \rho\sigma_1\sigma_2 \\
\rho\sigma_1\sigma_2 &\sigma_2^2
\end{array}\right]
\]
if and only if $\Var[X_1] = \sigma_1^2$, $\Var[X_2] = \sigma_2^2$, and the
linear correlation between $X_1$ and $X_2$ is $\rho$, where $\sigma_1 > 0$,
 $\sigma_2 > 0$, and $-1 \le \rho \le 1$.


%%%%%%%%%%%%%%%%%%%%%%%%%%%%%%%%%%%%%%%%%%%%%%%%%%%%%%%%%%%%%%%

\bigskip\hrule

\begin{code}
\begin{hide}
/*
 * Class:        MultinormalGen
 * Description:  multivariate normal random variable generator
 * Environment:  Java
 * Software:     SSJ 
 * Copyright (C) 2001  Pierre L'Ecuyer and Université de Montréal
 * Organization: DIRO, Université de Montréal
 * @author       
 * @since

 * SSJ is free software: you can redistribute it and/or modify it under
 * the terms of the GNU General Public License (GPL) as published by the
 * Free Software Foundation, either version 3 of the License, or
 * any later version.

 * SSJ is distributed in the hope that it will be useful,
 * but WITHOUT ANY WARRANTY; without even the implied warranty of
 * MERCHANTABILITY or FITNESS FOR A PARTICULAR PURPOSE.  See the
 * GNU General Public License for more details.

 * A copy of the GNU General Public License is available at
   <a href="http://www.gnu.org/licenses">GPL licence site</a>.
 */
\end{hide}
package umontreal.iro.lecuyer.randvarmulti;
\begin{hide}
import umontreal.iro.lecuyer.probdist.NormalDist;
import umontreal.iro.lecuyer.randvar.NormalGen;
import umontreal.iro.lecuyer.rng.RandomStream;
import cern.colt.matrix.DoubleMatrix2D;
import cern.colt.matrix.linalg.CholeskyDecomposition;
import cern.colt.matrix.impl.DenseDoubleMatrix2D;
\end{hide}

public class MultinormalGen extends RandomMultivariateGen\begin{hide} {
   protected double[] mu;
   protected DoubleMatrix2D sigma;
   protected DoubleMatrix2D sqrtSigma;
   protected double[] temp;
   protected static final double MYINF = 37.54;


   private void initMN (NormalGen gen1, double[] mu, int d) {
      if (gen1 == null)
         throw new NullPointerException ("gen1 is null");

      NormalDist dist = (NormalDist) gen1.getDistribution();
      if (dist.getMu() != 0.0)
         throw new IllegalArgumentException ("mu != 0");
      if (dist.getSigma() != 1.0)
         throw new IllegalArgumentException ("sigma != 1");
      this.gen1 = gen1;
      dist = null;

      if (mu == null) {    // d is the dimension
         dimension = d;
         this.mu = new double[d];
      } else {      // d is unused
         dimension = mu.length;
         this.mu = (double[])mu.clone();
      }
      temp = new double[dimension];
     }\end{hide}\end{code}

\subsubsection* {Constructors}
\begin{code}

   public MultinormalGen (NormalGen gen1, int d)\begin{hide} {
      initMN (gen1, null, d);
      sigma = new DenseDoubleMatrix2D (d, d);
      sqrtSigma = new DenseDoubleMatrix2D (d, d);
      for (int i = 0; i < d; i++) {
         sigma.setQuick (i, i, 1.0);
         sqrtSigma.setQuick (i, i, 1.0);
      }
   }\end{hide}
\end{code}
\begin{tabb} Constructs a generator with the standard multinormal distribution
  (with $\bmu=\bzero$ and $\bSigma = \bI$) in $d$ dimensions.
  Each vector $\bZ$ will be generated via $d$ successive calls to
  \texttt{gen1}, which must be a \emph{standard normal} generator.
\end{tabb}
\begin{htmlonly}
   \param{gen1}{the one-dimensional generator}
   \param{d}{the dimension of the generated vectors}
   \exception{IllegalArgumentException}{if the one-dimensional normal
    generator uses a normal distribution with $\mu$ not equal to 0, or
    $\sigma$ not equal to 1.}
   \exception{IllegalArgumentException}{if \texttt{d}
    is negative.}
   \exception{NullPointerException}{if \texttt{gen1} is \texttt{null}.}
\end{htmlonly}
\begin{code}

   protected MultinormalGen (NormalGen gen1, double[] mu,
                             DoubleMatrix2D sigma)\begin{hide} {
      initMN (gen1, mu, -1);
      this.sigma = sigma.copy();
   }\end{hide}
\end{code}
\begin{tabb}   Constructs a multinormal generator with mean vector
 \texttt{mu} and covariance matrix \texttt{sigma}.
 The mean vector must have the same length as the dimensions
 of the covariance matrix, which must be symmetric and positive-definite.
 If any of the above conditions is violated, an exception is thrown.
 The vector $\bZ$ is generated by calling $d$ times the generator \texttt{gen1},
 which must be \emph{standard normal}.
\end{tabb}
\begin{htmlonly}
   \param{gen1}{the one-dimensional generator}
   \param{mu}{the mean vector.}
   \param{sigma}{the covariance matrix.}
   \exception{NullPointerException}{if any argument is \texttt{null}.}
   \exception{IllegalArgumentException}{if the length of the mean
    vector is incompatible with the dimensions of the covariance matrix.}
\end{htmlonly}
\begin{code}

   protected MultinormalGen (NormalGen gen1, double[] mu, double[][] sigma) \begin{hide} {
      initMN (gen1, mu, -1);
      this.sigma = new DenseDoubleMatrix2D (sigma);
   }\end{hide}
\end{code}
\begin{tabb}   Equivalent to
 \method{MultinormalGen}{(NormalGen, double[], DoubleMatrix2D)} \texttt{(gen1, mu, new DenseDoubleMatrix2D (sigma))}.
\end{tabb}



%%%%%%%%%%%%%%%%%%%%%%%%%%%%%%%%%%%%%%%%%%%%%%%%5
\subsubsection* {Methods}
\begin{code}
   public double[] getMu()\begin{hide} {
      return mu;
   }\end{hide}
\end{code}
\begin{tabb}   Returns the mean vector used by this generator.
\end{tabb}
\begin{htmlonly}
   \return{the current mean vector.}
\end{htmlonly}
\begin{code}

   public double getMu (int i)\begin{hide} {
      return mu[i];
   }\end{hide}
\end{code}
\begin{tabb}   Returns the $i$-th component of the mean vector
 for this generator.
\end{tabb}
\begin{htmlonly}
   \param{i}{the index of the required component.}
   \return{the value of $\mu_i$.}
   \exception{ArrayIndexOutOfBoundsException}{if
    \texttt{i} is negative or greater than or equal to \method{getDimension}{()}.}
\end{htmlonly}
\begin{code}

   public void setMu (double[] mu)\begin{hide} {
      if (mu.length != this.mu.length)
         throw new IllegalArgumentException
            ("Incompatible length of mean vector");
      this.mu = mu;
   }\end{hide}
\end{code}
\begin{tabb}   Sets the mean vector to \texttt{mu}.
\end{tabb}
\begin{htmlonly}
   \param{mu}{the new mean vector.}
   \exception{NullPointerException}{if \texttt{mu} is \texttt{null}.}
   \exception{IllegalArgumentException}{if the length of \texttt{mu}
    does not correspond to \method{getDimension}{()}.}
\end{htmlonly}
\begin{code}

   public void setMu (int i, double mui)\begin{hide} {
      mu[i] = mui;
   }\end{hide}
\end{code}
\begin{tabb}   Sets the $i$-th component of the mean vector to \texttt{mui}.
\end{tabb}
\begin{htmlonly}
   \param{i}{the index of the modified component.}
   \param{mui}{the new value of $\mu_i$.}
   \exception{ArrayIndexOutOfBoundsException}{if \texttt{i}
    is negative or greater than or equal to \method{getDimension}{()}.}
\end{htmlonly}
\begin{code}

   public DoubleMatrix2D getSigma()\begin{hide} {
      return sigma.copy();
   }\end{hide}
\end{code}
\begin{tabb}   Returns the covariance matrix $\boldSigma$
 used by this generator.
\end{tabb}
\begin{htmlonly}
   \return{the used covariance matrix.}
\end{htmlonly}
\begin{code}

   public void nextPoint (double[] p)\begin{hide} {
      int n = dimension;
      for (int i = 0; i < n; i++) {
         temp[i] = gen1.nextDouble();
         if (temp[i] == Double.NEGATIVE_INFINITY)
            temp[i] = -MYINF;
         if (temp[i] == Double.POSITIVE_INFINITY)
            temp[i] = MYINF;
      }
      for (int i = 0; i < n; i++) {
         p[i] = 0;
         for (int c = 0; c < n; c++)
            p[i] += sqrtSigma.getQuick (i, c)*temp[c];
         p[i] += mu[i];
      }
   }\end{hide}
\end{code}
\begin{tabb}   Generates a point from this multinormal distribution.
\end{tabb}
\begin{htmlonly}
   \param{p}{the array to be filled with the generated point}
\end{htmlonly}

\begin{code}\begin{hide}
}\end{hide}
\end{code}
