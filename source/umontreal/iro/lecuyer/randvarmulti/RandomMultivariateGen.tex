\defclass {RandomMultivariateGen}

This class is the multivariate counterpart of
\externalclass{umontreal.iro.lecuyer.randvar}{RandomVariateGen}.
It is the base class for general random variate generators over
the $d$-dimensional real space $\RR^d$.
It specifies the signature of the \method{nextPoint}{} method, which is
normally called to generate a random vector from a given distribution.
Contrary to univariate distributions and generators, here the inversion method
is not well defined, so we cannot construct a multivariate generator simply
by passing a multivariate distribution and a stream; we must specify a
generating method as well.  For this reason, this class is abstract.
Generators can be constructed only by invoking the constructor of a subclass.
This is an important difference with
\externalclass{umontreal.iro.lecuyer.randvar}{RandomVariateGen}.

\hpierre{Je pense qu'il faut enlever les constructeurs de cette classe-ci.}
\hrichard{Je pense que le \texttt{RandomMultivariateGen(RandomStream)}
  doit rester.}

%\externalclass{umontreal.iro.lecuyer.rng}{RandomStream} objects.
%The multi-dimensional generator normally uses one or more one-dimensional
%generators or a primitive stream to generate the components of the points one
%at a time.

\bigskip\hrule
\begin{code}
\begin{hide}
/*
 * Class:        RandomMultivariateGen
 * Description:  base class for multidimensional random variate generators
 * Environment:  Java
 * Software:     SSJ 
 * Copyright (C) 2001  Pierre L'Ecuyer and Université de Montréal
 * Organization: DIRO, Université de Montréal
 * @author       
 * @since

 * SSJ is free software: you can redistribute it and/or modify it under
 * the terms of the GNU General Public License (GPL) as published by the
 * Free Software Foundation, either version 3 of the License, or
 * any later version.

 * SSJ is distributed in the hope that it will be useful,
 * but WITHOUT ANY WARRANTY; without even the implied warranty of
 * MERCHANTABILITY or FITNESS FOR A PARTICULAR PURPOSE.  See the
 * GNU General Public License for more details.

 * A copy of the GNU General Public License is available at
   <a href="http://www.gnu.org/licenses">GPL licence site</a>.
 */
\end{hide}
package umontreal.iro.lecuyer.randvarmulti;\begin{hide}

import umontreal.iro.lecuyer.probdistmulti.ContinuousDistributionMulti;
import umontreal.iro.lecuyer.rng.RandomStream;
import umontreal.iro.lecuyer.randvar.RandomVariateGen;
\end{hide}


public abstract class RandomMultivariateGen \begin{hide} {
   protected int dimension;
   // Careful here: there is also a RandomStream inside gen1. But only one
   // of these two is used in a given class.
   protected RandomStream stream;  // stream used to generate random numbers
   protected RandomVariateGen gen1; // 1-dim generator used to generate random variates
\end{hide}\end{code}

%%%%%%%%%%%%%%%%%%%%%%%%%%%%%%%%%%%%%%%%%%%%%
% \subsubsection* {Constructor}
\begin{hide}\begin{code}
// This constructor is needed for subclasses with no associated distribution.
//   protected RandomMultivariateGen() {}
\end{code}\end{hide}


%%%%%%%%%%%%%%%%%%%%%%%%%%%%%%%%%%%%%%%%%%%%%%
\subsubsection* {Methods}
\begin{code}

   abstract public void nextPoint (double[] p);
\end{code}
  \begin{tabb} Generates a random point $p$ using the
  the stream contained in this object.
 \end{tabb}
\begin{code}

   public void nextArrayOfPoints (double[][] v, int start, int n) \begin{hide} {
      if (n <= 0)
         throw new IllegalArgumentException ("n must be positive.");
      for (int i = 0; i < n; i++)
         nextPoint(v[start + i]);
   }\end{hide}
\end{code}
\begin{tabb}
   Generates $n$ random points. These points are stored in
   the array \texttt{v}, starting at index \texttt{start}. Thus
   \texttt{v[start][i]} contains
   coordinate $i$ of the first generated point.
   By default, this method calls \method{nextPoint}{} $n$
   times, but one can override it in subclasses for better efficiency.
   The array argument \texttt{v[][d]} must have $d$ elements reserved
   for each generated point before calling this method.
 \end{tabb}
\begin{htmlonly}
   \param{v}{array in which the variates will be stored}
   \param{start}{starting index, in \texttt{v}, of the new variates}
   \param{n}{number of variates to generate}
\end{htmlonly}
\begin{code}

   public int getDimension()\begin{hide} {
      return dimension;
   }\end{hide}
\end{code}
\begin{tabb}
   Returns the dimension of this multivariate generator
  (the dimension of the random points).
\end{tabb}
\begin{code}

   public RandomStream getStream() \begin{hide} {
      if (null != gen1)
         return gen1.getStream();
      return stream;
   }\end{hide}
\end{code}
  \begin{tabb}  Returns the \class{RandomStream} used by this object.
 \end{tabb}
\begin{htmlonly}
   \return{the stream associated to this object}
\end{htmlonly}
\begin{code}

   public void setStream (RandomStream stream)\begin{hide} {
      if (null != gen1)
         gen1.setStream(stream);
      else
         this.stream = stream;
   }\end{hide}
\end{code}
\begin{tabb}
   Sets the \class{RandomStream} used by this object to \texttt{stream}.
\end{tabb}

\begin{code}\begin{hide}
}\end{hide}
\end{code}
