\defclass {DiscreteDistributionInt}

Classes implementing discrete distributions over the integers should
inherit from this class.
It specifies the signatures of methods for computing the mass function
(or probability) $p(x) = P[X=x]$, distribution function $F(x)$,
complementary distribution function $\bar F(x)$,
and inverse distribution function $F^{-1}(u)$, for a random variable $X$
with a discrete distribution over the integers.

\emph{WARNING:} the complementary distribution function is defined as
$\bar F(j) = P[X \ge j]$ (for integers $j$, so that for discrete distributions
in SSJ, $F(j) + \bar F(j) \neq 1$ since both include the term $P[X = j]$.

The implementing classes provide both static and non-static methods
to compute the above functions.
The non-static methods require the creation of an object of
class \externalclass{umontreal.iro.lecuyer.probdist}{DiscreteDistributionInt};
all the non-negligible terms of the mass and distribution functions will be
precomputed by the constructor and kept in arrays. Subsequent accesses
will be very fast.
The static methods do not require the construction of an object.
These static methods are not specified in this abstract class because
the number and types of their parameters depend on the distribution.
When methods have to be called several times
with the same parameters  for the distributions,
it is usually more efficient to create an object and use its non-static
methods instead of the static ones.
This trades memory for speed.

\bigskip\hrule

\begin{code}
\begin{hide}
/*
 * Class:        DiscreteDistributionInt
 * Description:  discrete distributions over the integers
 * Environment:  Java
 * Software:     SSJ 
 * Copyright (C) 2001  Pierre L'Ecuyer and Université de Montréal
 * Organization: DIRO, Université de Montréal
 * @author       
 * @since

 * SSJ is free software: you can redistribute it and/or modify it under
 * the terms of the GNU General Public License (GPL) as published by the
 * Free Software Foundation, either version 3 of the License, or
 * any later version.

 * SSJ is distributed in the hope that it will be useful,
 * but WITHOUT ANY WARRANTY; without even the implied warranty of
 * MERCHANTABILITY or FITNESS FOR A PARTICULAR PURPOSE.  See the
 * GNU General Public License for more details.

 * A copy of the GNU General Public License is available at
   <a href="http://www.gnu.org/licenses">GPL licence site</a>.
 */
\end{hide}
package umontreal.iro.lecuyer.probdist;

public abstract class DiscreteDistributionInt implements Distribution\begin{hide} {\end{hide}

   public static double EPSILON = 1.0e-16;\end{code}
 \begin{tabb} Environment variable that determines what probability terms can
  be considered as negligible when building precomputed tables for
  distribution and mass functions.  Probabilities smaller than \texttt{EPSILON}
  are not stored in the
  \externalclass{umontreal.iro.lecuyer.probdist}{DiscreteDistribution} objects
  (such as those of class \class{PoissonDist}, etc.), but are computed
  directly each time they are needed (which should be very seldom).
  The default value is set to $10^{-16}$.
 \end{tabb}
\begin{code}\begin{hide}
  /*
     For better precision in the tails, we keep the cumulative probabilities
     (F) in cdf[x] for x <= xmed (i.e. cdf[x] is the sum off all the probabi-
     lities pdf[i] for i <= x),
     and the complementary cumulative probabilities (1 - F) in cdf[x] for
     x > xmed (i.e. cdf[x] is the sum off all the probabilities pdf[i]
     for i >= x).
  */
   protected final static double EPS_EXTRA = 1.0e-6;
   protected double cdf[] = null;    // cumulative probabilities
   protected double pdf[] = null;    // probability terms or mass distribution
   protected int xmin = 0;           // pdf[x] < EPSILON for x < xmin
   protected int xmax = 0;           // pdf[x] < EPSILON for x > xmax

   // xmed is such that cdf[xmed] >= 0.5 and cdf[xmed - 1] < 0.5.
   protected int xmed = 0;           /* cdf[x] = F(x) for x <= xmed, and
                                        cdf[x] = bar_F(x) for x > xmed */
   protected int supportA = Integer.MIN_VALUE;
   protected int supportB = Integer.MAX_VALUE;
\end{hide}

   public abstract double prob (int x);
\end{code}
\begin{tabb}  Returns $p(x)$, the probability of $x$.
\end{tabb}
\begin{htmlonly}
   \param{x}{value at which the mass function must be evaluated}
   \return{the mass function evaluated at \texttt{x}}
\end{htmlonly}
\begin{code}

   public double cdf (double x)\begin{hide} {
     return cdf ((int) x);
   }\end{hide}
\end{code}
\begin{tabb}  Returns the distribution function $F$ evaluated at $x$
(see (\ref{eq:FDistDisc})).
  Calls the \method{cdf}{int}\texttt{(int)} method.
\end{tabb}
\begin{htmlonly}
   \param{x}{value at which the distribution function must be evaluated}
   \return{the distribution function evaluated at \texttt{x}}
\end{htmlonly}
\begin{code}

   public abstract double cdf (int x);
\end{code}
\begin{tabb}  Returns the distribution function $F$ evaluated at $x$
(see (\ref{eq:FDistDisc})).
%% \eq
%%   F(x) = P[X\le x] = \sum_{s=-\infty}^x p(s).
%% \endeq
\end{tabb}
\begin{htmlonly}
   \param{x}{value at which the distribution function must be evaluated}
   \return{the distribution function evaluated at \texttt{x}}
\end{htmlonly}
\begin{code}

   public double barF (double x)\begin{hide} {
      return barF ((int) x);
   }\end{hide}
\end{code}
\begin{tabb}   Returns $\bar F(x)$, the complementary distribution function.
  Calls the \method{barF}{int}\texttt{(int)} method.
\end{tabb}
\begin{htmlonly}
   \param{x}{value at which the complementary distribution function
    must be evaluated}
   \return{the complementary distribution function evaluated at \texttt{x}}
\end{htmlonly}
\begin{code}

   public double barF (int x)\begin{hide} {
      return 1.0 - cdf (x - 1);
   }\end{hide}
\end{code}
\begin{tabb}  Returns $\bar F(x)$, the complementary
   distribution function. \emph{See the WARNING above}.
\begin{detailed}%
   The default implementation returns \texttt{1.0 - cdf(x - 1)}, which
   is not accurate when $F(x)$ is near 1.
\end{detailed}%
\end{tabb}
\begin{htmlonly}
   \param{x}{value at which the complementary distribution function
    must be evaluated}
   \return{the complementary distribution function evaluated at \texttt{x}}
\end{htmlonly}
\begin{code}

   public int getXinf()\begin{hide} {
      return supportA;
   }\end{hide}
\end{code}
\begin{tabb} Returns the lower limit $x_a$ of the support of the probability
 mass function. The probability is 0 for all $x < x_a$.
\end{tabb}
\begin{htmlonly}
   \return{$x$ lower limit of support}
\end{htmlonly}
\begin{code}

   public int getXsup()\begin{hide} {
      return supportB;
   }\end{hide}
\end{code}
\begin{tabb} Returns the upper limit $x_b$ of the support of the  probability
 mass function. The probability is 0 for all $x > x_b$.
\end{tabb}
\begin{htmlonly}
   \return{$x$ upper limit of support}
\end{htmlonly}
\begin{code}

   public double inverseF (double u)\begin{hide} {
      return inverseFInt (u);
   }\end{hide}
\end{code}
\begin{tabb}  Returns the inverse distribution function
  $F^{-1}(u)$, where $0\le u\le 1$. Calls the \texttt{inverseFInt} method.
\end{tabb}
\begin{htmlonly}
   \param{u}{value in the interval $(0,1)$ for which
             the inverse distribution function is evaluated}
   \return{the inverse distribution function evaluated at \texttt{u}}
   \exception{IllegalArgumentException}{if $u$ is  not in the interval $(0,1)$}
   \exception{ArithmeticException}{if the inverse cannot be computed,
     for example if it would give infinity in a theoritical context}
\end{htmlonly}
\begin{code}

   public int inverseFInt (double u)\begin{hide} {
      int i, j, k;

      if (u < 0.0 || u > 1.0)
         throw new IllegalArgumentException ("u is not in [0,1]");
      if (u <= 0.0)
         return supportA;
      if (u >= 1.0)
         return supportB;


      // Remember: the upper part of cdf contains the complementary distribu-
      // tion for xmed < s <= xmax, and the lower part of cdf the
      // distribution for xmin <= x <= xmed

      if (u <= cdf[xmed - xmin]) {
         // In the lower part of cdf
         if (u <= cdf[0])
            return xmin;
         i = 0;
         j = xmed - xmin;
         while (i < j) {
            k = (i + j) / 2;
            if (u > cdf[k])
               i = k + 1;
            else
               j = k;
         }
      }
      else {
         // In the upper part of cdf
         u = 1 - u;
         if (u < cdf[xmax - xmin])
            return xmax;

         i = xmed - xmin + 1;
         j = xmax - xmin;
         while (i < j) {
            k = (i + j) / 2;
            if (u < cdf[k])
               i = k + 1;
            else
               j = k;
         }
         i--;
      }

      return i + xmin;
   }\end{hide}
\end{code}
\begin{tabb}  Returns the inverse distribution function
  $F^{-1}(u)$, where $0\le u\le 1$.
  The default implementation uses binary search.
\end{tabb}
\begin{htmlonly}
   \param{u}{value in the interval $(0,1)$ for which
             the inverse distribution function is evaluated}
   \return{the inverse distribution function evaluated at \texttt{u}}
   \exception{IllegalArgumentException}{if $u$ is  not in the interval $(0,1)$}
   \exception{ArithmeticException}{if the inverse cannot be computed,
     for example if it would give infinity in a theoritical context}
\end{htmlonly}
\begin{code}\begin{hide}
}\end{hide}
\end{code}

