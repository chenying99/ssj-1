\defclass {NegativeBinomialDist}

Extends the class \class{DiscreteDistributionInt} for
the \emph{negative binomial\/} distribution
\cite[page 324]{sLAW00a} with real
parameters $n$ and $p$, where $n > 0$ and $0\le p\le 1$.
Its mass function is
\begin{htmlonly}
\eq
   p(x) = \Gamma (n + x) / (\Gamma (n)\: x!) p^n (1 - p)^x,
    \qquad\mbox{for } x = 0, 1, 2, \ldots\label{eq:fmass-negbin}
\endeq
\end{htmlonly}
\begin{latexonly}
\eq
   p(x) = \frac{\Gamma(n + x)}{\Gamma (n)\; x!} p^n (1 - p)^x,
    \qquad\mbox{for } x = 0, 1, 2, \ldots \label{eq:fmass-negbin}
\endeq
\end{latexonly}
where $\Gamma(x)$ is the gamma function.

If $n$ is an integer, $p(x)$ can be interpreted as the probability
of having $x$ failures before the $n$-th success in a sequence of
independent Bernoulli trials with  probability of success $p$. This special
case is implemented as the Pascal distribution (see \class{PascalDist}).
% For $n=1$, this gives the \emph{geometric} distribution.


\bigskip\hrule

%%%%%%%%%%%%%%%%%%%%%%%%%%%%%%%%%%%%%%%%
\begin{code}
\begin{hide}
/*
 * Class:        NegativeBinomialDist
 * Description:  negative binomial distribution
 * Environment:  Java
 * Software:     SSJ 
 * Copyright (C) 2001  Pierre L'Ecuyer and Université de Montréal
 * Organization: DIRO, Université de Montréal
 * @author       
 * @since

 * SSJ is free software: you can redistribute it and/or modify it under
 * the terms of the GNU General Public License (GPL) as published by the
 * Free Software Foundation, either version 3 of the License, or
 * any later version.

 * SSJ is distributed in the hope that it will be useful,
 * but WITHOUT ANY WARRANTY; without even the implied warranty of
 * MERCHANTABILITY or FITNESS FOR A PARTICULAR PURPOSE.  See the
 * GNU General Public License for more details.

 * A copy of the GNU General Public License is available at
   <a href="http://www.gnu.org/licenses">GPL licence site</a>.
 */
\end{hide}
package umontreal.iro.lecuyer.probdist;
\begin{hide}
import umontreal.iro.lecuyer.util.*;
import umontreal.iro.lecuyer.functions.MathFunction;
import optimization.*;
\end{hide}

public class NegativeBinomialDist extends DiscreteDistributionInt\begin{hide} {
   protected double n;
   protected double p;
   private static final double EPS2 = 1000.0*EPSILON;

   private static class Func1 implements MathFunction {
      protected int m;
      protected int[] x;
      protected double p;

      public Func1 (double p, int[] x, int m) {
         this.p = p;
         this.m = m;
         this.x = x;
      }

      public double evaluate (double gam) {
         if (gam <= 0 ) return 1.0e100;

         double sum = 0.0;
         for (int j = 0; j < m; j++)
            sum += Num.digamma (gam + x[j]);
         return sum/m + Math.log (p) - Num.digamma (gam);
      }
   }


   private static class Function implements MathFunction {
      protected int m;
      protected int max;
      protected double mean;
      protected int[] Fj;

      public Function (int m, int max, double mean, int[] Fj) {
         this.m = m;
         this.max = max;
         this.mean = mean;
         this.Fj = new int[Fj.length];
         System.arraycopy(Fj, 0, this.Fj, 0, Fj.length);
      }

      public double evaluate (double s) {
         if (s <= 0 ) return 1.0e100;
         double sum = 0.0;
         double p = s / (s + mean);

         for (int j = 0; j < max; j++)
            sum += Fj[j] / (s + (double) j);

         return sum + m * Math.log (p);
      }
   }

/*
   // Class Optim seems to be useless
   private static class Optim implements Lmder_fcn
   {
      private double mean;
      private int N;
      private int max;
      private int [] Fj;

      public Optim (int N, int max, double mean, int[] Fj)
      {
         this.N = N;
         this.max = max;
         this.mean = mean;
         this.Fj = new int[max];
         System.arraycopy (Fj, 0, this.Fj, 0, max);
      }

      public void fcn (int m, int n, double[] x, double[] fvec, double[][] fjac,
                       int iflag[])
      {
         if (x[1] <= 0.0 || x[2] <= 0.0 || x[2] >= 1.0) {
             final double BIG = 1.0e100;
             fvec[1] = BIG;
             fvec[2] = BIG;
             fjac[1][1] = BIG;
             fjac[1][2] = 0.0;
             fjac[2][1] = 0.0;
             fjac[2][2] = BIG;
             return;
         }

         double trig;
         if (iflag[1] == 1)
         {
            double sum = 0.0;
            for (int j = 0; j < max; j++)
               sum += Fj[j] / (x[1] + j);
            fvec[1] = x[2] * mean  - x[1] * (1.0 - x[2]);
            fvec[2] =  N * Math.log (x[2]) + sum;

         } else if (iflag[1] == 2) {

            fjac[1][1] = x[2] - 1.0;
            fjac[1][2] = mean + x[1];
            double sum = 0.0;
            for (int j = 0; j < max; j++)
               sum += Fj[j] / ((x[1] + j)*(x[1] + j));
            fjac[2][1] = -sum;
            fjac[2][2] = N / x[2];
         }
      }
   }
*/
\end{hide}
\end{code}
\begin{detailed}
%%%%%%%%%%%%%%%%%%%%%%%%%%%%%%%%%%%%%%%%%
\unmoved\subsubsection* {Constant}

\begin{code}
   public static double MAXN = 100000;
\end{code}
 \begin{tabb} If the maximum term is greater than this constant,
   then the tables will {\em not\/} be precomputed.
\end{tabb}
\end{detailed}

%%%%%%%%%%%%%%%%%%%%%%%%%%%%%%%%%%%%%%%%%
\subsubsection* {Constructor}
\begin{code}
\begin{hide}
   protected NegativeBinomialDist() {}

\end{hide}

   public NegativeBinomialDist (double n, double p)\begin{hide} {
      setParams (n, p);
   }\end{hide}
\end{code}
 \begin{tabb}
   Creates an object that contains the probability
   terms (\ref{eq:fmass-negbin}) and the distribution function for
   the negative binomial distribution with parameters $n$ and $p$.
 \end{tabb}


%%%%%%%%%%%%%%%%%%%%%%%%%%%%%%%%%%%
\subsubsection* {Methods}
\begin{code}\begin{hide}
   public double prob (int x) {
      if (x < 0)
         return 0.0;

      if (p == 0.0)
         return 0.0;

      if (p == 1.0) {
         if (x > 0)
            return 0.0;
         else
            return 1.0;
      }

      if (pdf == null)
         return prob (n, p, x);

      if (x > xmax || x < xmin)
         return prob (n, p, x);

      return pdf[x - xmin];
   }

   public double cdf (int x) {
      if (x < 0)
         return 0.0;
      if (p >= 1.0)    // In fact, p == 1
         return 1.0;
      if (p <= 0.0)    // In fact, p == 0
         return 0.0;

      if (cdf != null) {
         if (x >= xmax)
            return 1.0;
         if (x < xmin)
            return cdf (n, p, x);
         if (x <= xmed)
            return cdf[x - xmin];
         else
            // We keep the complementary distribution in the upper part of cdf
            return 1.0 - cdf[x + 1 - xmin];

      }
      else
         return cdf (n, p, x);
   }

   public double barF (int x) {
      if (x < 1)
         return 1.0;
      if (p >= 1.0)   // In fact, p == 1
         return 0.0;
      if (p <= 0.0)   // In fact, p == 0
         return 1.0;

      if (cdf == null)
         //return BinomialDist.cdf (x - 1 + n, p, n - 1);
         return BetaDist.barF (n, x, 15, p);

      if (x > xmax)
         //return BinomialDist.cdf (x - 1 + n, p, n - 1);
         return BetaDist.barF (n, x, 15, p);

      if (x <= xmin)
         return 1.0;
      if (x > xmed)
         // We keep the complementary distribution in the upper part of cdf
         return cdf[x - xmin];
      else
         return 1.0 - cdf[x - 1 - xmin];
   }

   public int inverseFInt (double u) {
      if ((cdf == null) || (u <= EPS2))
         return inverseF (n, p, u);
      else
         return super.inverseFInt (u);
   }

   public double getMean() {
      return NegativeBinomialDist.getMean (n, p);
   }

   public double getVariance() {
      return NegativeBinomialDist.getVariance (n, p);
   }

   public double getStandardDeviation() {
      return NegativeBinomialDist.getStandardDeviation (n, p);
   }\end{hide}

   public static double prob (double n, double p, int x)\begin{hide} {
      final int SLIM = 15;           // To avoid overflow
      final double MAXEXP = (Num.DBL_MAX_EXP - 1)*Num.LN2;// To avoid overflow
      final double MINEXP = (Num.DBL_MIN_EXP - 1)*Num.LN2;// To avoid underflow
      double y;

      if (p < 0.0 || p > 1.0)
         throw new IllegalArgumentException ("p not in [0, 1]");
      if (n <= 0.0)
         throw new IllegalArgumentException ("n <= 0.0");
      if (x < 0)
         return 0.0;
      if (p >= 1.0) {                // In fact, p == 1
         if (x == 0)
            return 1.0;
         else
            return 0.0;
      }
      if (p <= 0.0)                  // In fact, p == 0
         return 0.0;

      y = Num.lnGamma (n + x) - (Num.lnFactorial (x) + Num.lnGamma (n))
          + n * Math.log (p) + x * Math.log1p (-p) ;

      if (y >= MAXEXP)
         throw new IllegalArgumentException ("term overflow");
      return Math.exp (y);
   }\end{hide}
\end{code}
  \begin{tabb}
   Computes  the probability $p(x)$%
\latex{ defined in (\ref{eq:fmass-negbin})}.
  \end{tabb}
\begin{code}

   public static double cdf (double n, double p, int x)\begin{hide} {
      final double EPSILON = DiscreteDistributionInt.EPSILON;
      final int LIM1 = 100000;
      double sum, term, termmode;
      int i, mode;
      final double q = 1.0 - p;

      if (p < 0.0 || p > 1.0)
        throw new IllegalArgumentException ("p not in [0, 1]");
      if (n <= 0.0)
        throw new IllegalArgumentException ("n <= 0.0");

      if (x < 0)
         return 0.0;
      if (p >= 1.0)                  // In fact, p == 1
         return 1.0;
      if (p <= 0.0)                  // In fact, p == 0
         return 0.0;

      // Compute the maximum term
      mode = 1 + (int) Math.floor ((n*q - 1.0)/p);
      if (mode < 0)
          mode = 0;
      else if (mode > x)
         mode = x;

      if (mode <= LIM1) {
         sum = term = termmode = prob (n, p, mode);
         for (i = mode; i > 0; i--) {
            term *= i/(q*(n + i - 1.0));
            if (term < EPSILON)
               break;
            sum += term;
         }

         term = termmode;
         for (i = mode; i < x; i++) {
            term *= q*(n + i)/(i + 1);
            if (term < EPSILON)
               break;
            sum += term;
         }
         if (sum <= 1.0)
            return sum;
         else
            return 1.0;
      }
      else
         //return 1.0 - BinomialDist.cdf (x + n, p, n - 1);
         return BetaDist.cdf (n, x + 1.0, 15, p);
    }\end{hide}
\end{code}
  \begin{tabb} Computes the distribution function.
 \end{tabb}
\begin{code}

   public static double barF (double n, double p, int x)\begin{hide} {
      return 1.0 - cdf (n, p, x - 1);
   }\end{hide}
\end{code}
\begin{tabb}  Returns $\bar F(x) = P[X \ge x]$, the complementary
   distribution function.
\end{tabb}
\begin{code}

   public static int inverseF (double n, double p, double u)\begin{hide} {
      if (u < 0.0 || u > 1.0)
         throw new IllegalArgumentException ("u is not in [0,1]");
      if (p < 0.0 || p > 1.0)
         throw new IllegalArgumentException ("p not in [0, 1]");
      if (n <= 0.0)
         throw new IllegalArgumentException ("n <= 0");
      if (p >= 1.0)                  // In fact, p == 1
         return 0;
      if (p <= 0.0)                  // In fact, p == 0
         return 0;
      if (u <= prob (n, p, 0))
         return 0;
      if (u >= 1.0)
         return Integer.MAX_VALUE;

      double sum, term, termmode;
      final double q = 1.0 - p;

      // Compute the maximum term
      int mode = 1 + (int) Math.floor ((n * q - 1.0) / p);
      if (mode < 0)
         mode = 0;
      int i = mode;
      term = prob (n, p, i);
      while ((term >= u) && (term > Double.MIN_NORMAL)) {
         i /= 2;
         term = prob (n, p, i);
      }

      if (term <= Double.MIN_NORMAL) {
         i *= 2;
         term = prob (n, p, i);
         while (term >= u && (term > Double.MIN_NORMAL)) {
            term *= i / (q * (n + i - 1.0));
            i--;
         }
      }

      mode = i;
      sum = termmode = prob (n, p, i);

      for (i = mode; i > 0; i--) {
         term *= i / (q * (n + i - 1.0));
         if (term < EPSILON)
            break;
         sum += term;
      }

      term = termmode;
      i = mode;
      double prev = -1;
      if (sum < u) {
         // The CDF at the mode is less than u, so we add term to get >= u.
         while ((sum < u) && (sum > prev)){
            term *= q * (n + i) / (i + 1);
            prev = sum;
            sum += term;
            i++;
         }
      } else {
         // The computed CDF is too big so we substract from it.
         sum -= term;
         while (sum >= u) {
            term *= i / (q * (n + i - 1.0));
            i--;
            sum -= term;
         }
      }
      return i;
   }\end{hide}
\end{code}
\begin{tabb}  Computes the inverse function without precomputing tables.
%%  If the mode is too large, it will
%%    use the fact that $F_{n,p}(x)=B_{x+n,p}(n-1)$, where $B_{x+n,p}$ is the
%%    distribution function of a binomial variable with parameters $(x+n, p)$,
%%    to perform linear search.
%%    The latter case is not efficient and therefore should be avoided.
\end{tabb}
\begin{code}

   public static double[] getMLE (int[] x, int m, double n)\begin{hide} {
      if (m <= 0)
         throw new IllegalArgumentException ("m <= 0");
      double mean = 0.0;
      for (int i = 0; i < m; i++) {
         mean += x[i];
      }
      mean /= (double) m;
      double[] param = new double[1];
      param[0] = n / (n + mean);
      return param;
   }\end{hide}
\end{code}
\begin{tabb}
   Estimates the parameter $p$ of the negative binomial distribution
   using the maximum likelihood method, from the $m$ observations
   $x[i]$, $i = 0, 1, \ldots, m-1$. The parameter
   $n$ is assumed known.
   The estimate $\hat{p}$ is returned in element 0
   of the returned array.
   The maximum likelihood estimator $\hat{p}$ satisfies the equation
   $\hat{p} = n /(n + \bar{x}_m)$,
   where  $\bar{x}_m$ is the average of $x[0], \ldots, x[m-1]$.
\end{tabb}
\begin{htmlonly}
   \param{x}{the list of observations used to evaluate parameters}
   \param{m}{the number of observations used to evaluate parameters}
   \param{n}{the first parameter of the negative binomial}
   \return{returns the parameters [$\hat{p}$]}
\end{htmlonly}
\begin{code}

   public static NegativeBinomialDist getInstanceFromMLE (int[] x, int m,
                                                          double n)\begin{hide} {
      double parameters[] = getMLE (x, m, n);
      return new NegativeBinomialDist (n, parameters[0]);
   }\end{hide}
\end{code}
\begin{tabb}
   Creates a new instance of a negative binomial distribution with parameters
  $n$ given and $\hat{p}$ estimated using the maximum
   likelihood method, from the $m$ observations $x[i]$,
   $i = 0, 1, \ldots, m-1$.
\end{tabb}
\begin{htmlonly}
   \param{x}{the list of observations to use to evaluate parameters}
   \param{m}{the number of observations to use to evaluate parameters}
   \param{n}{the first parameter of the negative binomial}
\end{htmlonly}
\begin{code}

   public static double[] getMLE1 (int[] x, int m, double p)\begin{hide} {
      if (m <= 0)
         throw new IllegalArgumentException ("m <= 0");
      double mean = 0.0;
      for (int i = 0; i < m; i++)
         mean += x[i];
      mean /= m;

      double gam0 = mean*p/(1.0 - p);
      double[] param = new double[1];
      Func1 f = new Func1 (p, x, m);
      param[0] = RootFinder.brentDekker (gam0/10.0, 10.0*gam0, f, 1e-5);
      return param;
   }\end{hide}
\end{code}
\begin{tabb}
   Estimates the parameter $n$ of the negative binomial distribution
   using the maximum likelihood method, from the $m$ observations
   $x[i]$, $i = 0, 1, \ldots, m-1$. The parameter $p$ is assumed known.
   The estimate $\hat{n}$ is returned in element 0 of the returned array.
\begin{detailed}
   The maximum likelihood estimator $\hat{p}$ satisfies the equation
 \[
  \frac1m\sum_{j=0}^{m-1} \psi(n +x_j) = \psi(n) - \ln(p)
 \]
   where $\psi(x)$ is the digamma function, i.e. the logarithmic derivative
  of the Gamma function $\psi(x) = \Gamma^\prime(x)/\Gamma(x)$.
 \end{detailed}
\end{tabb}
\begin{htmlonly}
   \param{x}{the list of observations used to evaluate parameters}
   \param{m}{the number of observations used to evaluate parameters}
   \param{p}{the second parameter of the negative binomial}
   \return{returns the parameters [$\hat{n}$]}
\end{htmlonly}
\begin{code}

   public static NegativeBinomialDist getInstanceFromMLE1 (int[] x, int m,
                                                           double p)\begin{hide} {
      double param[] = getMLE1 (x, m, p);
      return new NegativeBinomialDist (param[0], p);
   }\end{hide}
\end{code}
\begin{tabb}
   Creates a new instance of a negative binomial distribution with parameters
  $p$ given and $\hat{n}$ estimated using the maximum
   likelihood method, from the $m$ observations $x[i]$,
   $i = 0, 1, \ldots, m-1$.
\end{tabb}
\begin{htmlonly}
   \param{x}{the list of observations to use to evaluate parameters}
   \param{m}{the number of observations to use to evaluate parameters}
   \param{p}{the second parameter of the negative binomial}
\end{htmlonly}
\begin{code}

   public static double[] getMLE (int[] x, int m)\begin{hide} {
      double estimGamma;
      if (m <= 0)
         throw new IllegalArgumentException ("m<= 0");

      double sum = 0.0;
      double max = Integer.MIN_VALUE;
      for (int i = 0; i < m; i++)
      {
         sum += x[i];
         if (x[i] > max)
            max = x[i];
      }
      double mean = (double) sum / (double) m;

      double var = 0.0;
      for (int i = 0; i < m; i++)
         var += (x[i] - mean) * (x[i] - mean);
      var /= (double) m;

      if (mean >= var)
          throw new UnsupportedOperationException("mean >= variance");

      estimGamma = (mean * mean) / ( var - mean );

      int[] Fj = new int[(int) max];
      for (int j = 0; j < max; j++)
      {
         int prop = 0;
         for (int i = 0; i < m; i++)
            if (x[i] > j)
               prop++;

         Fj[j] = prop;
      }

      double[] param = new double[3];
      Function f = new Function (m, (int) max, mean, Fj);
      param[1] = RootFinder.brentDekker (estimGamma/10, estimGamma*10, f, 1e-5);

      param[2] = param[1] / (param[1] + mean);

/* Seems to be useless
      Optim system = new Optim (m, (int) max, mean, Fj);
      double[] fvec = new double [3];
      double[][] fjac = new double[3][3];
      int[] iflag = new int[2];
      int[] info = new int[2];
      int[] ipvt = new int[3];

      Minpack_f77.lmder1_f77 (system, 2, 2, param, fvec, fjac, 1e-5, info, ipvt);
*/
      double parameters[] = new double[2];
      parameters[0] = param[1];
      parameters[1] = param[2];

      return parameters;
   }\end{hide}
\end{code}
\begin{tabb}
   Estimates the parameter $(n, p)$ of the negative binomial distribution
   using the maximum likelihood method, from the $m$ observations
   $x[i]$, $i = 0, 1, \ldots, m-1$. The estimates are returned in a two-element
    array, in regular order: [$n$, $p$].
   \begin{detailed}
   The maximum likelihood estimators are the values $(\hat{n}$, $\hat{p})$
   satisfying the equations
   \begin{eqnarray*}
      \frac{\hat{n}(1 - \hat{p})}{\hat{p}} & = & \bar{x}_m\\
     \sum_{j=1}^{\infty} \frac{F_j}{(\hat{n} + j - 1)}  & = & -m\ln (\hat{p})
   \end{eqnarray*}
   where  $\bar x_m$ is the average of $x[0],\dots,x[m-1]$, and
   $F_j = \sum_{i=j}^{\infty} f_i$ = number of $x_i \ge j$ (see
   \cite[page 132]{tJOH69a}).
   \end{detailed}
\end{tabb}
\begin{htmlonly}
   \param{x}{the list of observations used to evaluate parameters}
   \param{m}{the number of observations used to evaluate parameters}
   \return{returns the parameters [$\hat{n}$, $\hat{p}$]}
\end{htmlonly}
\begin{code}

   public static NegativeBinomialDist getInstanceFromMLE (int[] x, int m)\begin{hide} {
      double parameters[] = getMLE (x, m);
      return new NegativeBinomialDist (parameters[0], parameters[1]);
   }\end{hide}
\end{code}
\begin{tabb}
   Creates a new instance of a negative binomial distribution with
   parameters $n$ and $p$ estimated using the maximum likelihood method
   based on the $m$ observations $x[i]$,   $i = 0, 1, \ldots, m-1$.
\end{tabb}
\begin{htmlonly}
   \param{x}{the list of observations to use to evaluate parameters}
   \param{m}{the number of observations used to evaluate parameters}
\end{htmlonly}
\begin{code}

   public static double getMean (double n, double p)\begin{hide} {
      if (p < 0.0 || p > 1.0)
         throw new IllegalArgumentException ("p not in [0, 1]");
      if (n <= 0.0)
         throw new IllegalArgumentException ("n <= 0");

      return (n * (1 - p) / p);
   }\end{hide}
\end{code}
\begin{tabb}  Computes and returns the mean $E[X] = n(1 - p)/p$
    of the negative binomial distribution with parameters $n$ and $p$.
\end{tabb}
\begin{htmlonly}
   \return{the mean of the negative binomial distribution
    $E[X] = n(1 - p) / p$}
\end{htmlonly}
\begin{code}

   public static double getVariance (double n, double p)\begin{hide} {
      if (p < 0.0 || p > 1.0)
         throw new IllegalArgumentException ("p not in [0, 1]");
      if (n <= 0.0)
         throw new IllegalArgumentException ("n <= 0");

      return (n * (1 - p) / (p * p));
   }\end{hide}
\end{code}
\begin{tabb}  Computes and returns the variance $\mbox{Var}[X] = n(1
- p)/p^2$
   of the negative binomial distribution with parameters $n$ and $p$.
\end{tabb}
\begin{htmlonly}
   \return{the variance of the negative binomial distribution
$\mbox{Var}[X] = n(1 - p) / p^2$}
\end{htmlonly}
\begin{code}

   public static double getStandardDeviation (double n, double p)\begin{hide} {
      return Math.sqrt (NegativeBinomialDist.getVariance (n, p));
   }\end{hide}
\end{code}
\begin{tabb}  Computes and returns the standard deviation of the negative
   binomial distribution with parameters $n$ and $p$.
\end{tabb}
\begin{htmlonly}
   \return{the standard deviation of the negative binomial distribution}
\end{htmlonly}
\begin{hide}\begin{code}

   @Deprecated
   public double getGamma() {
      return n;
   }
\end{code}
\begin{tabb} Returns the parameter $n$ of this object.
\end{tabb}\end{hide}
\begin{code}

   public double getN()\begin{hide} {
      return n;
   }\end{hide}
\end{code}
\begin{tabb} Returns the parameter $n$ of this object.
\end{tabb}
\begin{code}

   public double getP()\begin{hide} {
      return p;
   }\end{hide}
\end{code}
\begin{tabb} Returns the parameter $p$ of this object.
\end{tabb}
\begin{code}

   public void setParams (double n, double p)\begin{hide} {
      /**
      *  Compute all probability terms of the negative binomial distribution;
      *  start at the mode, and calculate probabilities on each side until they
      *  become smaller than EPSILON. Set all others to 0.
      */
      supportA = 0;
      int i, mode, Nmax;
      int imin, imax;
      double sum;
      double[] P;     // Negative Binomial mass probabilities
      double[] F;     // Negative Binomial cumulative

      if (p < 0.0 || p > 1.0)
         throw new IllegalArgumentException ("p not in [0, 1]");
      if (n <= 0.0)
         throw new IllegalArgumentException ("n <= 0");

      this.n  = n;
      this.p  = p;

      // Compute the mode (at the maximum term)
      mode = 1 + (int) Math.floor((n*(1.0 - p) - 1.0)/p);

      /**
       For mode > MAXN, we shall not use pre-computed arrays.
       mode < 0 should be impossible, unless overflow of long occur, in
       which case mode will be = LONG_MIN.
      */

      if (mode < 0.0 || mode > MAXN) {
         pdf = null;
         cdf = null;
         return;
      }

      /**
        In theory, the negative binomial distribution has an infinite range.
        But for i > Nmax, probabilities should be extremely small.
        Nmax = Mean + 16 * Standard deviation.
      */

      Nmax = (int)(n*(1.0 - p)/p + 16*Math.sqrt (n*(1.0 - p)/(p*p)));
      if (Nmax < 32)
         Nmax = 32;
      P = new double[1 + Nmax];

      double epsilon = EPSILON/prob (n, p, mode);

      // We shall normalize by explicitly summing all terms >= epsilon
      sum = P[mode] = 1.0;

      // Start from the maximum and compute terms > epsilon on each side.
      i = mode;
      while (i > 0 && P[i] >= epsilon) {
         P[i - 1] = P[i]*i/((1.0 - p)*(n + i - 1));
         i--;
         sum += P[i];
      }
      imin = i;

      i = mode;
      while (P[i] >= epsilon) {
         P[i + 1] = P[i]*(1.0 - p)*(n + i)/(i + 1);
         i++;
         sum += P[i];
         if (i == Nmax - 1) {
            Nmax *= 2;
            double[] nT = new double[1 + Nmax];
            System.arraycopy (P, 0, nT, 0, P.length);
            P = nT;
         }
      }
      imax = i;

      // Renormalize the sum of probabilities to 1
      for (i = imin; i <= imax; i++)
         P[i] /= sum;

      // Compute the cumulative probabilities for F and keep them in the
      // lower part of CDF.
      F = new double[1 + Nmax];
      F[imin] = P[imin];
      i = imin;
      while (i < imax && F[i] < 0.5) {
         i++;
         F[i] = F[i - 1] + P[i];
      }

      // This is the boundary between F (i <= xmed) and 1 - F (i > xmed) in
      // the array CDF
      xmed = i;

      // Compute the cumulative probabilities of the complementary
      // distribution 1 - F and keep them in the upper part of the array
      F[imax] = P[imax];
      i = imax - 1;
      do {
         F[i] = P[i] + F[i + 1];
         i--;
      } while (i > xmed);

     xmin = imin;
     xmax = imax;
     pdf = new double[imax + 1 - imin];
     cdf = new double[imax + 1 - imin];
     System.arraycopy (P, imin, pdf, 0, imax + 1 - imin);
     System.arraycopy (F, imin, cdf, 0, imax + 1 - imin);

   }\end{hide}
\end{code}
\begin{tabb} Sets the parameter $n$ and $p$ of this object.
\end{tabb}
\begin{code}

   public double[] getParams ()\begin{hide} {
      double[] retour = {n, p};
      return retour;
   }\end{hide}
\end{code}
\begin{tabb}
   Return a table containing the parameters of the current distribution.
   This table is put in regular order: [$n$, $p$].
\end{tabb}
\begin{hide}\begin{code}

   public String toString ()\begin{hide} {
      return getClass().getSimpleName() + " : n = " + n + ", p = " + p;
   }\end{hide}
\end{code}
\begin{tabb}
   Returns a \texttt{String} containing information about the current distribution.
\end{tabb}\end{hide}
\begin{code}\begin{hide}
}\end{hide}
\end{code}
