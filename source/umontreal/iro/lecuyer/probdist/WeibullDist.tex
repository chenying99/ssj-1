\defclass {WeibullDist}

This class extends the class \class{ContinuousDistribution} for
the {\em Weibull\/} distribution \cite[page 628]{tJOH95a} with shape parameter
$\alpha > 0$, location parameter $\delta$, and scale parameter
$\lambda > 0$.
The density function is
\eq
  f(x) = \alpha\lambda^{\alpha}(x-\delta)^{\alpha-1}
       e^{-(\lambda(x-\delta))^{\alpha}}
 \qquad\mbox{for }x>\delta\latex{,}\html{.} \eqlabel {eq:fweibull}
\endeq
the distribution function is
\eq
   F(x) = 1 - e^{-(\lambda (x - \delta))^\alpha}
 \qquad\mbox{for }x>\delta,               \eqlabel{eq:Fweibull}
\endeq
and the inverse distribution function is
$$
     F^{-1}(u) = (-\ln (1-u))^{1/\alpha}/\lambda
                 + \delta \qquad \mbox{for } 0 \le u < 1.
$$

\bigskip\hrule

%%%%%%%%%%%%%%%%%%%%%%%%%%%%%%%%%%%%%%%%
\begin{code}\begin{hide}
/*
 * Class:        WeibullDist
 * Description:  Weibull distribution
 * Environment:  Java
 * Software:     SSJ 
 * Copyright (C) 2001  Pierre L'Ecuyer and Université de Montréal
 * Organization: DIRO, Université de Montréal
 * @author       
 * @since

 * SSJ is free software: you can redistribute it and/or modify it under
 * the terms of the GNU General Public License (GPL) as published by the
 * Free Software Foundation, either version 3 of the License, or
 * any later version.

 * SSJ is distributed in the hope that it will be useful,
 * but WITHOUT ANY WARRANTY; without even the implied warranty of
 * MERCHANTABILITY or FITNESS FOR A PARTICULAR PURPOSE.  See the
 * GNU General Public License for more details.

 * A copy of the GNU General Public License is available at
   <a href="http://www.gnu.org/licenses">GPL licence site</a>.
 */
\end{hide}
package umontreal.iro.lecuyer.probdist;
\begin{hide}
import umontreal.iro.lecuyer.util.Num;
import umontreal.iro.lecuyer.util.RootFinder;
import umontreal.iro.lecuyer.functions.MathFunction;
\end{hide}
public class WeibullDist extends ContinuousDistribution\begin{hide} {
   private double alpha;
   private double lambda;
   private double delta;

   private static class Function implements MathFunction {
      private int n;
      private double xi[];
      private double lnXi[];
      private double sumLnXi = 0.0;
      private final double LN_EPS = Num.LN_DBL_MIN - Num.LN2;

      public Function (double x[], int n)
      {
         this.n = n;
         this.xi = new double[n];
         this.lnXi = new double[n];

         for (int i = 0; i < n; i++)
         {
            this.xi[i] = x[i];
            if (x[i] > 0.0)
               this.lnXi[i] = Math.log (x[i]);
            else
               this.lnXi[i] = LN_EPS;
            sumLnXi += this.lnXi[i];
         }
      }

      public double evaluate (double x)
      {
         if (x <= 0.0) return 1.0e200;
         double sumXiLnXi = 0.0;
         double sumXi = 0.0;
         double xalpha;

         for (int i = 0; i < n; i++)
         {
            xalpha = Math.pow (this.xi[i], x);
            sumXiLnXi += xalpha * lnXi[i];
            sumXi += xalpha;
         }

         return (x * (n * sumXiLnXi - sumLnXi * sumXi) - n * sumXi);
      }
   }

\end{hide}
\end{code}
%%%%%%%%%%%%%%%%%%%%%%%%%%%%%%%%%%%%%%%%%
\subsubsection* {Constructors}

\begin{code}

   public WeibullDist (double alpha)\begin{hide} {
      setParams (alpha, 1.0, 0.0);
   }\end{hide}
\end{code}
 \begin{tabb} Constructs a \texttt{WeibullDist} object with parameters
    $\alpha$ = \texttt{alpha}, $\lambda$ = 1, and $\delta$ = 0.
 \end{tabb}
\begin{code}

   public WeibullDist (double alpha, double lambda, double delta)\begin{hide} {
      setParams (alpha, lambda, delta);
   }\end{hide}
\end{code}
\begin{tabb} Constructs a \texttt{WeibullDist} object with parameters
     $\alpha =$ \texttt{alpha},
     $\lambda $ = \texttt{lambda}, and $\delta$ = \texttt{delta}.
 \end{tabb}

%%%%%%%%%%%%%%%%%%%%%%%%%%%%%%%%%%%
\subsubsection* {Methods}

\begin{code}\begin{hide}

   public double density (double x) {
      return density (alpha, lambda, delta, x);
   }

   public double cdf (double x) {
      return cdf (alpha, lambda, delta, x);
   }

   public double barF (double x) {
      return barF (alpha, lambda, delta, x);
   }

   public double inverseF (double u) {
      return inverseF (alpha, lambda, delta, u);
   }

   public double getMean() {
      return WeibullDist.getMean (alpha, lambda, delta);
   }

   public double getVariance() {
      return WeibullDist.getVariance (alpha, lambda, delta);
   }

   public double getStandardDeviation() {
      return WeibullDist.getStandardDeviation (alpha, lambda, delta);
   }\end{hide}

   public static double density (double alpha, double lambda,
                                 double delta, double x)\begin{hide} {
      if (alpha <= 0.0)
        throw new IllegalArgumentException ("alpha <= 0");
      if (lambda <= 0.0)
        throw new IllegalArgumentException ("lambda <= 0");
      if (x <= delta)
         return 0.0;
      double y = Math.log(lambda*(x - delta)) * alpha;
      if (y >= 7.0)
         return 0.0;
      y = Math.exp(y);

      return alpha * (y / (x - delta)) * Math.exp(-y);
   }\end{hide}
\end{code}
\begin{tabb} Computes the density function.
\end{tabb}
\begin{code}

   public static double density (double alpha, double x)\begin{hide} {
      return density (alpha, 1.0, 0.0, x);
   }\end{hide}
\end{code}
\begin{tabb} Same as \texttt{density (alpha, 1, 0, x)}.
\end{tabb}
\begin{code}

   public static double cdf (double alpha, double lambda,
                             double delta, double x)\begin{hide} {
      if (alpha <= 0.0)
        throw new IllegalArgumentException ("alpha <= 0");
      if (lambda <= 0.0)
        throw new IllegalArgumentException ("lambda <= 0");
      if (x <= delta)
         return 0.0;
      if ((lambda*(x - delta) >= XBIG) && (alpha >= 1.0))
         return 1.0;
      double y = Math.log(lambda*(x - delta)) * alpha;
      if (y >= 3.65)
         return 1.0;
      y = Math.exp(y);
      return -Math.expm1 (-y);   // in JDK-1.5
   }\end{hide}
\end{code}
 \begin{tabb}
  Computes the distribution function.
 \end{tabb}
\begin{code}

   public static double cdf (double alpha, double x)\begin{hide} {
      return cdf (alpha, 1.0, 0.0, x);
   }\end{hide}
\end{code}
\begin{tabb} Same as \texttt{cdf (alpha, 1, 0, x)}.
\end{tabb}
\begin{code}

   public static double barF (double alpha, double lambda,
                              double delta, double x)\begin{hide} {
      if (alpha <= 0)
        throw new IllegalArgumentException ("alpha <= 0");
      if (lambda <= 0)
        throw new IllegalArgumentException ("lambda <= 0");
      if (x <= delta)
         return 1.0;
      if (alpha >= 1.0 && x >= Num.DBL_MAX_EXP*2)
         return 0.0;

      double temp = Math.log (lambda*(x - delta)) * alpha;
      if (temp >= Num.DBL_MAX_EXP * Num.LN2)
         return 0.0;
      temp = Math.exp(temp);
      return Math.exp (-temp);
   }\end{hide}
\end{code}
  \begin{tabb}
 Computes  the complementary distribution function.
 \end{tabb}
\begin{code}

   public static double barF (double alpha, double x)\begin{hide} {
      return barF (alpha, 1.0, 0.0, x);
   }\end{hide}
\end{code}
\begin{tabb} Same as \texttt{barF (alpha, 1, 0, x)}.
\end{tabb}
\begin{code}

   public static double inverseF (double alpha, double lambda,
                                  double delta, double u)\begin{hide} {
        double t;
        if (alpha <= 0.0)
            throw new IllegalArgumentException ("alpha <= 0");
        if (lambda <= 0.0)
            throw new IllegalArgumentException ("lambda <= 0");

        if (u < 0.0 || u > 1.0)
            throw new IllegalArgumentException ("u not in [0, 1]");
        if (u <= 0.0)
           return 0.0;
        if (u >= 1.0)
           return Double.POSITIVE_INFINITY;

        t = -Math.log1p (-u);
        if (Math.log (t)/Math.log (10) >= alpha*Num.DBL_MAX_10_EXP)
           throw new ArithmeticException
              ("inverse function cannot be positive infinity");

        return Math.pow (t, 1.0/alpha)/lambda + delta;
   }\end{hide}
\end{code}
  \begin{tabb}
  Computes  the inverse of the distribution function.
 \end{tabb}
\begin{code}

   public static double inverseF (double alpha, double x)\begin{hide} {
      return inverseF (alpha, 1.0, 0.0, x);
   }\end{hide}
\end{code}
\begin{tabb} Same as \texttt{inverseF (alpha, 1, 0, x)}.
\end{tabb}
\begin{code}\begin{hide}
   private static double[] getMaximumLikelihoodEstimate (double[] x, int n,
                                                         double delta) {
      if (n <= 0)
         throw new IllegalArgumentException ("n <= 0");
      if (delta != 0.0)
         throw new IllegalArgumentException ("delta must be equal to 0");
// Verifier cette fonction si delta != 0.

      final double LN_EPS = Num.LN_DBL_MIN - Num.LN2;
      double sumLn = 0.0;
      double sumLn2 = 0.0;
      double lnxi;
      for (int i = 0; i < n; i++) {
         if (x[i] <= delta)
            lnxi = LN_EPS;
         else
            lnxi = Math.log (x[i]);
         sumLn += lnxi;
         sumLn2 += lnxi * lnxi;
      }

      double alpha0 = Math.sqrt ((double) n / ((6.0 / (Math.PI * Math.PI)) *
                  (sumLn2 - sumLn * sumLn / (double) n)));
      double a = alpha0 - 20.0;
      if (a <= delta)
         a = delta + 1.0e-5;

      double param[] = new double[3];
      param[2] = 0.0;
      Function f = new Function (x, n);
      param[0] = RootFinder.brentDekker (a, alpha0 + 20.0, f, 1e-5);

      double sumXalpha = 0.0;
      for (int i = 0; i < n; i++)
         sumXalpha += Math.pow (x[i], param[0]);
      param[1] = Math.pow ((double) n / sumXalpha, 1.0 / param[0]);

      return param;
   }\end{hide}

   public static double[] getMLE (double[] x, int n)\begin{hide}
   {
      return getMaximumLikelihoodEstimate (x, n, 0.0);
   }\end{hide}
\end{code}
\begin{tabb}
   Estimates the parameters $(\alpha, \lambda)$ of the Weibull  distribution,
   assuming that $\delta = 0$,
    using the maximum likelihood method, from the $n$ observations
   $x[i]$, $i = 0, 1, \ldots, n-1$. The estimates are returned in a two-element
    array, in regular order: [$\alpha$, $\lambda$].
   \begin{detailed}
   The maximum likelihood estimators are the values
   $(\hat{\alpha}$, $\hat{\lambda})$ that satisfy the equations
   \begin{eqnarray*}
      \frac{\sum_{i=1}^{n} x_i^{\hat{\alpha}} \ln(x_i)}{\sum_{i=1}^{n}
  x_i^{\hat{\alpha}}} - \frac{1}{\hat{\alpha}} & = & \frac{\sum_{i=1}^{n} \ln(x_i)}{n}\\
      \hat{\lambda} & = & \left( \frac{n}{\sum_{i=1}^{n} x_i^{\hat{\alpha}}}
    \right)^{1/\hat{\alpha}}
   \end{eqnarray*}
  See \cite[page 303]{sLAW00a}.
   \end{detailed}
\end{tabb}
\begin{htmlonly}
   \param{x}{the list of observations to use to evaluate parameters}
   \param{n}{the number of observations to use to evaluate parameters}
   \return{returns the parameter [$\hat{\alpha}$, $\hat{\lambda}$, $\hat{\delta}$ = 0]}
\end{htmlonly}
\begin{code}

   public static WeibullDist getInstanceFromMLE (double[] x, int n)\begin{hide} {
      double param[] = getMLE (x, n);
      return new WeibullDist (param[0], param[1], param[2]);
   }\end{hide}
\end{code}
\begin{tabb}
   Creates a new instance of a Weibull distribution with parameters $\alpha$,
   $\lambda$ and $\delta = 0$
   estimated using the maximum likelihood method based on the $n$ observations
   $x[i]$, $i = 0, 1, \ldots, n-1$.
\end{tabb}
\begin{htmlonly}
   \param{x}{the list of observations to use to evaluate parameters}
   \param{n}{the number of observations to use to evaluate parameters}
\end{htmlonly}
\begin{code}

   public static double getMean (double alpha, double lambda, double delta)\begin{hide} {
      if (alpha <= 0.0)
        throw new IllegalArgumentException ("alpha <= 0");
      if (lambda <= 0.0)
        throw new IllegalArgumentException ("lambda <= 0");

      return (delta + Math.exp (Num.lnGamma(1.0 + 1.0 / alpha)) / lambda);
   }\end{hide}
\end{code}
\begin{tabb}  Computes and returns the mean
\begin{latexonly}
   $E[X] = \delta + \Gamma(1 + 1/\alpha)/\lambda$
\end{latexonly}
   of the Weibull distribution with parameters $\alpha$, $\lambda$ and $\delta$.
\end{tabb}
\begin{htmlonly}
   \return{the mean of the Weibull distribution
    $E[X] = \delta + \Gamma(1 + 1/\alpha) / \lambda$}
\end{htmlonly}
\begin{code}

   public static double getVariance (double alpha, double lambda,
                                     double delta)\begin{hide} {
      double gAlpha;

      if (alpha <= 0.0)
        throw new IllegalArgumentException ("alpha <= 0");
      if (lambda <= 0.0)
        throw new IllegalArgumentException ("lambda <= 0");

      gAlpha = Math.exp (Num.lnGamma (1.0 / alpha + 1.0));

      return (Math.abs (Math.exp (Num.lnGamma(2 / alpha + 1)) - gAlpha * gAlpha) / (lambda * lambda));
   }\end{hide}
\end{code}
\begin{tabb}  Computes and returns the variance
\begin{latexonly}
   $\mbox{Var}[X] = | \Gamma(2/\alpha + 1) - \Gamma^2(1/\alpha + 1) | /\lambda^2$
\end{latexonly}
   of the Weibull distribution with parameters $\alpha$, $\lambda$ and $\delta$.
\end{tabb}
\begin{htmlonly}
   \return{the variance of the Weibull distribution
    $\mbox{Var}[X] = 1 / \lambda^2 | \Gamma(2/\alpha + 1) - \Gamma^2(1/\alpha + 1) |$}
\end{htmlonly}
\begin{code}

   public static double getStandardDeviation (double alpha, double lambda,
                                              double delta)\begin{hide} {
      return Math.sqrt (WeibullDist.getVariance (alpha, lambda, delta));
   }\end{hide}
\end{code}
\begin{tabb}  Computes and returns the standard deviation
   of the Weibull distribution with parameters $\alpha$, $\lambda$ and $\delta$.
\end{tabb}
\begin{htmlonly}
   \return{the standard deviation of the Weibull distribution}
\end{htmlonly}
\begin{code}

   public double getAlpha()\begin{hide} {
      return alpha;
   }\end{hide}
\end{code}
  \begin{tabb} Returns the parameter $\alpha$.
  \end{tabb}
\begin{code}

   public double getLambda()\begin{hide} {
      return lambda;
   }\end{hide}
\end{code}
  \begin{tabb} Returns the parameter $\lambda$.
  \end{tabb}
\begin{code}

   public double getDelta()\begin{hide} {
      return delta;
   }\end{hide}
\end{code}
  \begin{tabb} Returns the parameter $\delta$.
  \end{tabb}
\begin{code}

   public void setParams (double alpha, double lambda, double delta)\begin{hide} {
      if (alpha <= 0.0)
        throw new IllegalArgumentException ("alpha <= 0");
      if (lambda <= 0.0)
        throw new IllegalArgumentException ("lambda <= 0");

      this.alpha  = alpha;
      this.lambda = lambda;
      this.delta  = delta;
      supportA = delta;
   }\end{hide}
\end{code}
  \begin{tabb} Sets the parameters $\alpha$, $\lambda$ and $\delta$ for this
   object.
  \end{tabb}
\begin{code}

   public double[] getParams ()\begin{hide} {
      double[] retour = {alpha, lambda, delta};
      return retour;
   }\end{hide}
\end{code}
\begin{tabb}
   Return a table containing the parameters of the current distribution.
   This table is put in regular order: [$\alpha$, $\lambda$, $\delta$].
\end{tabb}
\begin{hide}\begin{code}

   public String toString ()\begin{hide} {
      return getClass().getSimpleName() + " : alpha = " + alpha + ", lambda = " + lambda + ", delta = " + delta;
   }\end{hide}
\end{code}
\begin{tabb}
   Returns a \texttt{String} containing information about the current distribution.
\end{tabb}\end{hide}
\begin{code}\begin{hide}
}\end{hide}
\end{code}
