\latex{\section*{Overview}\addcontentsline{toc}{subsection}{Overview}}

This package contains a set of Java classes providing methods to 
compute mass, density, distribution, complementary
distribution, and inverse distribution functions for some discrete
and continuous probability distributions.
It also provides methods to estimate the parameters of some distributions
from empirical data.
It does not generate random variates;
for that, see the package \externalclass{umontreal.iro.lecuyer}{randvar}.
It is possible to plot the density or the cumulative probabilities of a
distribution function either on screen, or in a \LaTeX{} file, but for this, 
one has to use the package 
\externalclass{umontreal.iro.lecuyer}{charts}.



%%%%%%%%%%%%%%%%%%%%%%%%%%%%%%
\subsection* {Distributions}

We recall that the {\em distribution function\/} of a {\em continuous\/} 
random variable $X$ with {\em density\/} $f$ over the real line is
\eq
  F(x) = P[X\le x] = \int_{-\infty}^x f(s)ds    \label{eq:FDist}
\endeq
while that of a {\em discrete\/} random variable $X$ with 
{\em mass function\/} $p$ over a fixed set of real numbers 
$x_0 < x_1 < x_2 < \cdots$ is
\eq
  F(x) = P[X\le x] = \sum_{x_i\le x} p(x_i),     \label{eq:FDistDisc}
\endeq
where $p(x_i) = P[X = x_i]$.
For a discrete distribution over the set of integers, one has
\eq
  F (x) = P[X\le x] = \sum_{s=-\infty}^x p(s),   \label{eq:FDistDiscInt}
\endeq
where $p(s) = P[X=s]$.

We define $\bar{F}$, the {\em complementary distribution function\/} 
of $X$, by 
\eq
 \bar{F} (x) = P[X\ge x].
\endeq
With this definition of $\bar{F}$, one has
$\bar{F}(x) = 1 - F (x)$ for continuous distributions and
$\bar{F}(x) = 1 - F (x-1)$ for discrete distributions over the integers.
This definition is \emph{non-standard} for the discrete case: we have 
$\bar{F} (x) = P[X\ge x]$ instead of $\bar{F} (x) = P[X > x] = 1-F(x)$.
We find it more convenient especially for computing $p$-values in 
goodness-of-fit tests.

The {\em inverse distribution function\/} is defined as 
\eq
   F^{-1}(u) = \inf \{x\in\RR : F (x)\ge u\}, \label{eq:inverseF}
\endeq
for $0\le u\le 1$.
This function $F^{-1}$ is often used, among other things, to generate
the random variable $X$ by inversion, by passing a
$U (0,1)$ random variate as the value of $u$.

The package \texttt{probdist} offers two types of tools for computing 
$p$, $f$, $F$, $\bar{F}$, and $F^{-1}$: 
\emph{static methods}, for which no object needs to
be created, and methods associated with \emph{distribution objects}.
% Both types of methods are provided by the class that corresponds to
% the desired distribution.
% A number of static methods for computing $F$, $\bar{F}$, and $F^{-1}$,
% are regrouped in the static classes \texttt{FDist}, \texttt{FBar}, and
% \texttt{FInverse}, respectively.
Standard distributions are implemented each in their own class.
% (Poisson, binomial, normal, chi-square, etc.) 
Constructing an object from one of these classes can be convenient 
if $F$, $\bar{F}$, etc., has to be evaluated several times for the same
distribution. In certain cases (for the Poisson distribution, for example),
creating the distribution object would precompute tables that
would speed up significantly all subsequent method calls for computing
$F$, $\bar{F}$, etc. 
This trades memory, plus a one-time setup cost, for speed.
In addition to the non-static methods, the distribution classes also
provide static methods that do not require the creation of an object.

The distribution classes extend one of the (abstract) classes
\externalclass{umontreal.iro.lecuyer.probdist}{DiscreteDistribution} and
\externalclass{umontreal.iro.lecuyer.probdist}{ContinuousDistribution}
(which both implement the interface
\externalclass{umontreal.iro.lecuyer.probdist}{Distribution})
for discrete and continuous distributions over the real numbers,
or \externalclass{umontreal.iro.lecuyer.probdist}{DiscreteDistributionInt},
for discrete distributions over the non-negative integers.

For example, the class
\externalclass{umontreal.iro.lecuyer.probdist}{PoissonDist}  extends 
% \externalclass{umontreal.iro.lecuyer.probdist}{BinomialDist} extends 
\externalclass{umontreal.iro.lecuyer.probdist}{DiscreteDistributionInt}.
Calling a static method from this class
% \externalclass{umontreal.iro.lecuyer.probdist}{PoissonDist} 
will compute the corresponding probability from scratch.
Constructing a \externalclass{umontreal.iro.lecuyer.probdist}{PoissonDist} 
object, on the other hand,
will precompute tables that contain the probability terms and the 
distribution function for a given parameter $\lambda$ (the mean of the
Poisson distribution).  These tables will then be used whenever
a method is called for the corresponding object.
This second approach is recommended if some of $F$, $\bar{F}$, etc., 
has to be computed several times for the same parameter $\lambda$.
As a rule of thumb, creating objects and using their methods
is faster than just using static methods as soon as two or three
calls are made, unless the parameters are large.

In fact, only the non-negligible probability terms 
(those that exceed the threshold
\clsexternalmethod{umontreal.iro.lecuyer.probdist}{Discrete\-DistributionInt}{EPSILON}{}) 
are stored in the tables.  For $F$ and $\bar{F}$, a single table actually
contains $F (x)$ for $F (x) \le 1/2$ and $1-F (x)$ for $F (x) > 1/2$.
When the distribution parameters are so large that the tables would
take too much space, these are not created and the methods
automatically call their static equivalents instead of using tables.

% For the continuous distributions, 
Objects that implement the interface
\externalclass{umontreal.iro.lecuyer.probdist}{Distribution} 
(and sometimes the abstract class
\externalclass{umontreal.iro.lecuyer.probdist}{ContinuousDistribution}) 
are required by some methods in 
package \externalclass{umontreal.iro.lecuyer}{randvar}
and also in classes
\externalclass{umontreal.iro.lecuyer.gof}{GofStat} and
\externalclass{umontreal.iro.lecuyer.gof}{GofFormat} of
 package \externalclass{umontreal.iro.lecuyer}{gof}.


Some of the classes also provide methods that compute parameter estimations
of the corresponding distribution from a set of empirical observations, 
in most cases based on the maximum likelihood method.

