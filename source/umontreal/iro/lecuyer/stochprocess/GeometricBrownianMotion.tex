\defclass {GeometricBrownianMotion}

Represents a \emph{geometric Brownian motion} (GBM) process $\{S(t),\, t\ge 0\}$,  
which evolves according to the stochastic differential equation 
\begin{equation}
   dS(t) = \mu S(t) dt + \sigma S(t) dB(t),
                                               \label{eq:GBM}
\end{equation}
where $\mu$ and $\sigma$ are the drift and volatility parameters,
and $\{B(t),\, t\ge 0\}$ is a standard Brownian motion 
(for which $B(t)\sim N(0,t)$).
This process can also be written as the exponential of a Brownian motion:
\begin{equation}
  S(t) = S(0) \exp \left[ (\mu - \sigma^{2}/2) t + \sigma t B(t) \right]
       = S(0) \exp \left[ X(t) \right],
                                               \label{eq:GBM2}
\end{equation}
where $X(t) = (\mu - \sigma^{2}/2) t + \sigma t B(t)$.
The GBM process is simulated by simulating the BM process $X$ and taking the exponential.
This BM process is stored internally.

\bigskip\hrule\bigskip

%%%%%%%%%%%%%%%%%%%%%%%%%%%%%%%%%%%%%%%%%%%%%%%%%%%%%%%%%%%%%%%%%%
\begin{code}
\begin{hide}
/*
 * Class:        GeometricBrownianMotion
 * Description:  
 * Environment:  Java
 * Software:     SSJ 
 * Copyright (C) 2001  Pierre L'Ecuyer and Université de Montréal
 * Organization: DIRO, Université de Montréal
 * @author       
 * @since

 * SSJ is free software: you can redistribute it and/or modify it under
 * the terms of the GNU General Public License (GPL) as published by the
 * Free Software Foundation, either version 3 of the License, or
 * any later version.

 * SSJ is distributed in the hope that it will be useful,
 * but WITHOUT ANY WARRANTY; without even the implied warranty of
 * MERCHANTABILITY or FITNESS FOR A PARTICULAR PURPOSE.  See the
 * GNU General Public License for more details.

 * A copy of the GNU General Public License is available at
   <a href="http://www.gnu.org/licenses">GPL licence site</a>.
 */
\end{hide}
package umontreal.iro.lecuyer.stochprocess;\begin{hide}
import umontreal.iro.lecuyer.rng.*;
import umontreal.iro.lecuyer.probdist.*;
import umontreal.iro.lecuyer.randvar.*;

\end{hide}

public class GeometricBrownianMotion extends StochasticProcess \begin{hide} {

    protected NormalGen      gen;
    protected BrownianMotion bm;   // The underlying BM process X.
    protected double         mu,
                             sigma;
    protected double[]       mudt;
\end{hide}
\end{code}
%%%%%%%%%%%%%%%%%%%%%%%%%%%%%%%%%%%%%%%%%%%%%%%%%%%%%%%%%%%%%%%%
\subsubsection* {Constructors}
\begin{code}

   public GeometricBrownianMotion (double s0, double mu, double sigma,
                                   RandomStream stream) \begin{hide} {
        this (s0, mu, sigma, new BrownianMotion (0.0, 0.0, 1.0, stream));
    }\end{hide}
\end{code}
\begin{tabb} 
Same as \texttt{GeometricBrownianMotion (s0, mu, sigma,
new BrownianMotion (0.0, 0.0, 1.0, stream))}.
\end{tabb}
\begin{code}

   public GeometricBrownianMotion (double s0, double mu, double sigma,
                                   BrownianMotion bm) \begin{hide} {
        this.bm = bm;
        setParams (s0, mu, sigma);
    }\end{hide}
\end{code}
\begin{tabb} 
Constructs a new \texttt{GeometricBrownianMotion} with parameters
 $\mu = \texttt{mu}$, $\sigma = \texttt{sigma}$, and $S(t_0) = \texttt{s0}$,
using \texttt{bm} as the underlying \class{BrownianMotion}.
The parameters of \texttt{bm} are automatically reset to
$\mu-\sigma^2/2$ and $\sigma$, regardless of the original parameters
 of \texttt{bm}.
The observation times are the same as those of \texttt{bm}. The generation
method depends on that of \texttt{bm} (sequential, bridge sampling, PCA, etc.).
\end{tabb}

%%%%%%%%%%%%%%%%%%%%%%%%%%%%%%%%%%%%%%
\subsubsection* {Methods}

\begin{code}\begin{hide}
   public void setObservationTimes (double[] t, int d) {
        this.d = d;
        super.setObservationTimes (t, d);
        bm.setObservationTimes (t, d);
    }

   public double nextObservation() {
        // Note : this implementation is general, to deal with
        // the possibility of generating bm with bridge sampling, for example.  ???

        double s = x0 * Math.exp (bm.nextObservation());
        observationIndex = bm.getCurrentObservationIndex();
        path[observationIndex] = s;
        // Could be different than simply 'observationCounter++' because of the
        // possibility of Brownian bridge

        return s;
    }

   public double[] generatePath() {
        path[0] = x0;
        bm.generatePath ();
        for (int i = 1; i <= d; ++i)
            path[i] = x0 * Math.exp (bm.getObservation(i));
        observationCounter = d;
        return path;
    }

   public double[] generatePath (RandomStream stream) {
        setStream (stream);
        return generatePath();
    }\end{hide}

   public void resetStartProcess() \begin{hide} {
        observationCounter = 0;
        bm.resetStartProcess();
    }\end{hide}
\end{code}
\begin{tabb} Same as in \texttt{StochasticProcess}, but also invokes 
\texttt{resetStartProcess} for the underlying \texttt{BrownianMotion} object.
\end{tabb}
\begin{code}

   public void setParams (double s0, double mu, double sigma) \begin{hide} { 
        this.x0    = s0;
        this.mu    = mu;
        this.sigma = sigma;
        bm.setParams (0.0, mu - 0.5 * sigma * sigma, sigma);
        if (observationTimesSet) init(); // Otherwise not needed.
    }\end{hide}
\end{code}
\begin{tabb} 
Sets the parameters $S(t_{0}) = \texttt{s0}$, $\mu = \texttt{mu}$ and 
$\sigma = \texttt{sigma}$ of the process. 
\emph{Warning}: This method will recompute some quantities stored internally, 
which may be slow if called repeatedly.
\end{tabb}
\begin{code}

   public void setStream (RandomStream stream) \begin{hide} { (bm.gen).setStream (stream); }\end{hide}
\end{code}
\begin{tabb} 
Resets the \externalclass{umontreal.iro.lecuyer.rng}{RandomStream} 
for the underlying Brownian motion to \texttt{stream}.
\end{tabb}
\begin{code}

   public RandomStream getStream() \begin{hide} { return (bm.gen).getStream (); }\end{hide}
\end{code}
\begin{tabb} 
Returns the \externalclass{umontreal.iro.lecuyer.rng}{RandomStream}
for the underlying Brownian motion.
\end{tabb}
\begin{code}

   public double getMu() \begin{hide} { return mu; }\end{hide}
\end{code}
\begin{tabb} Returns the value of $\mu$.
\end{tabb}
\begin{code}

   public double getSigma() \begin{hide} { return sigma; }\end{hide}
\end{code}
\begin{tabb} Returns the value of $\sigma$. 
\end{tabb}
\begin{code}

   public NormalGen getGen() \begin{hide} { return gen; }\end{hide}
\end{code}
\begin{tabb} 
Returns the \externalclass{umontreal.iro.lecuyer.randvar}{NormalGen} used.
\end{tabb}
\begin{code}

   public BrownianMotion getBrownianMotion() \begin{hide} { 
        return bm;
    }\end{hide}
\end{code}
\begin{tabb} Returns a reference to the \class{BrownianMotion} object
used to generate the process.
\end{tabb}

\begin{code} \begin{hide}

    protected void init() {
        super.init();   // Maybe useless...
    }

}
\end{hide}\end{code}
