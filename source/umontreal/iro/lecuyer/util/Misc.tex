\defclass{Misc}

This class provides miscellaneous functions that are hard to classify.
Some may be moved to another class in the future.

\bigskip\hrule

\begin{code}\begin{hide}
/*
 * Class:        Misc
 * Description:  Miscellaneous functions that are hard to classify.
 * Environment:  Java
 * Software:     SSJ 
 * Copyright (C) 2001  Pierre L'Ecuyer and Université de Montréal
 * Organization: DIRO, Université de Montréal
 * @author       
 * @since

 * SSJ is free software: you can redistribute it and/or modify it under
 * the terms of the GNU General Public License (GPL) as published by the
 * Free Software Foundation, either version 3 of the License, or
 * any later version.

 * SSJ is distributed in the hope that it will be useful,
 * but WITHOUT ANY WARRANTY; without even the implied warranty of
 * MERCHANTABILITY or FITNESS FOR A PARTICULAR PURPOSE.  See the
 * GNU General Public License for more details.

 * A copy of the GNU General Public License is available at
   <a href="http://www.gnu.org/licenses">GPL licence site</a>.
 */
\end{hide}
package umontreal.iro.lecuyer.util;\begin{hide}

import umontreal.iro.lecuyer.functions.MathFunction;\end{hide}


public class Misc\begin{hide} {
   private Misc() {}\end{hide}
\end{code}

%%%%%%%%%%%%%%%%%%%%%%%%%%
\subsubsection* {Methods}

\begin{code}

   public static double quickSelect (double[] t, int n, int k)\begin{hide} {
      double[] U = new double[n];
      double[] V = new double[n];
      double p = t[k - 1];
      int u = 0;
      int v = 0;
      int indV = 0;

      for (int i = 0; i < n; i++) {
         if (t[i] <= p) {
            v++;
            if (t[i] != p) {
               U[u++] = t[i];
            }
         } else
            V[indV++] = t[i];
      }

      if (k <= u)
         return quickSelect (U, u, k);
      else if (k > v)
         return quickSelect (V, indV, k - v);
      else return p;
   }\end{hide}
\end{code}
\begin{tabb}
   Returns the $k^{\mbox{th}}$ smallest item of the array $t$ of size $n$.
   Array $t$ is unchanged by the method.
\end{tabb}
\begin{htmlonly}
   \param{t}{the array which contain the items}
   \param{n}{the number of items in the array}
   \param{k}{the index of the smallest item}
   \return{the kth smallest item of the array $t$}
\end{htmlonly}
\begin{code}

   public static int quickSelect (int[] t, int n, int k)\begin{hide} {
      int[] U = new int[n];
      int[] V = new int[n];
      int p = t[k - 1];
      int u = 0;
      int v = 0;
      int indV = 0;

      for (int i = 0; i < n; i++) {
         if (t[i] <= p) {
            v++;
            if (t[i] != p) {
               U[u++] = t[i];
            }
         } else
            V[indV++] = t[i];
      }

      if (k <= u)
         return quickSelect (U, u, k);
      else if (k > v)
         return quickSelect (V, indV, k - v);
      else return p;
   }\end{hide}
\end{code}
\begin{tabb}
   Returns the $k^{\mbox{th}}$ smallest item of the array $t$ of size $n$.
   Array $t$ is unchanged by the method.
\end{tabb}
\begin{htmlonly}
   \param{t}{the array which contain the items}
   \param{n}{the number of items in the array}
   \param{k}{the index of the smallest item}
   \return{the kth smallest item of the array $t$}
\end{htmlonly}
\begin{code}

   public static int getTimeInterval (double[] times, int start, int end,
                                      double t)\begin{hide} {
      if (start < 0)
         throw new IllegalArgumentException
            ("The starting index must not be negative");
      int n = end - start;
      if (n < 0)
         throw new IllegalArgumentException
            ("The ending index must be greater than or equal to the starting index");
      if (t < times[start])
         return -1;
      if (t >= times[end])
         return n;

      int start0 = start;
      // Perform binary search to find the interval index
      int mid = (start + end)/2;
      // Test if t is inside the interval mid.
      // The interval mid starts at times[mid],
      // and the interval mid+1 starts at times[mid + 1].
      while (t < times[mid] || t >= times[mid + 1]) {
         if (start == end)
            // Should not happen, safety check to avoid infinite loops.
            throw new IllegalStateException();
         if (t < times[mid])
            // time corresponds to an interval before mid.
            end = mid - 1;
         else
            // time corresponds to an interval after mid.
            start = mid + 1;
         mid = (start + end)/2;
      }
      return mid - start0;
   }\end{hide}
\end{code}
\begin{tabb}   Returns the index of the time interval corresponding to time \texttt{t}.
 Let $t_0\le\cdots\le t_n$ be simulation times stored in a subset of
 \texttt{times}.  This method uses binary search to determine the
 smallest value $i$ for which $t_i\le t < t_{i+1}$, and returns $i$.
 The value of $t_i$ is stored in \texttt{times[start+i]} whereas
 $n$ is defined as \texttt{end - start}.
 If $t<t_0$, this returns $-1$.  If $t\ge t_n$, this returns $n$.
 Otherwise, the returned value is greater than or equal to 0, and
 smaller than or equal to $n-1$. \texttt{start} and \texttt{end} are only used
 to set lower and upper limits of the search in the \texttt{times}
 array; the index space of the returned value always starts at 0.
 Note that if the elements of \texttt{times} with indices \texttt{start},
 \ldots, \texttt{end} are not sorted in non-decreasing order,
 the behavior of this method is undefined.
\end{tabb}
\begin{htmlonly}
   \param{times}{an array of simulation times.}
   \param{start}{the first index in the array to consider.}
   \param{end}{the last index (inclusive) in the array to consider.}
   \param{t}{the queried simulation time.}
   \return{the index of the interval.}
   \exception{NullPointerException}{if \texttt{times} is \texttt{null}.}
   \exception{IllegalArgumentException}{if \texttt{start} is negative,
    or if \texttt{end} is smaller than \texttt{start}.}
   \exception{ArrayIndexOutOfBoundsException}{if \texttt{start + end}
    is greater than or equal to the length of \texttt{times}.}
\end{htmlonly}
\begin{code}

   public static void interpol (int n, double[] X, double[] Y, double[] C)\begin{hide} {
      int j;
      // Compute divided differences for the Newton interpolating polynomial
      for (j = 0; j <= n; ++j)
         C[j] = Y[j];
      for (int i = 1; i <= n; ++i)
         for (j = n; j >= i; --j) {
            if (X[j] == X[j-i])
               C[j] = 0;
            else
               C[j] = (C[j] - C[j-1]) / (X[j] - X[j-i]);
         }
   }\end{hide}
\end{code}
\begin{tabb} Computes the Newton interpolating polynomial.  Given the $n+1$
 real distinct points $(x_0, y_0),$ $(x_1, y_1),\ldots, (x_n, y_n)$,
  with \texttt{X[i]} $= x_i$, \texttt{Y[i]} $= y_i$, this function computes
  the $n+1$ coefficients \texttt{C[i]} $= c_i$ of the Newton
  interpolating polynomial $P(x)$ of degree $n$ passing through these points,
  i.e. such that $y_i= P(x_i)$, given by
\begin{equation}
\qquad  P(x) = c_0 + c_1(x-x_0) + c_2(x-x_0)(x-x_1) + \cdots +
    c_n(x-x_0)(x-x_1) \cdots(x-x_{n-1}).    \label{eq.newton.interpol}
\end{equation}
\end{tabb}
\begin{htmlonly}
   \param{n}{degree of the interpolating polynomial}
   \param{X}{$x$-coordinates of points}
   \param{Y}{$y$-coordinates of points}
   \param{C}{Coefficients of the interpolating polynomial}
\end{htmlonly}
\begin{code}

   public static double evalPoly (int n, double[] X, double[] C, double z) \begin{hide} {
      double v = C[n];
      for (int j = n-1; j >= 0; --j)
         v = v*(z - X[j]) + C[j];
      return v;
   }\end{hide}
\end{code}
\begin{tabb} Given $n$, $X$ and $C$ as described in
 \method{interpol}{int,double[],double[],double[]}\texttt{(n, X, Y, C)}, this
function returns the value of the interpolating polynomial $P(z)$ evaluated
 at $z$ (see eq. \ref{eq.newton.interpol}).
\end{tabb}
\begin{htmlonly}
   \param{n}{degree of the interpolating polynomial}
   \param{X}{$x$-coordinates of points}
   \param{C}{Coefficients of the interpolating polynomial}
   \param{z}{argument where polynomial is evaluated}
   \return{Value of the interpolating polynomial $P(z)$}
\end{htmlonly}
\begin{code}

   public static double evalPoly (double[] C, int n, double x) \begin{hide} {
      double v = C[n];
      for (int j = n-1; j >= 0; --j)
         v = v*x + C[j];
      return v;
   }\end{hide}
\end{code}
\begin{tabb} Evaluates the polynomial 
\begin{equation}
\qquad  P(x) = c_0 + c_1 x + c_2 x^2 + \cdots + c_n x^n  \label{eq.horner}
\end{equation}
of degree $n$ with coefficients $c_j =$ \texttt{C[j]} at $x$. 
\end{tabb}
\begin{htmlonly}
   \param{C}{Coefficients of the polynomial}
   \param{n}{degree of the polynomial}
   \param{x}{argument where polynomial is evaluated}
   \return{Value of the polynomial $P(x)$}
\end{htmlonly}

\begin{code}\begin{hide}
}\end{hide}
\end{code}
