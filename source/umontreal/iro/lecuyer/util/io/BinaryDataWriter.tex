\defclass{BinaryDataWriter}

Binary data writer.

Stores a sequence of fields in binary file, which can be either atoms or arrays,
each of which having the following format:
\begin{itemize}
   \item Field label:
      \begin{itemize}
         \item Pipe character (\texttt{|})
         \item Label length (32-bit integer, big endian)
         \item Label string (array of bytes of the specified length)
      \end{itemize}
   
   \item Field type (byte):
      \begin{itemize}
         \item \texttt{i} (32-bit integer)
         \item \texttt{f} (32-bit float)
         \item \texttt{d} (64-bit double)
         \item \texttt{S} (string)
      \end{itemize}
   
   \item Number of dimensions (8-bit integer)
   \item Dimensions (array of 32-bit integers, big endian)
   \item Field data (in the specified format, big endian)
\end{itemize}

In the case of an atomic field, the number of dimensions is set to zero.

A string field is stored in the following format:
\begin{itemize}
   \item String length (32-bit integer)
   \item Array of bytes of the specified length
\end{itemize}

Also supports anonymous fields (fields with an empty label).

Arrays up to two dimensions are supported.

Modules for reading data exported with this class are available in Java (\class{BinaryDataReader}), Matlab and Python (numpy).

\begin{comment}
Provide links for the import modules.
\end{comment}

\bigskip\hrule
%%%%%%%%%%%%%%%%%%%%%%%%%%%%%%%%%%%%%%%%%%%%%%%%%%%%%%%%%%%%%%%%%%

\begin{code}
\begin{hide}
/*
 * Class:        BinaryDataWriter
 * Description:  Binary data writer
 * Environment:  Java
 * Software:     SSJ 
 * Copyright (C) 2001  Pierre L'Ecuyer and Université de Montréal
 * Organization: DIRO, Université de Montréal
 * @author       David Munger 
 * @since        August 2009

 * SSJ is free software: you can redistribute it and/or modify it under
 * the terms of the GNU General Public License (GPL) as published by the
 * Free Software Foundation, either version 3 of the License, or
 * any later version.

 * SSJ is distributed in the hope that it will be useful,
 * but WITHOUT ANY WARRANTY; without even the implied warranty of
 * MERCHANTABILITY or FITNESS FOR A PARTICULAR PURPOSE.  See the
 * GNU General Public License for more details.

 * A copy of the GNU General Public License is available at
   <a href="http://www.gnu.org/licenses">GPL licence site</a>.
 */
\end{hide}
package umontreal.iro.lecuyer.util.io;
\begin{hide}
import java.io.*;
\end{hide}

public class BinaryDataWriter extends AbstractDataWriter \begin{hide} {
   protected DataOutputStream out;
   
   /**
    * Utility method to write string data.
    *
    */    
   protected void writeStringData(String s) throws IOException {
      if (s != null) {
         out.writeInt(s.length());
         out.writeBytes(s);
      }
      else {
         out.writeInt(0);
      }
   }
   
   /**
    * Starts a new field by writing its label.
    *
    * @param label   name of the field (can be {@code null})
    *
    */
   protected void writeLabel(String label) throws IOException {
      out.writeByte(TYPECHAR_LABEL);
      writeStringData(label);
   }
\end{hide}
\end{code}

%%%%%%%%%%%%%%%%%%%%%%%%%%%%
\subsubsection*{Fields}

\begin{code}
   public final static byte TYPECHAR_LABEL   = '|';
\end{code}
\begin{tabb}
\emph{Field-type} symbol indicating a label (it more accurately a field separator symbol).
\end{tabb}
\begin{code}

   public final static byte TYPECHAR_STRING  = 'S';
\end{code}
\begin{tabb}
\emph{Field-type} symbol indicating \texttt{String} data.
\end{tabb}
\begin{code}

   public final static byte TYPECHAR_INTEGER = 'i';
\end{code}
\begin{tabb}
\emph{Field-type} symbol indicating \texttt{int} data.
\end{tabb}
\begin{code}

   public final static byte TYPECHAR_FLOAT   = 'f';
\end{code}
\begin{tabb}
\emph{Field-type} symbol indicating \texttt{float} data.
\end{tabb}
\begin{code}

   public final static byte TYPECHAR_DOUBLE  = 'd';
\end{code}
\begin{tabb}
\emph{Field-type} symbol indicating \texttt{double} data.
\end{tabb}


%%%%%%%%%%%%%%%%%%%%%%%%%%%%
\subsubsection*{Constructors}

\begin{code}

   public BinaryDataWriter (String filename, boolean append)
         throws IOException \begin{hide} {
      this.out = new DataOutputStream(new FileOutputStream(filename, append));
   }
   \end{hide}
\end{code}
\begin{tabb}
 Data will be output to the file with the specified name.
\end{tabb}
\begin{htmlonly}
\param{filename}{name of the file to be created or appended to}
\param{append}  {an existing file with the specified name will be appended to if \texttt{true} or truncated if \texttt{false}}
\end{htmlonly}
\begin{code}
   
   public BinaryDataWriter (File file, boolean append) throws IOException \begin{hide} {
      this.out = new DataOutputStream(new FileOutputStream(file, append));
   }
   \end{hide}
\end{code}
\begin{tabb}
 Data will be output to the specified file.
\end{tabb}
\begin{htmlonly}
\param{file}    {file to be created or appended to}
\param{append}  {an existing file with the specified name will be appended to if \texttt{true} or truncated if \texttt{false}}
\end{htmlonly}
\begin{code}
   
   public BinaryDataWriter (String filename) throws IOException \begin{hide} {
      this.out = new DataOutputStream(new FileOutputStream(filename));
   }
   \end{hide}
\end{code}
\begin{tabb}
 Truncates any existing file with the specified name.
\end{tabb}
\begin{htmlonly}
\param{filename}{name of the file to be created}
\end{htmlonly}
\begin{code}

   public BinaryDataWriter (File file) throws IOException \begin{hide} {
      this.out = new DataOutputStream(new FileOutputStream(file));
   }
   \end{hide}
\end{code}
\begin{tabb}
 Truncates any existing file with the specified name.
\end{tabb}
\begin{htmlonly}
\param{file}{file to be created}
\end{htmlonly}
\begin{code}

   public BinaryDataWriter (OutputStream outputStream) throws IOException \begin{hide} {
      this.out = new DataOutputStream(outputStream);
   }
   \end{hide}
\end{code}
\begin{tabb}
Constructor.
\end{tabb}
\begin{htmlonly}
\param{outputStream} {output stream to write to}
\end{htmlonly}

   
%%%%%%%%%%%%%%%%%%%%%%%%%%%%
\subsubsection*{Writing atomic data}

\begin{code}
   public void write (String label, String s) throws IOException \begin{hide} {
      writeLabel(label);
      out.writeByte(TYPECHAR_STRING);
      out.writeByte(0);
      writeStringData(s);
   }
   \end{hide}
\end{code}
\begin{tabb}
Writes an atomic string field.
Writes an anonymous field if \texttt{label} is \texttt{null}.
\end{tabb}
\begin{code}

   public void write (String label, int a) throws IOException \begin{hide} {
      writeLabel(label);
      out.writeByte(TYPECHAR_INTEGER);
      out.writeByte(0);
      out.writeInt(a);
   }
   \end{hide}
\end{code}
\begin{tabb}
Writes an atomic 32-bit integer (big endian).
Writes an anonymous field if \texttt{label} is \texttt{null}.
\end{tabb}
\begin{code}

   public void write (String label, float a) throws IOException \begin{hide} {
      writeLabel(label);
      out.writeByte(TYPECHAR_FLOAT);
      out.writeByte(0);
      out.writeFloat(a);
   }
   \end{hide}
\end{code}
\begin{tabb}
Writes an atomic 32-bit float  (big endian).
Writes an anonymous field if \texttt{label} is \texttt{null}.
\end{tabb}
\begin{code}

   public void write (String label, double a) throws IOException \begin{hide} {
      writeLabel(label);
      out.writeByte(TYPECHAR_DOUBLE);
      out.writeByte(0);
      out.writeDouble(a);
   }
   \end{hide}
\end{code}
\begin{tabb}
Writes an atomic 64-bit double (big endian).
Writes an anonymous field if \texttt{label} is \texttt{null}.
\end{tabb}


%%%%%%%%%%%%%%%%%%%%%%%%%%%%
\subsubsection*{Writing one-dimensional arrays}

\begin{code}

   public void write (String label, String[] a, int n) throws IOException \begin{hide} {
      writeLabel(label);
      out.writeByte(TYPECHAR_STRING);
      out.writeByte(1);
      out.writeInt(n);
      for (int i = 0; i < n; i++)
         writeStringData(a[i]);
   }
   \end{hide}
\end{code}
\begin{tabb}
Writes the first \texttt{n} elements of a one-dimensional array
of strings.
Writes an anonymous field if \texttt{label} is \texttt{null}.
\end{tabb}
\begin{code}

   public void write (String label, int[] a, int n) throws IOException \begin{hide} {
      writeLabel(label);
      out.writeByte(TYPECHAR_INTEGER);
      out.writeByte(1);
      out.writeInt(n);
      for (int i = 0; i < n; i++)
         out.writeInt(a[i]);
   }
   \end{hide}
\end{code}
\begin{tabb}
Writes the first \texttt{n} elements of a one-dimensional array
of 32-bit integers (big endian).
Writes an anonymous field if \texttt{label} is \texttt{null}.
\end{tabb}
\begin{code}

   public void write (String label, float[] a, int n) throws IOException \begin{hide} {
      writeLabel(label);
      out.writeByte(TYPECHAR_FLOAT);
      out.writeByte(1);
      out.writeInt(n);
      for (int i = 0; i < n; i++)
         out.writeFloat(a[i]);
   }
   \end{hide}
\end{code}
\begin{tabb}
Writes the first \texttt{n} elements of a one-dimensional array
of 32-bit floats (big endian).
Writes an anonymous field if \texttt{label} is \texttt{null}.
\end{tabb}
\begin{code}

   public void write (String label, double[] a, int n) throws IOException \begin{hide} {
      writeLabel(label);
      out.writeByte(TYPECHAR_DOUBLE);
      out.writeByte(1);
      out.writeInt(n);
      for (int i = 0; i < n; i++)
         out.writeDouble(a[i]);
   }
   \end{hide}
\end{code}
\begin{tabb}
Writes the first \texttt{n} elements of a one-dimensional array
of 64-bit doubles (big endian).
Writes an anonymous field if \texttt{label} is \texttt{null}.
\end{tabb}


%%%%%%%%%%%%%%%%%%%%%%%%%%%%
\subsubsection*{Writing two-dimensional arrays}

\begin{code}

   public void write (String label, String[][] a) throws IOException \begin{hide} {
      writeLabel(label);
      out.writeByte(TYPECHAR_STRING);
      out.writeByte(2);
      out.writeInt(a.length);
      out.writeInt(a[0].length);
      for (int i = 0; i < a.length; i++)
         for (int j = 0; j < a[i].length; j++)
            writeStringData(a[i][j]);
   }
   \end{hide}
\end{code}
\begin{tabb}
Writes a two-dimensional array of strings.
Writes an anonymous field if \texttt{label} is \texttt{null}.
\end{tabb}
\begin{code}

   public void write (String label, int[][] a) throws IOException \begin{hide} {
      writeLabel(label);
      out.writeByte(TYPECHAR_INTEGER);
      out.writeByte(2);
      out.writeInt(a.length);
      out.writeInt(a[0].length);
      for (int i = 0; i < a.length; i++)
         for (int j = 0; j < a[i].length; j++)
            out.writeInt(a[i][j]);
   }
   \end{hide}
\end{code}
\begin{tabb}
Writes a two-dimensional array of 32-bit integers (big endian).
Writes an anonymous field if \texttt{label} is \texttt{null}.
\end{tabb}
\begin{code}

   public void write (String label, float[][] a) throws IOException \begin{hide} {
      writeLabel(label);
      out.writeByte(TYPECHAR_FLOAT);
      out.writeByte(2);
      out.writeInt(a.length);
      out.writeInt(a[0].length);
      for (int i = 0; i < a.length; i++)
         for (int j = 0; j < a[i].length; j++)
            out.writeFloat(a[i][j]);
   }
   \end{hide}
\end{code}
\begin{tabb}
Writes a two-dimensional array of 32-bit floats (big endian).
Writes an anonymous field if \texttt{label} is \texttt{null}.
\end{tabb}
\begin{code}

   public void write (String label, double[][] a) throws IOException \begin{hide} {
      writeLabel(label);
      out.writeByte(TYPECHAR_DOUBLE);
      out.writeByte(2);
      out.writeInt(a.length);
      out.writeInt(a[0].length);
      for (int i = 0; i < a.length; i++)
         for (int j = 0; j < a[i].length; j++)
            out.writeDouble(a[i][j]);
   }
   \end{hide}
\end{code}
\begin{tabb}
Writes a two-dimensional array of 64-bit doubles (big endian).
Writes an anonymous field if \texttt{label} is \texttt{null}.
\end{tabb}


%%%%%%%%%%%%%%%%%%%%%%%%%%%%
\subsubsection*{Other methods}
\begin{code}

   public void close() throws IOException \begin{hide} {
      out.close();
   }
   \end{hide}
\end{code}
\begin{tabb}
Flushes any pending data and closes the file.
\end{tabb}

\begin{code}\begin{hide}
}
\end{hide}\end{code}
